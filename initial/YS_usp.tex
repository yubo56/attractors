    \documentclass[11pt,
        usenames, % allows access to some tikz colors
        dvipsnames % more colors: https://en.wikibooks.org/wiki/LaTeX/Colors
    ]{article}
    \usepackage{
        amsmath,
        amssymb,
        fouriernc, % fourier font w/ new century book
        fancyhdr, % page styling
        lastpage, % footer fanciness
        hyperref, % various links
        setspace, % line spacing
        amsthm, % newtheorem and proof environment
        mathtools, % \Aboxed for boxing inside aligns, among others
        float, % Allow [H] figure env alignment
        enumerate, % Allow custom enumerate numbering
        graphicx, % allow includegraphics with more filetypes
        wasysym, % \smiley!
        upgreek, % \upmu for \mum macro
        listings, % writing TrueType fonts and including code prettily
        tikz, % drawing things
        booktabs, % \bottomrule instead of hline apparently
        xcolor, % colored text
        cancel % can cancel things out!
    }
    \usepackage[margin=1in]{geometry} % page geometry
    \usepackage[
        labelfont=bf, % caption names are labeled in bold
        font=scriptsize % smaller font for captions
    ]{caption}
    \usepackage[font=scriptsize]{subcaption} % subfigures

    \newcommand*{\scinot}[2]{#1\times10^{#2}}
    \newcommand*{\dotp}[2]{\left<#1\,\middle|\,#2\right>}
    \newcommand*{\rd}[2]{\frac{\mathrm{d}#1}{\mathrm{d}#2}}
    \newcommand*{\pd}[2]{\frac{\partial#1}{\partial#2}}
    \newcommand*{\rdil}[2]{\mathrm{d}#1 / \mathrm{d}#2}
    \newcommand*{\pdil}[2]{\partial#1 / \partial#2}
    \newcommand*{\rtd}[2]{\frac{\mathrm{d}^2#1}{\mathrm{d}#2^2}}
    \newcommand*{\ptd}[2]{\frac{\partial^2 #1}{\partial#2^2}}
    \newcommand*{\md}[2]{\frac{\mathrm{D}#1}{\mathrm{D}#2}}
    \newcommand*{\pvec}[1]{\vec{#1}^{\,\prime}}
    \newcommand*{\svec}[1]{\vec{#1}\;\!}
    \newcommand*{\bm}[1]{\boldsymbol{\mathbf{#1}}}
    \newcommand*{\uv}[1]{\hat{\bm{#1}}}
    \newcommand*{\ang}[0]{\;\text{\AA}}
    \newcommand*{\mum}[0]{\;\upmu \mathrm{m}}
    \newcommand*{\at}[1]{\left.#1\right|}
    \newcommand*{\bra}[1]{\left<#1\right|}
    \newcommand*{\ket}[1]{\left|#1\right>}
    \newcommand*{\abs}[1]{\left|#1\right|}
    \newcommand*{\ev}[1]{\left\langle#1\right\rangle}
    \newcommand*{\p}[1]{\left(#1\right)}
    \newcommand*{\s}[1]{\left[#1\right]}
    \newcommand*{\z}[1]{\left\{#1\right\}}

    \newtheorem{theorem}{Theorem}[section]

    \let\Re\undefined
    \let\Im\undefined
    \DeclareMathOperator{\Res}{Res}
    \DeclareMathOperator{\Re}{Re}
    \DeclareMathOperator{\Im}{Im}
    \DeclareMathOperator{\Log}{Log}
    \DeclareMathOperator{\Arg}{Arg}
    \DeclareMathOperator{\Tr}{Tr}
    \DeclareMathOperator{\E}{E}
    \DeclareMathOperator{\Var}{Var}
    \DeclareMathOperator*{\argmin}{argmin}
    \DeclareMathOperator*{\argmax}{argmax}
    \DeclareMathOperator{\sgn}{sgn}
    \DeclareMathOperator{\diag}{diag\;}

    \colorlet{Corr}{red}

    % \everymath{\displaystyle} % biggify limits of inline sums and integrals
    \tikzstyle{circ} % usage: \node[circ, placement] (label) {text};
        = [draw, circle, fill=white, node distance=3cm, minimum height=2em]
    \definecolor{commentgreen}{rgb}{0,0.6,0}
    \lstset{
        basicstyle=\ttfamily\footnotesize,
        frame=single,
        numbers=left,
        showstringspaces=false,
        keywordstyle=\color{blue},
        stringstyle=\color{purple},
        commentstyle=\color{commentgreen},
        morecomment=[l][\color{magenta}]{\#}
    }

\begin{document}

\onehalfspacing

\section{Two Planet Mutual Precession}

Consider two planets mutually precessing. We calculate this in two ways below,
which are equivalent: the vector formulation, and as the eigenvalue of
Laplace-Lagrange theory. We compare these to the output of the RINGS code as a
third calculation.

\subsection{Vector Formulation}

We have
\begin{align}
    \rd{\uv{l}_1}{t} &= \omega_{21}\p{\uv{l}_1 \cdot \uv{l}_2}
            \p{\uv{l}_1 \times \uv{l}_2}\nonumber\\
        &= \omega_{21}\cos I_{12}
            \p{\uv{l}_1 \times \uv{l}_2},\\
    \rd{\uv{l}_2}{t} &= \frac{L_1}{L_2}\omega_{21}\cos I_{12}
        \p{\uv{l}_2 \times \uv{l}_1},\\
    \omega_{21}
        &= \frac{3 m_2}{4 M_\star}\p{\frac{a_1}{a_2}}^3 n_1
            f\p{\frac{a_1}{a_2}}\label{eq:w21},\\
    f\p{\alpha} &= \frac{b_{3/2}^{(1)}}{3\alpha}\nonumber\\
        &= \frac{1}{3\alpha}
            \frac{1}{\pi}\int\limits_0^{2\pi}
                \frac{\cos \p{t}}{
                \p{\alpha^2 + 1 - 2\alpha \cos t}^{3/2}}\;\mathrm{d}t\nonumber\\
        &\approx 1 + \frac{15}{8}\alpha^2 + \mathcal{O}\p{\alpha^4}.
\end{align}
In all numerical work, we calculate $f(\alpha)$ numerically, via direct
integration.

To get the precession rate of $\uv{l}_1$, we have shown many times that it
precesses around $\uv{j}$ with $\bm{J} = J\uv{j} = \bm{L}_1 + \bm{L}_2$ such
that
\begin{align}
    \rd{\uv{l}_1}{t} &= \omega_{21}\cos I_{21}
            \p{\uv{l}_1 \times \frac{\bm{L}_1 + \bm{L}_2}{L_2}}\nonumber\\
        &= \frac{J}{L_2}
            \omega_{21}\p{\uv{l}_1 \cdot \uv{l}_2}
                \p{\uv{l}_1 \times \uv{j}}\label{eq:2p_vec},
\end{align}
where again $\omega_{21}$ is given by Eq.~\eqref{eq:w21}.

\subsection{Laplace-Lagrange Formulation}\label{ss:LL}

Defining $\mathcal{I}_i = \abs{I_i} e^{i\Omega_i}$, the Laplace-Lagrange secular
theory says that:
\begin{align}
    \rd{}{t}\begin{bmatrix}
        \mathcal{I}_1\\\dots\\\mathcal{I}_N
    \end{bmatrix}
        &= \tilde{\bm{B}} \begin{bmatrix}
            \mathcal{I}_1\\\dots\\\mathcal{I}_N
        \end{bmatrix},\\
    B_{jk} &=
        -\frac{3 m_k}{4 M_\star}\p{\frac{a_i}{a_j}}^2
            \min\p{\frac{a_i}{a_j}, 1}
            n_j f\p{\frac{a_i}{a_j}},\\
    B_{jj} &= \sum\limits_{k \neq j}
        -B_{jk}.\label{eq:2p_LL}
\end{align}
In the two-planet case, $\tilde{\bm{M}}$ is just a $2 \times 2$ matrix, and the
only nonzero eigenvalue is the precession frequency (the other has eigenvalue
zero and corresponds to the total angular momentum). It is believable that this
is in agreement with Eq.~\eqref{eq:2p_vec} if we construct the corresponding
eigenvector and calculate its eigenvalue, and this appears to be the case.

\subsection{RINGS calculation}

For a given set of two-planet parameters, we can use RINGS to calculate their
dynamical evolution. The fiducial parameters we choose are:
\begin{align}
    a_1 &= 0.035\;\mathrm{AU} & m_1 &= M_{\oplus} & I_1 &= 1^\circ\\
    && m_2 &= 10M_\oplus & I_2 &= 5^\circ.
\end{align}
We choose $\Omega_i = \omega_i = 0$ for simplicity. We can then run RINGS and
try to extract the precession frequencies. To do this, I took the Fourier
Transform of $I_1(t)$ and found the frequency with the largest amplitude.

The results of the comparison among these three methods where $a_2$ is varied is
shown in Fig.~\ref{fig:scan}.
\begin{figure}
    \centering
    \includegraphics[width=0.7\columnwidth]{4_usp_rings/scan_alpha.png}
    \caption{(Top \& middle) plot of precession frequencies obtained using three
    methods as a function of $a_2$ for two and three-planet systems. Good
    agreement is observed in two-planet cases, while the three-planet systems
    have some interesting behavior. The bottom panel shows the quotient of the
    largest vector-calculated frequency and the numerical value for the 3-planet
    case; convincing agreement is observed. Frequencies are }\label{fig:scan}
\end{figure}

\section{Three-Planet Case}

\subsection{Vector Calculation}

We now add a third planet. For simplicity, we will neglect the precession
induced on $\bm{L}_2$ and $\bm{L}_3$ by $\bm{L}_1$, then:
\begin{align}
    \rd{\uv{l}_1}{t} &= \omega_{21}\cos I_{12} \p{\uv{l}_1 \times \uv{l}_2}
        + \omega_{31}\cos I_{13} \p{\uv{l}_1 \times \uv{l}_3}
        \nonumber\\
    \rd{\uv{l}_2}{t} &= \omega_{32}\cos I_{23} \p{\uv{l}_2 \times \uv{l}_3},\\
    \rd{\uv{l}_3}{t} &= \frac{L_2}{L_3}
        \omega_{32}\cos I_{23} \p{\uv{l}_3 \times \uv{l}_2},\\
    \omega_{jk}
        &= \frac{3 m_k}{4 M_\star}\p{\frac{a_j}{a_k}}^3 n_j
            f\p{\frac{a_j}{a_k}}.
\end{align}
We expect two eigenvectors, one where $\uv{l}_1$ is evolving and one where the
other two are evolving. The latter has the precession frequency
\begin{equation}
    \Omega_{\rm 32} = \frac{J_{23}}{L_3}\omega_{32}.
\end{equation}
The former has the precession frequency:
\begin{align}
    \rd{\uv{l}_1}{t} &= -\s{
        \omega_{21}\cos I_{12} \uv{l}_2 + \omega_{31}\cos I_{13} \uv{l}_3
    } \times \uv{l}_1 \nonumber\\
        &\equiv \bm{\Omega}_{32,1} \times \uv{l}_1,\\
    \Omega_{32, 1}
        &\approx \omega_{21} \cos I_{12}
            + \omega_{31}\cos I_{13}
            + \mathcal{O}\p{\cos I_{23}}.
\end{align}
Here, we have simply assmued that $\uv{l}_2$ is approximately aligned with
$\uv{l}_3$; by the law of cosines, the deviation scales with $\cos I_{23} \sim
2^\circ$ for compact architectures excepting the innermost planet.

We want to know which of $\Omega_{32}$ and $\Omega_{32, 1}$ are larger. To
calculate scalings, let's approximate $I_{12} \approx I_{13} \approx I_{23}
\approx 0$. If we assume the semimajor axis ratios are constant, $a_3 / a_2 =
a_2 / a_1 = \alpha$, then
\begin{align}
    \frac{\Omega_{32, 1}}{\Omega_{32}}
        &= \frac{m_2 \alpha^3 n_1 f(\alpha)
            + m_3 \alpha^6 n_1 f(\alpha^2)}{m_3 \alpha^3 n_2
            f(\alpha)}\frac{L_3}{J_{23}}\nonumber\\
        &= \frac{L_3}{J_{23}}\p{
            \frac{m_2}{m_3}\alpha^{-3/2}
                + \alpha^{3/2}\frac{f\p{\alpha^2}}{f(\alpha)}}.
\end{align}
Note that $J_{23} / L_3 \approx (m_2/m_3)\alpha^{1/2} + 1$. There are thus two
ways to obtain $\Omega_{32, 1} < \Omega_{32}$:
\begin{itemize}
    \item If we have $\alpha \approx m_2/m_3 \approx f\p{\alpha^2} / f(\alpha)
        \approx 1$, then $J_{23} / L_3 \approx 2$ and $\Omega_{32, 1} /
        \Omega_{32} \approx 0.5$, so not a very large ratio.

    \item A much larger ratio can be obtained if $m_2/m_3 \ll \alpha^{3/2} \ll
        1$.
\end{itemize}
Thus, in general, we find that the $\Omega_{32, 1}$ precession frequency should
generally be larger or comparable except for very small $a_{j + 1} / a_j$.

\subsection{Comparison with Other Results}

The LL results follow straightforwardly from Section~\ref{ss:LL}. The RINGS
simulations can yield two precession frequencies if we seek the first two
non-commensurate, non-adjacent frequencies in the FT of $I_1(t)$; note that
choosing $\Omega_{i} = \omega_i = 0$ means that $\uv{l}_2 = \uv{l}_3$ initially,
but due to different backreaction torques from $\uv{l}_1$ they eventually
misalign and give rise to the $\Omega_{32}$ mode. This is shown in the middle
panel of Fig.~\ref{fig:scan}. Finally, in the bottom panel, we show the quotient
of $\Omega_{32, 1}$ and the numerically-determined maximum precession frequency.
We see fractional deviations, showing that $\Omega_{32, 1}$ is a good estimate
of $g_{\max}$.

\subsection{Inner Mass Dependence}

In the above, we have considered the case where $m_1 = M_{\odot}$ and $m_2 = m_3
= 10M_{\odot}$, satisfying our ``no backreaction'' assumption. However, we can
also consider the case closer to the Millholland paper, where $m_i =
5M_{\odot}$. These results are shown in Fig.~\ref{fig:scan5}. For some reason,
the agreement \emph{improves}?
\begin{figure}
    \centering
    \includegraphics[width=0.7\columnwidth]{4_usp_rings/scan_alpha5.png}
    \caption{Same as Fig.~\ref{fig:scan} but for equal masses $m_i =
    5M_{\odot}$.}\label{fig:scan5}
\end{figure}

\section{Resulting Impact on USP Formation Story}

In the 3p systems, we see that $\abs{g_{\max}} \lesssim 0.2 \;\mathrm{yr^{-1}}$.
At the same time, we can calculate the spin-orbit precession frequency:
\begin{align}
    \alpha &= \frac{3k_q}{2k}\frac{M_\star}{m}
        \p{\frac{R}{a}}^3 \Omega_{\rm s},\\
        &\approx 0.86\;\mathrm{yr^{-1}}
            \frac{k_q}{k}
            \p{\frac{M_{\star}}{M_{\odot}}}^{3/2}
            \p{\frac{m}{M_{\oplus}}}^{-1}
            \p{\frac{R}{R_{\oplus}}}^3
            \p{\frac{a}{0.035\;\mathrm{AU}}}^{-9/2}
            \p{\frac{\Omega_{\rm s}}{n}}.
\end{align}
This shows that $\eta_{\rm sync} \equiv \abs{g_{\max}} / \alpha \lesssim 0.15$
very optimistically.

Using Millholland's formula, we have yet another small correction. They use
$3k_q = 0.4$ and $k = 0.35$, so their $\alpha$ is depressed by a factor of
$8/21$, leaving $\eta_{\rm sync} \sim 0.4$ for extremal values (note that $a_2 /
a_1 = 1.2$ corresponds to a period ratio of $1.3$). So indeed, we should expect
that $\eta_{\rm sync}$ is only $\gtrsim 1$ if $a_2 / a_1 \lesssim 1.2$.

\subsection{Double Checking $\eta_{\rm sync}$ for 2p}

Recall that we had:
\begin{align}
    \eta_{\rm sync} &= \frac{m_pm}{2M_\star^2}\p{\frac{a}{a_{\rm p}}}^3
            \p{\frac{a}{R}}^3,\\
% (8 * (Mearth)^2 / (Msun)^2) / 2 * (2)^(-3) * ((0.04 AU) / (2 Rearth))^3
% 0.00467
        &= 4.67 \times 10^{-4}
            \cos I \frac{k}{k_{\rm q}}f\p{\frac{1}{2}}
            \frac{m_pm}{8(M_{\odot}^2)}
            \p{\frac{M_\star}{M_{\odot}}}^{-2}
            \p{\frac{a/a_{\rm p}}{1/2}}^3
            \p{\frac{a}{0.04\;\mathrm{AU}}}^3
            \p{\frac{R}{2R_{\oplus}}}^{-3},\\
% (10 * (Mearth)^2 / (Msun)^2) / 2 * (1.2)^(-3) * ((0.035 AU) / (Rearth))^3
% 0.014480
        &= 0.0145
            \cos I \frac{k}{k_{\rm q}}f\p{\frac{1}{1.2}}
            \p{\frac{m_p}{10M_{\odot}}}
            \p{\frac{\rho}{\rho_{\oplus}}}
            \p{\frac{M_\star}{M_{\odot}}}^{-2}
            \p{\frac{a/a_{\rm p}}{1/1.2}}^3
            \p{\frac{a}{0.035\;\mathrm{AU}}}^3.
\end{align}
Note that $f(0.5) = 1.72$ while $f(1/1.2) = 9.69$ (NB\@: the power series for
$\alpha$ converges way too slowly for the latter result, have to evaluate
numerically), so this is already a bit more promising.

\subsection{Towards $\eta_{\rm sync}$ for 3p? No}

We showed above that
\begin{align}
    \abs{g_{\max}} &\approx \Omega_{\rm 32, 1}\nonumber\\
        &\approx \omega_{21}\cos I_{12}
            \p{1 + \frac{m_3}{m_2}\alpha^3\frac{f\p{\alpha^2}}{f(\alpha)}
                \frac{\cos I_{13}}{\cos I_{12}}}.
\end{align}
Trying to express this in any sort of scaling way is difficult since $f(\alpha)$
cannot be expressed analytically for sufficiently large $\alpha$. Nevertheless,
for the fiducial $\alpha = 1/1.2$ and $m_3 = m_2$, we find that this enhancement
is only $1.2\times$, and if we account for the fact that our naive prediction
is small by about $15\%$, the third planet only contributes $40\%$ of the
precession rate of the inner. This is unsurprising, and is not enough to lift
$\eta_{\rm sync}$ to values of order unity; it raises it only to $\sim 0.2
k/k_{\rm q}$, which is in line with our calculation above. So everything seems
consistent!

\end{document}

