    \documentclass[11pt,
        usenames, % allows access to some tikz colors
        dvipsnames % more colors: https://en.wikibooks.org/wiki/LaTeX/Colors
    ]{article}
    \usepackage{
        amsmath,
        amssymb,
        fouriernc, % fourier font w/ new century book
        fancyhdr, % page styling
        lastpage, % footer fanciness
        hyperref, % various links
        setspace, % line spacing
        amsthm, % newtheorem and proof environment
        mathtools, % \Aboxed for boxing inside aligns, among others
        float, % Allow [H] figure env alignment
        enumerate, % Allow custom enumerate numbering
        graphicx, % allow includegraphics with more filetypes
        wasysym, % \smiley!
        upgreek, % \upmu for \mum macro
        listings, % writing TrueType fonts and including code prettily
        tikz, % drawing things
        booktabs, % \bottomrule instead of hline apparently
        cancel % can cancel things out!
    }
    \usepackage[margin=1in]{geometry} % page geometry
    \usepackage[
        labelfont=bf, % caption names are labeled in bold
        font=scriptsize % smaller font for captions
    ]{caption}
    \usepackage[font=scriptsize]{subcaption} % subfigures

    \newcommand*{\scinot}[2]{#1\times10^{#2}}
    \newcommand*{\dotp}[2]{\left<#1\,\middle|\,#2\right>}
    \newcommand*{\rd}[2]{\frac{\mathrm{d}#1}{\mathrm{d}#2}}
    \newcommand*{\pd}[2]{\frac{\partial#1}{\partial#2}}
    \newcommand*{\rtd}[2]{\frac{\mathrm{d}^2#1}{\mathrm{d}#2^2}}
    \newcommand*{\ptd}[2]{\frac{\partial^2 #1}{\partial#2^2}}
    \newcommand*{\md}[2]{\frac{\mathrm{D}#1}{\mathrm{D}#2}}
    \newcommand*{\pvec}[1]{\vec{#1}^{\,\prime}}
    \newcommand*{\svec}[1]{\vec{#1}\;\!}
    \newcommand*{\bm}[1]{\boldsymbol{\mathbf{#1}}}
    \newcommand*{\ang}[0]{\;\text{\AA}}
    \newcommand*{\mum}[0]{\;\upmu \mathrm{m}}
    \newcommand*{\at}[1]{\left.#1\right|}

    \newtheorem{theorem}{Theorem}[section]

    \let\Re\undefined
    \let\Im\undefined
    \DeclareMathOperator{\Res}{Res}
    \DeclareMathOperator{\Re}{Re}
    \DeclareMathOperator{\Im}{Im}
    \DeclareMathOperator{\Log}{Log}
    \DeclareMathOperator{\Arg}{Arg}
    \DeclareMathOperator{\Tr}{Tr}
    \DeclareMathOperator{\E}{E}
    \DeclareMathOperator{\Var}{Var}
    \DeclareMathOperator*{\argmin}{argmin}
    \DeclareMathOperator*{\argmax}{argmax}
    \DeclareMathOperator{\sgn}{sgn}
    \DeclareMathOperator{\diag}{diag\;}

    \DeclarePairedDelimiter\bra{\langle}{\rvert}
    \DeclarePairedDelimiter\ket{\lvert}{\rangle}
    \DeclarePairedDelimiter\abs{\lvert}{\rvert}
    \DeclarePairedDelimiter\ev{\langle}{\rangle}
    \DeclarePairedDelimiter\p{\lparen}{\rparen}
    \DeclarePairedDelimiter\s{\lbrack}{\rbrack}
    \DeclarePairedDelimiter\z{\lbrace}{\rbrace}

    % \everymath{\displaystyle} % biggify limits of inline sums and integrals
    \tikzstyle{circ} % usage: \node[circ, placement] (label) {text};
        = [draw, circle, fill=white, node distance=3cm, minimum height=2em]
    \definecolor{commentgreen}{rgb}{0,0.6,0}
    \lstset{
        basicstyle=\ttfamily\footnotesize,
        frame=single,
        numbers=left,
        showstringspaces=false,
        keywordstyle=\color{blue},
        stringstyle=\color{purple},
        commentstyle=\color{commentgreen},
        morecomment=[l][\color{magenta}]{\#}
    }

\begin{document}

\def\Snospace~{\S{}} % hack to remove the space left after autorefs
\renewcommand*{\sectionautorefname}{\Snospace}
\renewcommand*{\appendixautorefname}{\Snospace}
\renewcommand*{\figureautorefname}{Fig.}
\renewcommand*{\equationautorefname}{Eq.}
\renewcommand*{\tableautorefname}{Tab.}

\section{Constant $\eta$}

\subsection{Toy Problem}

Consider simplest spin Hamiltonian $H = -\vec{B} \cdot \vec{s}$. It's clear that
if we set up initial conditions $\vec{s}$ misaligned from $\vec{B}$, it will
simply spin around $\vec{B}$, which is fixed. Thus, let $\hat{B} \cdot \hat{s} =
\cos \theta$ the angle between the two, and let $\phi$ measure the azimuthal
angle.

We claim that $\cos \theta, \phi$ are canonical variables. Since $\phi$ is
ignorable, immediately $\rd{\theta}{t} = \rd{\cos \theta}{t} = -\pd{H}{\phi} =
0$, while $\rd{\phi}{t} = \pd{H}{(\cos \theta)} = Bs$ tells us the rate at which
the spin precesses around $\vec{B}$.

\subsection{Cassini State Hamilttonian}

This Hamiltonian is Kassandras Eq.\ 13, in the co-rotating frame with the
perturber's angular momentum:
\begin{equation}
    \mathcal{H} = \frac{1}{2}\p*{\hat{s} \cdot \hat{l}}^2
        - \eta \p*{\hat{s} \cdot \hat{l}_p}.
\end{equation}
In this frame, we can choose $\hat{l} \equiv \hat{z}$ fixed, and $\hat{l}_p =
\cos I\hat{z} + \sin I\hat{x}$ fixed as well. Then
\begin{equation*}
    \hat{s} = \cos\theta \hat{z}
        - \sin\theta\p*{\sin \phi \hat{y} + \cos \phi \hat{x}}.
\end{equation*}
We can choose the convention for $\phi = \phi$ azimuthal angle requiring $\phi =
0, \pi$ mean coplanarity between $\hat{s}, \hat{l}, \hat{l}_p$ in the $\hat{x},
\hat{z}$ plane such that $\hat{l}_p, \hat{s}$ lie on the same side of
$\hat{l}$. Then we can evaluate in coordinates
\begin{align*}
    \hat{s} \cdot \hat{l} &= \cos \theta,\\
    \hat{s} \cdot \hat{l}_p
        &= \cos \theta \cos I - \sin I \sin \theta \cos \phi,\\
    \mathcal{H} &= -\frac{1}{2}\cos^2\theta
        + \eta \p*{\cos \theta \cos I - \sin I \sin \theta \cos \phi}.
\end{align*}
Note that if we take $\cos\theta$ to be our canonical variable, $\sin\theta =
\sqrt{1 - \cos^2\theta}$ can be used.

\subsection{Equation of Motion}

The correct EOM comes from Kassandra's Eq.\ 12:
\begin{align*}
    \rd{\hat{s}}{t} &=
        \p*{\hat{s} \cdot \hat{l}}\p*{\hat{s} \times \hat{l}}
            -\eta\p*{\hat{s} \times \hat{l}_p},\\
        &= \p*{s_ys_z - \eta s_y\cos(I)}\hat{x}
            - \p*{s_xs_z + \eta \p*{s_x\cos I - s_z\sin I}}\hat{y}
            + \eta s_y \sin(I)\hat{z}.
\end{align*}

Alternatively, consider Hamilton's equations applied to the Hamiltonian:
\begin{align}
    \pd{\phi}{t} = \pd{\mathcal{H}}{(\cos\theta)}
        &= -\cos\theta + \eta\p*{\cos I + \sin I \cot \theta \cos \phi},\\
    \pd{(\cos \theta)}{t} = -\pd{\mathcal{H}}{\phi}
        &= -\eta \sin I \sin \theta \sin \phi.\label{eq:H_eom}
\end{align}
This produces the same trajectories as the Cartesian EOM, so this is correct.
However, since $\pd{\phi}{t} \propto 1/\sin\theta$, this is not a desirable
system of equations to use, as they are very stiff near $\theta \approx 0$.

\subsection{Cassini States}

The zeros to \autoref{eq:H_eom} are the Cassini states; we will go to canonical
variables $\mu = \cos\theta$. We can immediately see that $\sin\phi = 0$ is
necessary, so $\cos \phi = \pm 1$ and we need only solve for $\pd{\phi}{t} = 0$.
We can furthermore separate the problem into two regimes, $\eta \ll 1$ and $\eta
\gg 1$.

For $\eta \ll 1$, it is clear that there will be two solutions near $\mu^2 = 1$
and two solutions near $\mu = 0$:
\begin{itemize}
    \item For $\mu = 1 - \frac{\theta^2}{2}$, the dominant terms are
        $\pd{\phi}{t} \approx -1 + \eta \sin I \frac{1}{\theta} = 0$,
        where we've taken $\cos \phi = +1$ and $\phi = 0$. This forces $\theta
        = \eta \sin I$.

    \item Similarly, for $\mu = -1 + \frac{\epsilon^2}{2}$, $\phi = 0$ and
        $\epsilon = \eta \sin I$ again. This actually corresponds to $\theta =
        \pi - \eta \sin I$.

    \item For $\mu \approx 0$, we have instead $\pd{\phi}{t} = -\mu\p*{1 - \eta
        \sin I \cos \phi} + \eta \cos I = 0$. This forces $\mu_\pm = \frac{\eta
        \cos I}{1 \pm \eta \sin I}$, where $\phi_{\pm} = \pi, 0$ respectively.

        Note that $\phi = 0, \mu \approx 0$ is conventionally CS4. The
        linearization locally has form $\pd{\delta \phi}{t} = -\delta \mu\p*{1 -
        \eta \sin I}$ and $\pd{\delta \mu}{t} = -\eta \sin I \delta \phi$, so
        the eigenvalues are $\approx \mp \sqrt{\eta \sin I}$, and the two
        eigenvectors are $\p*{1, \pm \sqrt{\eta \sin I}}$.
\end{itemize}

For $\eta \gg 1$, the solutions obviously just come from $\cos I \pm \sin I \cot
\theta = 0$, which are just $\sin (I \pm \theta) = 0$

\subsection{Separatrix Area}

We can estimate the area enclosed by the separatrix, as shown in
\autoref{fig:1contours}. Note that the separatrix joins Cassini State 4 to its
$+ 2\pi$ image.
\begin{figure}[t]
    \centering
    \includegraphics[width=0.6\textwidth]{0_eta/1contours.png}
    \caption{Separatrix for various values of $\eta$.}\label{fig:1contours}
\end{figure}

We notate $\mu = \cos\theta$; note that CS4 is $\mu_4 \approx \frac{\eta \cos
I}{1 - \eta \sin I} \approx \eta \cos I$. Setting the Hamiltonian equal to its
value at CS4 gives
\begin{align*}
    H_4 &\equiv H\p*{\mu_4, \phi_4}
        \approx -\frac{\mu_4^2}{2} + \eta \mu_4 \cos I - \eta \sin I,\\
        &= +\eta^2\cos^2 I - \eta \sin I,\\
    H(\mu_{sep}, \phi_{sep})
        &= H_4 = -\eta \sin I \cos \phi_{sep} - \frac{\mu_{sep}^2}{2} + \eta
            \mu_{sep} \cos I + \mathcal{O}(\eta^3),\\
    0 &\approx \frac{\mu_{sep}^2}{2} - \eta \mu_{sep} \cos I
        - \eta \sin I\p*{1 - \cos \phi_{sep}} + \eta^2\cos^2I,\\
    \mu_{sep}\p*{\phi} &\approx \sqrt{2\eta \sin I\p*{1 - \cos \phi}}
        + \mathcal{O}(\eta).
\end{align*}

We can then easily compute the area enclosed by the separatrix
\begin{align}
    A_{sep} &= \int\limits_0^{2\pi}2\mu_{sep}\;\mathrm{d}\phi,\nonumber\\
        &\approx 16\sqrt{\eta \sin I}.
\end{align}
For $\eta = 0.1, I = 20^\circ$, this predicts $\frac{A_{sep}}{A_{T}} \approx
0.235$, which is pretty close to my numerically calculated $\frac{A_{sep}}{A_T}
= 0.229$.

\subsection{Tidal Dissipation}

We can add a tidal dissipation term; we write it in form
$\p*{\rd{\hat{s}}{t}}_{tide} = \epsilon \hat{s} \times \p*{\hat{l} \times
\hat{s}} = \epsilon\p*{\vec{l} - \p*{\vec{s} \cdot \vec{l}}\vec{s}}$. Expanding,
\begin{align}
    \p*{\rd{\hat{s}}{t}}_{tide} &= \epsilon \p*{\hat{z} - s_z\hat{s}}
        ,\nonumber\\
        &= \epsilon \p*{-s_z s_x\hat{x} - s_zs_y\hat{y} + \p*{1 - s_z^2}
            \hat{z}}.
\end{align}
We run numerical simulations for weaker $\epsilon \ll \eta \ll 1$ and stronger
$\epsilon \lesssim \eta \ll 1$.

We can seek equilibria of the the system including tides, which requires
\begin{align*}
    0 &= s_ys_z - \eta s_y\cos I - \epsilon s_zs_x,\\
    0 &= -s_xs_z - \eta\p*{s_x \cos I - s_z \sin I} - \epsilon s_zs_y,\\
    0 &= \eta s_y\sin(I) + \epsilon \p*{1 - s_z^2}.
\end{align*}
We expect at least two equilibria, based on the simulations: one near $s_z
\approx 1$ and one $s_z \approx 0$.

For near alignment/near Cassini state $1$, $1 - s_z \sim 1 - s_{\perp}^2$, so we
can set $s_z = 1$ to first order: $s_y - \epsilon s_x - \eta s_y \cos I = -s_x -
\eta\p*{s_x \cos I - \sin I} - \epsilon s_y = \eta s_y \sin I = 0$. This can be
satisfied if we set $s_x = \tan(I) \ll 1, s_y = \mathcal{O}\p*{\epsilon s_x}$;
this coarsely corresponds to Cassini state $1$.

The other solution should be near Cassini state $2$, where $s_x \approx 1$;
dropping second order terms forces $\eta s_y + \epsilon s_z = -s_z -
\eta\p*{\cos I - s_z\sin I} = \eta s_y \sin(I) + \epsilon = 0$. This can thus be
satisfied for $s_y \approx -\frac{\epsilon}{\eta \sin(I)}$. Thus, this explains
why as $\epsilon$ is increased, we first start to get points that don't converge
to Cassini state $2$ in the absence of tides, before starting to see points that
fail to converge to Cassini state $1$.

\section{Separatrix Hopping}

Inspired by G\&H, heteroclinic orbits are topologically unstable for any nonzero
perturbation, but opened width $\sim$ perturbation parameter.

\subsection{Consideration 1: Qualitative}

We zoom in on Cassini State $4$, which has $\theta_4 =
-\frac{\pi}{2} + \frac{\eta \cos I}{1 - \eta \sin I}, \mu_4 = \frac{\eta \cos
I}{1 - \eta \sin I}, \phi_4 = 0$. Then, using
equations of motion
\begin{align}
    \pd{\phi}{t} &= \mu - \eta\p*{\cos I + \sin I \frac{\mu}{\sqrt{1 - \mu^2}}
        \cos \phi} ,\\
    \pd{\mu}{t} &= -\eta \sin I \sin \phi + \s*{\epsilon \p*{1 - \mu^2}},
\end{align}
we can perturbatively require $\pd{\theta}{t} = 0$ for $\epsilon \neq 0$. This
corresponds to $\eta \sin I \sin \p*{\phi_4 + \delta \phi} \approx \epsilon$, or
$\delta \phi_4 = +\frac{\epsilon}{\eta \sin I}$. This is in agreement with
Dong's result.

This implies that the stable manifolds of the two saddle points, which once
overlapped with each other's unstable manifolds (creating a heteroclinic orbit)
now are offset from one another by distance $D \sim \frac{\epsilon}{\eta \sin
I}$. But since $\epsilon$ also sets $\dot{\mu}$ in a precession orbit-averaged
sense, the effective cross section is constant in some sense: there will be one
orbit where $\mu$ goes from below CS4 to above CS4, during which it will make
jump of size $\epsilon$, and if it hits a particular interval of size $\epsilon$
then it will enter the separatrix. Thus, separatrix hopping should $\propto
\epsilon^0$.

\subsection{Consideration 2: Melnikov Distance}

We notice that the separatrix is a heteroclinic orbit, or a saddle connection,
in the dissipation free problem. Introducing dissipation breaks the saddle
connection by a distance that can be estimated with the Melnikov distance. This
is G\&H Equation 4.5.11 or something:
\begin{align}
    d(t_0) &= \frac{\epsilon M(t_0)}{\abs*{f(q^0(0))}} + \mathcal{O}(\epsilon^2)
        ,\\
    M(t_0) &= \int\limits_{-\infty}^\infty
        \s*{f \times g}_{hetero}\;\mathrm{d}t.
\end{align}

This is not a hard formula to understand; along the separatrix, motion is
dominated by $f$, but the perpendicular component adds up to contribute to a
total ``perpendicular distance away from the original separatrix'' necessary to
hit the saddle point, at least intuitively.

We evaluate the Melnikov integral $M(t_0)$ on the heteroclinic orbit. Note that
since in our problem our perturbation $g$ is time-independent, so too is the
Melnikov integral $M(t_0) = M$.

Let's apply this to the Cassini state Hamiltonian w/ dissipation. We first write
down our EOM in Melnikov form (we use canonical variables $\mu, \phi$):
\begin{align}
    \rd{\hat{s}}{t}
        &= \underbrace{
            \pd{\mathcal{H}}{\mu}\hat{\phi} - \pd{\mathcal{H}}{\phi}\hat{\mu}}_f
            + \epsilon \underbrace{\p*{1 - \mu^2}\hat{\mu}}_g.
\end{align}
Then $f \times g = f_\phi g_\mu = \pd{\mathcal{H}}{\mu}\p*{1 - \mu^2}$. We then
want to integrate this along the heteroclinic orbit. We can make change of
variables
\begin{equation}
    M = \int\limits_0^{2\pi}\pd{\mathcal{H}}{\mu}
        \p*{1 - \mu^2}\p*{\pd{\phi}{t}}^{-1}\;\mathrm{d}\phi.
\end{equation}
But thankfully, $\pd{\mathcal{H}}{\mu} = \pd{\phi}{t}$ in the absence of
dissipation, and so $M = 2\pi\p*{1 - \mu^2} \approx 2\pi$.
Thus, the Melnikov distance at point $q^0$, a point on the heteroclinic
orbit of the unperturbed Hamiltonian, is just
\begin{equation}
    d(q^0) = \frac{2\pi \epsilon}{\abs*{f(q^0)}}.
\end{equation}
Note that the maximum value $\abs*{f(q^0)}$, which occurs at $\phi = \pi$, is
just $f_{\max} \approx \sqrt{4\eta \sin I}$.

It proves to be a bit difficult to make quantitative predictions though, since
the phase diagram is very smushed where $f$ is large, and $d$ is rather
inaccurate where $f$ is small. Let's think about a Poincar\'e map instead.

\subsection{Consideration 3: Poincar\'e Section}

Let's consider the Poincar\'e section every time $\phi = \phi_4$ as the
trajectory subject to tidal dissipation is moving $\theta < \theta_4 \to
\theta_4$. To provide an estimate of $\Delta \theta(\theta) = \theta_{n - 1} -
\theta_n$, this is just $\epsilon T$ where $T$ is the time elapsed between
$\theta_n, \theta_{n + 1}$, the period of the orbit. $T$ is dominated by when
$\pd{\phi}{t} \ll 1$ though, or where the orbit is close to the saddle point.

Note that $T$ is dominated by the time it spends near the saddle point. We
showed earlier that near CS4, $\pd{\phi}{t} \approx \delta \mu$ where $\delta
\mu = \mu - \mu_4$. Thus, we might surmise $\Delta \theta(\theta) \propto
\theta^{-1}$ for sufficiently small $\theta - \theta_4$. Far away, $T$ is
roughly constant and $\Delta \theta(\theta)$ is roughly constant.

What is ``far away''? Well, it probably depends on how affected our trajectory
is by the separatrix; far away from the saddle point, we go along contours of
roughly constant $\theta$, while close by we follow the separatrix pretty well.
We computed earlier that $\mu_{sep} \sim \sqrt{4\eta \sin I}$, so we might
expect $\mu > \mu_{sep}, \Delta \mu \sim C$, while $\mu < \mu_{sep}, \Delta
\mu \sim \delta \mu^{-1}$.

My $\mu > \mu_{sep}$ simulations don't seem to work very well, so I'll focus on
the $\delta \mu^{-1}$ case. In this case, define $\delta \mu_c: \Delta
\mu\p*{\delta \mu_c} = -\delta \mu_c$, i.e.\ the point that jumps immediately to
the saddle point. Furthermore, assume the inbound distribution is flat between
$\delta \mu_c, f^{-1}\p*{\delta \mu_c}$. TODO\@: empirically, $\mu_c \sim
\epsilon T$ is \emph{flat} with $\eta$, probably just because we're not getting
sufficiently close to the saddle point for the $\propto \sqrt{\eta}$ to kick in.

Then, we can compare the empirical Poincar\'e section of the points that cross
the separatrix versus the total predicted interval width $\delta \mu_c,
f^{-1}\p*{\delta \mu_c}$; this would predict $7.2\%, 18\%$. This does alright!

\subsection{Consideration 4: Plotting Stable/Unstable manifolds}

We can plot the stable/unstable manifolds of two Cassini States as in
\autoref{fig:CS4s}. Then, since phase space is roughly flat near $\phi = \pi$
(near $\phi = 0$, $\dot{\phi}$ varies drastically and so phase space is
``squished'' a bit via Liouville's Theorem), we just need to compare the
distance between $\mathcal{W}^{(0)}_S$ and $W_U^{(0)}$, the capture gap, to the
distance between $\mathcal{W}^{(1)}_S$ and $W_U^{(0)}$ the Melnikov gap, to
estimate the capture probability.
\begin{figure}[t]
    \centering
    \includegraphics[width=0.6\textwidth]{0_eta/6manifolds0_10.png}
    \caption{Stable/Unstable manifolds of the two Cassini State
    4s.}\label{fig:CS4s}
\end{figure}

The Melnikov gap is predicted above as $d(q^0)$ or approximately
\begin{equation}
    \Delta_{M} \approx \frac{2\pi \epsilon}{\sqrt{4\eta\sin I}}.
\end{equation}

The capture gap is much trickier to predict, since it depends on the separation
between $\mathcal{W}^{(0)}_S$ \emph{after passing through} CS$_4^{(1)}$.
Instead, let's consider the closed orbit in the absence of dissipation that
starts at CS4', the modified CS\@. This orbit has a finite period set by
equating $\int\limits_0^T \pd{\phi}{t}\;\mathrm{d}t =
\int\limits_0^{2\pi}\;\mathrm{d}\phi + \int\limits_{2\pi}^0\;\mathrm{d}\phi$.

Now, let's reconsider the Melnikov integral when perturbing this finite
(non-homoclinic orbit); this may no longer be an exact result but should give
the correct scaling:
\begin{equation}
    M_c = \int\limits_0^T \pd{\phi}{t}\epsilon \p*{1 - \mu^2}\;\mathrm{d}t.
\end{equation}
Naively, we might claim that, since $\pd{\phi}{t}$ changes signs halfway through
the interval of integration, that the only surviving component is $2\epsilon
\bar{\mu}\mu'$, where $\bar{\mu} = \mu_4$ is the average value of $\mu$ and
$\mu'$ is the fluctuation. This gives
\begin{equation}
    M_c = 2\int\limits_0^{2\pi}2\epsilon \frac{\eta \cos I}{1 + \eta \sin I}
        \sqrt{2\eta \sin I \p*{1 - \cos \phi}}\;\mathrm{d}\phi.
\end{equation}
Note that $M_c \propto \eta^{3/2}$, and since the gap opened $\Delta_c =
\frac{M_c}{\pd{\phi}{t}} \propto \eta$, it seems like we're on the right track.
Specifically:
\begin{align}
    M_c &\approx \epsilon 2\eta \cos I A_{sep},\\
    \Delta_c &\approx 2\epsilon \eta \cos I \p*{16 \sqrt{\eta \sin I}}
        \frac{1}{\sqrt{4\eta \sin I}},\\
        &\approx 8\epsilon \eta \cos I.
\end{align}
This also agrees exceedingly well with our simulations This then gives us
hopping probability
\begin{equation}
    P_{hop} = \frac{\Delta_c}{\Delta_M} \approx
        \frac{16\eta^{3/2}\cos I \sqrt{\sin I}}{\pi}.
\end{equation}
This agrees perfectly with the cases we've run.

\section{Weak Tidal Friction, changing $\eta$}

Previously, we took the effect of tides to simply be $\rd{\hat{s}}{t} = \epsilon
\hat{s} \times \p*{\hat{l} \times \hat{s}}$, but in reality, tides will spin
down the body (in our case, planet) at the same rate as aligning $\hat{s}$ to
$\hat{l}$. We must treat more carefully.

\subsection{Equations of Motion}

We first write out the full forms of the EOM without tidal friction. These are
taken from Kassandra's Equations 1--3 except I replace subscript $\star$ with
subscript $s$ since we are interested in the case where the spin of planet $1$
evolves with its coupling to its orbital angular momentum and perturber. We
obtain (maybe?)
\begin{align}
    \rd{\hat{s}}{t}
        &= \omega_{s1}\p*{\hat{s} \cdot \hat{l}_1}\p*{\hat{s} \times \hat{l}_1}
            - \omega_{1p}\cos(I)\p*{\hat{s} \times \hat{l}_p},\\
    \omega_{s1} &= \frac{3k_q}{2k} \p*{\frac{R_1}{a_1}}^3 s,\\
    \omega_{1p} &= \frac{3m_p}{4m_1}\p*{\frac{a_1}{a_p\p*{1 - e_p^2}}} \Omega_1.
\end{align}
Note here that $s$ is the spin frequency and $\Omega_1 =
\sqrt{GM_1/a_1^3}$ is the Keplerian orbital frequency.

In the presence of tides, and further assuming $s \ll l_1$, we may write (Lai
2012, Equations 43--44, also Ward 1975 Equation 9 \& 13)
\begin{align}
    \frac{1}{s}\rd{s}{t}
        = \frac{1}{s}\rd{s}{t}
        &= \frac{1}{t_s}\frac{L}{2S}\s*{\cos\theta
            - \frac{s}{2\Omega_1}\p*{1 + \cos^2\theta}},\\
    \rd{\theta}{t} &= -\frac{1}{t_s}\frac{L}{2S}
        \sin\theta\p*{1 - \frac{s}{2\Omega_1}\cos\theta}.
\end{align}
Note that $L = \mu a^2\Omega_1, S = Is$ are the orbital and spin angular momenta
espectively.

It is perhaps easiest to define $\frac{s}{s_c} =
\frac{\omega_{s1}}{\omega_{1p}\cos I}$ and $\epsilon\frac{2\Omega_1}{s} =
\frac{L}{2St_s \omega_{1p}\cos I}$ while rescaling time $\tau =
\omega_{1p}\cos(I) t$, so that we obtain equations of motion
\begin{align}
    \rd{\hat{s}}{\tau}
        &= \frac{s}{s_c}\p*{\hat{s} \cdot \hat{l}_1}\p*{\hat{s} \times \hat{l}_1}
            - \hat{s} \times \hat{l}_p
            + \frac{\epsilon 2\Omega_1}{s}
                \p*{1 - \frac{s}{2\Omega_1}\p*{\hat{l}_1 \cdot \hat{s}}}
                    \hat{s} \times \p*{\hat{l}_1 \times \hat{s}},\\
    \rd{s}{\tau}
        &= \epsilon 2\Omega_1 \p*{\hat{s} \cdot \hat{l}_1 - \frac{s}{2\Omega_1}\p*{1
            + \p*{\hat{s} \cdot \hat{l}_1}^2}}.
\end{align}
$s_c$ has the interpretation of being the critical spin such that the $s1$
coupling is roughly equal strength to the $1p$ coupling. There then seem to be a
few outcomes that we might expect:
\begin{itemize}
    \item Fast evolution towards CS1, then tides will slowly change $s$ without
        changing $\hat{s}$.

    \item Fast evolution towards CS2, then tides are strong while state lives
        inside separatrix maybe? Then will spin down rapidly near CS2 until
        spin-orbit coupling is weak.

    \item Slow evolution that trails behind separatrix, expect state to converge
        somewhere below separatrix? Would probably stay on level curve of
        high-$\eta$ $H$ from earlier? Includes anything that doesn't make it to
        separatrix, including almost fully anti-aligned.
\end{itemize}

\subsection{Crude Analytic Estimate}

To make the equations more anemable to analytic analysis (not simulation), let's
write down the EOM in $(\mu, \phi)$ coordinates again. The $\phi$ EOM does not
change from the tide-free case, so we can reuse earlier equation:
\begin{align}
    \pd{\phi}{\tau} &= \frac{s}{s_c}\mu
        - \p*{\cos I + \sin I \frac{\mu}{\sqrt{1 - \mu^2}} \cos \phi} ,\\
    \pd{\mu}{\tau} &= -\sin I \sin \phi +
        \epsilon \frac{2\Omega_1}{s}\p*{1 - \mu^2}
            \p*{1 - \frac{s}{2\Omega_1}\mu},\\
    \rd{s}{\tau}
        &= \epsilon 2\Omega_1 \p*{\mu - \frac{s}{2\Omega_1}\p*{1 + \mu^2}}.
\end{align}
Assuming $s \gg s_c$ the strong spin-orbit coupling regime, let's first try
assuming $\mu$ is roughly constant over the course of a precession period, then
we can average out the $\phi$ dependencies. Then $\phi$ drops out of the EOM,
and we have approximate averaged equations
\begin{align}
    \pd{\mu}{(\epsilon\tau)} &\approx \frac{2\Omega_1}{s}\p*{1 - \mu^2}
            \p*{1 - \frac{s}{2\Omega_1}\mu},\nonumber\\
        &\approx \p*{1 - \mu^2}
            \p*{\frac{2\Omega_1}{s} - \mu},\\
    \frac{1}{2\Omega_1}\rd{s}{(\epsilon\tau)}
        &\approx \mu - \frac{s}{2\Omega_1}\p*{1 + \mu^2},\nonumber\\
        &\approx - \frac{s}{2\Omega_1}.
\end{align}
In the last term, we note $s \gtrsim 2\Omega_1$ initially, while $\mu \leq 1$,
so we drop both the linear contribution from $\mu$ and approximate $\p*{1 +
\mu^2} \approx 1$.

With all these approximations, we clearly obtain $s(\tau) \approx
s(0)e^{-\epsilon \tau}$, so the critical synchronization
timescale is $\tau_{sync} \sim \frac{1}{\epsilon}$.

The other timescale of interest is how long it takes $\hat{s}$ to reach a
Cassini State. We may present crude estimates in the $s \gg s_c$ strong
spin-orbit coupling limit; consider starting $\mu = -1 + \delta$, then to
leading order
\begin{align}
    \pd{\delta}{\tau} &\approx +\epsilon \delta,\nonumber\\
    \delta(\epsilon\tau) &\approx \delta_0e^{\epsilon \tau}
\end{align}
Setting $\delta \sim 1$ gives $\tau_{CS} = -\frac{1}{\epsilon}\ln \delta_0$ the
timescale it takes to reach $\mu = 0$ which is of order the timescale to reach
the CS\@. This reinforces our expectation that synchronization is probably no
faster than reaching the Cassini state, but that a significant fraction of
initial conditions below the separatrix should be able to reach a Cassini State
before tides synchronize.

Under this assumption, it would appear that points that start inside the
separatrix (small) stay inside the separatrix, points that start above the
separatrix stay above and undergo resonant obliquity excitation ($\sim 50\%$),
and finally points that start outside the separatrix probably end up at either
CS1 or CS2 probabilistically depending on $\eta(\Delta T(\mu_0))$ where $\Delta
T(\mu_0)$ is the arrival time for a state starting at $\mu_0$.

\end{document}

