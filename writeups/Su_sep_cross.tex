    \documentclass[
        fleqn,
        usenatbib,
    ]{mnras}
    \usepackage{
        amsmath,
        amssymb,
        newtxtext,
        newtxmath,
        ae, aecompl,
        graphicx,
        booktabs,
    }
    \usepackage[T1]{fontenc}
    \usepackage[
        labelfont=bf, % caption names are labeled in bold
        font=scriptsize % smaller font for captions
    ]{caption}
    \usepackage[font=scriptsize]{subcaption} % subfigures
    \captionsetup{compatibility=false}

    \newcommand*{\rd}[2]{\frac{\mathrm{d}#1}{\mathrm{d}#2}}
    \newcommand*{\pd}[2]{\frac{\partial#1}{\partial#2}}
    \newcommand*{\md}[2]{\frac{\mathrm{D}#1}{\mathrm{D}#2}}
    \newcommand*{\at}[1]{\left.#1\right|}
    \newcommand*{\abs}[1]{\left|#1\right|}
    \newcommand*{\ev}[1]{\langle#1\rangle}
    \newcommand*{\p}[1]{\left(#1\right)}
    \newcommand*{\s}[1]{\left[#1\right]}
    \newcommand*{\z}[1]{\left\{#1\right\}}
    \DeclareMathOperator*{\argmin}{argmin}
    \DeclareMathOperator*{\argmax}{argmax}
    \DeclareMathOperator*{\med}{med}

\title[Cassini State Capture]{Cassini State Capture}
\author[Y. Su et\ al.]{
Yubo Su$^1$,
Dong Lai$^1$
\\
$^1$ Cornell Center for Astrophysics and Planetary Science, Department of
Astronomy, Cornell University, Ithaca, NY 14853, USA
}

\date{Accepted XXX\@. Received YYY\@; in original form ZZZ}

\pubyear{2019}

\begin{document}\label{firstpage}
\pagerange{\pageref{firstpage}--\pageref{lastpage}}
\renewcommand*{\sectionautorefname}{Section}
\maketitle


\begin{abstract}
    Abstract
\end{abstract}

\begin{keywords}
planet--star interactions % chktex 8
\end{keywords}

\section{Introduction}

Introduction, test citation \citep{henrard1982}.

\section{Cassini States}\label{s:cs}

Denote $\hat{s}$ spin of planet, $\hat{l}$ angular momentum of planet, and
$\hat{l}_p$ angular momentum of perturber. The Cassini state Hamiltonian in the
frame corotating with $\hat{l}_p$ about $\hat{l}$ is:
\begin{equation}
    H = -\frac{\alpha}{2}\p{\hat{s} \cdot \hat{l}}^2
        -g \p{\hat{s} \cdot \hat{l}_p}.
\end{equation}
$\alpha > 0, g < 0$ depend on the particular dynamics of the system. Frequently,
parameter
\begin{equation}
    \eta \equiv \frac{\abs{g}}{\alpha} \label{eq:eta}
\end{equation}
is defined; we refrain from doing so immediately.

Choose $\hat{l} = \hat{z}$ and $\hat{l}_p = \cos I \hat{z} + \sin I \hat{x}$.
Furthermore, choose standard convention where $\phi = 0$ corresponds to
$\hat{l}_p, \hat{s}$ lying on \emph{opposite} sides of $\hat{l}$. This allows us
to evaluate Hamiltonian
\begin{equation}
    H = -\frac{\alpha}{2}\cos^2\theta
        + \abs{g} \p{\cos \theta \cos I - \sin I \sin \theta \cos \phi}
        \label{eq:H}
\end{equation}
Here, $\mu \equiv \cos \theta, \phi$ are canonically conjugate.

\subsection{Equilibria}

The evolution of $\hat{s}$ in this corotating frame is governed by:
\begin{equation}
    \rd{\hat{s}}{t} = \alpha \p{\hat{s} \cdot \hat{l}}
            \p{\hat{s} \times \hat{l}}
        - \abs{g}\p{\hat{s} \times \hat{l}_p}.
        \label{eq:dsdt_base}
\end{equation}

Spin states satisfying $\rd{\hat{s}}{t} = 0$ are referred to as \emph{Cassini
States} (CS). When $\eta < \eta_c$, there are four CSs, and when $\eta > \eta_c$
there are only two; $\eta_c$ is
\begin{equation}
    \eta_c \equiv \p{\sin^{2/3}I + \cos^{2/3}I}^{3/2}.
\end{equation}
Using the standard numbering given in \autoref{fig:0_1contours}, CSs 1, 2, 3 are
stable while CS4 is unstable.

\subsection{Separatrix}

In the four-CS regime, one of the CSs is a saddle point, conventionally denoted
Cassini State 4 (CS4). On the cylindrical phase space parameterized by $(\mu,
\phi)$, all trajectories are periodic with finite period except two critical (or
heteroclinic) trajectories. These are asymptotic in the past and future to CS4.
Together, these two heteroclinic are referred to as the \emph{separatrix} and
divide phase space into three zones.

The area enclosed by the separatrix is known exactly in literature
\citep{henrard1987}, but a serviceable approximation for small $\eta$ can be
derived as follows. Call $H_{sep}$ the value of $H$ along the separatrix (also
the value of $H$ at CS4), then the separatrix is defined implicitly by solutions
to $H\p{\mu_{sep}(\phi), \phi} = H_{sep}$. This may be evaluated and we obtain
\begin{equation}
    \mu_{sep}(\phi) \approx \pm \sqrt{2\eta \sin I \p{1 - \cos \phi}}
        + \eta \cos I + \mathcal{O}\p{\eta^{3/2}}.\label{eq:mu_sep}
\end{equation}
The enclosed area between the two solutions can be obtained via explicit
integration. It can be expressed as a fraction of the total phase space area:
\begin{equation}
    \frac{A_{sep}}{4\pi} \approx \frac{4\sqrt{\eta \sin I}}{\pi}
        + \mathcal{O}\p{\eta^{3/2}}.\label{eq:a_sep}
\end{equation}

Contours of equal $H$, CSs and the separatrix are plotted in
\autoref{fig:0_1contours}.
\begin{figure}
    \centering
    \includegraphics[width=\columnwidth]{../initial/0_eta/1contours.png}
    \caption{Contours of equal $H$, given by \autoref{eq:H}, at different values
    of $\eta \equiv \frac{\abs{g}}{\alpha}$. Labeled are CS1 (red), CS2
    (purple), CS3 (green) and CS4 (cyan). The thick black dashed line is the
    separatrix (which disappears for $\eta > \eta_c$). Finally, the fractional
    area enclosed by the separatrix is noted in the plot subtitles, in good
    agreement with \autoref{eq:a_sep}.}\label{fig:0_1contours}
\end{figure}

\section{Separatrix Crossing: Theory}\label{s:sep_cross}

In an exactly Hamiltonian system, $H$ is a conserved quantity and trajectories
in phase space coincide with level curves of $H$. If a system is only slightly
non-Hamiltonian, trajectories will generally cross level curves. Of particular
interest is the behavior of trajectories near the separatrix (also a level
curve), as it frequently governs the dynamics of resonance capture (CITE MMR,
Cassini, others). The intersection of system trajectories with the separatrix is
referred to as \emph{separatrix crossing}.

In previous work by~\cite{henrard1982}, separatrix crossing was studied for systems
that are Hamiltonian save for an adiabatically-varying parameter. In related work
surveyed by~\cite{g_and_h}, the impact of small perturbations to the equations
of motion of a Hamiltonian system are studied (referred to as \emph{saddle
connection bifurcations}). However, in some astrophysical systems of interest,
leading-order perturbations to a Hamiltonian system contribute in both of the
above ways. Such systems resist characterization via either existing technique.
We suggest a possible generalization unifying the above techniques. [Probably
eventually move this paragraph to the introduction]

\subsection{Derivation}\label{ss:henrard}

Consider our Hamiltonian described in \autoref{eq:H} as a unperturbed
Hamiltonian $H^{(0)}(\mu, \phi; \eta)$ where $\eta$ is to be varied
adiabatically. As $\eta$ varies, the location of the separatrix also changes. We
follow~\cite{henrard1982} by defining
\begin{equation}
    h(\mu, \phi; \eta) \equiv H^{(0)}(\mu, \phi; \eta) - H_{sep}(\eta).
\end{equation}
The significance of $h$ is that it always vanishes along the separatrix. Note
that $h$ corresponds to $K$ in~\cite{henrard1982}, though in our problem
$H^{(0)} > H_{sep}$ corresponds to the \emph{interior} of the separatrix. The
evolution of $h$ over time then determines whether and when a trajectory
experiences separatrix crossing. Denoting time derivatives with dots, we write
\begin{align}
    \rd{h}{t} &= \rd{H^{(0)}}{t} - \pd{H_{sep}}{\eta}\dot{\eta},\nonumber\\
        &= \s{\pd{H^{(0)}}{\mu}\dot{\mu}
                + \pd{H^{(0)}}{\phi}\dot{\phi}
                + \pd{H^{(0)}}{\eta}\dot{\eta}}
            - \pd{H_{sep}}{\eta}\dot{\eta},\nonumber\\
        &= \s{\pd{H^{(0)}}{\mu}\dot{\mu}
                + \pd{H^{(0)}}{\phi}\dot{\phi}
                + \pd{H^{(0)}}{\eta}\dot{\eta}}
            - \pd{H_{sep}}{\eta}\dot{\eta},\nonumber\\
        &= \s{\dot{\phi}^{(0)}\dot{\mu}^{(1)}
                - \dot{\mu}^{(0)}\dot{\phi}^{(1)}
                + \pd{H^{(0)}}{\eta}\dot{\eta}} -
            \pd{H_{sep}}{\eta}\dot{\eta}.\label{eq:dhdt}
\end{align}
We've notated $\dot{\phi}^{(0)} \equiv \pd{H^{(0)}}{\mu}$ and $\dot{\phi}^{(1)}
= \dot{\phi} - \dot{\phi}^{(0)}$ and similarly for $\mu$. The scenario studied
in~\cite{henrard1982} corresponds to $\dot{\mu}^{(1)} = \dot{\phi}^{(1)} = 0$
while that in~\cite{g_and_h} sets $\dot{\eta} = 0$.

In the neighborhood of $\mathcal{C}_{\pm}$, $\rd{h}{t}$ can be
approximated by its value on $\mathcal{C}_{\pm}$. This is the \emph{guiding
orbit} approximation; its accuracy is proven in~\cite{henrard1982} for their
specific application, while we will take it on good faith in ours. Denote then
\begin{equation}
    \Delta_{\pm} \equiv \int_{\mathcal{C}_{\pm}} \rd{h}{t}\;\mathrm{d}t.
        \label{eq:delta}
\end{equation}
These approximate $\Delta h$ for trajectories near $\mathcal{C}_{\pm}$ over one
orbit. They are identical to the $B_i$ defined in~\cite{henrard1982}.

\begin{figure}
    \centering
    \includegraphics[width=\columnwidth]{../initial/1_weaktide/3zones.png}
    \caption{Nomenclature of the two legs of the separatrix $\mathcal{C}_{\pm}$
    and three zones of the domain. For both $\mathcal{C}_{\pm}$, we will take
    the positive direction to be anti-clockwise.}\label{fig:1_3zones}
\end{figure}

\subsection{Zone I/III Initial Conditions}\label{ss:z_out_ic}

For concreteness, we consider a specific example first. We assign labels to our
phase space as shown in \autoref{fig:1_3zones}. Consider a trajectory that
begins in zone III, which is circulating with $\dot{\phi} > 0$ and has $h < 0$,
and consider if $\rd{h}{t} > 0$ in all of zone III, such that the trajectory
eventually experiences separatrix crossing at $\mathcal{C}_-$. The following
sequence of events unfolds:
\begin{enumerate}
    \item[\textbf{(i)}] At the beginning ($\phi = 0$) of its separatrix-crossing
        orbit, $h_i \equiv h(\mu, \phi = 0; \eta) \in (\Delta_-, 0)$. Note
        that $h_i < 0$ to still be in zone III at the start of the orbit, while
        $h_i > \Delta_-$ to be separatrix crossing during the final orbit.

    \item[\textbf{(ii)}] After traversing the length of $\mathcal{C}_-$, the
        value of $h$ is now $h_i + \Delta_- > 0$ (recall $\Delta_- > 0$), and so
        the trajectory has entered zone II\@. Since zone II hosts only librating
        solutions, the trajectory will then ``turn around'' and track
        $\mathcal{C}_+$ from inside zone II\@.

    \item[\textbf{(iii)}] At the end of following $\mathcal{C}_+$, $h$ now takes
        value $h_f \equiv h_i + \Delta_- + \Delta_+$. There are now two
        possibilities:

    \item[\textbf{(iii-a)}] If $h_f > 0$, the trajectory remains inside zone II
        accrues further multiples of $\Delta_- + \Delta_+ = h_f - h_i > 0$.
        Thus, the trajectory must securely enter zone II\@. This outcome
        commonly corresponds to \emph{resonance capture}, a transition into zone
        II\@.

    \item[\textbf{(iii-b)}] If $h_f < 0$, the trajectory exits zone II and
        enters zone I\@. Since $\Delta_+ = h_f - h_i - \Delta_- < 0$ (as $h_f
        < 0, h_i + \Delta_- > 0$), the trajectory will continue circulation in
        zone I and accrue further multiples of $\Delta_+$, securely entering
        zone I\@. This corresponds to a zone III-zone I transition,
        \emph{escape}.
\end{enumerate}

Thus, the result of separatrix crossing depends on the sign of $h_i +
\Delta_- + \Delta_+$ where $\Delta_{\pm}$ are given by \autoref{eq:delta} and
$h_i \in [-\Delta_-, 0]$. The result of this analysis reproduces the conclusion
of~\cite{henrard1982}:
\begin{itemize}
    \item If $\Delta_+ < -\Delta_-$, then $h_f < 0$ necessarily, secure escape
        ensues.

    \item If $\Delta_+ > 0$, then $h_f > 0$ necessarily, and secure capture
        ensues.

    \item If $\Delta_+ + \Delta_- \in [0, \Delta_-]$, then it is clear that the
        sign of $h_f$ depends on the precise value of $h_i$. Generally,
        $\Delta_-$ can be assumed to be small as perturbations are weak, and
        so $\Delta_-$ is usually much smaller than the range of $h$ of interest.
        Thus, $h_i$ can be treated as uniformly distributed on interval
        $[-\Delta_-, 0]$, and a probability of resonance capture can be
        computed:

        \begin{equation}
            P_c \equiv \frac{\Delta_+ + \Delta_-}{\Delta_-}.
                \label{eq:p_c_gen}
        \end{equation}
\end{itemize}
With the exception of \autoref{eq:dhdt}, the above outlines the argument given
in~\cite{henrard1982}. Similar conclusions can be drawn for trajectories
originating in zone I.

\subsection{Zone II Initial Conditions}\label{ss:z2_ic}

The fates of trajectories originating in zone II, which librate about the
CS2, also depend on $\Delta_{\pm}$. This implies $h_i > 0$. We identify the
following possibilities:
\begin{itemize}
    \item $\Delta_{+} + \Delta_- > 0$: trajectories near the separatrix are
        driven back into zone II over the course of a full libration, therefore
        any trajectory originating in zone II is bound to zone II\@. Note that
        this alone need not imply convergence to CS2, e.g.\ for a
        non-dissipative system.

    \item $\Delta_+ + \Delta_- < 0, \Delta_+ > 0, \Delta_- < 0$: trajectories
        near the separatrix are driven back into zone II along $\mathcal{C}_+$
        but out of zone II along $\mathcal{C}_-$, the net effect amounting to
        expulsion. Since expulsion cannot occur along $\mathcal{C}_+$, the only
        permitted transition is II $\to$ III\@. Vice versa of course if the signs
        of $\Delta_{\pm}$ are swapped while still satisfying $\Delta_+ +
        \Delta_- < 0$.

    \item $\Delta_\pm < 0$: trajectories near the separatrix tend to be
        ejected along both $\mathcal{C}_{\pm}$. Which zone the trajectory enters
        is then probabilistic: it enters zone I with probability
        $\frac{\Delta_+}{\Delta_+ + \Delta_-}$ and zone III with probability
        $\frac{\Delta_-}{\Delta_+ + \Delta_-}$.

        It bears noting that in the non-dissipative limit, this result is
        equivalent to \citep{henrard1987}, which computes these probabilities
        using change in enclosed phase space area.
\end{itemize}

\subsection{Equivalence to Stable/Unstable Manifold Splitting}\label{ss:meln}

We now establish equivalence to an alternative formulation popular in dynamical
systems that is more graphically intuitive. For simplicity, we will take
$\dot{\eta} = 0$.

If we return to the unperturbed Hamiltonian system, recall that CS4 is joined to
itself by an infinite-period orbit. For a given saddle point, we may define the
\emph{stable manifold} $W_s$ of a saddle point to be the set of points whose
forward evolution converges to the saddle point. Similarly, we may define the
\emph{unstable manifold} $W_u$ of a saddle point to be the set of points that
originated \emph{from} the saddle point, or whose backwards evolution converges
to the saddle point. A \emph{saddle connection} is when the unstable manifold of
one saddle point and the unstable manifold of another saddle point are
degenerate. It is evident that the separatrix is formed of two saddle
connections: $\mathcal{C}_+$ is the unstable manifold of CS4 at $\phi = 2\pi$
and the stable manifold of CS4 at $\phi = 0$.

In the presence of a small perturbation, the stable and unstable manifolds
generally separate, as the perturbations along the saddle connection will
generally be different going forwards and backwards in time. The separation
distance as a function of time along the \emph{unperturbed} trajectory can be
computed via \emph{Melnikov's Method}
\begin{equation}
    d(t_0) = \frac{\epsilon}{\abs{f(q^0(-t_0))}}
        \int\limits_{-\infty}^\infty
            f(q^0(t - t_0)) \wedge g(q^0(t - t_0), t)\;\mathrm{d}t.
\end{equation}
The notation used here is that of~\cite{g_and_h}, where $d(t_0)$ is the distance
between the stable/unstable manifolds at time $t_0$ along the heteroclinic
orbit, $f = \p{\dot{\mu}^{(0)}, \dot{\phi}^{(0)}}$, $\epsilon g =
\p{\dot{\mu}^{(1)}, \dot{\phi}^{(1)}}$ a general time-dependent perturbation,
$q^0(t)$ is the unperturbed trajectory along the saddle connection (time
coordinate defined such that $t = 0$ coincides with the ``middle'' of the
infinite trajectory), and $\wedge$ is the wedge product.

We will assume $\epsilon g$ is not explicitly time-dependent, characteristic of
secular/averaged equations in astrophysics, then the integral above is simply
the familiar
\begin{equation}
    d(t_0) \abs{\vec{\nabla}H^{(0)}}(t_0) = \epsilon \int\limits_{-\infty}^\infty
        \dot{\mu}^{(0)}\dot{\phi}^{(1)} - \dot{\phi}^{(0)}\dot{\mu}^{(1)}
            \;\mathrm{d}t.\label{eq:melnikov}
\end{equation}
The right hand side is just the total change in $H^{(0)}$ over the saddle
connection though, equivalent to our \autoref{eq:delta} in the $\dot{\eta} = 0$
limit. The left hand side on the other hand is exactly $\abs{\vec{d} \cdot
\vec{\nabla}H^{(0)}}$ where the displacement is along $\vec{\nabla}H^{(0)}$.
Thus, the effect of the $\Delta_{\pm}$ is to split the stable/unstable manifolds
along the saddle connections.

Visually, this is depicted in \autoref{fig:0_6manifolds}. Note that:
trajectories starting in zone III below $W_s^{R}$ fail to separatrix cross;
those between $W_s^{R}$ and $W_s^{L}$ continue between $W_u^{R}$ and $W_s^{L}$,
eventually escaping; those between $W_s^{L}$ and $W_u^{L}$ are securely
captured, and no trajectories can begin above $W_u^{L}$ at $\phi = 0$.

The saddle connection splitting is easiest to measure where $\vec{\nabla}H^{(0)}
\propto \hat{\phi}$, where $\dot{\mu} = 0$, and \autoref{eq:melnikov} reduces to
just $\Delta \mu \pd{H^{(0)}}{\mu} = \Delta H^{(0)}$. Then the capture
probability must be given by ratio of $\Delta \mu$ separating $W_s^{L},
W_u^{L}$ to $\Delta \mu$ separating $W_s^{R}, W_u^{L}$. These $\Delta \mu$
values are those given in the title, and the resultant $P_c$ capture probability
is also quoted.
\begin{figure}[t]
    \centering
    \includegraphics[width=\columnwidth]{../initial/0_eta/6manifolds0_20.png}
    \caption{Sample plot of saddle connection breaking in the presence of weakly
    non-Hamiltonian dissipation. Equations are used from \autoref{s:p2}. Stable
    and unstable manifolds are indexed such that superscript $L/R$ belongs to
    the saddle point at left/right. $\Delta \mu$ represent the manifold
    separations, resulting in capture probability $P_c$ prediction per
    \autoref{eq:p_c_dmu}. Numerical simulations predict $P_c \approx
    0.252$.}\label{fig:0_6manifolds}
\end{figure}

Finally, the connection between \autoref{ss:henrard} and \autoref{ss:meln} is
evident: the ratio of the $\Delta \mu$ values is simply
\begin{align}
    P_c &= \frac{\Delta \mu(W_s^{L}, W_u^{L})}{
            \Delta \mu\p{W_s^{R}, W_u^{L}}},\label{eq:p_c_dmu},\\
        &= \frac{\Delta H^{(0)}_- + \Delta H^{(0)}_+}{\Delta H^{(0)}_-}.
\end{align}
This is exactly \autoref{eq:p_c_gen} when $\dot{\eta} = 0$.

\section{Problem 1: Adiabatic Capture}\label{s:p1}

Here we will adiabatically vary $\eta$. This problem is extensively covered in
existing work e.g.\ \citep{henrard1987}.

\subsection{Spindown Model, $\dot{\eta} > 0$ (few sims)}\label{ss:p1_spindown}

We may first consider the unperturbed Hamiltonian
system with $\rd{\eta}{t} = \frac{\eta}{t_\eta}$ growth with characteristic
timescale $t_\eta$. Again rescaling time $\tau \equiv \alpha t$ and defining
$\epsilon \equiv \frac{1}{\alpha t_\eta}$, we obtain
\begin{equation}
    \rd{\eta}{\tau} = \epsilon \eta,\label{eq:p1_eom}
\end{equation}
where $\epsilon > 0$. Such a model is motivated e.g.\ when a planet's
spin $\hat{s}$ orbits a host star with angular momentum $\hat{l}$ while the star
spins down; $\dot{\alpha} < 0$ here, and $\eta' > 0$.

We then seek $\Delta_{\pm}$, determined by
\begin{align}
    \Delta_{\pm} &= \int\limits_{\mathcal{C}_{\pm}}
            \pd{h}{\eta}\eta'\;\mathrm{d}\tau,\nonumber\\
        &= \int\limits_{\mathcal{C}_{\pm}}
            \frac{\pd{h}{\eta}}{\phi'}\eta'\;\mathrm{d}\phi.
\end{align}
We must proceed carefully since $\phi' \to 0$ at the endpoints of
$\mathcal{C}_{\pm}$; careful work shows these zeros cancel against zeros of
$\pd{h}{\eta}$, and
\begin{align}
    \Delta_{\pm} &= \int\limits_{\mathcal{C}_{\pm}} \eta'
        \p{-\cos I \mp \sin I \sqrt{\frac{1 - \cos \phi}{2\eta\sin I}}}
            \;\mathrm{d}\phi,\label{eq:p1_delta_eq}\\
        &= \mp 2\pi \epsilon \eta \cos I
            + 4\epsilon\sqrt{\eta \sin I}.\label{eq:p1_delta}
\end{align}
Call $\eta_\star$ the value of $\eta$ at which a trajectory encounters the
separatrix. There are then three possibilities:
\begin{itemize}
    \item For $\eta_\star < \frac{4\sin I}{\pi^2 \cos^2 I}$, we obtain
        $\Delta_{\pm} > 0$. In this regime, all separatrix encounters
        originating in zones I/III result in secure capture (see
        \autoref{ss:z_out_ic}), while all trajectories originating in zone II do
        not experience separatrix encounter. For $I = 20^\circ$, this condition
        equates to $\eta_\star \lesssim 0.16$. This condition is not to be taken
        too exactly, as all our calculations are performed in the small $\eta$
        limit. Select simulations in $\eta_\star \lesssim 0.1$ are in agreement
        with the secure capture prediction. An example of such a simulation is
        presented in \autoref{fig:2_1testo4}.
        \begin{figure}[t]
            \centering
            \includegraphics[width=\columnwidth]{../initial/2_toy2/1testo4.png}
            \caption{Simulation with initial condition in zone I subject to
            $\eta' = \epsilon \eta$, where $\epsilon = 10^{-4}$. The top panel
            shows the trajectory (red) starting with initial condition $\mu_0 =
            0.140, \phi_0 = 0$ (green dot) and separatrix at separatrix crossing
            (dashed black) in the labelled coordinates. The end of the
            simulation is labelled in the blue cyan dot. The bottom panel shows
            the evolution of the enclosed area by the trajectory (red dots),
            area of the separatrix (blue) and $\eta$ (dotted green) over
            time.}\label{fig:2_1testo4}
        \end{figure}

    \item If $\eta_\star > \frac{4\sin I}{\pi^2 \cos^2 I}, \eta_\star < \eta_c$,
        then the separatrix still exists, and $\Delta_- < 0, \Delta_+ > 0$,
        $\Delta_+ + \Delta_- > 0$. This implies: zone III + zone II trajectories
        do not experience separatrix encounters, while zone I trajectories are
        captured into zone II with probability
        \begin{equation}
            P_c = \frac{8\sqrt{\eta_\star \sin I}}{2\pi \eta_\star \cos I +
                4\sqrt{\eta_\star \sin I}}.
        \end{equation}

    \item Further evolution from either of the above two cases, as well as the
        case where no separatrix encounter occurs before $\eta > \eta_c$, is
        governed simply by conservation of adiabatic invariant/action variable
        \begin{equation}
            I \equiv \oint \cos \theta \;\mathrm{d}\phi.
        \end{equation}
\end{itemize}

\subsection{Spinup Model, $\dot{\eta} < 0$ (no sims)}\label{ss:p1_spinup}

If instead $\epsilon < 0$, then the separatrix area shrinks over time. For
initial values $\eta > \eta_c$, the separatrix is not initially present and only
appears later in the simulation.

If $\eta$ decreases past critical value $\frac{4\sin I}{\pi^2 \cos^2 I}$, i.e.\
$\Delta_{\pm} < 0$, then the separatrix appears. Here, trajectories in zones
I/III do not experience separatrix encounter while trajectories in zone II
transition into zones I/III with probabilities given respectively (see
\autoref{ss:z2_ic}):
\begin{equation}
    \begin{split}
        P_{II \to I} &= \frac{-2\pi \eta_\star \cos I + 4\sqrt{\eta_\star \sin I}}
            {8\sqrt{\eta_\star \sin I}}, \\
        P_{II \to III} &= \frac{2\pi \eta_\star \cos I + 4\sqrt{\eta_\star \sin
            I}} {8\sqrt{\eta_\star \sin I}}.
    \end{split}
\end{equation}

\subsection{Nontrivial Test Problem, ``Problem 1.5'' (sims)}\label{ss:p1_toy}

It is clear that $\eta'$ in \autoref{eq:p1_delta_eq} being the same sign along
both $\mathcal{C}_{\pm}$ is the source of guaranteed capture during
spindown. In order to test that \autoref{eq:p_c_gen} is not only qualitatively
correct but also quantitatively exact, we seek to simulate a problem with
probabilistic behavior. \textbf{Note: This section is probably superceded by
\autoref{ss:p1_spinup} or \autoref{s:p3}.}. We investigate test problem where
$\eta$ is varied following
\begin{equation}
    \eta' = \epsilon \eta \mu.
\end{equation}
This is a pure mathematical exercise, but resembles weak tidal friction in the
subsynchronous rotation regime.

Straightforward evaluation of \autoref{eq:p1_delta_eq} yields
\begin{equation}
    \Delta_{\pm} = \pm \epsilon \eta_\star \s{
        2\pi \sin I \pm 12 \sqrt{\eta_\star \sin I}\cos I
            + 2\pi \eta_\star \cos^2 I}.
\end{equation}
It bears noting that we must evaluate these integrals at $\eta = \eta_\star$ the
value at separatrix crossing. Since $\eta$ evolves over time, $\eta_{\star}$
depends on initial conditions. While one might expect the adiabatic invariance
of the action variable determines $\eta_\star$, this turns out to be slightly
inaccurate for reasons (I haven't worked this out convincingly yet).
% not be the case; we address this in \autoref{ss:noncons}.

Since $\abs{\Delta_+} > \abs{\Delta_-}$, separatrix encounters originating in
zone III always result in capture. However, encounters originating in zone I
experience capture probability
\begin{equation}
    P_c(\eta_\star) = \frac{24\cos I \sqrt{\eta_\star \sin I}}{
        2\pi \sin I + 12\sqrt{\eta_\star \sin I}\cos I
            + 2\pi \eta_\star \cos^2 I}.\label{eq:p_c_p1}
\end{equation}

To test this capture probability, we test capture probability at a variety of
$I$ for similar ICs. For fixed initial condition $(\mu_0, \phi_0)$ in zone I, we
vary the initial $\eta \in [0.01, 0.02]$ and $\epsilon \in 10^{[-3, -2]}$. In
this way, every simulation experiences probabilistic separatrix crossing at
roughly comparable times. We then repeated this procedure for three different
values $I = 10^\circ, 20^\circ, 25^\circ$. A natural spread in $\eta_\star$ was
observed at each $I$ value, which permitted comparison to \autoref{eq:p_c_p1}
independently. The results for the three $I$ values are presented in
\autoref{fig:2_1hist}, where good agreement with
\autoref{eq:p_c_p1} is shown.
\begin{figure}
    \centering
    \begin{subfigure}{\columnwidth}
        \centering
        \includegraphics[width=\columnwidth]{../initial/2_toy2/1hist10.png}
    \end{subfigure}

    \begin{subfigure}{\columnwidth}
        \centering
        \includegraphics[width=\columnwidth]{../initial/2_toy2/1hist20.png}
    \end{subfigure}

    \begin{subfigure}{\columnwidth}
        \centering
        \includegraphics[width=\columnwidth]{../initial/2_toy2/1hist25.png}
    \end{subfigure}
    \caption{Simulations depicting $P_c(\eta_\star)$ at three different values
    $I = 10^\circ, 20^\circ, 25^\circ$ compared to analytical estimate
    \autoref{eq:p_c_p1}. The top panel is a stacked histogram showing the
    capture (into separatrix) and escape occurence counts at each $\eta_\star$.
    The bottom panel are the inferred estimates of $P_c(\eta_{\star})$ (capture
    probability) in blue and analytical curve in red. Uncertainties in
    $P_c(\eta_\star)$ from data are estimated as $\sqrt{N}$ where $N$ is the
    number of captures per bin. Also reported is the average value of
    $\eta_\star$.}\label{fig:2_1hist}
\end{figure}

% \subsection{Conservation of Adiabatic Invariant}\label{ss:noncons}

% Under the theory of adiabatic invariants (see e.g.\ \cite{henrard1982,
% l_and_l}), trajectories following Hamiltonian equations of motion save for small
% $\eta'$ conserve action variable $I$ except at separatrix crossing, where
% \begin{equation}
%     I \equiv \frac{1}{2\pi}\oint p\;\mathrm{d}q.
% \end{equation}
% $I$ has physical interpretation the enclosed phase space area of trajectories.
% Its evolution can be seen in \autoref{fig:2_1testo4}. Under the assumption $I$
% is conserved except at separatrix crossing, $\eta_\star$ is fixed implicitly by
% $I = A_{sep}(\eta_\star)$ for $I$ the initial value of the action variable.
% Since $A_{sep}$ is given by \autoref{eq:a_sep}, it would appear $\eta_\star$ can
% be predicted exactly.

% \textbf{NOTE:} The rest of this section is not very complete.

% This turns out to be accurate in \autoref{ss:p1_spindown} but not in
% \autoref{ss:p1_toy}, so it's probably non-physical/has to do with $\eta' = 0$
% when $\mu = 0$. Not sure whether this is physically useful. Useful formulas for
% $\rd{I}{t}$ can be taken from~\cite{henrard1982, l_and_l}:
% \begin{equation}
%     \rd{I}{\tau} = \s{\pd{H}{\eta}T(\eta) - \int\limits_0^{T(\eta)}
%             \pd{H}{\eta}\;\mathrm{d}\tau} \eta'
%         = -\pd{}{w}\p{\pd{S}{\eta}}\eta'.
% \end{equation}
% Here, $T(\eta)$ is the period of the orbit, $w$ is the angle variable conjugate
% to $I$, and $S$ is the generating function for canonical transformation $\mu =
% \pd{S}{\phi}, w = \pd{S}{I}$. These expressions are hard to evaluate, but maybe
% they will tell us how much to expect $\rd{I}{\tau}$ to deviate from zero near
% the separatrix?

\section{Problem 2: Weakly Dissipitive Hamiltonian}\label{s:p2}

\subsection{$\eta < \eta_c$, Separatrix Encounters (sims)}\label{ss:p2_lt}

We next consider a toy separatrix crossing problem in the $\dot{\eta} = 0$
limit, the case discussed in \autoref{ss:meln}. Consider adding to the
Hamiltonian Cassini State problem an additional aligning term
$\p{\rd{\theta}{t}}^{(1)} = -\frac{1}{t_s}\sin\theta$ that favors alignment
$\theta = 0$ on synchronization timescale $t_s$. This translates to equation of
motion
\begin{equation}
    \rd{\hat{s}}{\tau} = \p{\hat{s} \cdot \hat{l}}\p{\hat{s} \times \hat{l}}
        - \eta\p{\hat{s} \times \hat{l}_p}
        - \epsilon \sin \theta \hat{\theta}.
\end{equation}
We have divided \autoref{eq:dsdt_base} by $\alpha$ and defined $\tau \equiv
\alpha t, \epsilon \equiv \frac{1}{\alpha t_s}$. The equations of motion for
canonical variables $\mu \equiv \cos\theta, \phi$ are
\begin{subequations}\label{se:p2_eom}
    \begin{align}
        \phi' &= -\cos\theta
            + \eta\p{\cos I + \sin I \frac{\mu}{\sqrt{1 - \mu^2}} \cos \phi},\\
        \mu' &= -\eta \sin I \sin \theta \sin \phi
            \s{+ \epsilon\p{1 - \mu^2}}.
    \end{align}
\end{subequations}
Primes denote $\rd{}{\tau}$. The bracketed term in $\mu'$ is the perturbation
component. We will take $\alpha, \epsilon > 0$, which will result in separatrix
encounters originating in zone III, but an identical procedure for $\epsilon <
0$ describes separatrix encounters originating in zone I.

With a dissipitive term that favors alignment, we expect that separatrix
encounters can only occur from zone II and zone III\@. We will focus on the
outcomes of trajectories origiating in zone III for now. We now aim to compute
\autoref{eq:p_c_gen}, for which we need:
\begin{align}
    \Delta_\pm &= \int\limits_{\mathcal{C}_\pm}
            \phi'\epsilon\p{1 - \mu^2}\;\mathrm{d}t,\nonumber\\
        &= \mp 2\pi\p{1 - 2\eta \sin I - \eta^2 \cos^2 I}
                + 16\cos I \eta^{3/2}\sqrt{\sin I}.\label{eq:p2_delta}
\end{align}
Note that along $\mathcal{C}_{\pm}$ that $\mathrm{d}\phi$ has sign $\mp$
respectively. From these two integrals, we obtain
\begin{equation}
    P_c = \frac{16\eta^{3/2}\cos I \sqrt{\sin I}}{\pi
        \p{1 - 2\eta \sin I - \eta^2 \cos^2 I}
        + 8\eta^{3/2}\cos I \sqrt{\sin I}}.\label{eq:p1pc}
\end{equation}

\begin{table*}
    \centering
    \begin{tabular}{c|c c c c}
        $\eta$ & $0.025$ & $0.05$ & $0.1$ & $0.2$\\\midrule
        Numerical $P_c$ & $0.010 \pm 0.002$ & $0.029 \pm 0.003$
            & $0.082 \pm 0.004$ & $0.256 \pm 0.006$ \\
        Analytical $P_c$ & $0.0112$ & $0.0320$ & $0.0915$ & $0.2627$.
    \end{tabular}
    \caption{Capture probability for four different values of $\eta$, all using
    $\epsilon = 3 \times 10^{-4}$. Different values of $\epsilon$ were tried
    $\epsilon \in 10^{[-2, -4]}$ with no effect on $P_c$. $10000$ random initial
    conditions uniformly distributed in $(\mu, \phi)$ were used for the
    numerical calculations, of which roughly $1/2$ experience separatrix
    crossing. Numerical uncertainties are estimated as $\sqrt{N}$ where $N$ is
    the count per bin.}\label{tab:p_c_num}
\end{table*}
As evidence, we provide the following table of separatrix capture probabilities
in \autoref{tab:p_c_num}.

It may be noted that the numerical $P_c$ are systematically low compared to
their analytical estimates. This has a clean explanation: initial conditions
that start too close to the separatrix violate the assumption where $h_i$ is
uniformly distributed on $[-\Delta_-, 0]$. In particular, most of the points
near the separatrix and in zone III are escaping points, as can be visually seen
in \autoref{fig:0_6manifolds}. A systematic underdensity of points converging to
CS2 (separatrix capture) can be seen on the right hand side of the plot near the
separatrix in \autoref{fig:0_3stats}.

A further interesting feature in \autoref{fig:0_3stats} is a small portion of
points in zone II near CS4  at $\phi = 2\pi$ that escape. These points accrue
$\Delta_+$ on their first orbit and escape without librating. Visually, this
zone is obvious in \autoref{fig:0_6manifolds} as the separation between
$W_s^{(1)} \cup W_u^{(1)}$ and $W_s^{(0)} \cup W_u^{(0)}$ near $\phi = 2\pi$.
\begin{figure}
    \centering
    \includegraphics[width=\columnwidth]{../initial/0_eta/3stats3_5_0_1.png}
    \caption{Plot of the CSs that initial conditions converge to. Each subplot
    has a set of initial conditions (black dots) and the CS (red dot) that each
    of these initial conditions converge to. Initial conditions are chosen
    uniformly in $(\mu, \phi)$ coordinates. Key features to note are secure
    convergence to CS1 for all zone I initial conditions, probabilistic
    convergence to CS1 or CS2 for zone III initial conditions (first half of
    \autoref{ss:p2_lt}), and secure convergence to CS2 for all zone II initial
    conditions (second half of \autoref{ss:p2_lt}). The parameters for the
    simulation are in the title.}\label{fig:0_3stats}
\end{figure}

For trajectories originating in zone II, numerical simulations indicate they can
never leave the separatrix. This can be understood quantitatively (excepting the
few initial conditions in \autoref{fig:0_3stats} discussed above). For any orbit
in zone II that completes a full libration over contour $\mathcal{C}$, the total
change in $h$ is given
\begin{equation}
    \Delta h = \epsilon \oint_{\mathcal{C}} (1 - \mu^2)\;\mathrm{d}\phi.
\end{equation}
Since $\mathcal{C}$ is a closed contour, any terms that are symmetric on the top
and bottom legs of $\mathcal{C}$ vanish. If we decompose $\mu(\phi) \equiv
\bar{\mu} \pm \mu'(\phi)$ where $\bar{\mu}$ is the average value of $\mu(\phi)$
over $\mathcal{C}$ and $\mu'(\phi) \geq 0$ everywhere, then it is clear
\begin{equation}
    \Delta h = 2\epsilon \int\limits_0^{2\pi} 2\bar{\mu}\mu'(\phi)
            \;\mathrm{d}\phi.
\end{equation}
Now, since $\mu'(\phi) \geq 0$ everywhere, and $\bar{\mu} \in [\mu_2, \mu_4] >
0$ as well, $\Delta h \geq  0$ necessarily, equality only exactly at CS2. And
so, \emph{all full librations in zone II will be driven towards CS2} (the point
of maximum $h$ and $H^{(0)}$).

\subsection{$\eta > \eta_c$, Convergence to CS2 (no sims)}

While not strictly within the purview of this paper, we can also prove that once
$\eta > \eta_c$ and the separatrix vanishes, all trajectories converge to CS2.
This follows naturally by considering the change $\Delta H^{(0)}$ over any orbit
\begin{equation}
    \Delta H^{(0)} = \oint \epsilon\p{1 - \mu^2}\;\mathrm{d}\phi.
\end{equation}
Visual inspection of \autoref{fig:0_1contours} shows that all orbits either
circulate $\phi' > 0$ or librate counterclockwise very near CS2. For any
circulating $\phi' > 0$ orbit, $\Delta H^{(0)} > 0$ is enforced. For any
counterclockwise libration near CS2, the same argument as the zone II initial
conditions from \autoref{ss:p2_lt} above apply, where $\Delta H^{(0)} \geq 0$
with equality only at CS2 itself.

\section{Problem 3: Combined Perturbations (sims)}\label{s:p3}

We now consider the combined problem, where the EOMs for Problem 2
(\autoref{se:p2_eom}) are used in conjunction with the simplest $\eta' = \delta
\eta$, the same as \autoref{eq:p1_eom}. Evaluation of $\Delta h$ along
$\mathcal{C}_{\pm}$ is simply combining the results of \autoref{eq:p1_delta} and
\autoref{eq:p2_delta}. The result is
\begin{align}
    \Delta_{\pm} ={}& \epsilon\s{
        \mp 2\pi \p{1 - 2\eta \sin I - \eta^2 \cos^2 I}
            + 16\cos I\eta^{3/2}\sqrt{\sin I}}\nonumber\\
        &+ \delta \s{\pm 2\pi \eta \cos I + 4\sqrt{\eta \sin I}}.
\end{align}
Thus, the capture probability from a zone III initial condition into the
separatrix is
\begin{equation}
    P_c \approx \frac{\p{32 \eta_\star \cos I + 8f}\sqrt{\eta_\star \sin I}}{
        2\pi\p{1 - 2\eta_{\star}\sin I - f\eta_\star \cos I}
            + \p{16\eta_\star\cos I + 4f} \sqrt{\eta_\star\sin I}}.
            \label{eq:p3_p_c}
\end{equation}
An $\eta_\star^2$ term has been dropped from the denominator, and $f \equiv
\frac{\delta}{\epsilon}$ measures the relative strength of the two
perturbations. Note that $f = 0$ reduces to \autoref{s:p2}.

The good accuracy of \autoref{eq:p3_p_c} is illustrated in
\autoref{fig:p3_hist}. Both a red line from \autoref{eq:p3_p_c} and a black line
from numerically evaluating $\Delta h$ along the numerically-determined
separatrix (obtained by solving $H(\mu_{sep}, \phi_{sep}) = H_4$) are presented.
Good agreement is observed.
\begin{figure}
    \centering
    \begin{subfigure}{\columnwidth}
        \centering
        \includegraphics[width=\textwidth]{../initial/3_toy3/1hist20_0_00.png}
        \caption{$\delta/\epsilon = 0$}
    \end{subfigure}

    \begin{subfigure}{\columnwidth}
        \centering
        \includegraphics[width=\textwidth]{../initial/3_toy3/1hist20_0_30.png}
        \caption{$\delta/\epsilon = 0.3$}
    \end{subfigure}

    \begin{subfigure}{\columnwidth}
        \centering
        \includegraphics[width=\textwidth]{../initial/3_toy3/1hist20_0_70.png}
        \caption{$\delta/\epsilon = 0.7$}
    \end{subfigure}
    \caption{Histogram of simulation and inferred $P_c(s_\star)$ compared with
    \autoref{eq:p3_p_c}, where $f = \delta/\epsilon$ is varied. $3 \times 10^4$
    initial conditions were used for both plots. The solid red line denotes the
    prediction of \autoref{eq:p3_p_c}, while the black dotted line is estimated
    via numerical integration instead.}\label{fig:p3_hist}
\end{figure}

\section{Problem 4: Weak Tidal Dissipation (sims)}\label{s:p4}

\subsection{Solution}

We consider now the full Cassini State system with under weak tidal friction
(see e.g.\ \cite{lai2012}). Note that generally $\abs{g}$ is a constant under
weak tidal friction, as it is an orbit-orbit coupling term, while $\alpha
\propto s$. Defining then $s_c$ critical spin such that $\frac{s}{s_c} \equiv
\frac{\alpha}{\abs{g}}$\footnote{$s_c$ has interpretation \emph{critical spin}
such that $\alpha, \abs{g}$ are equal.} and $\Omega_1$ the spin that $\hat{s}$
is coupled to, we can write down fully coupled evolution equations for $(\mu,
\phi, s)$:
\begin{subequations}\label{se:p4_eom}
    \begin{align}
        \rd{\phi}{\tau} &= \frac{s}{s_c}\mu
            - \p{\cos I + \sin I \frac{\mu}{\sqrt{1 - \mu^2}} \cos \phi},\\
        \rd{\mu}{\tau} &= -\sin I \sin \phi +
            \epsilon \p{1 - \mu^2} \p{\frac{2\Omega_1}{s} - \mu},\\
        \rd{s}{\tau}
            &= \epsilon 2\Omega_1 \p{\mu - \frac{s}{2\Omega_1}\p{1 + \mu^2}}.
    \end{align}
\end{subequations}
The expressions for $H^{(0)}, H_{sep}$ can also be written down
\begin{align}
    H(\mu, \phi; s) &= -\frac{s}{s_c}\frac{\mu^2}{2}
        + \mu \cos I - \sin I \sqrt{1 - \mu^2}\cos \phi.\\
    H_{sep}(s) &= -\sin I + \frac{s_c}{2s}\cos^2 I.
\end{align}

Note that here $s$ takes the role of adiabatically-varying parameter. Thus, when
computing $\Delta_{\pm}$ along the separatrix using \autoref{eq:delta}, there
are two contributions. In particular, we write
\begin{equation}
    \Delta_{\pm} = \int_{\mathcal{C}_{\pm}} \dot{\mu}^{(1)} +
        \frac{\dot{s}}{\dot{\phi}^{(0)}} \pd{h}{s}
        \;\mathrm{d}\phi.
\end{equation}
Both of these integrals were performed in preceeding sections, and we may
express in closed form (dropping some terms for simplicity)
\begin{align}
    \frac{\Delta_{\pm}}{\epsilon} ={}&
        -2\cos I\p{\pm 2\pi \eta \cos I + 8\sqrt{\eta \sin I}}
        \pm 2\pi s\cos I\nonumber\\
        &+ \eta \cos I \p{-8\sqrt{\sin I / \eta}}
            + \frac{s}{2}8\sqrt{\sin I/\eta}\nonumber\\
        &+ \frac{2\Omega}{s}\p{\mp 2\pi\p{1 - 2\eta \sin I}
            + 16\cos I \eta^{3/2}\sqrt{\sin I}}\nonumber\\
        &+ 8\sqrt{\eta \sin I}
            \pm 2 \pi \eta \cos I
            - \frac{64}{3} \p{\eta \sin I}^{3/2},\label{eq:delta_p4}\\
    P_{c} &= \frac{\Delta_+ + \Delta_-}{\Delta_-}. \label{eq:p_c_p4}
\end{align}
Here, $P_c$ describes the separatrix capture probability from zone III\@. Note
that for $\frac{s_c}{s_\star} < \eta_c$ that no separatrix (or resonance) exists
anymore, so we will generally assume $s_\star \gtrsim s_c$ and correspeondingly
$\Omega_1$.

To test \autoref{eq:p_c_p4}, we generate an evenly spaced grid of initial
conditions $(\mu_0, \phi_0)$ and initialize $s_0 = 10$. We then record $s_\star$
and the final outcome of the trajectory over many simulations. The agreement of
\autoref{eq:p_c_p4} with data can be observed in \autoref{fig:1_4stats}. As
before, a red solid line depicting \autoref{eq:p_c_p4} and a black dotted line
using numerical evaluation of $\Delta h$ and a numerically-determined separatrix
are depicted.
\begin{figure}
    \centering
    \begin{subfigure}{\columnwidth}
        \centering
        \includegraphics[width=\columnwidth]{../initial/1_weaktide/4_cross_hist0_05.png}
    \end{subfigure}

    \begin{subfigure}{\columnwidth}
        \centering
        \includegraphics[width=\columnwidth]{../initial/1_weaktide/4_cross_hist0_20.png}
    \end{subfigure}

    \begin{subfigure}{\columnwidth}
        \centering
        \includegraphics[width=\columnwidth]{../initial/1_weaktide/4_cross_hist0_40.png}
    \end{subfigure}

    \caption{Histogram of simulation and inferred $P_c(s_\star)$ compared with
    \autoref{eq:p_c_p4}. $s_c$ used are numbered at the top of each plot. $10^4$
    simulations were used for each plot, and a single $I = 20^\circ$ was
    used. The solid red and dotted black lines denote N\autoref{eq:p_c_p4} and a
    full numerical evaluation of $P_c$ from $\Delta_{\pm}$ respectively. Note
    that small values of $s$ cannot be evaluated as the separatrix has
    disappeared $\eta > \eta_c$.}\label{fig:1_4stats}
\end{figure}

\subsection{Estimating $s_\star$}

In \autoref{s:p1}, the capture probability $P_c(\eta_\star)$ depended on
$\eta_\star$ which could be determined by conservation of the action varible
$I$. Such an argument is not possible in the weak tidal prescription as the
system is dissipitive thanks to $\dot{\mu}^{(1)}$ (see \autoref{se:p4_eom}).

However, instead we can write down $\rd{h}{\tau}$, just as \autoref{eq:dhdt}.
Since the initial value of $h$ can be computed from initial conditions, the ODE
can simply be integrated numerically to find $\tau_\star$ the separatrix
crossing time. In fact, for $\mu$ far from the separatrix, $\rd{h}{\tau} \approx
\frac{3s\mu^2}{2s_c}$, predicting
\begin{equation}
    \tau_\star = \frac{h(t = 0)}{3s\mu^2/2s_c} + C,\label{eq:tau_star}
\end{equation}
where $C$ depends on the detairs of $\rd{h}{\tau}$ near the separatrix. This is
in good agreement with measurement (see top panel of \autoref{fig:1_4cross}).
\begin{figure}
    \centering
    \includegraphics[width=\columnwidth]{../initial/1_weaktide/4_cross0_20.png}
    \caption{TODO make this plot actually look reasonable. The top and bottom
    panels show $t_\star, s_\star$ as a function of $-h = H_{sep} - H$. The
    dotted red line in the top panel is $t_\star = \frac{h}{3s\mu^2/2s_c} + C$
    where $C$ is a fitting parameter determined by the behavior of
    $\rd{h}{\tau}$ near the separatrix. This demonstrates that the separatrix
    arrival time is well-described by
    \autoref{eq:tau_star}.}\label{fig:1_4cross}
\end{figure}

\subsection{High Misalignment as an Attractor}

Careful examination of \autoref{eq:delta_p4} reveals that for sufficiently large
$s_c$, $\Delta_+$ changes sign and becomes positive. When $\Delta_+ < 0$,
trajectories in zone I above the separatrix cannot enter the separatrix, as $h$
decreases over an orbit. However, if $\Delta_+ > 0$, then trajectories in zone I
can also enter the separatrix. The critical value of $s_c$ where $\Delta_+$
changes sign as a function of $s$ can thus be determined analytically. TODO
write down the formula and check it against my numerical solution earlier.

Furthermore, $\rd{\mu}{\tau} < 0$ for $\mu > \frac{2\Omega_1}{s}$, it is clear
that if $s$ is initially large, then aligned states $\mu > 0$ are driven towards
misalignment and the separatrix, where they may then be captured for
sufficiently large $s_c$. We find that $s_c \gtrsim 0.5 \Omega_1$ is sufficient
that an initial spin of $s_0 = 10$ will \emph{always} be captured into CS2.

TODO Maybe this is worth a real plot?

\bibliographystyle{mnras}
\bibliography{Su_sep_cross}

% \clearpage
% \onecolumn
% \appendix

\label{lastpage} % chktex 24
\end{document}
