    \documentclass[
        fleqn,
        usenatbib,
    ]{mnras}
    \usepackage{
        amsmath,
        amssymb,
        newtxtext,
        newtxmath,
        ae, aecompl,
        graphicx,
        booktabs,
    }
    \usepackage[T1]{fontenc}
    \usepackage[
        labelfont=bf, % caption names are labeled in bold
        font=scriptsize % smaller font for captions
    ]{caption}

    \newcommand*{\rd}[2]{\frac{\mathrm{d}#1}{\mathrm{d}#2}}
    \newcommand*{\pd}[2]{\frac{\partial#1}{\partial#2}}
    \newcommand*{\md}[2]{\frac{\mathrm{D}#1}{\mathrm{D}#2}}
    \newcommand*{\at}[1]{\left.#1\right|}
    \newcommand*{\abs}[1]{\left|#1\right|}
    \newcommand*{\ev}[1]{\langle#1\rangle}
    \newcommand*{\p}[1]{\left(#1\right)}
    \newcommand*{\s}[1]{\left[#1\right]}
    \newcommand*{\z}[1]{\left\{#1\right\}}
    \DeclareMathOperator*{\argmin}{argmin}
    \DeclareMathOperator*{\argmax}{argmax}
    \DeclareMathOperator*{\med}{med}

\title[Cassini State Capture]{Cassini State Capture}
\author[Y. Su et\ al.]{
Yubo Su$^1$,
Dong Lai$^1$
\\
$^1$ Cornell Center for Astrophysics and Planetary Science, Department of
Astronomy, Cornell University, Ithaca, NY 14853, USA
}

\date{Accepted XXX\@. Received YYY\@; in original form ZZZ}

\pubyear{2019}

\begin{document}\label{firstpage}
\pagerange{\pageref{firstpage}--\pageref{lastpage}}
\renewcommand*{\sectionautorefname}{Section}
\maketitle


\begin{abstract}
    Abstract
\end{abstract}

\begin{keywords}
planet--star interactions % chktex 8
\end{keywords}

\section{Introduction}

Introduction, test citation \citep{henrard1982}.

\section{Cassini States}\label{s:cs}

Denote $\hat{s}$ spin of planet, $\hat{l}$ angular momentum of planet, and
$\hat{l}_p$ angular momentum of perturber. The Cassini state Hamiltonian in the
frame corotating with $\hat{l}_p$ about $\hat{l}$ is:
\begin{equation}
    H = -\frac{\alpha}{2}\p{\hat{s} \cdot \hat{l}}^2
        -g \p{\hat{s} \cdot \hat{l}_p}.
\end{equation}
$\alpha > 0, g < 0$ depend on the particular dynamics of the system. Frequently,
parameter
\begin{equation}
    \eta \equiv \frac{\abs*{g}}{\alpha} \label{eq:eta}
\end{equation}
is defined; we refrain from doing so immediately.

Choose $\hat{l} = \hat{z}$ and $\hat{l}_p = \cos I \hat{z} + \sin I \hat{x}$.
Furthermore, choose standard convention where $\phi = 0$ corresponds to
$\hat{l}_p, \hat{s}$ lying on \emph{opposite} sides of $\hat{l}$. This allows us
to evaluate Hamiltonian
\begin{equation}
    H = -\frac{\alpha}{2}\cos^2\theta
        + \abs{g} \p{\cos \theta \cos I - \sin I \sin \theta \cos \phi}
        \label{eq:H}
\end{equation}
Here, $\mu \equiv \cos \theta, \phi$ are canonically conjugate.

\subsection{Equilibria}

The evolution of $\hat{s}$ in this corotating frame is governed by:
\begin{equation}
    \rd{\hat{s}}{t} = \alpha \p{\hat{s} \cdot \hat{l}}
            \p{\hat{s} \times \hat{l}}
        - \abs{g}\p{\hat{s} \times \hat{l}_p}.
        \label{eq:dsdt_base}
\end{equation}

Spin states satisfying $\rd{\hat{s}}{t} = 0$ are referred to as \emph{Cassini
States} (CS). When $\eta < \eta_c$, there are four CSs, and when $\eta > \eta_c$
there are only two; $\eta_c$ is
\begin{equation}
    \eta_c \equiv \p{\sin^{2/3}I + \cos^{2/3}I}^{3/2}.
\end{equation}
Using the standard numbering given in \autoref{fig:0_1contours}, CSs 1, 2, 3 are
stable while CS4 is unstable.

\subsection{Separatrix}

In the four-CS regime, one of the CSs is a saddle point, conventionally denoted
Cassini State 4 (CS4). On the cylindrical phase space parameterized by $(\mu,
\phi)$, all trajectories are periodic with finite period except two critical (or
heteroclinic) trajectories. These are asymptotic in the past and future to CS4.
Together, these two heteroclinic are referred to as the \emph{separatrix} and
divide phase space into three zones.

The area enclosed by the separatrix is known exactly in literature
\citep{henrard1987}, but a serviceable approximation for small $\eta$ can be
derived as follows. Call $H_4$ the value of $H$ at CS4, then the separatrix is
defined implicitly by solutions to $H\p{\mu_{sep}(\phi), \phi} = H_4$. This may
be evaluated and we obtain
\begin{equation}
    \mu_{sep}(\phi) \approx \pm \sqrt{2\eta \sin I \p{1 - \cos \phi}}
        + \eta \cos I + \mathcal{O}\p{\eta^{3/2}}.\label{eq:mu_sep}
\end{equation}
The enclosed area between the two solutions can be obtained via explicit
integration. It can be expressed as a fraction of the total phase space area:
\begin{equation}
    \frac{A_{sep}}{4\pi} \approx \frac{4\sqrt{\eta \sin I}}{\pi}
        + \mathcal{O}\p{\eta^{3/2}}.\label{eq:a_sep}
\end{equation}

Contours of equal $H$, CSs and the separatrix are plotted in
\autoref{fig:0_1contours}.
\begin{figure}
    \centering
    \includegraphics[width=\columnwidth]{../initial/0_eta/1contours.png}
    \caption{Contours of equal $H$, given by \autoref{eq:H}, at different values
    of $\eta \equiv \frac{\abs{g}}{\alpha}$. Labeled are CS1 (red), CS2
    (purple), CS3 (green) and CS4 (cyan). The thick black dashed line is the
    separatrix (which disappears for $\eta > \eta_c$). Finally, the fractional
    area enclosed by the separatrix is noted in the plot subtitles, in good
    agreement with \autoref{eq:a_sep}.}\label{fig:0_1contours}
\end{figure}

\section{Separatrix Crossing: Theory}\label{s:sep_cross}

In an exactly Hamiltonian system, $H$ is a conserved quantity and trajectories
in phase space coincide with level curves of $H$. If a system is only slightly
non-Hamiltonian, trajectories will generally cross level curves. Of particular
interest is the behavior of trajectories near the separatrix (also a level
curve), as it frequently governs the dynamics of resonance capture (CITE MMR,
Cassini, others). The intersection of system trajectories with the separatrix is
referred to as \emph{separatrix crossing}.

In previous work by~\cite{henrard1982}, separatrix crossing was studied for systems
that are Hamiltonian save for an adiabatically-varying parameter. In related work
surveyed by~\cite{g_and_h}, the impact of small perturbations to the equations
of motion of a Hamiltonian system are studied (referred to as \emph{saddle
connection bifurcations}). However, in some astrophysical systems of interest,
leading-order perturbations to a Hamiltonian system contribute in both of the
above ways. Such systems resist characterization via either existing technique.
We suggest a possible generalization unifying the above techniques. [Probably
eventually move this paragraph to the introduction]

\subsection{Derivation}\label{ss:henrard}

Consider our Hamiltonian described in \autoref{eq:H} as a unperturbed
Hamiltonian $H^{(0)}(\mu, \phi; \eta)$ where $\eta$ is to be varied
adiabatically. As $\eta$ varies, the location of the separatrix also changes. We
follow~\cite{henrard1982} by defining
\begin{equation}
    h(\mu, \phi; \eta) \equiv H^{(0)}(\mu, \phi; \eta) - H_4(\eta).
\end{equation}
The significance of $h$ is that it always vanishes along the separatrix. Note
that $h$ corresponds to $K$ in~\cite{henrard1982}, though in our problem
$H^{(0)} > H_4$ corresponds to the \emph{interior} of the separatrix. The
evolution of $h$ over time then determines whether and when a trajectory
experiences separatrix crossing. Denoting time derivatives with dots, we write
\begin{align}
    \rd{h}{t} &= \rd{H^{(0)}}{t} - \pd{H_4}{\eta}\dot{\eta},\nonumber\\
        &= \s{\pd{H^{(0)}}{\mu}\dot{\mu}
                + \pd{H^{(0)}}{\phi}\dot{\phi}
                + \pd{H^{(0)}}{\eta}\dot{\eta}}
            - \pd{H_4}{\eta}\dot{\eta},\nonumber\\
        &= \s{\pd{H^{(0)}}{\mu}\dot{\mu}
                + \pd{H^{(0)}}{\phi}\dot{\phi}
                + \pd{H^{(0)}}{\eta}\dot{\eta}}
            - \pd{H_4}{\eta}\dot{\eta},\nonumber\\
        &= \s{\dot{\phi}^{(0)}\dot{\mu}^{(1)}
                - \dot{\mu}^{(0)}\dot{\phi}^{(1)}
                + \pd{H^{(0)}}{\eta}\dot{\eta}} -
            \pd{H_4}{\eta}\dot{\eta}.\label{eq:dhdt}
\end{align}
We've notated $\dot{\phi}^{(0)} \equiv \pd{H^{(0)}}{\mu}$ and $\dot{\phi}^{(1)}
= \dot{\phi} - \dot{\phi}^{(0)}$ and similarly for $\mu$. The scenario studied
in~\cite{henrard1982} corresponds to $\dot{\mu}^{(1)} = \dot{\phi}^{(1)} = 0$
while that in~\cite{g_and_h} sets $\dot{\eta} = 0$.

In the neighborhood of $\mathcal{C}_{\pm}$, $\rd{h}{t}$ can be
approximated by its value on $\mathcal{C}_{\pm}$. This is the \emph{guiding
orbit} approximation; its accuracy is proven in~\cite{henrard1982} for their
specific application, while we will take it on good faith in ours. Denote then
\begin{equation}
    \Delta_{\pm} \equiv \int_{\mathcal{C}_{\pm}} \rd{h}{t}\;\mathrm{d}t.
        \label{eq:delta}
\end{equation}
These approximate $\Delta h$ for trajectories near $\mathcal{C}_{\pm}$ over one
orbit. They are identical to the $B_i$ defined in~\cite{henrard1982}.

\begin{figure}
    \centering
    \includegraphics[width=\columnwidth]{../initial/1_weaktide/3zones.png}
    \caption{Nomenclature of the two legs of the separatrix $\mathcal{C}_{\pm}$
    and three zones of the domain. For both $\mathcal{C}_{\pm}$, we will take
    the positive direction to be anti-clockwise.}\label{fig:1_3zones}
\end{figure}
For concreteness, we consider a specific example first. We assign labels to our
phase space as shown in \autoref{fig:1_3zones}. Consider a trajectory that
begins in zone III, which is circulating with $\dot{\phi} > 0$ and has $h < 0$,
and consider if $\rd{h}{t} > 0$ in all of zone III, such that the trajectory
eventually experiences separatrix crossing at $\mathcal{C}_-$. The following
sequence of events unfolds:
\begin{enumerate}
    \item[\textbf{(i)}] At the beginning ($\phi = 0$) of its separatrix-crossing
        orbit, $h_i \equiv h(\mu, \phi = 0; \eta) \in (\Delta_-, 0)$. Note
        that $h_i < 0$ to still be in zone III at the start of the orbit, while
        $h_i > \Delta_-$ to be separatrix crossing during the final orbit.

    \item[\textbf{(ii)}] After traversing the length of $\mathcal{C}_-$, the
        value of $h$ is now $h_i + \Delta_- > 0$ (recall $\Delta_- > 0$), and so
        the trajectory has entered zone II\@. Since zone II hosts only librating
        solutions, the trajectory will then ``turn around'' and track
        $\mathcal{C}_+$ from inside zone II\@.

    \item[\textbf{(iii)}] At the end of following $\mathcal{C}_+$, $h$ now takes
        value $h_f \equiv h_i + \Delta_- + \Delta_+$. There are now two
        possibilities:

    \item[\textbf{(iii-a)}] If $h_f > 0$, the trajectory remains inside zone II
        accrues further multiples of $\Delta_- + \Delta_+ = h_f - h_i > 0$.
        Thus, the trajectory must securely enter zone II\@. This outcome
        commonly corresponds to \emph{resonance capture}, a transition into zone
        II\@.

    \item[\textbf{(iii-b)}] If $h_f < 0$, the trajectory exits zone II and
        enters zone I\@. Since $\Delta_+ = h_f - h_i - \Delta_- < 0$ (as $h_f
        < 0, h_i + \Delta_- > 0$), the trajectory will continue circulation in
        zone I and accrue further multiples of $\Delta_+$, securely entering
        zone I\@. This corresponds to a zone III-zone I transition,
        \emph{escape}.
\end{enumerate}

Thus, the result of separatrix crossing depends on the sign of $h_i +
\Delta_- + \Delta_+$ where $\Delta_{\pm}$ are given by \autoref{eq:delta} and
$h_i \in [-\Delta_-, 0]$. The result of this analysis reproduces the conclusion
of~\cite{henrard1982}:
\begin{itemize}
    \item If $\Delta_+ < -\Delta_-$, then $h_f < 0$ necessarily, secure escape
        ensues.

    \item If $\Delta_+ > 0$, then $h_f > 0$ necessarily, and secure capture
        ensues.

    \item If $\Delta_+ + \Delta_- \in [0, \Delta_-]$, then it is clear that the
        sign of $h_f$ depends on the precise value of $h_i$. Generally,
        $\Delta_-$ can be assumed to be small as perturbations are weak, and
        so $\Delta_-$ is usually much smaller than the range of $h$ of interest.
        Thus, $h_i$ can be treated as uniformly distributed on interval
        $[-\Delta_-, 0]$, and a probability of resonance capture can be
        computed:

        \begin{equation}
            P_c \equiv \frac{\Delta_+ + \Delta_-}{\Delta_-}.
                \label{eq:p_c_gen}
        \end{equation}
\end{itemize}
With the exception of \autoref{eq:dhdt}, the above outlines the argument given
in~\cite{henrard1982}.

Similar derivations can be given for trajectories originating in zone I.
TODO say something about zone-II originating systems.

\subsection{Equivalence to Stable/Unstable Manifold Splitting}\label{ss:meln}

We now establish equivalence to an alternative formulation popular in dynamical
systems that is more graphically intuitive. For simplicity, we will take
$\dot{\eta} = 0$.

If we return to the unperturbed Hamiltonian system, recall that CS4 is joined to
itself by an infinite-period orbit. For a given saddle point, we may define the
\emph{stable manifold} $W_s$ of a saddle point to be the set of points whose
forward evolution converges to the saddle point. Similarly, we may define the
\emph{unstable manifold} $W_u$ of a saddle point to be the set of points that
originated \emph{from} the saddle point, or whose backwards evolution converges
to the saddle point. A \emph{saddle connection} is when the unstable manifold of
one saddle point and the unstable manifold of another saddle point are
degenerate. It is evident that the separatrix is formed of two saddle
connections: $\mathcal{C}_+$ is the unstable manifold of CS4 at $\phi = 2\pi$
and the stable manifold of CS4 at $\phi = 0$.

In the presence of a small perturbation, the stable and unstable manifolds
generally separate, as the perturbations along the saddle connection will
generally be different going forwards and backwards in time. The separation
distance as a function of time along the \emph{unperturbed} trajectory can be
computed via \emph{Melnikov's Method}
\begin{equation}
    d(t_0) = \frac{\epsilon}{\abs{f(q^0(-t_0))}}
        \int\limits_{-\infty}^\infty
            f(q^0(t - t_0)) \wedge g(q^0(t - t_0), t)\;\mathrm{d}t.
\end{equation}
The notation used here is that of~\cite{g_and_h}, where $d(t_0)$ is the distance
between the stable/unstable manifolds at time $t_0$ along the heteroclinic
orbit, $f = \p{\dot{\mu}^{(0)}, \dot{\phi}^{(0)}}$, $\epsilon g =
\p{\dot{\mu}^{(1)}, \dot{\phi}^{(1)}}$ a general time-dependent perturbation,
$q^0(t)$ is the unperturbed trajectory along the saddle connection (time
coordinate defined such that $t = 0$ coincides with the ``middle'' of the
infinite trajectory), and $\wedge$ is the wedge product.

We will assume $\epsilon g$ is not explicitly time-dependent, characteristic of
secular/averaged equations in astrophysics, then the integral above is simply
the familiar
\begin{equation}
    d(t_0) \abs{\vec{\nabla}H^{(0)}}(t_0) = \epsilon \int\limits_{-\infty}^\infty
        \dot{\mu}^{(0)}\dot{\phi}^{(1)} - \dot{\phi}^{(0)}\dot{\mu}^{(1)}
            \;\mathrm{d}t.\label{eq:melnikov}
\end{equation}
The right hand side is just the total change in $H^{(0)}$ over the saddle
connection though, equivalent to our \autoref{eq:delta} in the $\dot{\eta} = 0$
limit. The left hand side on the other hand is exactly $\abs{\vec{d} \cdot
\vec{\nabla}H^{(0)}}$ where the displacement is along $\vec{\nabla}H^{(0)}$.
Thus, the effect of the $\Delta_{\pm}$ is to split the stable/unstable manifolds
along the saddle connections.

Visually, this is depicted in \autoref{fig:0_6manifolds}. Note that:
trajectories starting in zone III below $W_s^{(1)}$ fail to separatrix cross;
those between $W_s^{(1)}$ and $W_s^{(0)}$ continue between $W_u^{(1)}$ and
$W_s^{(0)}$, eventually escaping; those between $W_s^{(0)}$ and $W_u^{(0)}$ are
securely captured, and no trajectories can begin above $W_u^{(0)}$ at $\phi =
0$.

The saddle connection splitting is easiest to measure where $\vec{\nabla}H^{(0)}
\propto \hat{\phi}$, where $\dot{\mu} = 0$, and \autoref{eq:melnikov} reduces to
just $\Delta \mu \pd{H^{(0)}}{\mu} = \Delta H^{(0)}$. Then the capture
probability must be given by ratio of $\Delta \mu$ separating $W_s^{(0)},
W_u^{(0)}$ to $\Delta \mu$ separating $W_s^{(1)}, W_u^{(0)}$. These $\Delta \mu$
values are those given in the title, and the resultant $P_c$ capture probability
is also quoted.
\begin{figure}[t]
    \centering
    \includegraphics[width=\columnwidth]{../initial/0_eta/6manifolds0_20.png}
    \caption{Sample plot of saddle connection breaking in the presence of weakly
    non-Hamiltonian dissipation. Equations are used from \autoref{s:p1}. Stable
    and unstable manifolds are indexed such that superscript $(0)$/$(1)$
    belongs to the saddle point at left/right. $\Delta \mu$ represent the
    manifold separations, resulting in capture probability $P_c$
    prediction per \autoref{eq:p_c_dmu}. Numerical simulations predict $P_c
    \approx 0.252$.}\label{fig:0_6manifolds}
\end{figure}

Finally, the connection between \autoref{ss:henrard} and \autoref{ss:meln} is
evident: the ratio of the $\Delta \mu$ values is simply
\begin{align}
    P_c &= \frac{\Delta \mu(W_s^{(0)}, W_u^{(0)})}{
            \Delta \mu\p{W_s^{(1)}, W_u^{(0)}}},\label{eq:p_c_dmu},\\
        &= \frac{\Delta H^{(0)}_- + \Delta H^{(0)}_+}{\Delta H^{(0)}_-}.
\end{align}
This is exactly \autoref{eq:p_c_gen} when $\dot{\eta} = 0$.

\section{Problem 1: Weakly Dissipitive Hamiltonian}\label{s:p1}

We first consider a toy separatrix crossing problem in the $\dot{\eta} = 0$
limit, the case discussed in \autoref{ss:meln}. Consider adding to the
Hamiltonian Cassini State problem an additional aligning term
$\p{\rd{\theta}{t}}^{(1)} = -\frac{1}{t_s}\sin\theta$ that favors alignment
$\theta = 0$ on synchronization timescale $t_s$. This translates to equation of
motion
\begin{equation}
    \rd{\hat{s}}{\tau} = \p{\hat{s} \cdot \hat{l}}\p{\hat{s} \times \hat{l}}
        - \eta\p{\hat{s} \times \hat{l}_p}
        - \epsilon \sin \theta \hat{\theta}.
\end{equation}
We have divided \autoref{eq:dsdt_base} by $\alpha$ and defined $\tau \equiv
\alpha t, \epsilon \equiv \frac{1}{\alpha t_s}$. The equations of motion for
canonical variables $\mu \equiv \cos\theta, \phi$ are
\begin{subequations}\label{se:p1_eom}
    \begin{align}
        \mu' &= -\cos\theta
            + \eta\p{\cos I + \sin I \frac{\mu}{\sqrt{1 - \mu^2}} \cos \phi},\\
        \mu' &= -\eta \sin I \sin \theta \sin \phi
            \s{+ \epsilon\p{1 - \mu^2}}.
    \end{align}
\end{subequations}
Primes denote $\rd{}{\tau}$. The bracketed term in $\mu'$ is the perturbation
component.

With a dissipitive term that favors alignment, we expect that separatrix
encounters can only occur from zone II and zone III\@. We will focus on the
outcomes of trajectories origiating in zone III for now. We now aim to compute
\autoref{eq:p_c_gen}, for which we need:
\begin{subequations}
    \begin{align}
        \Delta_- &= \int\limits_{\mathcal{C}_-}
                \phi'\epsilon\p{1 - \mu^2}\;\mathrm{d}t,\nonumber\\
            &= \int\limits_{\mathcal{C}_s}
                \p{1 - \mu^2}\;\mathrm{d}\phi,\\
        \Delta_- + \Delta_+ &= \int\limits_{\mathcal{C}_- + \mathcal{C}_+}
            \epsilon\p{1 - \mu^2}\;\mathrm{d}\phi.
    \end{align}
\end{subequations}
Note that along $\mathcal{C}_{\pm}$ that $\mathrm{d}\phi$ has sign $\mp$
respectively, thanks to our sign convention. We proceed to integrate:
\begin{itemize}
    \item $\Delta_-$: Recalling the parameterization for the separatrix
        $\mu_{sep}(\phi)$ given by \autoref{eq:mu_sep}, we can approximate the
        integrand as $1 + \mathcal{O}(\eta^1)$. This yields simply $\Delta_-
        \approx 2\pi \epsilon + \mathcal{O}(\eta^1)$.

    \item $\Delta_- + \Delta_+$: Since $\mathrm{d}\phi$ changes signs from
        $\mathcal{C}_-$ to $\mathcal{C}_+$, only terms in $(1 - \mu^2)$ that
        also change sign over $\mathcal{C}_{\mp}$ contribute to the integral.
        The leading order term that does so can be read off of
        \autoref{eq:mu_sep}, so the integral evaluates as
        \begin{align}
            \Delta_- + \Delta_+ &\approx
                2\int\limits_0^{2\pi}
                    \s{(\eta \cos I) \sqrt{2\eta \sin I(1 - \cos \phi)}
                        + \mathcal{O}(\eta^{5/2})} \;\mathrm{d}\phi,\\
                &\approx 32\epsilon \eta^{3/2} \cos I \sqrt{\sin I}
                    + \mathcal{O}(\eta^{5/2}).
        \end{align}
\end{itemize}
From these two integrals, we obtain
\begin{equation}
    P_c = \frac{16\eta^{3/2}\cos I \sqrt{\sin I}}{\pi}.\label{eq:p1pc}
\end{equation}

As evidence, we provide the following table of separatrix capture probabilities
in \autoref{tab:p_c_num} (TODO make a plot and more robust parameter
exploration):
\begin{table*}
    \centering
    \begin{tabular}{c|c c c c}
        $\eta$ & $0.025$ & $0.05$ & $0.1$ & $0.2$\\\midrule
        Numerical $P_c$ & $0.010 \pm 0.002$ & $0.028 \pm 0.003$
            & $0.085 \pm 0.004$ & $0.251 \pm 0.006$ \\
        Analytical $P_c$ & $0.0111$ & $0.0313$ & $0.0886$ & $0.2503$.
    \end{tabular}
    \caption{Capture probability for four different values of $\eta$, all using
    $\epsilon = 3 \times 10^{-4}$. Different values of $\epsilon$ were tried
    $\epsilon \in 10^{[-2, -4]}$ with no effect on $P_c$. $10000$ random initial
    conditions were used for the numerical calculations, of which roughly $1/2$
    experience separatrix crossing. Numerical uncertainties are estimated as
    $\sqrt{N}$ errors.}\label{tab:p_c_num}
\end{table*}

For trajectories originating in zone II, numerical simulations indicate they can
never leave the separatrix. This can be intuited by considering trajectories
that are near the separatrix; over one orbit, they accumulate $\Delta h =
\Delta_+ + \Delta_- > 0$, which acts to drive them \emph{towards} CS2 and away
from the separatrix. Thus, separatrix crossing is forbidden in zone II, and our
above characterization of separatrix encounters with $P_c$ in \autoref{eq:p1pc}
is complete.

TODO prove the above completely.

\section{Problem 2: Adiabatic Capture}\label{s:p2}

Here, instead of introducing an aligning perturbation, we will instead
adiabatically vary $\eta$.

\subsection{Spindown Model}

We may first consider the unperturbed Hamiltonian
system with $\rd{\eta}{t} = \frac{\eta}{t_\eta}$ growth with characteristic
timescale $t_\eta$. Again rescaling time $\tau \equiv \alpha t$ and defining
$\epsilon \equiv \frac{1}{\alpha t_\eta}$, we obtain $\eta' = \epsilon
\eta$, where $\epsilon > 0$. Such a model is motivated e.g.\ when a planet's
spin $\hat{s}$ orbits a host star with angular momentum $\hat{l}$ while the star
spins down; $\dot{\alpha} < 0$ here, and $\eta' > 0$.

We then seek $\Delta_{\pm}$, determined by
\begin{align}
    \Delta_{\pm} &= \int\limits_{\mathcal{C}_{\pm}}
            \pd{h}{\eta}\eta'\;\mathrm{d}\tau,\nonumber\\
        &= \int\limits_{\mathcal{C}_{\pm}}
            \frac{\pd{h}{\eta}}{\phi'}\eta'\;\mathrm{d}\phi.
\end{align}
We must proceed carefully since $\phi' \to 0$ at the endpoints of
$\mathcal{C}_{\pm}$; careful work shows these zeros cancel against zeros of
$\pd{h}{\eta}$, and
\begin{equation}
    \Delta_{\pm} = \int\limits_{\mathcal{C}_{\pm}} \eta'
        \p{\cos I \mp \sin I \sqrt{\frac{1 - \cos \phi}{2\eta\sin I}}}
            \;\mathrm{d}\phi.\label{eq:p2_delta}
\end{equation}
Immediately, we see that if we use $\eta' = \epsilon \eta$, then $\Delta_{\pm} >
0$ and \emph{all separatrix encounters result in capture} regardless of whether
the trajectory originates in zone I or zone III\@. Since increasing $\eta$
results in larger separatrix, zone II trajectories will not experience
separatrix crossing. Indeed, an example of such a simulation is presented in
\autoref{fig:2_1testo4}. Further simulations with other initial conditions
support the guaranteed capture prediction in this system.
\begin{figure}[t]
    \centering
    \includegraphics[width=\columnwidth]{../initial/2_toy2/1testo4.png}
    \caption{Simulation with initial condition in zone I subject to $\eta' =
    \epsilon \eta$, where $\epsilon = 10^{-4}$. The top panel shows the
    trajectory (red) and separatrix at separatrix crossing (dashed black) in
    the labelled coordinates. The bottom panel shows the evolution of the
    enclosed area by the trajectory (red dots), area of the separatrix (blue)
    and $\eta$ (dotted green) over time.}\label{fig:2_1testo4}
\end{figure}

If instead $\epsilon < 0$, then only zone II initial conditions experience
separatrix encounter, and since $\Delta_{\pm} < 0$, all trajectories are
ejected as the separatrix shrinks.

\subsection{Toy Model}

It is clear that $\eta'$ in \autoref{eq:p2_delta} being the same sign along both
$\mathcal{C}_{\pm}$ is the source of this guaranteed capture. To assess the
accuracy of our formalism, we choose to further study $\eta' = \epsilon \eta
\mu$. Straightforward evaluation of \autoref{eq:p2_delta} yields
\begin{equation}
    \Delta_{\pm} = \pm \epsilon \eta_\star \s{
        2\pi \sin I \pm 12 \sqrt{\eta_\star \sin I}\cos I
            + 2\pi \eta_\star \cos^2 I}.
\end{equation}
It bears noting that we must evaluate these integrals at $\eta = \eta_\star$ the
value at separatrix crossing. Since $\eta$ evolves over time, $\eta_{\star}$
depends on initial conditions. While one might expect the adiabatic invariance
of the action variable determines $\eta_\star$, this turns out to not be the
case; we address this in \autoref{ss:noncons}.

Since $\abs{\Delta_+} > \abs{\Delta_-}$, separatrix encounters originating in
zone III always result in capture. However, encounters originating in zone I
experience capture probability
\begin{equation}
    P_c = \frac{24\cos I \sqrt{\eta_\star \sin I}}{
        2\pi \sin I + 12\sqrt{\eta_\star \sin I}\cos I
            + 2\pi \eta_\star \cos^2 I}.
\end{equation}
This is in good agreement
% TODOYUBO plots

\subsection{Non-conservation of Adiabatic Invariant}\label{ss:noncons}

Best evidence is $\eta_{\star}$ when changing $I$, blatantly wrong scaling

\section{Problem 3: Weak Tidal Dissipation}\label{s:p3}

% TODOYUBO show that zone II stays in zone II?

\bibliographystyle{mnras}
\bibliography{paper}

% \clearpage
% \onecolumn
% \appendix

\label{lastpage} % chktex 24
\end{document}
