    \documentclass[
        fleqn,
        usenatbib,
        % referee,
    ]{mnras}
    \usepackage{
        amsmath,
        amssymb,
        newtxtext,
        newtxmath,
        graphicx,
        ae, aecompl,
        booktabs,
        caption,
        subcaption,
    }
    \usepackage[T1]{fontenc}
    \captionsetup{compatibility=false}

    \newcommand*{\rd}[2]{\frac{\mathrm{d}#1}{\mathrm{d}#2}}
    \newcommand*{\rtd}[2]{\frac{\mathrm{d}^2#1}{\mathrm{d}#2^2}}
    \newcommand*{\pd}[2]{\frac{\partial#1}{\partial#2}}
    \newcommand*{\pdil}[2]{\partial#1 / \partial#2}
    \newcommand*{\md}[2]{\frac{\mathrm{D}#1}{\mathrm{D}#2}}
    \newcommand*{\at}[1]{\left.#1\right|}
    \newcommand*{\abs}[1]{\left|#1\right|}
    \newcommand*{\ev}[1]{\langle#1\rangle}
    \newcommand*{\bm}[1]{\boldsymbol{\mathbf{#1}}}
    \newcommand*{\uv}[1]{\hat{\bm{#1}}}
    \newcommand*{\p}[1]{\left(#1\right)}
    \newcommand*{\s}[1]{\left[#1\right]}
    \newcommand*{\z}[1]{\left\{#1\right\}}
    \DeclareMathOperator*{\argmin}{argmin}
    \DeclareMathOperator*{\argmax}{argmax}
    \DeclareMathOperator*{\med}{med}

\title[Exoplanet Obliquities]{Dynamics of Colombo's Top: Generating Exoplanet
Obliquties from Planet-Disc Interactions}
\author[Y. Su and D. Lai]{
Yubo Su$^1$,
Dong Lai$^1$
\\
$^1$ Cornell Center for Astrophysics and Planetary Science, Department of
Astronomy, Cornell University, Ithaca, NY 14853, USA
}


% for i in ../initial/99_misc/2_3vec.png ../initial/99_misc/2_cs_locs.png ../initial/0_eta/1contours_flip.png ../initial/99_misc/1_areas.png ../initial/2_toy2/3_ensemble_05_35.png ../initial/2_toy2/3_ensemble_10_35.png ../initial/2_toy2/3_ensemble_20_35.png ../initial/2_toy2/3testo21.png ../initial/2_toy2/3testo21_subplots.png ../initial/2_toy2/3testo23.png ../initial/2_toy2/3testo23_subplots.png ../initial/2_toy2/3testo31.png ../initial/2_toy2/3testo31_subplots.png ../initial/2_toy2/3testo321.png ../initial/2_toy2/3testo321_subplots.png ../initial/2_toy2/3_ensemble_05_25.png ../initial/2_toy2/3_ensemble_05_15.png ../initial/2_toy2/3testo_nonad.png ../initial/2_toy2/3testo_nonad_subplots.png ../initial/2_toy2/3_ensemble_05_05.png ../initial/2_toy2/3scan.png ../initial/2_toy2/3scan_20.png ../initial/99_misc/2_lambdas.png; do cp $i plots_diskdisp; done

\date{Accepted XXX\@. Received YYY\@; in original form ZZZ}

\pubyear{2020}

\begin{document}\label{firstpage}
\pagerange{\pageref{firstpage}--\pageref{lastpage}}
\maketitle

\begin{abstract}
%DL: Not done yet
    Large planetary spin-orbit misalignments (obliquities) are thought to play a
    key role in some planets' atmospheric circulation and tidal heating. A
    promising avenue to generate high obliquities is via spin-orbit resonances,
    where the spin and orbital precession frequencies of the planet become and
    remain commensurate as the system evolves. One such proposed mechanism
    involves a dissipating protoplanetary disk driving orbital precession of an
    interior planet \citep{millholland_disk}. This previous work is limited to
    small initial misalignment angles between the spin axis of the planet and
    orbital axis of the massive disk. In the present paper, this scenario is
    analytically characterized and extended to arbitrary misalignment angles. It
    is shown that (i) under adiabatic evolution, final planetary obliquities
    bifurcate into distinct tracks governed by interactions with the resonance,
    while (ii) under nonadiabatic evolution, a broad range of obliquities can
    be excited. Analytical formulae are provided for most interesting regimes of
    parameter space.
\end{abstract}

\begin{keywords}
planet--star interactions % chktex 8
\end{keywords}

\section{Introduction}\label{s:intro}

\subsection{Colombo's Top}

A rotating planet is subjected to gravitational torque from its host star,
making its spin axis precess around its orbital (angular momentum) axis.  Now
suppose the orbital axis precesses around another fixed axis---such orbital
precession could arise from gravitational interactions with other masses in the
system (e.g.\ planets, external discs, or binary stellar companion). What is the
dynamics of the planetary spin axis?  How does the spin axis evolve as the spin
precession rate, the orbital precession rate, or their ratio, gradually changes
in time?

\citet{colombo1966} was the first to point out the importance of the above
simple model in the study of the obliquity (the angle between the spin axis and
orbital axis) of planets and satellites. Subsequent works
\citep{peale1969,peale1974possible,ward1975tidal,henrard1987} have revealed rich
dynamics of this model. With appropriate modification,
% (e.g. including the nutation of the orbital axis and multi-frequency
% precessions due to multiple planets),
this model can be used as a basis for understanding the evolution of rotation
axes of celestial bodies.  Indeed, many contemporary problems in
planetary/exoplanetary dynamics can be casted into a form analogous to this
simple model or its variant \citep[e.g.][]{ward2004I, fabrycky_otides,
batygin2013magnetic, lai2014star, anderson2018teeter}.

In this paper we present a systematic investigation on the secular evolution of
Colombo's top, starting from general initial conditions. Our study includes
several new analytical results that go beyond previous works. While our results
are general, we frame our study in the context of generating exoplanet
obliquities from planet-disc interaction with in a dissipating disc.

\subsection{Planetary Obliquities from Planet-Disc Interaction}

It is well recognized that the obliquity of a planet may provide important clue
to its dynamical history. In the the Solar System, a wide range of planetary
obliquities are observed, from nearly zero for Mercury and $3.1^\circ$ for
Jupiter, to $23^\circ$ for Earth and $26.7^\circ$ for Saturn, to $98^\circ$ for
Uranus. Multiple giant impacts are traditionally invoked to generate the large
obliquities of ice giants \citep{original_gi, benz1989tilting,
korycansky1990one, morbidelli_gi}. For gas giants, obliquity excitations may be
achieved via spin-orbit resonances, where the spin and orbital precession
frequencies of the planet become commensurate as the system evolves
\citep{ward2004I, ward2004II, vokrouhlicky2015tilting}. Such resonances may also
play a role in generating Uranus's obliquity \citep{hamilton_tilting_ice}. For
terrestrial planets, multiple spin-orbit resonances and their overlaps can make
the obliquity vary chaotically over a wide range
\citep[e.g.][]{laskar1993chaotic, touma1993chaotic, correia2003long}

Obliquities of extrasolar planets are difficult to measure. So far only loose
constraints have been obtained for the obliquity of a faraway ($\gtrsim 50$~au)
planetary-mass companion \citep{bryan2020obliquity}. But there are prospects for
constraining exoplanetary obliquities in the coming years, such as using
high-resolution spectroscopy to obtain $v\sin i$ of the planet
\citep{snellen2014fast, bryan2018constraints} and using high-precision
photometry to measure asphericity of the planet \citep{seager2002constraining}.
Finite planetary obliquities have been indirectly inferred to explain the
peculiar thermal phase curves \citep[see e.g.][]{millholland_signatures,
ohno_infer_obl} and tidal dissipation in hot Jupiters
\citep{millholland_wasp12b} and in super-Earths
\citep{millholland2019obliquity}.

It is natural to imagine some of the mechanisms that generate planetary
obliquities in the Solar System may also operate in exoplanetary systems.
Recently, \citet{millholland_disk} studied the production of planet obliquities
via spin-orbit resonance, where a dissipating protoplanetary disk causes
resonance capture and advection. In their work, a planet is accompanied by an
inclined exterior disk; as the disk graduatelly dissipates, the resulting
planetary obliquity is generally substantial, reaching $90^\circ$ for what they
characterize as adiabatic resonance crossings.

The Millholland \& Batygin study assumes a negligible initial planetary
obliquity. This assumption is intuitive, since the planet attains its spin
angular momentum from the disk. But it may not always be satisfied. In
particular, the formation of rocky planets through planetesimal accretion can
lead to a wide range of obliquities, especially if the final spin is imparted by
a few large bodies \citep{dones1993does, lissauer1997accretion,
miguel2010planet}. Such ``stocastic'' accretion likely happened for terrestrial
planets in the Solar System. Giant impacts may have also played a role in the
formation of the close-in multiple-planet systems discovered by the Kepler
satellite \citep[e.g.][]{inamdar2015formation, izidoro2017breaking}.

\subsection{Goal of This Paper}

In this paper, we consider a wide range of initial planetary obliquities in the
Millholland-Batygin dissipating disk scenario, and examine how the obliquity
evolves toward the ``final'' value as the exterior disk dissipates. We provide
an analytical framework for understanding the final planetary obliquity for
arbitrary initial spin-disk misalignment angles. We also consider various
dissipation timescales, and examine both ``adiabatic'' (slow disk dissipation)
and ``non-adiabatic'' evolution. We calibrate our analytical results with
numerical calculations.

It is important to note that while we focus on a specific scenario of
generating/modifying planetary obliquities from planet-disk interactions, our
analysis and results have a wide range of applicability. For example, a
dissipating disk is dynamically equivalent to an outward migrating external
companion.

The paper is organized as follows. In Section~\ref{s:eq}, we review the relevant
spin-orbit dynamics, equations and key concepts that are used in the remainder
of the paper. In Sections~\ref{s:ad} and~\ref{s:nonad}, we study the evolution
of the system when the disk dissipates on different timescales, from highly
adiabatic to nonadiabatic. Analytical results are presented to explain the
numerical results in both limits. We discuss the implications of our results in
Section~\ref{s:disc}. Our primary physical results consist of
Fig.~\ref{fig:ad_ensemble} in the adiabatic limit and
Fig.~\ref{fig:3_ensemble_05_15} in the nonadiabatic limit. Some detailed
calculations are relegated to the appendicies, including a leading-order
estimate of the final planetary obliquities given small initial spin-disk
misalignment angles in Appendix~\ref{s:ad_approx}.

%%%%%%%%%%%%%%%%%%%%%%%%%%%%%
\section{Theory}\label{s:eq}

\subsection{Equations of Motion}

We consider a star of mass $M_\star$ hosting an oblate planet (mass $M_{\rm p}$,
radius $R_{\rm p}$ and spin angular frequency $\Omega_{\rm p}$) at semimajor
axis $a_{\rm p}$, and a protoplanetary disk of mass $M_{\rm d}$. For simplicity,
we treat the disk as a ring of radius $r_{\rm d}$, but it is simple to
generalize to a disk with finite extent \citep[see][]{millholland_disk}. Denote
$\bm{S}$ the spin angular momentum and $\bm{L}$ the orbital angular momentum of
the planet, and $\bm{L}_{\rm d}$ the angular momentum of the disk. The
corresponding unit vectors are $\uv{s} \equiv \bm{S} / S$, $\uv{l} \equiv \bm{L}
/ L$, and $\uv{l}_{\rm d} \equiv \bm{L}_{\rm
d} / L_{\rm d}$.

The spin axis $\uv{s}$ of the planet tends to precess around its orbital
(angular momentum) axis $\uv{l}$, driven by the gravitational torque from the
host star acting on the planet's rotational bulge. On the other hand, $\uv{l}$
and the disc axis $\uv{l}_{\rm d}$ precess around each other due to
gravitational interactions. We assume $S \ll L \ll L_{\rm d}$, so $\uv{l}_{\rm
d}$ is nearly constant and $\uv{l}$ experiences negligible backreaction torque
from $\uv{s}$. The
equations of motion for $\uv{s}$ and $\uv{l}$ are
\begin{align}
    \rd{\uv{s}}{t} &= \omega_{\rm sl} \p{\uv{s} \cdot \uv{l}}
            \p{\uv{s} \times \uv{l}}
        \equiv \alpha \p{\uv{s} \cdot \uv{l}}
            \p{\uv{s} \times \uv{l}},\label{eq:dsdt}\\
    \rd{\uv{l}}{t} &= \omega_{\rm ld}\p{\uv{l} \cdot \uv{l}_{\rm d}}
            \p{\uv{l} \times \uv{l}_{\rm d}}
        \equiv -g\p{\uv{l} \times \uv{l}_{\rm d}},
            \label{eq:dldt}
\end{align}
where
\begin{align}
    \omega_{\rm sl} &\equiv \frac{3GJ_2 M_pR_p^2 M_\star}{2a_p^3 I_p\Omega_p}
        = \frac{3k_{\rm qp}}{2k_{\rm p}} \frac{M_\star}{m_{\rm p}}
            \p{\frac{R_{\rm p}}{a_{\rm p}}}^3 \Omega_{\rm p},\label{eq:wsl}\\
    \omega_{\rm ld} &\equiv \frac{3M_{\rm d}}{4M_\star}\p{\frac{a_{\rm
            p}}{r_{\rm d}}}^3 n .\label{eq:wld}
\end{align}
In Eq.~\eqref{eq:wsl}, $I_p = k_p M_pR_p^2$ (with $k_p$ a constant) is the
moment of inertia and $J_2=k_{\rm qp}\Omega_p^2 (R_p^3/GM_p)$ (with $k_{qp}$ a
constant) the rotation-induced (dimensionless) quadrupole of the planet [for a
body with uniform density, $k_p=0.4, k_{qp}=0.5$; for giant planets, $k_p\simeq
0.25$ and $k_{qp}\simeq 0.17$ \citep[e.g.][]{lainey2016quantification}]. In
Eq.~\eqref{eq:wld}, $n \equiv \sqrt{GM_\star/a_{\rm p}^3}$ is the planet's
orbital mean motion, and we have assumed $r_d\gg a_p$ and included only the
leading-order (quadrupole) interaction between the planet and disc. We define
three relative inclination angles via
\begin{equation}
  \uv{s} \cdot \uv{l}\equiv \cos\theta,\quad
  \uv{s} \cdot \uv{l}_{\rm d}\equiv \cos\theta_{\rm sd},\quad
  \uv{l} \cdot \uv{l}_{\rm d}\equiv \cos I.
\end{equation}
In our model, $I$ is a constant.
Following standard notation \citep[e.g.][]{colombo1966,peale1969,ward2004I}, we
have defined $\alpha \equiv \omega_{\rm sl}$ and $g \equiv -\omega_{\rm ld}\cos I$.

We can combine Eqs.~\eqref{eq:dsdt} and~\eqref{eq:dldt} into a single equation
by transforming into a frame rotating about $\uv{l}_{\rm d}$ with frequency $g$.
In this frame, $\uv{l}_{\rm d}$ and $\uv{l}$ are fixed, and
$\uv{s}$ evolves as:
\begin{equation}
    \p{\rd{\uv{s}}{t}}_{\rm rot} = \alpha \p{\uv{s} \cdot \uv{l}}
            \p{\uv{s} \times \uv{l}}
        + g\p{\uv{s} \times \uv{l}_{\rm d}}.\label{eq:dsdt_rot}
\end{equation}

We define the dimensionless time $\tau$ as
\begin{equation}
    \tau \equiv \alpha t,
\end{equation}
and the frequency ratio $\eta$
\begin{align}
    \eta \equiv{}& -\frac{g}{\alpha}\nonumber\\
        ={}& 2.08 \p{\frac{k_{\rm p}}{k_{\rm qp}}}
            \p{\frac{\rho_{\rm p}}{\mathrm{g/cm}^3}}
            \p{\frac{M_{\rm d}}{0.01 M_{\odot}}}
            \p{\frac{a_{\rm p}}{5\;\mathrm{AU}}}^{9/2}\nonumber\\
        &\times
            \p{\frac{r_{\rm d}}{30 \;\mathrm{AU}}}^{-3}
            \p{\frac{M_\star}{M_{\odot}}}^{-3/2}
            \p{\frac{P_p}{10\;\mathrm{hrs}}}
            \cos I ,\label{eq:eta}
\end{align}
% g = 1 / (51249 yr) 1 / (3 / 4 * (0.01) / (30 AU)^3 * sqrt(G * (solar mass) * (5AU)^3))
% alpha = 1/(106700 yr) 1 / (3/2 * (1 solar mass / (5AU)^3) / (4*pi / 3 * 1g/cm^3) * (2pi / (10 hours)))
% ((3 * 0.01 / 4) * (1 / 30AU)^3 * (10 hours * sqrt(G * (solar mass)) / (2pi)) / (1.5 * (1 solar mass / (5AU)^(9/2)) / (4 * pi / 3 * (1g/cm^3))))
where $\rho_p = 3M_p/(4\pi R_p^3)$ and $P_p = 2\pi/\Omega_p$ is the planet's
rotation period. . In Eq.~\eqref{eq:eta}, we have introduced the fiducial values
of variable parameters for the appciation considered in this paper.
Eq.~\eqref{eq:dsdt_rot} then becomes
\begin{equation}
    \p{\rd{\uv{s}}{\tau}}_{\rm rot} = \p{\uv{s} \cdot \uv{l}}
            \p{\uv{s} \times \uv{l}}
        - \eta\p{\uv{s} \times \uv{l}_{\rm d}}. \label{eq:dsdt_base}
\end{equation}

Throughout this paper, we consider $\alpha$ constant, but allow $g$ to vary in
time. In the dispersing disk scenario of \citet{millholland_disk}, $\abs{g}$
decreases in time due to the decreasing disk mass. We consider a simple
exponential decay model
\begin{equation}
    M_{\rm d}(t) = M_{\rm d}(0)e^{-t/t_{\rm d}},\label{eq:dmd_dt}
\end{equation}
with $t_{\rm d}$ constant. This implies
\begin{equation}
    \rd{\eta}{t} = -\eta /t_{\rm d},\;\; \mathrm{or}\;\;
    \rd{\eta}{\tau} = -\epsilon \eta\label{eq:deta_dt},
\end{equation}
where
% 1 / (1Myr * 3 / 4 * (0.01) / (30 AU)^3 * sqrt(G * (solar mass) * (5AU)^3))
\begin{align}
    \epsilon \equiv{}& \frac{1}{\alpha t_{\rm d}}\nonumber\\
    ={}& 0.051 \p{\frac{k_{\rm p}}{k_{\rm qp}}}
        \p{\frac{\rho_{\rm p}}{\mathrm{g/cm}^3}}
        \p{\frac{a_{\rm p}}{5\;\mathrm{AU}}}^3
        \p{\frac{P_{\rm p}}{10\;\mathrm{hrs}}}
        \p{\frac{t_{\rm d}}{\mathrm{Myr}}}^{-1}.\label{eq:eps_def}
\end{align}
Eqs.~\eqref{eq:dsdt_base} and~\eqref{eq:deta_dt} together constitute our system
of study.

In the next two subsections, we summarize the theoretical background relevant to
our analysis of the evolution of the system.

\subsection{Cassini States}\label{ss:cs}

Spin states satisfying $\p{d\uv{s}/d\uv{\tau}}_{\rm rot} = 0$ are referred to as
\emph{Cassini States} (CSs) \citep{colombo1966,peale1969}. They require that
$\uv{s}$, $\uv{l}$, and $\uv{l}_{\rm d}$ be coplanar. There are either two or
four CSs, depending on the value of $\eta$. They are specified by the obliquity
$\theta$ and the precessional phase of $\uv{s}$ around $\uv{l}$, denoted by
$\phi$. Following the standard convention and nomenclature (see
Figs.~\ref{fig:cs_vecs} and~\ref{fig:cs_locs}), CSs 1, 3, 4 have $\phi = 0$ and
$\theta < 0$, corresponding to $\uv{s}$ and $\uv{l}_{\rm d}$ being on opposite
sides of $\uv{l}$, while CS2 has $\phi = \pi$ and $\theta > 0$, corresponding to
$\uv{s}$ and $\uv{l}_{\rm d}$ being on the same side of $\uv{l}$. When $\eta <
\eta_{\rm
c}$, where
\begin{equation}
    \eta_{\rm c} \equiv \p{\sin^{2/3}\!I + \cos^{2/3}\!I}^{-3/2},\label{eq:etac}
\end{equation}
all four CSs exist, and when $\eta > \eta_{\rm c}$, only CSs 2, 3 exist. The CS
obliquities as a function of $\eta$ are shown in Fig.~\ref{fig:cs_locs}.

\begin{figure}
    \centering
    \includegraphics[width=0.3\textwidth]{plots_diskdisp/2_3vec.png}
    \caption{Definition of angles in the Cassini state configuration and the
    adopted sign convention for $\theta$. Traditionally, $\theta \in [-\pi,
    \pi]$.}\label{fig:cs_vecs}
\end{figure}

\begin{figure}
    \includegraphics[width=0.5\textwidth]{plots_diskdisp/2_cs_locs.png}
    \caption{Cassini state obliquities as a function of $\eta$ for $I =
    5^\circ$. The vertical dashed line indicates $\eta_{\rm c}$ ($= 0.766$ for
    $I = 5^\circ$), where CS1 and CS4 merge and annihilate, and the horizontal
    dashed lines indicate $\theta = I$ and $I - 180^\circ$, the asymptotic
    values for CSs 2 and 3 for $\eta \gg \eta_{\rm c}$.}\label{fig:cs_locs}
\end{figure}

Of the four CSs, 1, 2, 3 are stable while 4 is unstable.
Appendix~\ref{s:local_dynamics} gives the libration frequencies and growth rates,
respectively, near these CSs.

\subsection{Separatrix}

The Hamiltonian (in the rotating frame) of the system is
\begin{align}
    \mathcal{H}\p{\phi, \cos \theta} &= -\frac{1}{2}\p{\uv{s} \cdot \uv{l}}^2
            + \eta \p{\uv{s} \cdot \uv{l}_{\rm d}}\nonumber\\
        &= -\frac{1}{2}\cos^2\theta
            + \eta \p{\cos \theta \cos I - \sin I \sin \theta \cos \phi}
                \label{eq:H}.
\end{align}
Here, $\phi$ and $\cos \theta$ are canonically conjugate variables. Trajectories
in the phase space $\p{\phi, \cos \theta}$ satisfy $H = $ constant (see
Fig.~\ref{fig:eq_1contours}).

When $\eta < \eta_{\rm c}$, CS4 exists and is a saddle point. The two
trajectories originating and ending at CS4 are the only two infinite-period
orbits in the phase space. Together, these two critical trajectories are
referred to as the \emph{separatrix} and divide phase space into three zones. In
Fig.~\ref{fig:eq_1contours}, we show the separatrix, the three zones, and their
relations to the CSs. Trajectories in zone II librate about CS2 while those in
zones I and III circulate.
\begin{figure*}
    \centering
    \includegraphics[width=0.8\textwidth]{plots_diskdisp/1contours_flip.png}
    \caption{Level curves of $\mathcal{H}\p{\phi, \cos \theta}$
    [Eq.~\eqref{eq:H}] for $I = 5^\circ$, where warmer colors denote more
    positive values. The black solid line is the separatrix, which only exists
    for $\eta < \eta_{\rm c} = 0.766$. The three zones (I, II, III), divided by
    the separatrix, are labeled. The Cassini states are denoted by filled
    circles and have the same colors as in Fig.~\ref{fig:cs_locs}. The interior
    of the separatrix, shaded in grey, is formally only defined for $\eta <
    \eta_{\rm c}$, but we may identify the points in phase space that flow into
    zone II when evolved forward in time (decreasing $\eta$ adiabatically); this
    is the shaded region in panel (a), bounded by the black dotted
    line.}\label{fig:eq_1contours}
\end{figure*}

Since $\p{\phi, \cos \theta}$ are canonically conjugate, the integral $\oint
\cos \theta\;\mathrm{d}\phi$ along a trajectory is an adiabatic invariant (see
Section~\ref{s:ad}). The unsigned areas $\p{\abs{\int \cos \theta
\;\mathrm{d}\phi}}$ of the three zones (as defined in
Fig.~\ref{fig:eq_1contours}) can be computed analytically
\citep{henrard1987,ward2004I}. Define
\begin{subequations}
    \begin{align}
        z_0 &= \eta\cos I, &
        \chi &= \sqrt{-\frac{\tan^3\theta_4}{\tan I} - 1},\\
        \rho &= \chi \frac{\sin^2 \theta_4\cos \theta_4}{
            \chi^2 \cos^2\theta_4 + 1},&
        T &= 2\chi \frac{\cos \theta_4}{
            \chi^2 \cos^2\theta_4 - 1}.
    \end{align}
\end{subequations}
The areas for $\eta < \eta_{\rm c}$ are given by
\begin{subequations}\label{se:area_ward}
    \begin{align}
        A_{\rm I} &= 2\pi\p{1 - z_0} - \frac{A_2}{2}, \label{eq:A1}\\
        A_{\rm II} &= 8\rho + 4\arctan T - 8z_0 \arctan \frac{1}{\chi},
            \label{eq:A2}\\
        A_{\rm III} &= 2\pi\p{1 + z_0} - \frac{A_2}{2}\label{eq:A3}.
    \end{align}
\end{subequations}
These are plotted as a function of $\eta$ in Fig.~\ref{fig:eq_areas}. While the
zones are not formally defined for $\eta > \eta_{\rm c}$ since the separatrix
disappears, a natural extension exists: evolve an initial phase space point $p$
under adiabatic decrease of $\eta$ until the separatrix appears at $\eta =
\eta_{\rm c}$, then identify $p$ with the zone it is in at $\eta_{\rm c}$. Since
phase space area is conserved under adiabatic evolution, this extension implies
$A_{\rm i}\p{\eta > \eta_{\rm c}} = A_{\rm i}(\eta_{\rm c})$. The boundary
between these extended zones is denoted by the dashed black line in panel (a) of
Fig.~\ref{fig:eq_1contours}, where no separatrix exists.
\begin{figure}
    \centering
    \includegraphics[width=0.5\textwidth]{plots_diskdisp/1_areas.png}
    \caption{The solid lines show the fractional areas of each of the zones
    $A_{\rm i}(\eta) / 4\pi$ as given by Eqs.~\eqref{se:area_ward}. The colored
    dashed lines correspond to small $\eta$ approximations used in
    Appendix~\ref{s:ad_approx}. The colored dashed lines for $\eta > \eta_{\rm
    c}$ are the effective values of $A_{\rm II}, A_{\rm III}$ for $\eta >
    \eta_{\rm c}$, denoting the points that would flow into either area under
    adiabatic decrease of $\eta$ from $\eta > \eta_{\rm c}$ (see the text). The vertical black
    dashed lines correspond to $\eta = \eta_{\rm c}$ [Eq.\eqref{eq:etac}] and
    the values of $\eta$ for which $A_{\rm II}$ is maximized ($\eta_{\rm \max,
    II}$) and for which $A_{\rm III}$ is minimized ($\eta_{\rm \min, III}$,
    Eq.~\eqref{eq:eta_minIII}).}\label{fig:eq_areas}
\end{figure}

\section{Adiabatic Evolution}\label{s:ad}

In this section, we study the evolution of the planetary obliquity $\theta$ when
the parameter $\eta$ [or the disk mass $M_{\rm d}$; see Eqs.~\eqref{eq:eta}
and~\eqref{eq:deta_dt}] decreases sufficiently slowly that the evolution is
adiabatic. Strictly speaking, this requires the disk evolution timescale $t_{\rm
d}$ [Eq.~\eqref{eq:dmd_dt}] be much larger than all timesccales of the dynamical
system governed by the Hamiltonian [Eq.~\eqref{eq:H}]. This is of course not
possible in all cases, as the motion along the separatrix has an infinite
period.

In practice, as $\eta$ evolves, the system only crosses the separatrix once or
twice, while it spends many orbits inside one of the three zones and far from
the separatrix. Thus, a \emph{weak adiabaticity criterion} is that the variation
timescale of $\eta$ be slower than all circulation/libration timescales near the
equilibria/fixed points. If this criterion is satisfied, then the system will
evolve adiabatically for most of its evolution save one or two separatrix
crossings.

As shown in Appendix~\ref{ss:eigens}, libration about CS2 is slower than that
about CS1 or CS3. As such, it has the smallest characteristic frequency in the
system. The weak adiabaticity criterion can be recast as requiring that $\eta$
vary more slowly than this libration frequency \citep[see][]{ward2004II}, i.e.\
$t_{\rm d}^{-1} \lesssim \omega_{\rm lib} / 2\pi$. In terms of the dimensionless
parameter $\epsilon$
[Eq.~\eqref{eq:eps_def}], this adiabaticity condition is
\begin{equation}
    \epsilon \lesssim \frac{1}{2\pi}\sqrt{\eta\sin I \sin \theta_2
            \p{1 + \eta \sin I \csc^3 \theta_2}},
            \label{eq:ad_constr}
\end{equation}
where $\theta_2$ is the obliquity at CS2. This formula differs from that given
in \citet{millholland_disk}, where the $\csc^3\!\theta_2$ term is neglected and
the square root is missing\footnote{The missing $\csc^3 \theta_2$ term can be
traced to a $\theta \gg I$ approximation made in Eq.~(3) of \citet{ward2004II}.
Since $\theta_2 \approx I$ for $\eta \gg 1$ (Fig.~\ref{fig:cs_locs}), this
approximation is not always valid.}.

For $I = 5^\circ$, $\eta \sim \eta_{\rm c}$ Eq.~\eqref{eq:ad_constr} gives
$\epsilon \lesssim 0.085$. Since our criterion is only a weak condition for
diabaticity, we use $\epsilon = 3 \times 10^{-4}$ in our ``adiabatic''
calculations below. We explore the consequences of nonadiabatic evolution in
Section~\ref{s:nonad}.

\subsection{Adiabatic Evolution Outcomes}\label{ss:ad_ensemble}

We consider the evolution of a system with arbitrary initial spin-disk
misalignment angle $\theta_{\rm sd, i}$ and initial $\eta_{\rm i} \gg 1$. We are
interested in the final spin obliquities $\theta_{\rm f}$ after $\eta$ gradually
decreases to $\eta_{\rm f} \ll 1$ (i.e.\ after the disk has dissipated to a
negligible mass). Note that when $\eta_{\rm i} \gg 1$, $\uv{l}$ precesses
around $\uv{l}_{\rm d}$ much faster than the spin-orbit precession
($\abs{\omega_{\rm ld}} \gg \abs{\omega_{\rm sl}}$), and the spin obliquity
$\theta$ varies rapidly. It is thus more appropriate to use $\theta_{\rm sd, i}$
rather than $\theta$ to specify the initial spin orientation. We explore the
entire range $\theta_{\rm sd, i} \in [0, \pi]$ and choose $\epsilon = 3 \times
10^{-4}$ so that the system evolves almost adiabatically (see above).

To obtain the distribution of the final obliquities $\theta_{\rm f}$, we evenly
sample $101$ values of $\theta_{\rm sd, i}$, and for each $\theta_{\rm sd, i}$
value, we pick $101$ evenly spaced orientations of $\uv{s}$ approximately from
the ring of initial conditions having angular distance $\theta_{\rm sd, i}$ to
$\uv{l}_{\rm d}$%
%
\footnote{The actual procedure we adopt to choose the initial conditions is the
natural extension of this description to finite $\eta_{\rm i}$. Note that the
center of libration of $\uv{s}$ is CS2, which, since $\eta_{\rm i}$ is finite,
is different from $\uv{l}_{\rm d}$. Furthermore, the libration is not exactly
circular. As a result, the libration trajectories for initial conditions on the
circular ring of points having angular distance $\theta_{\rm sd, i}$ from
$\uv{l}_{\rm d}$ are not the same and will each enclose slightly different
initial phase space areas $A_{\rm i}$. Since our analytical theory assumes exact
conservation of the initially enclosed phase space area $A_{\rm i}$ for each
$\theta_{\rm sd, i}$ (see Section~\ref{ss:zone_transitions}), this discrepancy
introduces an extra deviation from the analytical prediction. To guarantee all
points for a particular $\theta_{\rm sd, i}$ have the same $A_{\rm i}$, we
instead choose initial conditions on the libration cycle going through
$\p{\theta_2 + \theta_{\rm sd, i}, \phi_2}$ [where $\p{\theta_2, \phi_2}$ are
the coordinates of CS2]. This ensures that all initial conditions for a given
$\theta_{\rm sd, i}$ enclose the same initial $A_{\rm i}$. As $\eta_{\rm i} \to
\infty$, this procedure generates initial conditions on the ring having angular
distance $\theta_{\rm sd, i}$ to $\uv{l}_{\rm d}$, recovering the procedure
given in the text.}.
%
To be concrete, we choose $\eta_{\rm i} = 10\eta_{\rm c}$ where $\eta_{\rm c}$
is given by Eq.~\eqref{eq:etac} and evolve Eqs.~\eqref{eq:dsdt_base}
and~\eqref{eq:deta_dt} until $\eta$ reaches its final value $10^{-5}$. At such a
small $\eta$, $\uv{s}$ is strongly coupled to $\uv{l}$ and the final obliquity
$\theta_{\rm f}$ is frozen. The mapping between $\theta_{\rm sd, i}$ and
$\theta_{\rm f}$ is our primary result, and is shown for $I = 5^\circ, 10^\circ,
20^\circ$ in Figs.~\ref{fig:ad_ensemble},~\ref{fig:3_ensemble_10_35},
and~\ref{fig:3_ensemble_20_35} respectively. The blue dots represent the results
of the numerical calculation. The colored tracks are calculated
semi-analytically using the method discussed in the following subsection.

\begin{figure}
    \centering
    \includegraphics[width=0.5\textwidth]{plots_diskdisp/3_ensemble_05_35.png}
    \caption{Top: The final spin obliquity $\theta_{\rm f}$ as a function of the
    initial spin-disk misalignment angle $\theta_{\rm sd, i}$ for systems
    evolving from initial $\eta_{\rm i} \gg 1$ to $\eta_{\rm f} \ll 1$. The blue
    dots are results of numerical calculations (Section~\ref{ss:ad_ensemble}),
    and the colored tracks are semi-analytical results
    (Section~\ref{ss:zone_transitions}). Bottom: The probabilities of different
    outcomes. Where a particular $\theta_{\rm sd, i}$ corresponds to multiple
    tracks, the system evolves probabilistically. The track that a particular
    system evolves along in a numerical simulation can be measured by examining
    its final obliquity. The dots represent the inferred probabilities from
    measured final obliquities in our simulations, while the colored tracks
    denote the semi-analytic probability of the system evolving along each
    track. There are five regimes of $\theta_{\rm sd, i}$ values for which
    different tracks are accessible. In both plots, the vertical dashed black
    lines denote semi-analytical calculations of the boundaries of these regimes
    (see Section~\ref{ss:zone_transitions}), while the black dotted lines
    represent analytical approximations valid in the small-$\theta_{\rm sd, i}$
    limit (see Appendix~\ref{s:ad_approx}).}\label{fig:ad_ensemble}
\end{figure}
\begin{figure}
    \centering
    \includegraphics[width=0.5\textwidth]{plots_diskdisp/3_ensemble_10_35.png}
    \caption{Same as the top panel of Fig.~\ref{fig:ad_ensemble} but for $I =
    10^\circ$ and with fewer annotations.}\label{fig:3_ensemble_10_35}
\end{figure}
\begin{figure}
    \centering
    \includegraphics[width=0.5\textwidth]{plots_diskdisp/3_ensemble_20_35.png}
    \caption{Same as the top panel of Fig.~\ref{fig:3_ensemble_10_35} but for $I
    = 20^\circ$ and with fewer annotations.}\label{fig:3_ensemble_20_35}
\end{figure}

\subsection{Analytical Theory for Adiabatic Evolution}\label{ss:zone_transitions}

The evolutionary tracks that govern the $\theta_{\rm f}$-$\theta_{\rm sd, i}$
mapping correspond to various sequences of separatrix crossings. They can be
understood using the principle of adiabatic invariance, combined with (i) how
the enclosed phase space area by the trajectory evolves across each separatrix
crossing, and (ii) the associated probabilities with each separatrix crossing.

\subsubsection{Governing Principle: Evolution of Enclosed Phase Space
Area}\label{sss:a_evo}

First, we consider how the enclosed phase space area by a trajectory evolves
over time. In the absence of separatrix encounters, the enclosed phase space
area $\oint \cos\theta \;\mathrm{d}\phi$ is an adiabatic invariant. We adopt
convention where
\begin{equation}
    A \equiv \oint \p{1 - \cos \theta}\;\mathrm{d}\phi.\label{eq:a_oint}
\end{equation}
This definition of $A$ has two advantages: (i) it is continuous across
transitions from circulating to librating that cross the North pole ($\cos
\theta = 1$), and (ii) the areas of the three zones are equal in absolute value
to the expressions given in Eqs.~\eqref{se:area_ward}. The path over which the
integral is taken is either a libration or circulation cycle. When $\eta_{\rm i}
\gg 1$, trajectories librate about $\uv{l}_{\rm d}$ with constant $\theta_{\rm
sd}$, meaning they enclose initial phase space area
\begin{equation}
    A_{\rm i} = 2\pi\p{1 - \cos \theta_{\rm sd, i}}.\label{eq:ai_qsd}
\end{equation}
Complications arise when considering finite $\eta_{\rm i}$, as trajectories near
CS2 or CS3 librate about these equilibria, rather than $\uv{l}_{\rm d}$, and
Eq.~\eqref{eq:ai_qsd} is no longer exact. In practice, Eq.~\eqref{eq:ai_qsd}
holds very well when defining $\theta_{\rm sd, i}$ as the angular distance to
CS2; an exception is discussed in Section~\ref{sss:evol_traj}.

Beginning at the last separatrix crossing, the final enclosed phase space area
$A_{\rm f}$ will be conserved for all time. As $\eta \to 0$, trajectories
circulate about $\uv{l}$ at constant obliquity $\theta_{\rm f}$, related to
$A_{\rm f}$ by
\begin{equation}
    2\pi\p{1 - \cos \theta_{\rm f}} = A_{\rm f}. \label{eq:qfaf}
\end{equation}

The enclosed phase space area is not conserved when the trajectory encounters
the separatrix. Howover, the change is easily understood \citep{henrard1982}. In
essence, when the trajectory crosses the separatrix, it continues to evolve
adjacent to the separatrix. So if a separatrix crossing results in a zone I
trajectory (see Fig.~\ref{fig:eq_1contours}), the new area can be approximated
by integrating Eq.~\eqref{eq:a_oint} along the upper leg of the separatrix.
Pictorally, this can be seen in the bottom panels of Fig.~\ref{fig:ad_21}.

\subsubsection{Governing Principle: Probabilistic Separatrix Crossing}

When a trajectory experiences separatrix crossing, it transitions into nearby
zones probabilistically. This process is studied by \citet{henrard1982} and
\citet{henrard1987}. Their results may be summarized as follows: if zone $i$
is shrinking while adjacent zones $j, k$ are expanding such that the sum of
their areas is constant, the probabilities of transition from zone $i$ to zones
$j$ and $k$ are given by
\begin{subequations}\label{eq:henrard_hop}
    \begin{align}
        \Pr\p{i \to j} = -\frac{\pdil{A_{\rm j}}{\eta}}{\pdil{A_{\rm i}}{\eta}},
                \\
        \Pr\p{i \to k} = -\frac{\pdil{A_{\rm k}}{\eta}}{\pdil{A_{\rm i}}{\eta}}.
    \end{align}
\end{subequations}
Note that $\Pr \p{i \to j} + \Pr\p{i \to k} = 1$.
Eqs.~\eqref{eq:henrard_hop} can be used in conjunction with
Eqs.~\eqref{se:area_ward} to understand for what initial conditions each track
can be observed and with what probabilities.

As an example, consider a system in zone II in panel (d) of
Fig.~\ref{fig:eq_1contours}. As $\eta$ decreases, zone II will shrink while
zones I and III will expand until the trajectory crosses the separatrix. Suppose
the trajectory exits zone II at some $\eta_\star$, then the probability of the
II $\to$ I transition is $\Pr\p{\rm II \to I} = -\dot{A}_{\rm I} / \dot{A}_{\rm
II}$, while the II $\to$ III transition occurs with probability $\Pr\p{\rm II
\to III} = -\dot{A}_{\rm III} / \dot{A}_{\rm II}$.

\subsubsection{Evolutionary Trajectories}\label{sss:evol_traj}

Returning to the evolution of $\uv{s}$, we can classify trajectories by the
sequence of separatrix encounters. Initially, in the $\eta > \eta_{\rm c}$
regime, only zones II and III exist; as $\eta \to 0$, only zones I and III exist
(see Fig.~\ref{fig:eq_1contours}). There are five distinct evolutionary tracks:
\begin{enumerate}
    \item II $\to$ I (see Fig.~\ref{fig:ad_21} for an example). The spin axis
        $\uv{s}$ initially circulates in zone II (snapshot a), and then starts
        librating about CS2 as $\eta$ decreases (snapshot b), enclosing some
        initial phase space area $A_{\rm i}$. This libration continues until the
        separatrix expands (due to decreasing $\eta$) to ``touch'' the
        trajectory (snapshot c), at which $A_{\rm II}(\eta_\star) = A_{\rm i}$.
        As $\hat{s}$ moves to a circulating trajectory in zone I immediately
        bordering the separatrix, it will encompass $-A_{\rm I}(\eta_\star)$
        phase space area. The final obliquity $\theta_{\rm f}$ is then given by
        Eq.~\eqref{eq:qfaf}, with $A_{\rm f} = -A_{\rm I}\p{\eta_\star}$. An
        analytical approximation to $\theta_{\rm f}$ is derived in
        Appendix~\ref{s:ad_approx} and is
        \begin{equation}
            \p{\cos \theta_{\rm f}}_{\rm II \to I} \simeq
                \p{\frac{\pi \theta_{\rm sd, i}^2}{16}}^2 \cot I
                    + \frac{\theta_{\rm sd, i}^2}{4}.\label{eq:qf_21_approx}
        \end{equation}
        The transition probability is
        \begin{equation}
            \Pr\p{\mathrm{II} \to \mathrm{I}} = -\p{
                \frac{\pdil{A_{\rm I}}{\eta}}{\pdil{A_{\rm II}}{\eta}}}
                    _{\eta = \eta_{\star}}.
        \end{equation}
        This track can only occur when the initial condition begins in zone II,
        requiring $A_{\rm i} < A_{\rm II}(\eta_{\rm c})$, where $A_{\rm
        II}\p{\eta_c}$ is given by Eq.~\eqref{eq:A2} evaluated at $\eta =
        \eta_c$. Since $\pdil{A_{\rm I}}{\eta} < 0$ everywhere, while
        $\pdil{A_{\rm II}}{\eta} > 0$ at all possible $\eta_\star$ for an
        initial condition starting in zone II, this track always has nonzero
        probability.

    \item II $\to$ III (see Fig.~\ref{fig:ad_23}). This track is similar to the
        II $\to$ I track; the only difference is that, upon separatrix
        encounter, the trajectory follows the circulating trajectory in zone III
        bordering the separatrix, upon which it will encompass area $A_{\rm
        I}(\eta_\star) + A_{\rm II}(\eta_\star) = A_{\rm f}$. The final
        obliquity is still given by Eq.~\eqref{eq:qfaf}, and the analytical
        approximation derived in Appendix~\ref{s:ad_approx} is
        \begin{equation}
            \p{\cos \theta_{\rm f}}_{\rm II \to III} \simeq
                \p{\frac{\pi \theta_{\rm sd, i}^2}{16}}^2 \cot I
                    - \frac{\theta_{\rm sd, i}^2}{4}.\label{eq:qf_23_approx}
        \end{equation}
        The transition probability is
        \begin{equation}
            \Pr\p{\mathrm{II} \to \mathrm{III}} = -\p{
                \frac{\pdil{A_{\rm III}}{\eta}}{\pdil{A_{\rm II}}{\eta}}}
                    _{\eta = \eta_{\star}}.
        \end{equation}
        Again, this track can only occur when $A_{\rm i} < A_{\rm II}(\eta_{\rm
        c})$, but a further constraint arises when we consider the transition
        probability. Upon examination of Fig.~\ref{fig:eq_areas}, it is clear
        that $\pdil{A_{\rm III}}{\partial \eta} > 0$ for a large range of
        $\eta$, which would give a negative transition probability---implying
        a forbidden transition. Define
        \begin{equation}
            \eta_{\min, III} \equiv \argmin A_{\rm III}(\eta)
                \label{eq:eta_minIII},
        \end{equation}
        which is labeled in Fig.~\ref{fig:eq_areas}. Thus, the II $\to$ III
        track is permitted only if $\eta_\star <, \eta_{\rm \min, III}$.

    \item III $\to$ I (see Fig.~\ref{fig:ad_31}). The trajectory encounters the
        separatrix when $A_{\rm I}(\eta_\star) + A_{\rm II}(\eta_\star) =
        A_{\rm i}$, upon which it transitions to a zone I trajectory enclosing
        $A_{\rm f} = -A_{\rm I}$. The final obliquity is again given by
        Eq.~\eqref{eq:qfaf}.

        This track can only occur if $A_{\rm i} > A_{\rm II}(\eta_{\rm c})$,
        but is also constrained by requiring $A_{\rm i}$ be sufficiently small
        so that it will encounter the separatrix (if $A_{\rm i}$ is too large,
        it will never encounter the separatrix, and we simply have a III $\to$
        III transition). This condition is $A_{\rm i} < \max \p{A_{\rm I} +
        A_{\rm II}} = 4\pi - \min \p{A_{\rm III}}$.
        Since $\pdil{A_{\rm I}}{\eta} < 0$ and $\pdil{A_{\rm III}}{\eta} > 0$
        for all accessible $\eta_{\star}$, this track is always permitted.

    \item III $\to$ II $\to$ I (see Fig.~\ref{fig:ad_321}). That $A_{\rm
        II}(\eta)$ is not a monotonic function of $\eta$ (see
        Fig.~\ref{fig:eq_areas}) is key to the existence of this track. Consider
        a trajectory originating in zone III that first encounters the
        separatrix at $\eta_1$, when $A_{\rm I}(\eta_1) + A_{\rm II}(\eta_1) =
        A_{\rm i}$, such that it transitions into zone II enclosing intermediate
        phase space area $A_{\rm m} = A_{\rm II}(\eta_1)$. Such a transition has
        probability
        \begin{equation}
            \Pr\p{\rm III \to II} = -\p{
                \frac{\pdil{A_{\rm II}}{\eta}}{\pdil{A_{\rm III}}{\eta}}}
                    _{\eta = \eta_1},
        \end{equation}
        which is nonnegative (i.e.\ the transition is permitted) if $\eta_1 \in
        [\eta_{\rm \max, II}, \eta_{\rm c}]$. Equivalently, this requires
        $A_{\rm i} \in \s{A_{\rm II}\p{\eta_{\rm c}} , A_{\rm II, \max}}$. Then,
        as $\eta$ continues to decrease, a second $\eta_2$ value exists for
        which $A_{\rm m} = A_{\rm II}(\eta_2)$, upon which the trajectory is
        ejected to zone I and $A_{\rm f} = -A_{\rm I}(\eta_2)$. Note that
        $\eta_2 < \eta_{\rm \max, II}$ necessarily, as zone II must be shrinking
        in order for the trajectory to be ejected. The final obliquity is given
        by Eq.~\eqref{eq:qfaf}. Graphical inspection of Fig.~\ref{fig:eq_areas}
        shows that $\pdil{A_{\rm II}}{\eta}$ and $\pdil{A_{\rm III}}{\eta}$ have
        the same signs for $\eta < \eta_{\rm \max, II}$, and therefore the
        complementary II $\to$ I transition is guaranteed. Overall, the III
        $\to$ II $\to$ I track is permitted so long as the first transition is
        permitted, or $A_{\rm i} \in \s{A_{\rm II}\p{\eta_{\rm c}} , A_{\rm II,
        \max}}$.

    \item III $\to$ III\@. This track is the trivial case where no separatrix
        encounter occurs, and $A$ is constant throughout the evolution ($A_{\rm
        f} = A_{\rm i}$) except for a jump by $4\pi$ when crossing the South
        pole ($\cos \theta = -1$) due to the coordinate singularity. This
        requires $A_{\rm i} > \max \p{A_{\rm I} + A_{\rm II}}$. In the limit of
        $\eta_{\rm i} \to \infty$ and $ \eta_{\rm f} \to 0$ we have $\theta_{\rm
        f} = \theta_{\rm sd, i}$. For finite $\eta_{\rm i}$, the initial
        enclosed phase space area for III $\to$ III trajectories is not given
        exactly by using $\theta = \theta_{\rm sd, i}$ in Eq.~\eqref{eq:ai_qsd}.
        This is because the initial orbits for such trajectories are better
        described as librating about CS3 with angle of libration $\Delta \theta
        - \theta_{\rm sd, i}$ rather than about CS2 with angle of libration
        $\theta_{\rm sd, i}$. Here, $\Delta \theta$ is the angular distance
        between CS2 and CS3 and is not equal to $180^\circ$ except when
        $\eta_{\rm i} \to \infty$. This finite-$\eta_{\rm i}$ effect is
        responsible for the small cusp at the very right ($\theta_{\rm sd, i}
        \to 180^\circ$) of
        Figs.~\ref{fig:ad_ensemble},~\ref{fig:3_ensemble_10_35},
        and~\ref{fig:3_ensemble_20_35}.
\end{enumerate}

In summary, starting from an initial condition with phase space area $A_{\rm i}$
at $\eta = \eta_{\rm i} \gg 1$, the five evolutionary tracks are:
\begin{enumerate}
    \item $A_{\rm i} \in \s{0, A_{\rm II}\p{\eta_{\rm \min, III}}}$: Both the II
        $\to$ III and the II $\to$ I tracks are possible.

    \item $A_{\rm i} \in \s{A_{\rm II}\p{\eta_{\rm \min, III}}, A_{\rm
        II}(\eta_{\rm c})}$: Only the II $\to$ I track.

    \item $A_{\rm i} \in \s{A_{\rm II}(\eta_{\rm c}), A_{\rm II, \max}}$:
        Both the III $\to$ I and III $\to$ II $\to$ I are possible.

    \item $A_{\rm i} \in \s{A_{\rm II, \max}, \max \p{A_{\rm I} + A_{\rm II}}}$:
        Only the III $\to$ I track.

    \item $A_{\rm i} > \max \p{A_{\rm I} + A_{\rm II}}$: Only the III $\to$ III
        track.
\end{enumerate}
In all cases, the corresponding ranges for $\theta_{\rm sd,i}$ can be computed via
Eq.~\eqref{eq:ai_qsd}. The boundaries between these ranges are overplotted in
Fig.~\ref{fig:ad_ensemble}, where they can be seen to agree well with the
numerical results.
\begin{figure}
    \centering
    \includegraphics[width=0.5\textwidth]{plots_diskdisp/3testo21.png}

    \includegraphics[width=0.5\textwidth]{plots_diskdisp/3testo21_subplots.png}
    \caption{An example of the II $\to$ I evolutionary track for $I = 5^\circ$
    and $\theta_{\rm sd, i} = 17.2^\circ$. Upper panel: The thin green line shows
    $\cos \theta$ as a function of $\eta$, obtained by numerical integration
    (with $\epsilon = 3 \times 10^{-4}$). Overlaid are the location of Cassini
    State 2 (dashed red) and the upper and lower bounds on the separatrix
    (dotted black). The trajectory tracks CS2 to a final obliquity of
    $88.57^\circ$. The black vertical dashed lines denote instants in the
    simulation  portrayed in bottom panels. Middle panel: The enclosed
    separatrix area obtained by integrating the simulated trajectory (green
    dots) and adiabatic theory (red line). Lower plot: Snapshots in $\p{\cos
    \theta, \phi}$ phase space of one circulation/libration cycle of the
    trajectory, shown in dark green with an arrow indicating direction. The
    snapshots correspond to the start of the simulation (a), the appearance of
    the separatrix (b), two panels depicting the separatrix crossing process
    (c-d), and a final snapshot at $\eta = 10^{-3.5}$ (e). The separatricies at
    the beginning and end of the portrayed cycle in each snapshot are shown in
    solid/dashed black lines respectively. Also labeled is CS2 at the start of
    each cycle (filled red circle). Finally, the enclosed phase space area is
    shaded in grey ($A > 0$) and red ($A < 0$).}\label{fig:ad_21}
\end{figure}
\begin{figure}
    \centering
    \includegraphics[width=0.5\textwidth]{plots_diskdisp/3testo23.png}

    \includegraphics[width=0.5\textwidth]{plots_diskdisp/3testo23_subplots.png}
    \caption{Same as Fig.~\ref{fig:ad_21} but for the II $\to$ III track.
    $\theta_{\rm sd, i} = 17.2^\circ$ and $\epsilon = 3.01 \times
    10^{-4}$.}\label{fig:ad_23}
\end{figure}
\begin{figure}
    \centering
    \includegraphics[width=0.5\textwidth]{plots_diskdisp/3testo31.png}

    \includegraphics[width=0.5\textwidth]{plots_diskdisp/3testo31_subplots.png}
    \caption{Same as Fig.~\ref{fig:ad_21} but for the III $\to$ I track.
    $\theta_{\rm sd, i} = 89.1^\circ$, and $\epsilon = 3 \times
    10^{-4}$.}\label{fig:ad_31}
\end{figure}
\begin{figure}
    \centering
    \includegraphics[width=0.5\textwidth]{plots_diskdisp/3testo321.png}

    \includegraphics[width=0.5\textwidth]{plots_diskdisp/3testo321_subplots.png}
    \caption{Same as Fig.~\ref{fig:ad_21} but for the III $\to$ II $\to$ I
    track. $\theta_{\rm sd, i} = 60^\circ$, and $\epsilon = 3.14 \times
    10^{-4}$. Two separatrix crossings are shown, in panels (c-d) and
    (e-f).}\label{fig:ad_321}
\end{figure}


\section{Nonadiabatic Effects}\label{s:nonad}

In Section~\ref{s:ad}, we have examined the spin axis evolution in the limit
where $\epsilon \ll 1$ [see Eq.~\eqref{eq:ad_constr}] and the evolution is
mostly adiabatic (except at separatrix crossings). We now consider nonadiabatic
effects.

\subsection{Transition to Non-adiabaticity: Results for $\epsilon\lesssim
1$}\label{ss:transition}

To illustrate the transition to nonadiabaticity, we carried out a suite of
numerical calculations for several values of $\epsilon$. The results for two of these
values are shown in Figs.~\ref{fig:3_ensemble_05_25}
and~\ref{fig:3_ensemble_05_15}.

As $\epsilon$ increases (see Fig.~\ref{fig:3_ensemble_05_25}), nonadiabaticity
manifests as a larger scatter of final obliquities near the tracks predicted
from adiabatic evolution. This scatter first sets in for trajectories starting
in zone III, as these trajectories encounter the separatrix at larger $\eta$
compared to those originating in zone II\@. This means the obliquity of CS2
$\theta_2$ is smaller for these trajectories, and the adiabaticity criterion is
stricter [see Eq.~\eqref{eq:ad_constr}]. Physically, approaching the
adiabaticity criterion corresponds to the separatrix crossing process becoming
sensitive to the \emph{phase} of the libration/circulation cycle at the
crossing: if the trajectory crosses the separatrix when the obliquity is at its
maximum, the final obliquity will also be relatively larger.

As $\epsilon$ increases further (see Fig.~\ref{fig:3_ensemble_05_15}) but still
marginally satisfies the weak adiabaticity criterion [Eq.~\eqref{eq:ad_constr}],
the scatter in $\theta_{\rm f}$ continues to widen. The horizontal banded
structure of the final obliquities is a consequence of even stronger phase
sensitivity during separatrix crossing: trajectories cross the separatrix at
similar phases evolve to similar final obliquities that only depend weakly on
on $\theta_{\rm sd, i}$.

\begin{figure}
    \centering
    \includegraphics[width=0.5\textwidth]{plots_diskdisp/3_ensemble_05_25.png}
    \caption{Same as Fig.~\ref{fig:ad_ensemble} but for $\epsilon = 10^{-2.5}$
    and restricting $\theta_{\rm sd, i} < 90^\circ$ (blue dots). The colored
    solid lines are analytical adiabatic results (same as shown in
    Fig.~\ref{fig:ad_ensemble}). A larger spread from the adiabatic tracks is
    observed in the numerical results due to the non-adiabaticity effect.
    }\label{fig:3_ensemble_05_25}
\end{figure}

\begin{figure}
    \centering
    \includegraphics[width=0.5\textwidth]{plots_diskdisp/3_ensemble_05_15.png}
    \caption{Same as Fig.~\ref{fig:3_ensemble_05_25} but for $\epsilon =
    10^{-1.5}$ (i.e.\ larger non-adiabaticity effect). Some small resemblance to
    the adiabatic tracks remains, and the deviations appear to have a banded
    structure. }\label{fig:3_ensemble_05_15}
\end{figure}

A sample trajectory following in the style of Fig.~\ref{fig:ad_21} but for
$\epsilon = 0.3$ (violating even weak adiabaticity) is provided in
Fig.~\ref{fig:nonad_traj}. It is clear that the trajectory does not track the
level curves of the Hamiltonian during each individual snapshot. This results
from CS2 migrating more quickly than the trajectory can librate about CS2,
violating the weak adiabaticity criterion.
\begin{figure}
    \centering
    \includegraphics[width=0.5\textwidth]{plots_diskdisp/3testo_nonad.png}

    \includegraphics[width=0.5\textwidth]{plots_diskdisp/3testo_nonad_subplots.png}
    \caption{Same as Fig.~\ref{fig:ad_21} but for a nonadiabatic case, with $\epsilon =
    0.3$. In the top panel, it is evident that the libration cycle about CS2 is
    unable to keep up with the swift migration of CS2 as $\eta$ changes,
    decreasing the obliquity excitation compared to the adiabatic simulation. In
    the middle panel, the trajectory only undergoes six libration/circulation
    cycles before $\eta < 10^{-5}$, and the enclosed phase space area is
    clearly not conserved. In the bottom panel, we can see that individual
    trajectories do not lie along level curves of the Hamiltonian, as the
    Hamiltonian phase space changes quickly compared to the period of
    circulation cycles.}\label{fig:nonad_traj}
\end{figure}

\subsection{Non-adiabatic Evolution: Result for $\epsilon\gtrsim 1$}

In general, numerical calculations are needed to determine the non-adiabatic
obliquity evolution ($\epsilon\gtrsim 1$). However, some analytical results can
be still be obtained when the obliquity remains small throughout the evolution.

We start from Eq.~\eqref{eq:dsdt_base}, which governs the evolution of the spin
axis. We choose coordinate axes such that $\uv{l} = \uv{z}$ and $ \uv{l}_{\rm d}
= \uv{z} \cos I + \uv{x}\sin I$, giving
\begin{equation}
    \rd{\uv{s}}{\tau} = \s{
        \p{\eta \cos I - \cos\theta}\uv{z} + \eta \sin I \,\uv{x}} \times \uv{s}.
\end{equation}
%Now, let's assume that $s_{\rm z} \approx \cos I$ throughout the evolution of
%$\uv{s}$ (note that $\theta_{\rm sd, i} = 0$ implies the initial $s_{\rm z} = \cos I$).
Define $S = {\hat s}_{\rm x} + i{\hat s}_{\rm y}$, we find
\begin{equation}
    \rd{S}{\tau} = i\p{\eta\cos I - \cos\theta}S - i \eta \sin I\cos\theta.\label{eq:nonad_ode}
\end{equation}
To proceed, we assume $\theta$ is small and set $\cos\theta \simeq 1$. Equation
(\ref{eq:nonad_ode}) can then be solved explicitly, starting from the initial
value $S(\tau_{\rm i})$:
\begin{equation}
    S(\tau)e^{-i\Phi(\tau)} - S(\tau_{\rm i})
        \simeq -i\sin I\int_{\tau_{\rm i}}^\tau \eta(\tau')
            e^{-i\Phi(\tau')}\;\mathrm{d}\tau',
\end{equation}
where
\begin{equation}
    \Phi(\tau) \equiv \int_{\tau_{\rm i}}^\tau \p{\eta(\tau') \cos I -1}\;
        \mathrm{d}\tau'.
\end{equation}
We now invoke the stationary phase approximation, so that $\Phi(\tau)\simeq
\Phi(\tau_0) + (1/2)\ddot\Phi (\tau_0)(\tau - \tau_0)^2$, where $\tau_0$ is
determined by $\dot\Phi=0$ or $\eta_0=1/\cos I$. We then find, for $\tau\gg
\tau_0$,
\begin{equation}
    S(\tau)e^{-i\Phi(\tau)}-S(\tau_{\rm i})\simeq -i\sin I\,\eta(\tau_0)\, e^{-i\Phi(\tau_0)}
    \p{\frac{2\pi}{i \ddot\Phi(\tau_0)}}^{1/2}.
\end{equation}
Using $\dot\eta = -\epsilon \eta$ [Eq.~\eqref{eq:deta_dt}] and
$\ddot\Phi(\tau_0) = \dot\eta(\tau_0)\cos I = -\epsilon$, we have
\begin{equation}
    S(\tau)e^{-i\Phi(\tau)}-S(\tau_{\rm i})
        \simeq -i^{3/2}\tan I\, e^{-i\Phi(\tau_0)}
            \p{\frac{2\pi}{\epsilon}}^{1/2}.
\end{equation}
If the initial obliquity is negligible ($\theta_{\rm i} = \abs{S(\tau_{\rm
i})} \ll 1$), the final obliquity is given by
\begin{equation}
    \theta_{\rm f}\simeq \sqrt{\frac{2\pi}{\epsilon}}\tan I.\label{eq:nonad_q_f}
\end{equation}
This expression is valid only if $\theta_f$ is small (recall that we have used
the approximation $\cos\theta\simeq 1$ in the above derivation).

We can generalize the above analytic result to nonzero initial spin-disc
angle $\theta_{\rm sd, i}$ by assuming the amplitude
of libration of $\uv{s}$ about $\uv{l}_{\rm d}$ (namely $\theta_{\rm sd, i}$)
is equal to the range of final obliquities when the disk dissipates quickly.
Intuitively, this corresponds to ``freezing in'' the variations in initial
obliquity over a single libration cycle to the final obliquity. Thus we expect
\begin{equation}
  \theta_{\rm f} - \sqrt{\frac{2\pi}{\epsilon}} \tan I
  \in \s{-\theta_{\rm sd, i}, \theta_{\rm sd,i}}.\label{eq:nonad_q_f_dist}
\end{equation}

In Fig.~\ref{fig:nonad_3_ensemble}, we show the numerical result of $\theta_f$
vs $\theta_{\rm sd,i}$ for $I = 5^\circ$ and $\epsilon = 0.3$. We see that
Eq.~(\ref{eq:nonad_q_f_dist}) provides good lower and upper bounds of the final
obliquity for $\theta_{\rm sd,i} \lesssim 20^\circ$.

\begin{figure}
    \centering
    \includegraphics[width=0.5\textwidth]{plots_diskdisp/3_ensemble_05_05.png}
    \caption{$\theta_{\rm  f}$ vs $\theta_{\rm sd, i}$ for $I=5^\circ$ and
    $\epsilon = 0.3$ (firmly in the nonadiabatic regime). The blue dots
    represent numerical result, and the two red lines show the analytical lower
    and upper limits given by
    Eq.~\eqref{eq:nonad_q_f_dist}.}\label{fig:nonad_3_ensemble}
\end{figure}

Figures~\ref{fig:nonad_3_scan} and~\ref{fig:nonad_3_scan_20} show $\theta_f$ as
function of $\epsilon$ for $\theta_{\rm sd,i} = 0$ and $I = 5$, $20^\circ$. We
see that the agreement between the numerical results and
Eq.~\eqref{eq:nonad_q_f} is excellent, particularly for the $I = 5^\circ$ case.
The agreement when $I = 20^\circ$ is poorer as the initial obliquity
$\theta_{\rm i} = I$ is larger, violating one of our key assumptions. For
$\epsilon \ll 1$, we find $\theta_f \simeq 90^\circ$, in agreement with the
result of adiabatic evolution (see Fig.~\ref{fig:ad_ensemble}).

\begin{figure}
    \centering
    \includegraphics[width=0.5\textwidth]{plots_diskdisp/3scan.png}
    \caption{Final obliquity $\theta_{\rm  f}$ as a function of $\epsilon$ for
    $\theta_{\rm sd,i} = 0$ and $I = 5^\circ$. The shaded area, bordered by the
    black line, corresponds to the adiabatic regime estimated by
    Eq.~\eqref{eq:ad_constr}. The blue dots are numerical results, and the red
    line corresponds to Eq.~\eqref{eq:nonad_q_f}, which is in good agreement
    with numerical results for $\epsilon \gtrsim 0.1$ (the nonadiabatic regime);
    $\theta_{\rm f} \simeq  90^\circ$ in the adiabatic regime ($\epsilon\ll
    1$).}\label{fig:nonad_3_scan}
\end{figure}
\begin{figure}
    \centering
    \includegraphics[width=0.5\textwidth]{plots_diskdisp/3scan_20.png}
    \caption{Same as Fig.~\ref{fig:nonad_3_scan} but for $I=20^\circ$.
    }\label{fig:nonad_3_scan_20}
\end{figure}

\section{Summary and Discussion}\label{s:disc}

%DL: Not done yet

In this paper, we have studied the excitation of planetary obliquities by a
dissipating protoplanetary disk for arbitrary initial misalignment angles. We
present our result as a mapping from $\theta_{\rm sd, i}$ to $\theta_{\rm f}$,
where $\theta_{\rm f}$ is the final planetary obliquity and $\theta_{\rm sd, i}$
is the initial misalignment angle between the planet's spin axis and the disk's
orbital angular momentum. We have presented analytical results that capture the
behavior of this mapping in both the adiabatic and nonadiabatic limits:
\begin{enumerate}
    \item In the adiabatic limit, we reproduce the known result $\theta_{\rm f}
        = 90^\circ$ for $\theta_{\rm sd, i} \approx 0$ \citep{millholland_disk}.
        We demonstrate via numerical simulation and analytical argument the
        dual-valued behavior of $\theta_{\rm f}$ as nonzero initial spin-disk
        misalignment angles $\theta_{\rm sd, i}$ are permitted (see
        Fig.~\ref{fig:ad_ensemble}). We are able to capture both the exact final
        $\theta_{\rm f}$ values and the probabilities of observing each value
        via careful accounting of enclosed phase space area and separatrix
        crossing dynamics.

    \item As the system transitions more abruptly, the adiabatic prediction
        breaks down when criterion Eq.~\eqref{eq:ad_constr} is violated. We find
        a broad range of final obliquities can be reached for a given
        $\theta_{\rm sd, i}$ (see Fig.~\ref{fig:nonad_3_ensemble}). We provide
        an analytical expression of the bounds on $\theta_{\rm f}$ in
        Eq.~\eqref{eq:nonad_q_f_dist}.
\end{enumerate}

It is of interest to note the leading order dependence of $\theta_{\rm f}$ on
$\theta_{\rm sd, i}$ [given by Eq.~\eqref{eq:qf_21_approx} and
Eq.~\eqref{eq:qf_23_approx}] is second order. Therefore, if $\theta_{\rm sd, i}$
is generally small, as might be expected for planets that do not experience
strong collisions or scattering, resonance passage induced by an adiabatically
dissipating protoplanetary disk is expected to significantly narrow the final
$\theta_{\rm f}$ spread compared to the initial $\theta_{\rm sd, i}$ spread.

\bibliographystyle{mnras}
\bibliography{Su_sep_cross}

% \clearpage
% \onecolumn
\appendix

\section{Cassini State Local Dynamics}\label{s:local_dynamics}

In this section, we linearize the equations of motion near each CS\@. We
determine the stability of each CS and the local libration frequencies of the
stable CSs.

\subsection{Canonical Equations of Motion and Solutions}

For analytical work, we adopt spherical coordinate system where $\uv{l} =
\uv{z}$ and $\theta, \phi$ are the polar and azimuthal angle of $\uv{s}$. We
choose $\uv{l}_{\rm z}$ at coordinates $\theta = I, \phi = \pi$ (see
Figs.~\ref{fig:cs_vecs} and~\ref{fig:cs_locs}). Following modern mathematical
convention, we will use
$\theta \in [0, \pi)$% chktex 9
and $\phi \in [0, 2\pi)$.% chktex 9

The equations of motion in coordinates $\p{\phi, \cos \theta}$ follow by
applying Hamilton's equations to the Hamiltonian [Eq.~\eqref{eq:H}]:
\begin{subequations}\label{se:H_eom}
    \begin{align}
        \rd{\phi}{t} = \pd{\mathcal{H}}{(\cos\theta)}
            &= -\cos\theta + \eta\p{\cos I + \sin I \cot \theta \cos \phi},
                \label{seq:H_eom_phi_t}\\
        \rd{(\cos \theta)}{t} = -\pd{\mathcal{H}}{\phi}
            &= -\eta \sin I \sin \theta \sin \phi.
                \label{seq:H_eom_mu_t}
    \end{align}
\end{subequations}
These agree with Eq.~\eqref{eq:dsdt_base}. While these equations are numerically
stiff owing to the $\cot\theta$ term, they are the most intuitive description
for analytical work.

We next develop some approximate forms for the CS obliquities that will be
useful for guiding later discussion. Denote $\theta_{\rm cs}$ the obliquity of
some Cassini State (fixed point of Eqs.~\eqref{se:H_eom}).
\begin{enumerate}
    \item $\eta \ll 1, \cos \theta_{\rm cs} \ll 1$ --- This is the limiting case
        for CS2 and CS4 when $\eta \ll \eta_{\rm c}$. We examine
        Eqs.~\eqref{seq:H_eom_phi_t} and approximate $\cot \theta \approx \cos
        \theta$ to find
        \begin{equation}
            \cos \theta_{\rm cs} \approx \frac{\eta \cos I}
                {1 \mp \eta \sin I}.
        \end{equation}
        The two signs correspond to choices of $\cos \phi$. Each sign
        corresponds to one of CS2 and CS4.

    \item $\sin \theta_{\rm cs} \ll 1$ --- The $\eta \ll 1, \eta \gg 1$ cases
        can be solved together, and are the limiting cases both for CS1 and CS3
        when $\eta \ll \eta_{\rm c}$, and for CS2 and CS3 when $\eta \gg
        \eta_{\rm c}$. It proves easiest to rewrite Eqs.~\eqref{seq:H_eom_phi_t}
        as
        \begin{equation}
            0 = \cos \theta_{\rm cs}\p{-1 + 2\eta
                \frac{\sin \p{I \pm \theta_{\rm cs}}}{\sin \p{2\theta_{\rm
                    cs}}}}.\label{eq:rewritten_dphi}
        \end{equation}
        The $\pm$ choice again comes from choice of $\cos \phi$. Then assuming
        $\theta_{\rm cs}, I \ll 1$, we obtain $\frac{I \pm
        \theta_{\rm cs}}{2\theta_{\rm cs}} = \frac{1}{2\eta}$ or
        \begin{equation}
            \theta_{\rm cs} = \frac{\eta I}{1 \mp \eta}.
        \end{equation}

        In the limit $\eta \ll \eta_{\rm c}$, these two solutions describe CS1
        and CS3, while in the limit $\eta \gg \eta_{\rm c}$ these two solutions
        describe CS2 and CS3.
\end{enumerate}
Again, the $\theta_{\rm cs}$ values derived here do not follow the same
convention as Fig.~\ref{fig:cs_locs} due to our different $\theta, \phi$
convention.

\subsection{Stability and Frequency of Local Oscillations}\label{ss:eigens}

To examine stability of each CS, it proves very easy to handle them generally.
We linearize about an equilibrium located at $\phi_{\rm cs} = 0, \pi$ but
arbitrary $\theta_{\rm cs}$. Linearizing Eqs.~\eqref{se:H_eom} about $\phi =
\phi_{\rm cs} + \delta \phi, \theta = \theta_{\rm cs} + \delta \theta$ gives
\begin{subequations}\label{se:H_eom_lin}
    \begin{align}
        \rd{\delta \phi}{t} &= \sin \theta_{\rm cs} \delta \theta
            \mp \eta \frac{\sin I}{\sin^2\theta_{\rm cs}} \delta \theta,\\
        \rd{\delta \theta}{t} &= \pm \eta \sin I \delta \phi.
    \end{align}
\end{subequations}
Note that the positive sign corresponds to $\phi_{\rm cs} = 0$. Eliminating
$\delta \theta$ gives
\begin{align}
    \rtd{\delta \phi}{t} &= \p{\sin \theta_{\rm cs}
        \mp \eta \sin I\csc^2\theta}\p{\pm \eta \sin I} \delta
            \phi,\\
        &\equiv \lambda^2\delta \phi.\label{eq:lambda2}
\end{align}
A plot of $\lambda^2$ for each of the CSs is given in Fig.~\ref{fig:lambda2}.
From the plot, it is clear that only CS4 is a saddle point, while the other
three are centers (stable). The local libration frequency for these points is
just
\begin{align}
    \omega_{\rm lib} &= \sqrt{-\lambda^2},\\
        &= \sqrt{\p{\sin \theta_{\rm cs}
            \mp \eta \sin I \csc^2\theta}\p{\mp \eta \sin I}}.
\end{align}
The libration period is related simply $T_{\rm lib} = \frac{2\pi}{\omega_{\rm
lib}}$.
\begin{figure}
    \centering
    \includegraphics[width=0.5\textwidth]{plots_diskdisp/2_lambdas.png}
    \caption{$\lambda^2$, given by Eq.~\eqref{eq:lambda2}, evaluated at each of
    the Cassini States. The vertical axis is rescaled for clarity. Note that CS4
    is a saddle point ($\lambda^2 > 0$) when it exists while all others are
    stable ($\lambda^2 < 0$). The horizontal dashed line is the instability
    boundary $\lambda^2 = 0$ while the vertical dashed line labels $\eta =
    \eta_{\rm c}$.}\label{fig:lambda2}
\end{figure}

\section{Approximate Adiabatic Evolution}\label{s:ad_approx}

In this section, we will use approximations valid for small $\eta$ to derive
analytic expressions for the final obliquities at small $\theta_{\rm sd, i}$ and
associated probabilities for the II $\to$ I and II $\to$ III tracks that are
accessible.

We first seek a simple parameterization for the separatrix. We first solve for
equilibria of the equation of motion Eq.~\eqref{eq:dsdt_base} to compute the
coordinates for Cassini State 4:
\begin{equation}
    \cos \theta_4 \approx \frac{\mu \cos I}{1 - \eta \sin I}.
\end{equation}
Note that $\phi_4 = 0$. Then, the separatrix is the level curve of the
Hamiltonian intersecting CS4, so it satisfies $\mathcal{H}\p{\phi, \theta_{\rm
sep}(\phi)} = \mathcal{H}\p{\phi_4, \theta_4}$. To leading order in $\eta$, this
has simple expression
\begin{equation}
    \cos \theta_{\rm sep}(\phi) \approx \cos \theta_4 \pm
        \sqrt{2\eta \sin I\p{1 - \cos \phi}}.
\end{equation}
Integration of the phase area enclosed by the two legs of the separatrix then
yields
\begin{equation}
    A_{\rm II}(\eta) \approx 16\sqrt{\eta \sin I}.\label{eq:a_approx}
\end{equation}
This, in conjunction with Eqs.~\eqref{eq:henrard_hop}, is sufficient to compute
$\theta_{\rm f}\p{\theta_{\rm sd, i}}$ for zone II initial conditions.

\begin{enumerate}
    \item For a given $\theta_{\rm sd, i}$, we know that if $\eta \to \infty$
        then the trajectory executes simple libration about $\uv{l}_{\rm d}$,
        and so $A = 2\pi\p{1 - \cos \theta_{\rm sd, i}} \approx \pi \theta_{\rm
        sd, i}^2$. This then implies $\eta_\star$ must be the solution to
        $A_{\rm II}(\eta_\star) = A$, or
        \begin{align}
            \eta_\star &\approx \p{\frac{2\pi\p{1 - \cos \theta_{\rm sd,i}}}{
                        16}}^2 / \sin I,\\
                    &\approx \p{\frac{\pi \theta_{\rm sd, i}^2}{16}}^2/\sin I.
        \end{align}

    \item Upon separatrix encounter, a transition to either zone I or zone
        III occurs. These can be calculated to have associated probabilities
        (using the approximate area Eq.~\eqref{eq:a_approx})
        \begin{subequations}
            \begin{align}
                \Pr\p{\rm II \to I} &\approx \frac{2\pi
                    \eta_{\star} \cos I + 4\sqrt{\eta_{\star}\sin
                    I}}{8\sqrt{\eta_{\star}\sin I}},\\
                \Pr\p{\rm II \to III} &\approx \frac{-2\pi
                    \eta_{\star} \cos I + 4\sqrt{\eta_{\star}\sin
                    I}}{8\sqrt{\eta_{\star}\sin I}}.
            \end{align}
        \end{subequations}

    \item Upon a transition to zone I or zone III, the final obliquities can
        be predicted by observing the final adiabatic invariant $A_{\rm f} = -A_{\rm
        I}(\eta_\star)$ in the zone I case and $A_{\rm f} = A_{\rm I}(\eta_\star) +
        A_{\rm I}I(\eta_\star)$ in the zone III case. As $\eta \to 0$, these
        correspond to obliquities
        \begin{subequations}\label{se:q_f_approx}
            \begin{align}
                \p{\cos \theta_{\rm f}}_{\rm II \to I} &\approx
                    \p{\frac{\pi \theta_{\rm sp, i}^2}{16}}^2 \cot I
                        + \frac{\theta_{\rm sp, i}^2}{4},\\
                \p{\cos \theta_{\rm f}}_{\rm II \to III} &\approx
                    \p{\frac{\pi \theta_{\rm sp, i}^2}{16}}^2 \cot I
                        - \frac{\theta_{\rm sp, i}^2}{4}.
            \end{align}
        \end{subequations}
        These are the black dotted lines overplotted in
        Fig.~\ref{fig:ad_ensemble}.
\end{enumerate}


\bsp
\label{lastpage} % chktex 24
\end{document} % chktex 17
