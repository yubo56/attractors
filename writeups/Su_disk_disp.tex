    \documentclass[
        fleqn,
        usenatbib,
        referee,
    ]{mnras}
    \usepackage{
        amsmath,
        amssymb,
        newtxtext,
        newtxmath,
        graphicx,
        ae, aecompl,
        booktabs,
        caption,
        subcaption,
    }
    \usepackage[T1]{fontenc}
    \captionsetup{compatibility=false}

    \newcommand*{\rd}[2]{\frac{\mathrm{d}#1}{\mathrm{d}#2}}
    \newcommand*{\rtd}[2]{\frac{\mathrm{d}^2#1}{\mathrm{d}#2^2}}
    \newcommand*{\pd}[2]{\frac{\partial#1}{\partial#2}}
    \newcommand*{\md}[2]{\frac{\mathrm{D}#1}{\mathrm{D}#2}}
    \newcommand*{\at}[1]{\left.#1\right|}
    \newcommand*{\abs}[1]{\left|#1\right|}
    \newcommand*{\ev}[1]{\langle#1\rangle}
    \newcommand*{\p}[1]{\left(#1\right)}
    \newcommand*{\s}[1]{\left[#1\right]}
    \newcommand*{\z}[1]{\left\{#1\right\}}
    \DeclareMathOperator*{\argmin}{argmin}
    \DeclareMathOperator*{\argmax}{argmax}
    \DeclareMathOperator*{\med}{med}

\title[Analytical Exoplanet Obliquities]{Analytical Predictions of Explanet
Obliquities Generated by Planet-Disk Interactions}
\author[Y. Su et\ al.]{
Yubo Su$^1$,
Dong Lai$^1$
\\
$^1$ Cornell Center for Astrophysics and Planetary Science, Department of
Astronomy, Cornell University, Ithaca, NY 14853, USA
}

\date{Accepted XXX\@. Received YYY\@; in original form ZZZ}

\pubyear{2019}

\begin{document}\label{firstpage}
\pagerange{\pageref{firstpage}--\pageref{lastpage}}
\renewcommand*{\sectionautorefname}{Section}
% \renewcommand*{\subsectionautorefname}{Subsection}
\maketitle

\begin{abstract}
    Large planetary spin-orbit misalignments (obliquities) are thought to play a
    key role in some planets' atmospheric circulation and tidal heating. A
    promising avenue to generate high obliquities is via spin-orbit resonances,
    where the spin and orbital precession frequencies of the planet become and
    remain commensurate as the system evolves. One such proposed mechanism
    involves a dissipating protoplanetary disk driving orbital precession of an
    interior planet \citep{millholland_disk}. This previous work is limited to
    small initial misalignment angles between the spin axis of the planet and
    orbital axis of the massive disk. In the present paper, this scenario is
    analytically characterized and extended to arbitrary misalignment angles. It
    is shown that (i) under adiabatic evolution, final planetary obliquities
    bifurcate into distinct tracks governed by interactions with the resonance,
    while (ii) under non-adiabatic evolution, a broad range of obliquities can
    be excited. Analytical formulae are provided for most interesting regimes of
    parameter space.
\end{abstract}

\begin{keywords}
planet--star interactions % chktex 8
\end{keywords}

\section{Introduction}

Substantial planetary obliquities, defined as the angle between a planet's spin
and orbital axes, have featured in many astrophysical scenarios. They lie at the
heart of the obliquity tides scenario of radius enhancement in hot Jupiters
\citep{winn_otides, fabrycky_otides, millholland_wasp12b}. They have also been
invoked to explain peculiar thermal phase curves in hot Jupiters \citep[see
e.g.][]{millholland_signatures, ohno_infer_obl}. While exoplanet obliquities
have yet to be directly detected, techniques have been proposed to constrain
them \citep[see e.g.][]{schwarz_infer_obl, rauscher_infer_obl}.

Many mechanisms have been proposed to generate large planetary obliquities.
Giant impacts are traditionally used to generate the large obliquities of the
solar system ice giants \citep{original_gi, morbidelli_gi}. More recently,
obliquity excitations have been achieved via \emph{spin-orbit resonances}, where
the spin and orbital precession frequencies of the planet become and remain
commensurate as the system evolves (sometimes called \emph{resonance
advection}). Spin-orbit resonances have been studied extensively in solar system
dynamics as \emph{Cassini States} and are responsible for the significant
nonzero obliquities of both Jupiter ($3.1^\circ$) and Saturn ($26.7^\circ$)
\citep{colombo1966, henrard1987, ward2004I, ward_jupiter}. Recent work has
attempted to apply this mechanism to the ice giants as well
\citep{hamilton_tilting_ice}.

Producing exoplanet obliquities via spin-orbit resonances was the subject of
\citet{millholland_disk}, where a dissipating protoplanetary disk causes
resonance capture and advection. In their work, they begin with a planet whose
spin is nearly aligned with the angular momentum vector of the protoplanetary
disk. After the disk dissipates, the resulting planetary obliquity is generally
substantial, reaching $90^\circ$ for what they characterize as adiabatic
resonance crossings.

However, their study was limited to small initial misalignment angles between
the spin vector of the planet and the orbital angular momentum of the
protoplanetary disk. Such an assumption is intuitive, since the planet forms
from the disk, but prospects for it to be violated exist. In particular, giant
and planetesimal impacts are expected to be abundant during planet formation and
may perturb the spin axis of the protoplanet from its primordial value
\citep{yalinewich2019atmospheric, schlichting2015atmospheric}. Furthermore, the
abundance of compact multi-planet systems provides support for possible
scattering between close-in, newly-formed planets \citep{usp_compact1,
usp_review}. It is conceivable that such processes could generate significant
spin-disk misalignment within the lifetime of the disk.

In our work, we provide an analytical framework for understanding the final
planetary obliquity for arbitrary initial spin-disk misalignment angles. Our
analytical work and independent numerical simulations are in agreement with
\citet{millholland_disk} but provide a more complete characterization of
possible behavior.

The paper is organized as follows. In \autoref{s:eq}, we review relevant
spin-orbit dynamics and present the equations studied in the rest of the paper.
In \autoref{s:ad} and \autoref{s:nonad}, we study the evolution of the system
when the disk either dissipates adiabatically or non-adiabatically respectively.
Exhaustive yet simple analytical arguments are able to explain completely the
numerical results in both limits. We discuss potential consequences in
\autoref{s:disc}. Our primary physical results consist of
\autoref{fig:ad_ensemble} in the adiabatic limit and
\autoref{fig:3_ensemble_05_15} in the non-adiabatic limit. Longer mathematical
calculations are relegated to the appendicies, including a leading-order
estimate of final planetary obliquities given small initial spin-disk
misalignment angles in \autoref{s:ad_approx}.

\section{Theory}\label{s:eq}

\subsection{Equations of Motion}

Much of this treatment parallels that of \citet{anderson2018teeter} and
\citet{millholland_disk}. We consider an oblate star of mass $M_\star$ hosting a
planet of mass $M_p$, radius $R_p$, at semimajor axis $a_p$, and with spin
angular frequency $\Omega_p$, and a protoplanetary disk of mass $M_d$ at some
characteristic distance $r_d$ from the host star \citep[see][for a power-law
disk profile]{millholland_disk}. Denote $\hat{s} \equiv \vec{S} / S$ spin unit
vector of planet, $\hat{l} \equiv \vec{L} / L$ orbital angular momentum unit
vector of planet, and $\hat{l}_d \equiv \vec{L}_d / L_d$ angular momentum of the
surrounding disc. We approximate $S \ll L \ll L_d$, so $\hat{l}_d$ is
approximately constant and $\hat{l}$ experiences negligible backreaction torque
from $\hat{s}$. The quadrupole-order secular equations of motion for $\hat{s}$,
$\hat{l}$ and $\hat{l}_d$ then become very simple
\begin{align}
    \rd{\hat{s}}{t} &= \omega_{sl} \p{\hat{s} \cdot \hat{l}}
        \p{\hat{s} \times \hat{l}},\label{eq:dsdt}\\
    \rd{\hat{l}}{t} &= \omega_{ld}\p{\hat{l} \cdot \hat{l}_d}
        \p{\hat{l} \times \hat{l}_d}\label{eq:dldt}.
\end{align}
where
\begin{align}
    \omega_{sl} &\equiv \frac{3k_{qp}}{2k_p} \frac{M_\star}{m_p}
        \p{\frac{R_p}{a_p}}^3 \Omega_p,\label{eq:wsl}\\
    \omega_{ld} &\equiv \frac{3M_d}{4M_\star}\p{\frac{a_p}{r_d}}^3 n
        .\label{eq:wld}
\end{align}
% = 1/(152600 yr) (kp = 6kqp, Mstar = msun, mp = mJ, Rp = RJ, ap = AU, Wp = 2pi
% / (10 days))
% = 1 / (573362 yr) (Md/Mstar = 0.01, ap/rd = 1/30)
In \autoref{eq:wsl}, $k_{qp}$ and $k_p$ are related to the planet's quadrupole
moment and moment of inertia respectively \citep[see][]{lai2018}, and $n \equiv
\sqrt{GM_\star/a_p^3}$ is the planet's orbital mean motion. The three angles
formed by the three vectors are denoted
\begin{align}
    \theta &\equiv \arccos \hat{s} \cdot \hat{l},\\
    \theta_{sd} &\equiv \arccos \hat{s} \cdot \hat{l}_d,\\
    I &\equiv \arccos \hat{l} \cdot \hat{l}_d.
\end{align}

Let us transform to frame rotating about $\hat{l}_p$ with frequency $g \equiv
-\omega_{ld}\cos I < 0$, then $\hat{l}$ is fixed. In this frame, $\hat{l}_p$
also remains fixed and $\hat{s}$ evolves as:
\begin{equation}
    \rd{\hat{s}}{t} = \alpha \p{\hat{s} \cdot \hat{l}}
            \p{\hat{s} \times \hat{l}}
        + g\p{\hat{s} \times \hat{l}_d},
\end{equation}
where $\alpha \equiv \omega_{sl} > 0$ follows standard notation. By defining
$\eta$ and nondimensionalized time $\tau$ as
\begin{align}
    \eta \equiv{}& \frac{\abs{g}}{\alpha}\label{eq:eta},\\
        ={}& 3.75 \p{\frac{k_p}{6k_{qp}}}
            \p{\frac{m_p}{0.001 M_\odot}}
            \p{\frac{R_p}{R_J}}^{-3}
            \p{\frac{\Omega_p}{2\pi/\p{10\;\mathrm{days}}}}^{-1}\nonumber\\
        &\p{\frac{M_d}{0.01 M_{\odot}}}
            \p{\frac{a_p}{1\;\mathrm{AU}}}^{9/2}
            \p{\frac{r_d}{30 \;\mathrm{AU}}}^{-3}
            \p{\frac{M_\star}{1M_{\odot}}}^{-3/2}
            \cos I,\\
    \tau \equiv{}& \alpha t,
\end{align}
we can non-dimensionalize the equation of motion as
\begin{equation}
    \rd{\hat{s}}{\tau} = \p{\hat{s} \cdot \hat{l}}
            \p{\hat{s} \times \hat{l}}
        - \eta\p{\hat{s} \times \hat{l}_d}. \label{eq:dsdt_base}
\end{equation}

Following~\cite{millholland_disk}, we allow $g$ to vary in time owing to a
disk with decaying mass
\begin{equation}
    M_d(t) = M_d(0)e^{-t/t_d}.
\end{equation}
As $\alpha$ is held constant, this equates to $\eta$ decaying as $\rd{\eta}{t} =
-\frac{\eta}{t_d}$ as well. We nondimensionalize:
\begin{align}
    \epsilon \equiv{}& \frac{1}{\alpha t_d},\\
        ={}& 0.015 \p{\frac{k_p}{6k_{qp}}}
            \p{\frac{m_p/M_\star}{0.001}}
            \p{\frac{a_p}{1\;\mathrm{AU}}}^3\nonumber\\
        &\p{\frac{R_p}{R_J}}^{-3}
            \p{\frac{\Omega_p}{2\pi/\p{10\;\mathrm{days}}}}
            \p{\frac{t_d}{10\;\mathrm{Myr}}}^{-1},\\
    \rd{\eta}{\tau} ={}& -\epsilon \eta.\label{eq:deta_dt}
\end{align}
\autoref{eq:dsdt_base} and \autoref{eq:deta_dt} together constitute our system
of study.

\subsection{Cassini States}\label{ss:cs}

Spin states satisfying $\rd{\hat{s}}{t} = 0$ are referred to as \emph{Cassini
States} (CSs). There are either two or four solutions, depending on the values
of $\eta, I$. Following standard convention and nomenclature (see
\autoref{fig:cs_locs}), CSs 1, 3, 4 have $\phi = 0, \theta < 0$ and CS2 has
$\phi = \pi, \theta > 0$. Of these, CSs 1, 2, 3 are stable while CS4 is unstable
(see \autoref{s:local_dynamics}). When $\eta < \eta_c$, where
\begin{equation}
    \eta_c \equiv \p{\sin^{2/3}I + \cos^{2/3}I}^{-3/2},
\end{equation}
all four CSs exist, and when $\eta > \eta_c$, only CSs 2, 3 exist
\citep{henrard1987,ward2004I}. The CS obliquities as a function of $\eta$ are
shown in \autoref{fig:cs_locs}.
\begin{figure}
    \centering
    \includegraphics[width=0.5\textwidth]{../initial/99_misc/2_cs_locs.png}
    \caption{Cassini state obliquities as a function of $\eta$. Note that
    $\theta \in [-\pi, \pi]$ is the traditional definition of the polar angle
    \citep[see e.g.][]{colombo1966,peale1969,henrard1987}.}\label{fig:cs_locs}
\end{figure}

\subsection{Separatrix}

When $\eta < \eta_c$, CS4 exists and is a saddle point. The two trajectories
originating and ending at CS4 are the only two infinite-period orbits in phase
space. Together, these two critical trajectories are referred to as the
\emph{separatrix} and divide phase space into three zones. The separatrix, three
zones, and their relations to the CSs are shown via a contour plot of the
Hamiltonian of the system
\begin{equation}
    \mathcal{H} = -\frac{1}{2}\p{\hat{s} \cdot \hat{l}}^2
            + \eta \p{\hat{s} \cdot \hat{l}_d}.\label{eq:H}
\end{equation}
Such a contour plot is given in \autoref{fig:eq_1contours}. Note that
trajectories in zone $II$ librate about CS2 while those in zones $I$/$III$
circulate.
\begin{figure*}
    \centering
    \includegraphics[width=0.8\textwidth]{../initial/0_eta/1contours_flip.png}
    \caption{Contour plot of $\mathcal{H}\p{\phi, \cos \theta}$ as given in
    \autoref{eq:H}, where warmer colors denote more positive values. The black
    solid line is the separatrix, which only exists for $\eta < \eta_c$. The
    three zones, divided by the separatrix, are labeled. The Cassini states are
    labeled and have the same colors as \autoref{fig:cs_locs}. The interior of
    the separatrix, shaded in grey, is formally only defined for $\eta <
    \eta_c$, but we may identify the phase space trajectories that flow into
    zone $II$ when evolved forward in time; this is the shaded region in the top
    left plot, bounded by the dotted black line.}\label{fig:eq_1contours}
\end{figure*}

The unsigned areas of the three zones are known exactly
\citep{henrard1987,ward2004I}. Defining
\begin{align*}
    z_0 &= \eta\cos I, &
    \chi &= \sqrt{-\frac{\tan^3\theta_4}{\tan I} - 1},\\
    \rho &= \chi \frac{\sin^2 \theta_4\cos \theta_4}{
        \chi^2 \cos^2\theta_4 + 1},&
    T &= 2\chi \frac{\cos \theta_4}{
        \chi^2 \cos^2\theta_4 - 1},\\
\end{align*}
the areas for $\eta < \eta_c$ are given by
\begin{subequations}\label{se:area_ward}
    \begin{align}
        A_{II} &= 8\rho + 4\arctan T - 8z_0 \arctan \frac{1}{\chi},\\
        A_I &= 2\pi\p{1 - z_0} - \frac{A_2}{2},\\
        A_{III} &= 2\pi\p{1 + z_0} - \frac{A_2}{2}.
    \end{align}
\end{subequations}
These are plotted as a function of $\eta$ in \autoref{fig:eq_areas}. Note that
the zones are not formally defined for $\eta > \eta_c$, but a natural extension
exists: evolve an initial point $p$ under adiabatic decrease of $\eta > \eta_c$
until the separatrix appears at $\eta = \eta_c$, then identify $p$ with the zone
it is in at $\eta_c$. Since phase space area is conserved under adiabatic
evolution, this implies $A_i\p{\eta > \eta_c} = A_i(\eta_c)$. This extension is
also reflected in the dashed boundary in \autoref{fig:eq_1contours}.
\begin{figure}
    \centering
    \includegraphics[width=0.5\textwidth]{../initial/99_misc/1_areas.png}
    \caption{Plot of fractional areas of each of the zones $A_{i}(\eta) / 4\pi$
    as given by \autoref{se:area_ward}. Dotted lines correspond to small $\eta$
    approximations discussed in \autoref{s:ad_approx}. Dashed lines for $\eta >
    \eta_c$ effective values of $A_{II}, A_{III}$ for $\eta > \eta_c$, denoting
    the area of points that would flow into either area under adiabatic decrease
    of $\eta$ from $\eta > \eta_c$.}\label{fig:eq_areas}
\end{figure}

\section{Adiabatic Evolution}\label{s:ad}

A perturbation to a Hamiltonian system is considered \emph{adiabatic} if it is
sufficiently slow compared to all other relevant time scales in the problem.
In the current problem, the slowest time scale is the \emph{separatrix crossing
time scale}: as $\eta$ evolves, a trajectory may encounter the separatrix at
some value $\eta_\star$ which varies depending on its initial condition. If
$\eta$ changes sufficiently little during the separatrix-crossing orbit, the
evolution of the system may be considered adiabatic. The truly adiabatic limit
corresponds to $\epsilon \to 0$; we take $\epsilon = 3 \times 10^{-4}$ unless
otherwise noted.

One criterion for adiabaticity is that the separatrix crossing timescale be
longer than the the resonant libration period \citep{ward2004II}. The resonant
libration period for a nearly separatrix-crossing orbit can be approximated by
the libration period near CS2, resulting in constraint
\begin{align}
    -\at{\frac{1}{\eta}\rd{\eta}{\tau}}_{cross} = \epsilon &\lesssim
            \frac{1}{T_{lib}},\\
        &\lesssim \frac{1}{2\pi}\sqrt{\eta_\star\sin I \sin \theta_2
            \p{1 + \eta_\star \sin I \csc^3 \theta_2}},
            \label{eq:ad_constr}
\end{align}
where $\theta_2$ is the obliquity at CS2. This formula differs from that
presented in \citet{millholland_disk}, where the $\csc^3\theta_2$ term is
neglected and a square root is missing. See \autoref{s:local_dynamics} for a
detailed derivation of $T_{lib}$ and comparison to existing work.

\subsection{Adiabatic Evolution Outcomes}\label{ss:ad_ensemble}

We consider the evolution of a system with arbitrary initial $\theta_{sd, i}
\equiv \at{\theta_{sd}}_{t = 0}$ and $\eta_i \equiv \at{\eta}_{t = 0} \gg 1$. We
are interested in the distribution of final obliquities after $\eta$ decreases
to $\eta_f \ll 1$ (after the disk has dissipated to negligible mass). While it
is likely that $\theta_{sd, i} \approx 0$ as the planet forms in the disk, we
explore all $\theta_{sd, i} \in [0, \pi]$ for completeness.

We evenly sample $101$ values of $\theta_{sd, i}$, and for each $\theta_{sd, i}$
value, we pick $101$ $\hat{s}$ initial conditions approximately on the ring
fixed by $\theta_{sd, i}$\footnote{For each $\theta_{sd, i}$ we choose an
initial condition on the libration cycle going through $\p{\theta_2 +
\theta_{sd, i}, \phi_2}$ (where $\p{\theta_2, \phi_2}$ are the coordinates of
CS2), such that all initial conditions for a given $\theta_{sd, i}$ enclose the
same initial phase space; see \autoref{sss:a_evo}.}. To be concrete, we choose
$\eta_i = 10\eta_c$. For each initial condition, we evolve
Equations~\ref{eq:dsdt_base} and~\ref{eq:deta_dt} until $\eta = 10^{-5}$. For
such small $\eta$, $\hat{s}$ precesses about $\hat{l}$ with uniform final
obliquity $\theta_f$. This transfer function $\theta_f\p{\theta_{sd, i}}$ is our
primary result, and is shown for $I = 5^\circ, 10^\circ, 20^\circ$ in
\autoref{fig:ad_ensemble}, \autoref{fig:3_ensemble_10_35}, and
\autoref{fig:3_ensemble_20_35} respectively. The blue dots represent the results
of the simulation, which are in excellent agreement with the colored tracks.
These colored tracks are calculated semi-analytically using methods explained in
the following section.

\begin{figure}
    \centering
    \includegraphics[width=0.5\textwidth]{../initial/2_toy2/3_ensemble_05_35.png}
    \caption{Top: $\theta_{f}\p{\theta_{sd, i}}$, overlaid with semi-analytic
    predictions of the $\theta_{ f}$ for each of the four nontrivial
    dynamical tracks in colored lines. Black dotted lines correspond to
    small-$\theta_{sd, i}$ predictions derived in \autoref{s:ad_approx}.
    Bottom: Semi-analytic probabilities of each of the dynamical tracks for each
    $\theta_{sd, i}$. The five regimes of $\theta_{sd, i}$ values demarcated by
    the vertical dashed black lines correspond to the five regimes of $A_i$
    calculated in \autoref{ss:zone_transitions}. Black dotted lines correspond to
    simple analytic estimates given in
    \autoref{s:ad_approx}.}\label{fig:ad_ensemble}
\end{figure}
\begin{figure}
    \centering
    \includegraphics[width=0.5\textwidth]{../initial/2_toy2/3_ensemble_10_35.png}
    \caption{Same as \autoref{fig:ad_ensemble} but for $I =
    10^\circ$ and with fewer annotations.}\label{fig:3_ensemble_10_35}
\end{figure}
\begin{figure}
    \includegraphics[width=0.5\textwidth]{../initial/2_toy2/3_ensemble_20_35.png}
    \caption{Same as \autoref{fig:3_ensemble_10_35} but for $I = 20^\circ$. The
    $III \to II \to I$ track is very short since the maximum $A_{II}$ is very
    near $\eta_c$ for $I = 20^\circ$.}\label{fig:3_ensemble_20_35}
\end{figure}

\subsection{Adiabatic Evolution Theory}\label{ss:zone_transitions}

The evolutionary tracks that govern the behavior of $\theta_f\p{\theta_{sd,
i}}$ correspond to particular sequences of separatrix crossings. Coarsely, they
can be understood using a combination of (i) how the enclosed phase space area
by the trajectory evolves across each separatrix crossing, and (ii) the
associated probabilities with each separatrix crossing.

\subsubsection{Governing Principles: Evolution of $A$}\label{sss:a_evo}

First, we consider how the enclosed phase space area by a trajectory evolves
over time. In the absence of separatrix encounters, the enclosed phase space
area $\oint \cos\theta \;\mathrm{d}\phi$ is an adiabatic invariant. In
particular, we adopt convention where
\begin{equation}
    A \equiv \oint \p{1 - \cos \theta}\;\mathrm{d}\phi.\label{eq:a_oint}
\end{equation}
This definition of $A$ has the advantage of (i) being continuous at $\cos \theta
= 1$, and (ii) being easily expressible as combinations of the $A_i$ of
\autoref{se:area_ward}. The bounds of the integral are either a libration
($\phi$ returns to its original value with the same $\dot{q\phi}$ sign) or
circulation cycle ($\phi$ advances to $\phi \pm 2\pi$). Note lastly that $A$ has
simple physical interpretation of are enclosed on the unit sphere by $\hat{s}$
over one cycle measured relative to $+\hat{z}$:
\begin{equation*}
    \int \int\limits_{\cos \theta(\phi)}^1
        \mathrm{d}\cos\theta\;\mathrm{d}\phi = \int \p{1 - \cos \theta(\phi)}
            \;\mathrm{d}\phi.
\end{equation*}

Observe that when $\eta_i \gg 1$, trajectories librate about $\hat{l}_d$ with
constant $\theta_{sd}$, meaning they enclose initial phase space area
\begin{equation}
    A_i \equiv \at{A}_{t = 0}
        = 2\pi\p{1 - \cos \theta_{sd, i}}.\label{eq:ai_qsd}
\end{equation}
Complications arise when considering finite $\eta_i$, as trajectories near CS2 or
CS3 librate about these equilibria respectively, rather than $\hat{l}_d$, and
\autoref{eq:ai_qsd} is no longer exact. In practice, \autoref{eq:ai_qsd} holds
very well when defining $\theta_{sd, i}$ as the angular distance to CS2; an
exception is discussed in \autoref{sss:evol_traj}.

Beginning at the last separatrix crossing, the final enclosed phase space area
$A_f$ will be conserved for all time. As $\eta \to 0$, trajectories circulate at
constant obliquity $\theta_f$, related to $A_f$ by
\begin{equation}
    2\pi\p{1 - \cos \theta_f} = A_f. \label{eq:qfaf}
\end{equation}

$A$ is not conserved when the trajectory encounters the separatrix. Howover,
its change is easily understood \citep{henrard1982}. In essence, when the
trajectory crosses the separatrix, it continues to evolve adjacent to the
separatrix. So if a separatrix crossing results in a zone I trajectory, the
resultant $A$ can be approximated by integrating \autoref{eq:a_oint} along the
upper leg of the separatrix. Pictorally, this can be seen in the bottom plots of
\autoref{fig:ad_21}.

\subsubsection{Governing Principles: Probabilistic Separatrix Crossing}

When a trajectory experiences separatrix crossing, it transitions into nearby
zones probabilistically. This process is well understood by
\citealp{henrard1982,henrard1987}. Their results may be simply summarized as
follows: if a zone $i$ is shrinking while adjacent zones $j, k$ are expanding
such that the sum of their areas is constant, the probability of transition from
zone $i$ to zone $j$ is given
\begin{align}
    \Pr\p{i \to j} = -\frac{\pd{A_j}{t}}{ \pd{A_i}{t}}
        = -\frac{\pd{A_j}{\eta}}{ \pd{A_i}{\eta}},\\
    \Pr\p{i \to k}
        = -\frac{\pd{A_k}{\eta}}{ \pd{A_i}{\eta}}.\label{eq:henrard_hop}
\end{align}
Note that $\Pr \p{i \to j} + \Pr\p{i \to k} = 1$. This can be used directly in
conjunction with \autoref{se:area_ward} to understand for what initial
conditions each track can be observed and with what probabilities.

As a particular example, consider a system in zone $II$ in panel (d) of
\autoref{fig:eq_1contours}. As $\eta$ decreases, zone $II$ will shrink while
zones $I$ and $III$ will expand until the trajectory. Suppose the trajectory
exits zone $II$ at some $\eta_\star$, then the probability of a $II \to I$
transition is $\Pr\p{II \to I} = -\frac{\dot{A}_{I}}{\dot{A}_{II}}$, while a $II
\to III$ transition occurs with probability $\Pr\p{II \to III} =
-\frac{\dot{A}_{III}}{\dot{A}_{II}}$.

\subsubsection{Evolutionary Trajectories}\label{sss:evol_traj}

Returning to the evolution of $\hat{s}$, we can classify trajectories by the
separatrix encounters experienced. Initially, in the $\eta > \eta_c$ regime,
only zones $II, III$ exist. Conversely, at the end of the simulation when $\eta
\to 0$, only zones $I, III$ exist. Most simply, one might expect four sequences
of transitions between zones (we call these dynamical ``tracks'') to manifest:
from one of $\z{I, III}$ to one of $\z{II, III}$, where $III \to III$ is a
trivial case (no separatrix encounter). In addition to these four tracks, a
fifth track $III \to II \to I$ is observed. Below, we will describe the
evolution of $A$ throughout each of the five tracks, as well as determine the
initial conditions and probabilities associated with each track:
\begin{enumerate}
    \item $II \to I$ --- An example trajectory following this track is depicted
        in \autoref{fig:ad_21}. $\hat{s}$ starts librating about CS2 in zone
        $II$, enclosing some initial phase space area $A_i$. The trajectory is
        able to librate without separatrix encounter until $A_{II}(\eta_\star) =
        A_i$. As the trajectory transitions to a circulating trajectory in zone
        $I$ immediately bordering the separatrix, it will encompass
        $-A_I(\eta_\star)$ phase space area. The final $\theta_f$ is then given
        by \autoref{eq:qfaf}. An analytical approximation to $\theta_f$ is
        derived in \autoref{s:ad_approx} and is
        \begin{equation}
            \cos \theta_{f, II \Rightarrow I} \approx
                \p{\frac{\pi \theta_{sp, i}^2}{16}}^2 \cot I
                    + \frac{\theta_{sp, i}^2}{4}.\label{eq:qf_21_approx}
        \end{equation}

        This transition can only occur when the initial condition begins in zone
        II, requiring $A_i < A_{II}(\eta_c)$. Note $A_{II}(\eta_c)$ is given in
        closed form in \citet{ward2004I} but appears to be incorrect. Then, at
        some known $\eta_\star$ satisfying $A_{II}(\eta_\star) = A_i$, the
        trajectory is ejected from the separatrix following
        \autoref{eq:henrard_hop}. Note that since $\pd{A_i}{\eta} < 0$
        everywhere, while $\pd{A_{II}}{\eta} > 0$ at all possible $\eta_\star$
        for an initial condition starting in zone $II$, this track always has
        nonzero probability.

    \item $II \to III$ --- An example trajectory following this track is
        depicted in \autoref{fig:ad_23}. The only difference from the previous
        track is that, upon separatrix encounter, the trajectory follows the
        circulating trajectory in zone $III$ bordering the separatrix, upon
        which it will encompass $A_f = A_I(\eta_\star) + A_{II}(\eta_\star)$.
        The final obliquity is still given by \autoref{eq:qfaf}, and the
        analytical approximation derived in \autoref{s:ad_approx} is
        \begin{equation}
            \cos \theta_{f, II \Rightarrow III} \approx
                \p{\frac{\pi \theta_{sp, i}^2}{16}}^2 \cot I
                    - \frac{\theta_{sp, i}^2}{4}.\label{eq:qf_23_approx}
        \end{equation}

        Again, this track can only occur when $A_i < A_{II}(\eta_c)$, but a
        further constraint arises when we consider the transition probability
        given by \autoref{eq:henrard_hop}. Upon examination of
        \autoref{fig:eq_areas}, it is clear that $\pd{A_{III}}{\eta} > 0$ for
        many $\eta$. Call
        \begin{equation}
            \eta_{\min, III} \equiv \argmin A_{III}(\eta),
        \end{equation}
        which is labeled in \autoref{fig:eq_areas}, then if $\eta_\star >
        \eta_{\min, III}$ then $\Pr_{II \to III} < 0$. This is understood as a
        forbidden transition, and so $II \to III$ is only a permitted dynamical
        track if $\eta_\star <, \eta_{\min, III}$.

    \item $III \to I$ --- This track is depicted in \autoref{fig:ad_31}.
        The trajectory encounters the separatrix when $A_I(\eta_\star) +
        A_{II}(\eta_\star) = A_i$, upon which it transitions to a zone $I$
        trajectory enclosing $A_f = -A_I$. Then, as always, the final obliquity
        is given by \autoref{eq:qfaf}.

        This track can only occur if $A_i > A_{II}(\eta_c)$, but is also
        constrained by requiring its $A_i$ is sufficiently small it will
        encounter the separatrix (if it is too large, it will never encounter
        the separatrix, which is a $III \to III$ transition). This is written
        $A_i < \max A_I + A_{II} = 4\pi - \min A_{III}$.

        Note that since $\pd{A_I}{\eta} < 0$ always while $\pd{A_{III}}{\eta} >
        0$ for all accessible $\eta_{\star}$, this track is always permitted.

    \item $III \to II \to I$ --- This track is depicted in \autoref{fig:ad_321}.
        The first separatrix encounter occurs at $\eta_1$ when $A_I(\eta_1) +
        A_{II}(\eta_1) = A_i$, upon which the trajectory moves into zone $II$
        enclosing intermediate phase space area $A_m = A_{II}(\eta_1)$.
        Examining \autoref{fig:eq_areas}, there exist values $A$ for which
        $A_{II}(\eta) = A$ has multiple solutions, and $\eta_1 > \eta_{II,
        \max}$ as labeled. These are the only values for which the $III \to II
        \to I$ transition is permitted. Thus, as $\eta$ continues to decrease, a
        second $\eta_2$ value exists for which $A_m = A_{II}(\eta_2)$, upon
        which the trajectory is ejected to zone $I$ and $A_f = -A_I(\eta_2)$.
        The final obliquity again then obeys \autoref{eq:qfaf}.

        Similar to the $III \to I$ track, the $III \to II \to I$ track can only
        occur over the same $A_{II}(\eta_c) < A_i < \max A_I + A_{II}$ interval.
        The probability of encountering this track is set by the initial $III
        \to II$ transition. It bears noting that this probability is only
        nonnegative for a very small fraction of $A_i$ values, since it requires
        $\pd{A_{II}}{\eta}$ and $\pd{A_{III}}{\eta}$ to have different signs;
        this occurs only if $A_{II}\p{\eta_c} < A_i < A_{II, \max}$. Outside of
        these bounds, $\Pr_{III \to II} < 0$ which is interpreted again as a
        forbidden transition.

        Then, once a $III \to II$ transition occurs, the second $II \to I$
        transition occurs for some $\eta_2$ satisfying $A_{II}(\eta_2) =
        A_{II}(\eta_1), \eta_2 < \eta_1$. Graphical inspection shows that
        $\pd{A_{II}}{\eta}$ has the same sign as $\pd{A_{III}}{\eta}$ over all
        possible values, so the second $II \to I$ transition is guaranteed,
        completing the $III \to II \to I$ track.

    \item $III \to III$ --- This track depicts the trivial case where no
        separatrix encounter ever occurs, and $A$ is constant throughout the
        evolution. This constraint equates to $A_i > \max A_I + A_{II}$,
        whereupon no separatrix encounter occurs at all, and $A_f = A_i$, which
        for $\eta_i \to \infty, \eta_f \to 0$ results in $\theta = \theta_{sd,
        i}$.

        Note that a small deviation from $\theta = \theta_{sd, i}$ occurs
        because we use finite $\eta_i$, as hinted at in \autoref{sss:a_evo}.
        This is because for $\theta_{sd, i} \gtrsim \pi/2$, trajectories are
        better described as librating about CS3. Nevertheless, we can still
        approximate $A_i$ to good accuracy as the solid angle enclosed by
        libration about CS3 with polar angle $2\pi + \theta_3 - \theta_2 -
        \theta_{sd, i}$. This recovers the correct transfer function
        $\theta_f\p{\theta_{sd, i}}$ for the $III \to III$ regime, as can be
        seen visibly in \autoref{fig:3_ensemble_20_35}.
\end{enumerate}
In summary, the five regimes of track possibilities in increasing order of $A_i$
are:
\begin{enumerate}
    \item $A_i \in \s{0, A_{II}\p{\eta_{\min, III}}}$ --- $II \to III, II \to I$
        both possible.

    \item $A_i \in \s{A_{II}\p{\eta_{\min, III}}, A_{II}(\eta_c)}$ --- $II \to
        I$ only.

    \item $A_i \in \s{A_{II}(\eta_c), A_{II, \max}}$ --- $III \to I, III \to II
        \to I$ both possible.

    \item $A_i \in \s{A_{II, \max}, \max A_I + A_{II}}$ --- $III \to I$ only.

    \item $A_i > \max A_I + A_{II}$ --- $III \to III$ only.
\end{enumerate}
The corresponding ranges for for $\theta_{sd,i}$ can be computed via
\autoref{eq:ai_qsd}. The boundaries between these ranges are overplotted in
\autoref{fig:ad_ensemble}, where they can be seen to agree well with numerical
results.
\begin{figure}
    \centering
    \begin{subfigure}{\columnwidth}
        \centering
        \includegraphics[width=0.5\textwidth]{../initial/2_toy2/3testo21.png}
    \end{subfigure}
    \begin{subfigure}{\columnwidth}
        \centering
        \includegraphics[width=0.5\textwidth]{../initial/2_toy2/3testo21_subplots.png}
    \end{subfigure}
    \caption{Fiducial simulation following the $A_{II} \to A_{I}$ transition. An
    initial $\theta_{sd, i} = 0.3\;\mathrm{rad} \approx 17.2^\circ$ was used, as
    well as $\epsilon = 3 \times 10^{-4}$. Top plot upper panel: plot of $\cos
    \theta(t)$ (green) in an example simulation. Overlaid are the locations of
    Cassini State 2 (red), and upper and lower bounds on the separatrix (black).
    Note that the trajectory successfully tracks CS2 to an obliquity $\theta
    \approx 90^\circ$. Light vertical blue lines denote times portrayed in
    bottom plot. Top plot lower panel: plot of the enclosed separatrix area
    obtained by integrating the simulated trajectory (blue dots) and adiabatic
    theory (red line). Even though the enclosed area jumps at separatrix
    crossing, the resultant $A$ is still accurately predicted. Lower plot:
    snapshots in the $\p{\cos \theta, \phi}$ space of the simulation
    trajectories for the vertical blue lines in the top plots. These times
    correspond to the start of the simulation, the appearance of the separatrix,
    two panels depicting the separatrix crossing process, and a final snapshot
    at $\eta = 10^{-3.5}$. The green line denotes a full circulation or
    libration cycle at the selected time, including an arrow for directionality.
    The separatrix at the start/end of the portrayed cycle are shown in
    solid/dashed black lines respectively. Also labeled is CS2 at the start of
    the cycle (red). Finally, the enclosed phase space area is shaded in grey
    ($A > 0$) and red ($A < 0$).}\label{fig:ad_21}
\end{figure}
\begin{figure}
    \centering
    \begin{subfigure}{\columnwidth}
        \centering
        \includegraphics[width=0.5\textwidth]{../initial/2_toy2/3testo23.png}
    \end{subfigure}
    \begin{subfigure}{\columnwidth}
        \centering
        \includegraphics[width=0.5\textwidth]{../initial/2_toy2/3testo23_subplots.png}
    \end{subfigure}
    \caption{Same as \autoref{fig:ad_21} but for the $A_2 \to A_3$ track.
    $\theta_{sd, i} = 0.3\;\mathrm{rad}$ and $\epsilon = 3.01 \times
    10^{-4}$.}\label{fig:ad_23}
\end{figure}
\begin{figure}
    \centering
    \begin{subfigure}{\columnwidth}
        \centering
        \includegraphics[width=0.5\textwidth]{../initial/2_toy2/3testo31.png}
    \end{subfigure}
    \begin{subfigure}{\columnwidth}
        \centering
        \includegraphics[width=0.5\textwidth]{../initial/2_toy2/3testo31_subplots.png}
    \end{subfigure}
    \caption{Same as \autoref{fig:ad_21} but for the $A_{III} \to A_I$
    track. $\theta_{sd, i} = 89.1^\circ$, and $\epsilon = 3 \times
    10^{-4}$.}\label{fig:ad_31}
\end{figure}
\begin{figure}
    \centering
    \begin{subfigure}{\columnwidth}
        \centering
        \includegraphics[width=0.5\textwidth]{../initial/2_toy2/3testo321.png}
    \end{subfigure}
    \begin{subfigure}{\columnwidth}
        \centering
        \includegraphics[width=0.5\textwidth]{../initial/2_toy2/3testo321_subplots.png}
    \end{subfigure}
    \caption{Same as \autoref{fig:ad_21} but for the $A_{III} \to A_{II} \to A_I$
    track. $\theta_{sd, i} = 60^\circ$, and $\epsilon = 3.14 \times 10^{-4}$.
    Two separatrix crossings are shown.}\label{fig:ad_321}
\end{figure}

\section{Nonadiabatic Effects}\label{s:nonad}

\subsection{Transition to Non-adiabaticity}

To illustrate the transition to non-adiabaticity, we present two simulations
with increasingly larger values of $\epsilon$ than \autoref{fig:ad_ensemble}
below in \autoref{fig:3_ensemble_05_25} and \autoref{fig:3_ensemble_05_15}.

At first non-adiabaticity manifests as a larger scatter of final obliquities
near the tracks predicted from adiabatic evolution. These scatters first set in
for trajectories starting in zone $III$, as these cross the separatrix at higher
values of $\theta$ and have a stricter adiabaticity criterion (see
\autoref{eq:ad_constr}).

Later, these large scatters take on band-like structures that seem to persist
across values of $\theta_{sd, i}$; we attribute these to the separatrix crossing
process becoming sensitive to the \emph{phase} of the final
libration/circulation orbit at separatrix crossing. This is equivalent to
separatrix crossing no longer being significantly slower than the resonant
libration period and violating the adiabatic assumption.
\begin{figure}
    \centering
    \includegraphics[width=0.5\textwidth]{../initial/2_toy2/3_ensemble_05_25.png}
    \caption{Same as \autoref{fig:ad_ensemble} but for $\epsilon = 10^{-2.5}$,
    and restricting $\theta_{sd, i} < 90^\circ$. A larger spread from the
    dynamical tracks is observed in the data.}\label{fig:3_ensemble_05_25}
\end{figure}
\begin{figure}
    \centering
    \includegraphics[width=0.5\textwidth]{../initial/2_toy2/3_ensemble_05_15.png}
    \caption{Same as \autoref{fig:3_ensemble_05_25} but for $\epsilon =
    10^{-1.5}$. Some small semblance of the evolutionary tracks remains, and the
    deviations appear to have a banded structure. This can be attributed to
    non-adiabaticity ``freezing-in'' the phase of the obliquity variations over
    the final libration/circulation orbit prior to separatrix
    crossing.}\label{fig:3_ensemble_05_15}
\end{figure}

A sample trajectory following in the style of \autoref{fig:ad_21} but for
$\epsilon = 0.3$ is provided in \autoref{fig:nonad_traj}. It is clear the
trajectory does not track level curves of the Hamiltonian during each individual
snapshot (e.g.\ the third snapshot).
\begin{figure}
    \centering
    \begin{subfigure}{\columnwidth}
        \centering
        \includegraphics[width=0.5\textwidth]{../initial/2_toy2/3testo_nonad.png}
    \end{subfigure}
    \begin{subfigure}{\columnwidth}
        \centering
        \includegraphics[width=0.5\textwidth]{../initial/2_toy2/3testo_nonad_subplots.png}
    \end{subfigure}
    \caption{Same as \autoref{fig:ad_21} but for a non-adiabatic $\epsilon =
    0.3$. In the top panel of the top plot, it is evident that the libration
    cycle about CS2 is unable to keep up with the swift motion of CS2 as $\eta$
    changes, decreasing the obliquity jump. In the bottom plot, we can see that
    individual orbits do not lie along level curves of the Hamiltonian, as the
    Hamiltonian phase space changes quickly compared to the period of
    circulation cycles.}\label{fig:nonad_traj}
\end{figure}

\subsection{Non-adiabatic Evolution Outcomes}

A formula for $\theta_{f}$ assuming $\theta_{sd, i} = 0$ initially can be
given (see \autoref{s:ad_approx})
\begin{equation}
    \theta_{f}\p{\theta_{sd, i} = 0} = \sqrt{\frac{2\pi}{\epsilon}}
        \tan I\cos I.\label{eq:nonad_q_f}
\end{equation}
We can naively generalize this by recognizing that any nonzero $\theta_{sd, i}$
manifests as obliquity variations as $\hat{s}$ librates about $\hat{l}_d$. These
can be accounted for by simply allowing the final obliquity to vary over the
same range as the initial obliquity, or,
\begin{equation}
    \theta_{f}\p{\theta_{sd, i}} - \sqrt{\frac{2\pi}{\epsilon}}
        \tan I\cos I\in \s{-\theta_{sd, i}, \theta_{sd,i}}
        .\label{eq:nonad_q_f_dist}
\end{equation}

We present the results of simulations for using $\epsilon = 0.3$ in
\autoref{fig:nonad_3_ensemeble}.
\begin{figure}
    \centering
    \includegraphics[width=0.5\textwidth]{../initial/2_toy2/3_ensemble_05_05.png}
    \caption{$\theta_{ f}\p{\theta_{sd, i}}$ at $\epsilon = 0.3$, firmly in
    the non-adiabatic regime. Note the clear double-valuedeness has disappeared,
    as have distinct dynamical tracks. The red dotted line presents the
    analytical prediction given by
    \autoref{eq:nonad_q_f_dist}.}\label{fig:nonad_3_ensemeble}
\end{figure}

The agreement of \autoref{eq:nonad_q_f} at fixed $I$ for varying $\epsilon$ is
shown in \autoref{fig:nonad_3_scan}. Note that $\epsilon \to 0$ recovers the
adiabatic regime. The deviation of $\theta_f$ from the analytical prediction
within the adiabatic regime is indeed due to adiabatic effects becoming
dominant; compare to \autoref{fig:nonad_3_scan_20} where little deviation is
observed until $\theta_f \approx 90^\circ$.
\begin{figure}
    \centering
    \includegraphics[width=0.5\textwidth]{../initial/2_toy2/3scan.png}
    \caption{Plot of $\theta_{ f}\p{\theta_{sd, i} = 0}$ as a function of
    $\epsilon$, where $I = 5^\circ$. The shaded area, bordered by the black
    line, corresponds to the adiabatic regime estimated by
    \autoref{eq:ad_constr}. Overplotted in the red line is
    \autoref{eq:nonad_q_f}, which is in good agreement for $\epsilon \gtrsim
    0.1$ the non-adiabatic regime, while $\theta_{f} \approx 90^\circ$ in the
    adiabatic regime.}\label{fig:nonad_3_scan}
\end{figure}
\begin{figure}
    \centering
    \includegraphics[width=0.5\textwidth]{../initial/2_toy2/3scan_20.png}
    \caption{Same as \autoref{fig:nonad_3_scan} but for $I=20^\circ$. Note that
    \autoref{eq:nonad_q_f} is accurate until $\theta_f \approx 90^\circ$, which
    is much closer to the transition to adiabaticity. Comparison with
    \autoref{fig:nonad_3_scan} suggests that the breakdown of
    \autoref{eq:nonad_q_f} indeed is caused by the breakdown of the
    non-adiabatic assumption.}\label{fig:nonad_3_scan_20}
\end{figure}

\section{Summary and Discussion}\label{s:disc}

In this paper, we have studied the excitation of planetary obliquities by a
dissipating protoplanetary disk for arbitrary initial misalignment angles. This
is described by transfer function $\theta_f\p{\theta_{sd, i}}$, where $\theta_f$
is the final planetary obliquity and $\theta_{sd, i}$ is the initial
misalignment angle between the planet's spin axis and the disk's orbital angular
momentum. We have presented analytical results that capture the behavior of
$\theta_f\p{\theta_{sd, i}}$ in both the adiabatic and nonadiabatic limits:
\begin{enumerate}
    \item In the adiabatic limit, we are able to reproduce the known result
        $\theta(0) \approx 90^\circ$ \citep{millholland_disk}. We demonstrate
        via numerical simulation the dual-valued behavior of $\theta$ as nonzero
        values of $\theta_{sd, i}$ are permitted (see
        \autoref{fig:ad_ensemble}). We are able to capture both the exact final
        $\theta$ values and the probabilities of observing each value via
        careful accounting of enclosed phase space area and separatrix crossing
        dynamics \citep{henrard1982,henrard1987}.

    \item As the system transitions more abruptly, the adiabatic prediction
        breaks down when criterion \autoref{eq:ad_constr} is violated. We find a
        broad range of $\theta$ values can be excited for a given $\theta_{sd,
        i}$ (see \autoref{fig:nonad_3_ensemeble}). We provide an analytical
        expression of the bounds on $\theta_f\p{\theta_{sd, i}}$ in
        \autoref{eq:nonad_q_f_dist}.
\end{enumerate}

It is of interest to note the leading order behavior of $\theta$ in the
small $\theta_{sd, i}$ limit given by \autoref{eq:qf_21_approx} and
\autoref{eq:qf_23_approx}. Therefore, if $\theta_{sd, i}$ is generally small, as
would be expected for peacefully forming planets, resonance capture induced by
an adiabatically dissipating protoplanetary disk is expected to significantly
narrow the final $\theta$ spread compared to the initial $\theta_{sd, i}$
spread.

\bibliographystyle{mnras}
\bibliography{Su_sep_cross}

% \clearpage
% \onecolumn
\appendix

\section{Cassini State Local Dynamics}\label{s:local_dynamics}

In this section, we perform local linearizations about the CSs to determine the
stability of each CS and, when stable, the local libration frequencies.

\subsection{Canonical Equations of Motion and Solutions}

For analytical work, we adopt spherical coordinate system where $\hat{l} =
\hat{z}$ and $\theta, \phi$ are the polar and azimuthal angle of $\hat{s}$. By
convention, we choose $\hat{l}_z$ at coordinates $\theta = I, \phi = \pi$.
Note that $\p{\cos \theta, \phi}$ form a canonically conjugate pair of variables
describing $\hat{s}$. To derive equations of motion for these canonical
coordinates, we rewrite \autoref{eq:H}
\begin{equation}
    \mathcal{H}\p{\theta, \phi} = -\frac{1}{2}\cos^2\theta
            + \eta \p{\cos \theta \cos I - \sin I \sin \theta \cos \phi}.
\end{equation}

The equations of motion follow by taking derivatives of $\mathcal{H}$ and agree
with \autoref{eq:dsdt_base}:
\begin{subequations}\label{se:H_eom}
    \begin{align}
        \rd{\phi}{t} = \pd{\mathcal{H}}{(\cos\theta)}
            &= -\cos\theta + \eta\p{\cos I + \sin I \cot \theta \cos \phi},
                \label{seq:H_eom_phi_t}\\
        \rd{(\cos \theta)}{t} = -\pd{\mathcal{H}}{\phi}
            &= -\eta \sin I \sin \theta \sin \phi.
                \label{seq:H_eom_mu_t}
    \end{align}
\end{subequations}
While these equations are numerically stiff owing to the $\cot\theta$ term, they
are the most intuitive description for analytical work.

We next develop some approximate forms for the CS obliquities that will be
useful for guiding later discussion.
\begin{enumerate}
    \item $\eta \ll 1, \cos \theta_{CS} \ll 1$ --- This is the limiting case for
        CS2 and CS4 when $\eta \ll \eta_c$. We examine \autoref{seq:H_eom_phi_t}
        and approximate $\cot \theta \approx \cos \theta$ to find
        \begin{equation}
            \cos \theta_{CS} \approx \frac{\eta \cos I}
                {1 \mp \eta \sin I}.
        \end{equation}
        The two signs correspond to choices of $\cos \phi$. Each sign
        corresponds to one of CS2 and CS4.

    \item $\sin \theta_{CS} \ll 1$ --- The $\eta \ll 1, \eta \gg 1$ cases can be
        solved together, and are the limiting cases both for CS1 and CS3 when
        $\eta \ll \eta_c$, and for CS2 and CS3 when $\eta \gg \eta_c$. It proves
        easiest to rewrite \autoref{seq:H_eom_phi_t} as
        \begin{equation}
            0 = \cos \theta_{CS}\p{-1 + 2\eta
                \frac{\sin \p{I \pm \theta_{CS}}}{\sin \p{2\theta_{CS}}}}.
                \label{eq:rewritten_dphi}
        \end{equation}
        The $\pm$ choice again comes from choice of $\cos \phi$. Then assuming
        $\theta_{CS}, I \ll 1$, we obtain $\frac{I \pm
        \theta_{CS}}{2\theta_{CS}} = \frac{1}{2\eta}$ or
        \begin{equation}
            \theta_{CS} = \frac{\eta I}{1 \mp \eta}.
        \end{equation}

        In the limit $\eta \ll \eta_c$, these two solutions describe CS1 and
        CS3, while in the limit $\eta \gg \eta_c$ these two solutions describe
        CS2 and CS3.
\end{enumerate}
Note that the $\theta_{CS}$ values derived here do not follow the same
convention as \autoref{fig:cs_locs}, because the previous $\theta$ conventions
presume $\cos \phi = -1$ for all states and let $-\pi \leq \theta < \pi$.

\subsection{Stability}

To examine stability of each CS, it proves very easy to handle them generally.
We linearize about an equilibrium located at $\bar{\phi} = 0, \pi$ but arbitrary
$\bar{\theta}$. Linearizing \autoref{se:H_eom} about $\phi = \bar{\phi} + \delta
\phi, \theta = \bar{\theta} + \delta \theta$ gives
\begin{subequations}\label{se:H_eom_lin}
    \begin{align}
        \rd{\delta \phi}{t} &= \sin \bar{\theta} \delta \theta
            \mp \eta \frac{\sin I}{\sin^2\bar{\theta}} \delta \theta,\\
        \rd{\delta \theta}{t} &= \pm \eta \sin I \delta \phi.
    \end{align}
\end{subequations}
Note that the positive sign corresponds to $\cos \bar{\phi} = +1, \bar{\phi} =
0$. Eliminating $\delta \theta$ gives
\begin{align}
    \rtd{\delta \phi}{t} &= \p{\sin \bar{\theta}
        \mp \eta \sin I\csc^2\theta}\p{\pm \eta \sin I} \delta
            \phi,\\
        &\equiv \lambda^2\delta \phi.\label{eq:lambda2}
\end{align}
A plot of $\lambda^2$ for each of the CSs is given in \autoref{fig:lambda2}.
From the plot, it is clear that only CS4 is a saddle point, while the other
three are centers (stable). The local libration frequency for these points is
just
\begin{align}
    \omega_{lib} &= \sqrt{-\lambda^2},\\
        &= \sqrt{\p{\sin \bar{\theta}
            \mp \eta \sin I \csc^2\theta}\p{\mp \eta \sin I}}.
\end{align}
The libration period is related simply $T_{lib} = \frac{2\pi}{\omega_{lib}}$.
\begin{figure}[t]
    \centering
    \includegraphics[width=0.5\textwidth]{../initial/99_misc/2_lambdas.png}
    \caption{Plot of $\lambda^2$ per \autoref{eq:lambda2} evaluated at each
    of the Cassini States. It is clear that CS4 is a saddle point when it
    exists while all others are centers.}\label{fig:lambda2}
\end{figure}

Note that our expression is somewhat more complicated than in existing
literature \citep{millholland_disk,ward2004II}. This seems to stem from the
approximation $\theta \gg I$ made in deriving Equation 3 of
\citet{ward2004II}. This is invalid in the $\eta \gtrsim \eta_c$ limit for
CS2, since $\theta_{CS} \approx I$ in this regime as was shown above.

\subsection{Consequences for Adiabaticity}\label{ss:careful_ad}

In \autoref{s:ad}, we said the adiabaticity criterion is given by comparison of
the separatrix crossing timescale to the resonant libration period
\citep{ward2004II}, and we corrected the formula presented by
\citet{millholland_disk}. Two further comments can be made:
\begin{enumerate}
    \item The adiabaticity criterion compares how quickly the
        separatrix-crossing orbit is completed to how quickly the separatrix
        location shifts. However, it's not obvious that the libration period
        about CS2 captures the duration of the separatrix-crossing orbit, which
        as $\epsilon \to 0$ formally diverges. Thus, this criterion can only be
        taken as a necessary but not sufficient criterion to guarantee accurate
        predictions of $A$ post-encounter.

        We speculate a formally sufficient criterion may be able to be
        formulated as $\Delta t \lesssim \frac{\Delta \eta}{\dot{\eta}}$, where
        $\Delta t$ is the length of the separatrix-crossing orbit, and $\Delta
        \eta$ is the change in $\eta$ over the separatrix-crossing orbit. Note
        for all $\epsilon > 0$ that $\Delta t < \infty$. The exact form of such
        a criterion is inessential to our result.

    \item A different adiabaticity criterion can be defined for
        $III$-originating trajectories, as the separatrix-crossing orbit is much
        shorter (see e.g.\ \autoref{fig:ad_31}). However, the resultant
        criterion becomes extremely strict for the $III \to I$ track, as it goes
        through very small $\theta$. That this track is least adiabatic track
        can be seen by comparing \autoref{fig:ad_ensemble} and
        \autoref{fig:3_ensemble_05_25}, where the $III \to I$ track develops the
        largest spread about the analytical prediction.
\end{enumerate}

\section{Approximate Adiabatic Evolution}\label{s:ad_approx}

In this section, we will use approximations valid for small $\eta$ to derive
analytic expressions for the final obliquities at small $\theta_{sd, i}$ and
associated probabilities for the $II \to I, II \to III$ tracks that are
accessible.

We first seek a simple parameterization for the separatrix. We first solve for
equilibria of the equation of motion \autoref{eq:dsdt_base} to compute the
coordinates for Cassini State 4:
\begin{equation}
    \cos \theta_4 \approx \frac{\mu \cos I}{1 - \eta \sin I}.
\end{equation}
Note that $\phi_4 = 0$. Then, the separatrix is the level curve of the
Hamiltonian intersecting CS4, so it satisfies $\mathcal{H}\p{\theta_{sep}(\phi),
\phi} = \mathcal{H}\p{\theta_4, \phi_4}$. This solves to be approximately
\begin{equation}
    \cos \theta_{sep}(\phi) \approx \cos \theta_4 \pm
        \sqrt{2\eta \sin I\p{1 - \cos \phi}}.
\end{equation}
Integration of the phase area enclosed by the two legs of the separatrix then
yields
\begin{equation}
    A_{II}(\eta) \approx 16\sqrt{\eta \sin I}.\label{eq:a_approx}
\end{equation}
This, in conjunction with \autoref{eq:henrard_hop}, is sufficient to compute
$\theta_f\p{\theta_{sd, i}}$ for zone $II$ initial conditions.

\begin{enumerate}
    \item For a given $\theta_{sd, i}$, we know that if $\eta \to \infty$ then
        the trajectory executes simple libration about $\hat{l}_d$, and so $A =
        2\pi\p{1 - \cos \theta_{sd, i}} \approx \pi \theta_{sd, i}^2$. This
        then implies $\eta_\star$ must be the solution to $A_{II}(\eta_\star) =
        A$, or
        \begin{align}
            \eta_\star &\approx \p{\frac{2\pi\p{1 - \cos \theta_{sd,i}}}{
                        16}}^2 / \sin I,\\
                    &\approx \p{\frac{\pi \theta_{sd, i}^2}{16}}^2/\sin I.
        \end{align}

    \item Upon separatrix encounter, a transition to either zone $I$ or zone
        $III$ occurs. These can be calculated to have associated probabilities
        (using the approximate area \autoref{eq:a_approx})
        \begin{subequations}
            \begin{align}
                \Pr_{II \Rightarrow I} &\approx \frac{2\pi
                    \eta_{\star} \cos I + 4\sqrt{\eta_{\star}\sin
                    I}}{8\sqrt{\eta_{\star}\sin I}},\\
                \Pr_{II \Rightarrow III} &\approx \frac{-2\pi
                    \eta_{\star} \cos I + 4\sqrt{\eta_{\star}\sin
                    I}}{8\sqrt{\eta_{\star}\sin I}}.
            \end{align}
        \end{subequations}

    \item Upon a transition to zone $I$ or zone $III$, the final obliquities can be
        predicted by observing the final adiabatic invariant $A_f =
        -A_I(\eta_\star)$ in the zone $I$ case and $A_f = A_I(\eta_\star) +
        A_II(\eta_\star)$ in the zone $III$ case. As $\eta \to 0$, these
        correspond to obliquities
        \begin{subequations}\label{se:q_f_approx}
            \begin{align}
                \cos \theta_{f, II \Rightarrow I} &\approx
                    \p{\frac{\pi \theta_{sp, i}^2}{16}}^2 \cot I
                        + \frac{\theta_{sp, i}^2}{4},\\
                \cos \theta_{f, II \Rightarrow III} &\approx
                    \p{\frac{\pi \theta_{sp, i}^2}{16}}^2 \cot I
                        - \frac{\theta_{sp, i}^2}{4}.
            \end{align}
        \end{subequations}
        These are the black dotted lines overplotted in
        \autoref{fig:ad_ensemble}.
\end{enumerate}

\section{Derivation of Nonadiabatic Evolution}\label{s:nonad_app}

We present a single result, the derivation of \autoref{eq:nonad_q_f}. We take
\autoref{eq:dsdt_base} and choose coordinate axes such that $\hat{l} = \hat{z},
\hat{l}_d = \hat{z} \cos I + \hat{x}\sin I$, then we obtain
\begin{equation}
    \rd{\hat{s}}{t} = \s{
        \p{\eta \cos I - 1}\hat{z} + \eta \sin I \hat{x}} \times \hat{s}.
\end{equation}

Now, let's assume that $s_z \approx \cos I$ throughout the evolution of $\hat{s}$
(note that $\theta_{sd, i} = 0$ implies the initial $s_z = \cos I$). Then let's
examine the evolution of quantity $S = s_x + is_y$ instead:
\begin{equation}
    \rd{S}{t} = i\p{\eta\cos I - 1}S - i \eta \sin I\cos I.\label{eq:nonad_ode}
\end{equation}
Now this is a first-order ODE in $S$, albeit complex, which can be solved via
an integrating factor
\begin{align}
    \Phi(t) &\equiv \int_{-\infty}^t \p{1 - \eta(t') \cos I}
        \mathrm{d}t',\\
    \at{S(t) e^{i\Phi(t)}}_{-\infty}^t
        &= \int\limits_{-\infty}^t e^{i\Phi(t')}
            \p{-i\eta(t')\sin I\cos I}\;\mathrm{d}t'\label{eq:nonad_int}
\end{align}
We now invoke stationary phase, asserting that $e^{i\Phi(t)}$ is dominated by
its contribution where $\dot{\Phi} = 0$ (the phases add constructively). But
$\dot{\Phi} = 0$ is where $1 - \eta\cos I = 0$, or where $\eta\cos I = 1$.

Now at this point, let's choose $\eta(0) = 1/\cos I, \at{\rd{\eta}{t}}_{t=0} =
-\epsilon/\cos I$. Then we expand near $t = 0$ so
\begin{align*}
    \Phi(t) &\approx \Phi(0) + \frac{1}{2}\ddot{\Phi}t^2,\\
        &\approx \Phi(0) + \frac{1}{2}\epsilon t^2,\\
    \int\limits_{-\infty}^t e^{i\Phi(t')}\eta(t')\;\mathrm{d}t'
        &\approx
        \begin{cases}
            0 & t < 0,\\
            \frac{1}{\cos I}e^{i\Phi(0)}\int\limits_{-\infty}^\infty
                \exp\s{\frac{i}{2}\epsilon t^2}\;\mathrm{d}t
                & t > 0.
        \end{cases}\\
    \int\limits_{-\infty}^\infty
                \exp\s{\frac{i}{2}\epsilon t^2}\;\mathrm{d}t
        &= \int\limits_{-\infty}^\infty e^{-\tau^2}\;\mathrm{d}\tau
            \sqrt{\frac{2}{i\epsilon}},\\
        &= \sqrt{\frac{2\pi}{i\epsilon}}.
\end{align*}

Now, it should be noted that $e^{i\Phi}$ is just a phase; all we really care
about is $\abs{S} = \sqrt{1 - s_z^2}$. Thus, taking the absolute value of both
sides of \autoref{eq:nonad_int}, assuming $S\p{-\infty} \ll S\p{+\infty}$ and
noting $\theta_f \approx \abs{S}(+\infty)$, we obtain
\begin{equation}
    \theta_f = \tan I\cos I\sqrt{\frac{2\pi}{\epsilon}}.\label{eq:nonad_dong}
\end{equation}

It should be noted that this calculation breaks down in two ways:
\begin{enumerate}
    \item If the evolution of $\hat{s}$ is adiabatic, then it is invalid to
        assume $s_z$ is approximately constant in time, as many
        circulation/libration orbits can ensue. Only when the driving is
        sufficiently impulsive that the evolution dominates the change in $s_z$
        is this calculation valid.

    \item If $\abs{S\p{-\infty}} \sim \abs{S\p{+\infty}}$, then taking the
        absolute value of both sides of \autoref{eq:nonad_int} no longer simply
        yields $\abs{S}\p{+\infty}$. This corresponds to the extreme limit
        where $\eta$ changes so suddenly that $\hat{s}$ has no time to respond
        and remains roughly unchanged as $\eta \to 0$. This is accommodated by
        noting $\theta_f \geq \theta_{sd, i}$, so the correct estimate can be
        roughly amended
        \begin{equation}
            \theta_f \simeq \min\p{\tan I\cos I\sqrt{\frac{2\pi}{\epsilon}},
                \theta_{sd, i}}.
        \end{equation}
\end{enumerate}

\bsp
\label{lastpage} % chktex 24
\end{document}
