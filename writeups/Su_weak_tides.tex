% vim 0_eta/1*.py 0_eta/3*.py 0_eta/6*.py 3_toy3/1sim.py 99_misc/0*.py 99_misc/2*.py 99_misc/5*.py 1_weaktide/5*.py 1_weaktide/6*.py
% cd 0_eta && python 1*.py && python 3*.py && python 6*.py && cd .. && cd 3_toy3 && python 1sim.py && cd .. && cd 99_misc && python 0*.py && python 2*.py && python 5*.py && cd .. && cd 1_weaktide && python 5*.py && python 6*.py
    \documentclass[
        fleqn,
        usenatbib,
        % referee,
    ]{mnras}
    \usepackage{
        amsmath,
        amssymb,
        newtxtext,
        newtxmath,
        ae, aecompl,
        graphicx,
        booktabs,
        xcolor,
    }

    \newcommand*{\scinot}[2]{#1\times10^{#2}}
    \newcommand*{\rd}[2]{\frac{\mathrm{d}#1}{\mathrm{d}#2}}
    \newcommand*{\rtd}[2]{\frac{\mathrm{d}^2#1}{\mathrm{d}#2^2}}
    \newcommand*{\pd}[2]{\frac{\partial#1}{\partial#2}}
    \newcommand*{\ptd}[2]{\frac{\partial^2#1}{\partial#2^2}}
    % inline
    \newcommand*{\mdil}[2]{\mathrm{D}#1/\mathrm{D}#2}
    \newcommand*{\pdil}[2]{\partial#1/\partial#2}
    \newcommand*{\rdil}[2]{\mathrm{d}#1/\mathrm{d}#2}
    \newcommand*{\at}[1]{\left.#1\right|}
    \newcommand*{\abs}[1]{\left|#1\right|}
    \newcommand*{\ev}[1]{\left\langle#1\right\rangle}
    \newcommand*{\p}[1]{\left(#1\right)}
    \newcommand*{\s}[1]{\left[#1\right]}
    \newcommand*{\z}[1]{\left\{#1\right\}}
    \newcommand*{\bm}[1]{\mathbf{#1}}
    \newcommand*{\uv}[1]{\hat{\mathbf{#1}}}
    \newcommand*{\md}[0]{\mathrm{d}}
    \DeclareMathOperator*{\med}{med}
    \DeclareMathOperator*{\erf}{erf}

\title[Weak Tides and Cassini States]{Dynamics of Colombo's Top: Tidal
Dissipation and Resonance Capture}
\author[Y. Su and D. Lai.]{
Yubo Su,$^1$\thanks{E-mail: yubosu@astro.cornell.edu},
Dong Lai$^{1,2}$
\\
$^1$ Cornell Center for Astrophysics and Planetary Science, Department of
Astronomy, Cornell University, Ithaca, NY 14853, USA\\
$^2$ Tsung-Dao Lee Institute \& School of Physics and Astronomy, Shanghai Jiao
Tong University, 200240 Shanghai, China
}

\date{Accepted XXX\@. Received YYY\@; in original form ZZZ}

\pubyear{2021}

\begin{document}\label{firstpage}
\pagerange{\pageref{firstpage}--\pageref{lastpage}}
\maketitle

\begin{abstract}
    Abstract here
\end{abstract}

\begin{keywords}
planet-star interactions
\end{keywords}

\section{Introduction}\label{s:intro}

\begin{itemize}
    \item Studying planetary obliquities (define) is important. Cassini States
        are key. More introduction.

    \item Resonance capture via separatrix crossing was first considered by
        \citep{henrard1982} for non-dissipative perturbations
        \citep[e.g.][]{su2020}. However, tidal friction is dissipative, so this
        formalism does not apply. We generalize this calculation and show that
        it reproduces results.
\end{itemize}

In Section XXX\dots

\section{Spin Evolution Equations and Cassini States: Review}\label{s:theory}

In this section, we first briefly lay out the spin dynamics of the planet,
introducing the Cassini State spin-orbit resonance \citep[for more details,
see][]{su2020}. We then introduce the weak friction theory of equilibrium tides
used in this work \citep{lai2012}. While many different tidal effects may
dominate in different planetary systems, our qualitative conclusions do not
depend on the specific form of the tidal dissipation, so we use the classic weak
friction theory for simplicity.

\subsection{Spin Dynamics in the Absence of Tides}\label{ss:theory_spin}

\subsubsection{Equations of Motion}

We consider a star of mass $M_\star$ hosting an inner oblate planet of mass $m$
and radius $R$ on a circular orbit with semi-major axis $a$ and an outer
perturber of mass $m_{\rm p}$ on a circular orbit with semi-major axis $a_{\rm
p}$. We assume that the two orbits are mildly misaligned with mutual inclination
$I$. Denote $\bm{S}$ the spin angular momentum and $\bm{L}$ the orbital angular
momentum of the planet, and $\bm{L}_{\rm p}$ the angular momentum of the
perturber. The corresponding unit vectors are $\uv{s} \equiv \bm{S} / S$,
$\uv{l} \equiv \bm{L} / L$, and $\uv{l}_{\rm p} \equiv \bm{L}_{\rm p} / L_{\rm
p}$. The spin axis $\uv{s}$ of the planet tends to precess around its orbital
(angular momentum) axis $\uv{l}$, driven by the gravitational torque from the
host star acting on the planet's rotational bulge. On the other hand, $\uv{l}$
and the disk axis $\uv{l}_{\rm p}$ precess around each other due to
gravitational interactions. We assume $S \ll L \ll L_{\rm p}$, so $\uv{l}_{\rm
p}$ and $\uv{l}$ are nearly constant. The equations of motion for $\uv{s}$ and
$\uv{l}$ in this limit are \citep{anderson2018teeter, su2020}
\begin{align}
    \rd{\uv{s}}{t}
        &= \omega_{\rm sl}\p{\uv{s} \cdot \uv{l}}\p{\uv{s} \times \uv{l}}
        \equiv \alpha\p{\uv{s} \cdot \uv{l}}\p{\uv{s} \times
        \uv{l}},\label{eq:dsdt1}\\
    \rd{\uv{l}}{t} &= \omega_{\rm lp}\p{\uv{l} \cdot \uv{l}_{\rm p}}\p{\uv{l}
        \times \uv{l}_{\rm p}} \equiv -g\p{\uv{l} \times \uv{l}_{\rm p}},
        \label{eq:dldt1}
\end{align}
where
\begin{align}
    \omega_{\rm sl} &\equiv \frac{3GJ_2 mR^2 M_\star}{2a^3 I\Omega_{\rm s}}
        = \frac{3k_q}{2k}\frac{M_\star}{m}\p{\frac{R}{a}}^3 \Omega_{\rm s},
            \label{eq:wsl}\\
    \omega_{\rm lp} &= \frac{3m_{\rm p}}{4M_\star}\p{\frac{a}{a_{\rm p}}}^3
        n.\label{eq:wlp}
\end{align}
In Eq.~\eqref{eq:wsl}, $\Omega_{\rm s}$ is the spin frequency of the inner
planet, $I = k mR^2$ (with $k$ a constant) is its moment of inertia and $J_2 =
k_{\rm q}\Omega_{\rm s}^2 (R^3/Gm)$ (with $k_{q}$ a constant) is its rotation-induced
(dimensionless) quadrupole moment [for a body with uniform density, $k=0.4,
k_{\rm q}=0.5$; for rocky planets, $k\simeq 0.2$ and $k_{\rm q}\simeq 0.17$
\citep[e.g.][]{lainey2016quantification} \textcolor{red}{? not sure}]. In other
studies, $3k_{\rm q} / 2 k$ is often notated as $k_2 / 2C$
\citep[e.g.][]{millholland_disk}. In Eq.~\eqref{eq:wlp}, $n \equiv
\sqrt{GM_\star/a^3}$ is the inner planet's orbital mean motion,  and we have
assumed $a_{\rm p}\gg a$ and included only the leading-order (quadrupole)
interaction between the inner planet and perturber. Following standard notation,
we define $\alpha = \omega_{\rm sl}$ and $g \equiv -\omega_{\rm 1p} \cos I$
\citep[e.g.][]{colombo1966}.

As in \citet{su2020}, we combine Eqs.~(\ref{eq:dsdt1}--\ref{eq:dldt1}) into a
single equation by transforming into a frame rotating about $\uv{l}_{\rm p}$
with frequency $g$. In this frame, $\uv{l}_{\rm p}$ and $\uv{l}$ are both fixed,
and $\uv{s}$ evolves as:
\begin{equation}
    \p{\rd{\uv{s}}{t}}_{\rm rot}
        = \alpha\p{\uv{s} \cdot \uv{l}}\p{\uv{s} \times \uv{l}}
            + g\p{\uv{s} \times \uv{l}_{\rm p}}. \label{eq:dsdt_rot}
\end{equation}
In this reference frame, we choose the coordinate system such that $\uv{z} =
\uv{l}$ and $\uv{l}_{\rm p}$ lies in the $\uv{x}$-$\uv{z}$ plane. We describe
$\uv{s}$ in spherical coordinates using the polar angle $\theta$, the planet's
obliquity, and $\phi$, the precessional phase of $\uv{s}$ about $\uv{l}$.

The equilibria of Eq.~\eqref{eq:dsdt_rot} are referred to as \emph{Cassini
States} \citep[CSs;][]{colombo1966, peale1969}. We follow the notation of
\citet{su2020}, where the parameter
\begin{equation}
    \eta \equiv -\frac{g}{\alpha},\label{eq:def_eta}
\end{equation}
is used. For a given value of $\eta$, there can be either two or four CSs, all
of which require $\uv{s}$ lie in the plane of $\uv{l}$ and $\uv{l}_{\rm p}$.
Following the standard nomenclature, $\uv{s}$ and
$\uv{l}_{\rm p}$ are on opposite sides of $\uv{l}$ for CSs 1, 3, and 4, and are
on the same side for CS2. We depart from the standard convention
and simply label the CSs in spherical coordinates: figure~\ref{fig:cs_locs}
shows the CS obliquities as a function of $\eta$. CS1 and CS4 do not exist when
$\eta > \eta_{\rm c}$, where
\begin{equation}
    \eta_{\rm c} \equiv \p{\sin^{2/3}\!I + \cos^{2/3}\!I}^{-3/2}.
        \label{eq:def_etac}
\end{equation}
\begin{figure}
    \centering
    \includegraphics[width=\columnwidth]{../initial/99_misc/2_cs_locs_phi.png}
    \caption{Cassini State obliquities $\theta$ as a function of $\eta \equiv
    -g/\alpha$ (Eq.~\ref{eq:def_eta}) for $I = 20^\circ$. The vertical dashed
    line denotes $\eta_{\rm c}$, where the number of Cassini States changes from
    four to just two (Eq.~\ref{eq:def_etac}). The y-axis labels on the right of
    the plot show the asymptotic obliquities for CS2 and CS3, $I$ and $180^\circ
    - I$ respectively. Note that $\theta$ does not follow the standard
    convention \citep[e.g.][]{colombo1966} and is simply the angle between
    $\uv{s}$ and $\uv{l}$.}\label{fig:cs_locs}
\end{figure}

The Hamiltonian corresponding to Eq.~\eqref{eq:dsdt_rot} is
\begin{align}
    H &= -\frac{\alpha}{2}\p{\uv{s} \cdot \uv{l}}^2
            - g\p{\uv{s} \cdot \uv{l}_{\rm d}}\nonumber\\
        &= -\frac{\alpha}{2} \cos^2\theta
            - g\p{\cos\theta \cos I - \sin I \sin\theta \cos \phi}.\label{eq:H}
\end{align}
Here, $\cos \theta$ and $\phi$ form a canonically conjugate pair of variables.
Figure~\ref{fig:1contours} shows the level curves of this Hamiltonian for $I =
20^\circ$, for which $\eta_{\rm c} \approx 0.574$ (Eq.~\ref{eq:def_etac}). When $\eta
< \eta_{\rm c}$, CS4 exists and is a saddle point. The two trajectories
originating and ending at CS4 are the only two infinite-period orbits in the
phase space. Together, these two critical trajectories are referred to as the
\emph{separatrix} and divide phase space into three zones. Trajectories in zone
II librate about CS2 while those in zones I and III circulate. On the other
hand, when $\eta > \eta_{\rm c}$, all trajectories circulate. When the
separatrix exists, we divide it into two curves: $\mathcal{C}_+$, the boundary
between zones I and II, and $\mathcal{C}_-$, the boundary between zones II and
III\@.
\begin{figure}
    \centering
    \includegraphics[width=\columnwidth]{../initial/0_eta/1contours20.png}
    \caption{Level curves of the Cassini State Hamiltonian (Eq.~\ref{eq:H}) for
    $I = 20^\circ$, for which $\eta_{\rm c} \approx 0.57$
    (Eq.~\ref{eq:def_etac}). For $\eta < \eta_{\rm c}$, there are four Cassini
    States (labeled), while for $\eta > \eta_{\rm c}$ there are only two. In the
    former case, the existence of a \emph{separatrix} (solid black lines)
    separates phase space into three numbered zones (I/II/III, labeled). We
    denote the upper and lower legs of the separatrix by $\mathcal{C}_{\pm}$
    respectively, as shown in the upper two panels. }\label{fig:1contours}
\end{figure}

\section{Spin Evolution with Alignment Torque}\label{s:toy_model}

In this section, we consider a simplified model of equilibrium tides that
isolates the important new phenomenon presented in this paper. We assume that
the spin magnitude of the planet is constant, so $\alpha$ and $g$ are both
fixed, while the spin orientation $\uv{s}$ experiences an alignment torque
towards $\uv{l}$ on the alignment timescale $t_{\rm s}$:
\begin{equation}
    \p{\rd{\uv{s}}{t}}_{\rm tide}
        = \frac{1}{t_{\rm s}} \uv{s} \times \p{\uv{l} \times \uv{s}}.
        \label{eq:dsdt_tide_toy}
\end{equation}
The full equations of motion for $\uv{s}$ in the coordinates $\theta$ and $\phi$
can be written:
\begin{align}
    \rd{\theta}{t} &= -g\sin I \sin \phi - \frac{1}{t_{\rm s}} \sin \theta,
        \label{eq:dqdt_toy}\\
    \rd{\phi}{t} &= -\alpha \cos\theta
        - g\p{\cos I + \sin I \cot \theta \cos \phi}.\label{eq:dfdt_toy}
\end{align}

\subsection{Shifted Cassini States and Linear Stability
Analysis}\label{ss:tidal_equils}

If the alignment torque is weak ($\abs{gt_{\rm s}} \gg 1$), then the fixed points of
Eqs.~(\ref{eq:dqdt_toy}--\ref{eq:dfdt_toy}) are just slightly shifted CSs. This
shift can be calculated cleanly to leading order: all of the CS obliquities
$\theta_{\rm cs}$ are unchanged while the azimuthal angle $\phi_{\rm cs}$ for
each CS now satisfies
\begin{equation}
    \sin \phi_{\rm cs} = \frac{\sin\theta_{\rm cs}}{\sin I \abs{g}t_{\rm s}}.
        \label{eq:mcs_shift}
\end{equation}
We can see that if $t_{\rm s} > t_{\rm s, c}$, where
\begin{align}
    t_{\rm s, c} &= \frac{1}{\abs{g}\sin I},\label{eq:mcs_shift_crit}
\end{align}
then Eq.~\eqref{eq:mcs_shift} will always have solutions for $\phi_{\rm cs}$,
and the alignment torque never changes the number of fixed points of the system.
If $t_{\rm s}$ is decreased below $t_{\rm s, c}$, CS2 and CS4 cease to be
fixed points if $\eta \ll 1$ \citep[as first noted in][]{fabrycky_otides}, as
$\theta_{\rm cs} \approx 90^\circ$ for these (see Fig.~\ref{fig:cs_locs}), while
the other CSs have small $\sin \theta_{\rm cs}$ and are only slightly perturbed.
Figure~\ref{fig:mcs} shows the obliquity and azimuthal angles for each of the
CSs in the $\eta \ll 1$ case, where it can be seen that CS2 and CS4 collide and
annihilate. The phase shift $\phi_{\rm cs}$ for CS2 and CS4 for $t_{\rm s} >
t_{\rm s, c}$ can be predicted to good accuracy using Eq.~\eqref{eq:mcs_shift}
using $\theta_{\rm cs} \approx \pi/2 - \eta \cos (I) \approx 79^\circ$
\citep{su2020} for both, shown as the dashed lines in the bottom panel of
Fig.~\ref{fig:mcs}. For the remainder of this section, we will consider the case
where $t_{\rm s} \gg t_{\rm s, c}$ and the CSs only differ slightly from their
unperturbed locations.
\begin{figure}
    \centering
    \includegraphics[width=\columnwidth]{../initial/99_misc/0_stab.png}
    \caption{Modified CS obliquities (top) and azimuthal angles (bottom) for $I
    = 20^\circ$ and $\eta = 0.2$, where the CS1 and CS3 obliquities have been
    offset (labeled in top legend) to improve clarity of the plot. In both
    panels, the solid lines give the result when applying a numerical root
    finding algorithm to the full equations of motion,
    Eqs.~(\ref{eq:dqdt_toy}--\ref{eq:dfdt_toy}), while the dotted lines in the
    bottom panel give the CS2 and CS4 azimuthal angles according to
    Eq.~\eqref{eq:mcs_shift}. At $\abs{gt_{\rm s} \sin I} = 1$, CS2 and CS4
    collide and annihilate (Eq.~\ref{eq:mcs_shift_crit}).}\label{fig:mcs}
\end{figure}

We next seek to characterize the stability of small perturbations about each of
the CSs in the presence of the weak tidal alignment torque. We can linearize
Eqs.~(\ref{eq:dqdt_toy}--\ref{eq:dfdt_toy}) about a shifted CS, yelding
\begin{align}
    \rd{}{t}\begin{bmatrix}
        \theta\\ \phi
    \end{bmatrix} &= \begin{bmatrix}
        -\frac{\cos \theta}{t_{\rm s}} &
        -g\sin I \cos \phi \\
        \alpha \sin \theta + g\frac{\sin I \cos \phi}{\sin^2\theta} &
        0
    \end{bmatrix}_{\rm cs}\begin{bmatrix}
        \Delta \theta \\ \Delta \phi
    \end{bmatrix},\label{eq:dsdt_hessian}
\end{align}
where the cs subscript indicates evaluating at a CS, $\Delta \theta = \theta -
\theta_{\rm cs}$, and $\Delta \phi = \phi - \phi_{\rm cs}$. The eigenvalues
$\lambda$ of Eq.~\eqref{eq:dsdt_hessian} satisfy the equation
\begin{equation}
    0 = \p{\lambda + \frac{\cos \theta_{\rm cs}}{t_{\rm s}}}\lambda
        - \lambda_0^2,\label{eq:lambda_orig}
\end{equation}
where
\begin{equation}
    \lambda_0^2 \equiv \p{\alpha
        \sin \theta_{\rm cs} + g\sin I \csc^2\theta_{\rm cs}\cos \phi_{\rm cs}}
            \p{- g \sin I \cos \phi_{\rm cs}}\label{eq:def_l0_sq}.
\end{equation}
When $t_{\rm s}$ is large, we can simplify Eq.~\eqref{eq:lambda_orig} to
\begin{equation}
    \lambda \approx -\frac{\cos \theta_{\rm cs}}{t_{\rm s}}
        \pm \sqrt{\lambda_0^2}. \label{eq:lambda_approx}
\end{equation}
The stability depends only on the real part of $\lambda$ in
Eq.~\eqref{eq:lambda_approx}. $\lambda_0^2$ is a generalization of Eq.~(A4)
in, \citet{su2020} and generally has the same behavior: it is negative for CSs
1--3 and positive for CS4, as shown in Fig.~\ref{fig:lambda_full}. Thus, CS4 is
always unstable, as there will always be at least one positive solution for
$\lambda$, and the stability of CSs 1--3 are solely determined by the sign of
$\cos \theta_{\rm cs}$. Evaluating at each of the CSs (see
Fig.~\ref{fig:cs_locs}), we conclude that CS1 and CS2 are stable and attracting
while CS3, stable in the absence of the torque, becomes unstable. These
calculations justify results long used in CS literature
\citep[e.g.][]{ward1975tidal, fabrycky_otides}.
\begin{figure}
    \centering
    \includegraphics[width=\columnwidth]{../initial/99_misc/2_lambdas_full.png}
    \caption{$\lambda_0^2$ (Eq.~\ref{eq:def_l0_sq}) as a function of $\eta$ for
    the four CSs. The solid lines give $\lambda_0^2$ for the unperturbed
    $\phi_{\rm cs}$, and the dashed lines give the values for $\phi_{\rm cs}$
    shifted by $60^\circ$ ($\phi_{\rm cs} = 120^\circ$ for CS2 and $\phi_{\rm
    cs} = 60^\circ$ for CSs 1, 3, and 4).
    }\label{fig:lambda_full}
\end{figure}

\subsection{Spin Obliquity Evolution Driven by Alignment
Torque}\label{ss:toy_outcomes}

With the above results, we are equipped to ask questions about the dynamics of
Eqs.~(\ref{eq:dqdt_toy}--\ref{eq:dfdt_toy}): what is the long term behavior for
a general initial $\uv{s}$? If $\eta > \eta_{\rm c}$, then the only possible
final spin state is CS2, and all initial conditions will evolve asympotically
towards it. As such, we consider only $\eta < \eta_{\rm c}$, where an initial
condition can asymptotically evolve towards either CS1 or CS2. We numerically
integrate Eqs.~(\ref{eq:dqdt_toy}--\ref{eq:dfdt_toy}) for many random initial
conditions uniformly distributed in $\p{\cos \theta, \phi}$ with $\abs{g}t_{\rm
s} = 10^{-3}$ and record the nearest CS for each integration after $10t_{\rm
s}$. In Fig.~\ref{fig:toy_phop}, we show the results of this procedure for $\eta
= 0.2$, and $I = 20^\circ$. It is clear that initial conditions in zone I evolve
into CS1, those in zone II evolve into CS2, while those in zone III have a
probabilistic outcome. We aim to understand each of these in turn:
\begin{figure}
    \centering
    \includegraphics[width=\columnwidth]{../initial/0_eta/3stats3_5_0_2.png}
    \caption{Plot illustrating the asymptotic behavior of initial conditions for
    $\eta = 0.2$ and $I = 20^\circ$. Each dot represents an initial spin
    orientation, and the coloring of the dot indicates which Cassini State
    (legend) the system asymptotes towards.}\label{fig:toy_phop}
\end{figure}

For initial conditions in zone I, the spin circulates, and $\dot{\theta}$ is
negative everywhere during the cycle. Thus, for initial conditions in zone I,
$\theta$ decreases until the trajectory has converged to CS1. This is
intuitively reasonable, as CS1 is stable.

For initial conditions in zone II, our stability calculation in
Section~\ref{ss:tidal_equils} shows that initial conditions sufficiently near CS2
will converge to CS2. In fact, this result can be extended to all initial
conditions inside the separatrix; see Appendix~\ref{app:cs_stab2}.

For initial conditions in zone III, since there are no stable CSs in zone III,
the system must evolve through the separatrix to reach one of either CS1 or CS2.
The outcome of the separatrix encounter is effectively probabilistic and
determines the final CS\@. Intuitively, this can be understood as probabilistic
resonance capture: since $\eta \ll \eta_{\rm c}$, $\alpha \gg -g$ (the
spin-orbit precession rate, Eq.~\ref{eq:dsdt1}, and the orbit precession induced
by the perturber, Eq.~\ref{eq:dldt1}, respectively), but $\alpha \cos \theta$
can become commensurate with $-g$ if $\cos \theta \sim \eta$. This is achieved
as $\theta$ evolves from an initially retrograde obliquity through $90^\circ$
towards $0^\circ$ under the influence of the dissipative term in
Eq.~\eqref{eq:dqdt_toy}.

While similar in behavior to previous studies of probabilistic resonance capture
\citep{henrard1982, su2020}, the underlying mechanism is different: in these
previous studies, the phase space structure itself evolves and causes systems to
transition among phase space zones, while in the present scenario, a
non-Hamiltonian, dissipative perturbation causes systems to transition among
unchanging phase space zones.

\subsubsection{Analytical Calculation of Resonance Capture Probability}

The specific probabilities of the two outcomes upon separatrix encounter, a zone
III-II or a zone III-I transition, can be calculated analytically.
Figure~\ref{fig:toy_hop_manifolds} shows how the perturbative alignment torque
generates probabilistic outcomes upon separatrix encounter. We present the
interpretation of Figure~\ref{fig:toy_hop_manifolds} and the calculation of the
resonance capture probability by first giving a qualitative description of the
intuition behind the method, then presenting a calculation in good agreement with
numerical results.

We first describe the origin of the boundaries between regions of phase space
that shown in Fig.~\ref{fig:toy_hop_manifolds}. They are calculated numerically
by integrating a point infinitesimally close to CS4 forward and backward in
time. In the absence of the alignment torque, these trajectories would evolve
along the separatrix, but in the presence of the alignment torque they are
perturbed slightly and cease to overlap. It can be seen in
Fig.~\ref{fig:toy_hop_manifolds} that this splitting opens a path from zone III
into both zones I and II\@.

To understand this process quantitatively, as well as associate probabilities to
the two possible outcomes, it is important to be more quantitative. The correct
approach is to consider the evolution of the value of the \emph{unperturbed}
Hamiltonian (Eq.~\ref{eq:H}) as the spin evolves with the alignment torque. A
point in zone III evolves such that $H$ is increasing until $H \approx H_{\rm
sep}$ where $H_{\rm sep}$ is the value of $H$ along the separatrix, given by
\begin{align}
    H_{\rm sep} &\equiv H\p{\cos \theta_{\rm 4}, \phi_{\rm 4}},\nonumber\\
        &\approx g\sin I + \frac{g^2}{2\alpha}\cos^2 I +
            \mathcal{O}\p{\eta^2},\label{eq:def_Hsep}
\end{align}
where $\theta_4$ and $\phi_4$ are the coordinates of CS4. As the system evolves
closer to the separatrix, the change in $H$ over each circulation cycle can be
approximated by $\Delta H_-$, the change in $H$ along $\mathcal{C}_-$ (see
Fig.~\ref{fig:1contours}). In general, we can compute $\Delta H_{\pm}$ the
change over both legs of the separatrix with
\begin{equation}
    \Delta H_{\pm} \equiv \oint\limits_{\mathcal{C}_{\pm}}
        \rd{H}{t}\;\mathrm{d}t.\label{eq:def_dHpm}
\end{equation}
This can be simplified by using:
\begin{align}
    \rd{H}{t} &=
            \pd{H}{(\cos \theta)}\rd{(\cos \theta)}{t}
            + \pd{H}{\phi}\rd{\phi}{t},\nonumber\\
        &= \p{\rd{(\cos\theta)}{t}}_{\rm tide} \rd{\phi}{t},\\
    \Delta H_{\pm} &= \mp\frac{1}{t_{\rm s}}
        \int\limits_0^{2\pi} \sin^2\theta\;\mathrm{d}\phi.
\end{align}
Thus, if we evaluate $H$ every time that a trajectory crosses $\phi = 0$, we see
that it will initially be negative and increase for each circulation cycle until
the system encounters the separatrix. At the beginning of the
separatrix-crossing orbit, the initial value of $H$, denoted $H_{\rm i}$,
satisfies $H_{\rm sep} - \Delta H_- \leq H_{\rm i} \leq H_{\rm sep}$: if it is
less than the lower bound, it will not cross the separatrix on this orbit, and
if it is greater than the upper bound then it is already inside the separatrix.
The two endpoints of this range are denoted in Fig.~\ref{fig:6equils} by the
black dot and cross at the left and right edges of the plot.

During the separatrix-crossing orbit, the trajectory first evolves
approximately along $\mathcal{C}_-$ then along $\mathcal{C}_+$, after which two
possible outcomes can occur:
\begin{itemize}
    \item If the final value of $H$ at the end of this separatrix traversal,
        denoted $H_{\rm f}$, satisfies $H_{\rm f} > H_{\rm sep}$, then the
        trajectory enters the separatrix, following the red shaded region in
        Fig.~\ref{fig:6equils}, and executes a zone III $\to$ II transition.

    \item If $H_{\rm f} < H_{\rm sep}$, then the trajectory exits the separatrix
        above CS4, following the yellow shaded region in Fig.~\ref{fig:6equils}
        and executes a zone III $\to$ I transition.
\end{itemize}
Since $H_{\rm f} = H_{\rm i} + \Delta H_+ + \Delta H_-$, we find that if $H_{\rm
i}$ is in the interval $\s{H_{\rm sep} - \Delta H_-, H_{\rm sep} - \Delta H_{-}
- \Delta H_+}$, then the system executes a III $\to$ I transition, and if it
is in the interval $\s{H_{\rm sep} - \Delta H_- - \Delta H_+, H_{\rm sep}}$,
then the system executes a III $\to$ II transition. The values of $\cos
\theta$ for which $H$ is equal to $H_{\rm sep} - \Delta H_-$, $H_{\rm sep} - \Delta
H_- - \Delta H_+$, and $H_{\rm sep}$ (CS4) for $\phi = 0$ are shown in
Fig.~\ref{fig:6equils} as the black cross, star, and dot at $\phi = 0$
respectively. Finally, if the alignment torque is weak, then $\Delta H_- \sim
\mathcal{O}\p{t_{\rm s}^{-1}}$ is small compared to the characteristic values of
$H$, and $H_{\rm i}$ can be effectively considered as uniformly distributed over
$\s{H_{\rm sep} - \Delta H_-, H_{\rm sep}}$. As such, we obtain that
\begin{equation}
    P_{\rm III \to II} = \frac{\Delta H_- + \Delta H_+}{\Delta H_-}.
        \label{eq:def_P32_toy}
\end{equation}

\begin{figure}
    \centering
    \includegraphics[width=\columnwidth]{../initial/0_eta/6manifolds0_20.png}
    \caption{Plot illustrating the probabilistic origin of separatrix capture
    for $\eta = 0.2$, $I = 20^\circ$, and $\abs{gt_{\rm s}} = 10^{-3}$. Orange
    regions converge to CS1, and green to CS2, while CS4 is labeled with the
    purple dots. The boundaries separating the CS1 and CS2-approaching regions
    consist of the critical trajectories (labeled in the legend) that, when
    evolved either forwards or backwards in time (superscripts in the legend
    labels), asymptotically approach CS4 where either $\phi = 0^\circ$ or $\phi
    = 360^\circ$ (subscripts in the legend labels). Within zone III, the regions
    of phase space reaching CS1 and CS2 both become very thin (shown
    qualitatively as the light green lines of decreasing width), reflecting the
    fact that the outcome for a particular initial condition can be approximated
    as probabilistic sufficiently far from the separatrix. Labeled in the blue
    and black dots at the left edge of the plot are the two backwards-in-time
    critical trajectories intersect $\phi = 0$, and are where $H$ is equal to
    $H_{\rm sep} - \Delta H_- - \Delta H_+$ and $H_{\rm sep} - \Delta H_-$
    respectively (Eqs.~\ref{eq:def_Hsep} and~\ref{eq:def_dHpm}).
    }\label{fig:toy_hop_manifolds}
\end{figure}

To actually evaluate Eq.~\eqref{eq:def_P32_toy}, we use the parameterization for
the separatrix \citep{su2020} for $\eta \ll 1$:
\begin{equation}
    \p{\cos \theta}_{\mathcal{C}_{\pm}} \approx
        \eta \cos I \pm \sqrt{2\eta\sin I\p{1 - \cos \phi}}.
        \label{eq:sep_theta}
\end{equation}
It can then be shown that
\begin{align}
    \Delta H_- &\approx \frac{2\pi}{t_{\rm s}}\p{1
        - 2\eta \sin I} + \mathcal{O}(\eta^{3/2}),\\
    \Delta H_+ + \Delta H_- &\approx
        \frac{32 \eta^{3/2}\cos I \sqrt{\sin I}}{t_{\rm s}}
            + \mathcal{O}\p{\eta^{5/2}},\\
    P_{\rm III \to II} &\approx
        \frac{16 \eta^{3/2} \cos I \sqrt{\sin I}}{\pi
            \p{1  - 2\eta \sin I}}.\label{eq:P32_toy}
\end{align}

To directly compare Eq.~\eqref{eq:P32_toy} with numerical simulation, we
perform integrations of Eqs.~(\ref{eq:dqdt_toy}--\ref{eq:dfdt_toy}) while
restricting our attention to only outcomes of initial conditions in zone III\@.
In Fig.~\ref{fig:1hist_toy}, we display Eq.~\eqref{eq:P32_toy} alongside the
computed $P_{\rm III \to II}$ using $1000$ initial conditions in zone III for
each of $60$ values of $\eta$. Excellent agreement is observed.
\begin{figure}
    \centering
    \includegraphics[width=\columnwidth]{../initial/3_toy3/1hist_toy.png}
    \caption{Plot of $P_{\rm III \to II}$ as a function of $\eta$ for the
    constant alignment torque model considered in Section~\ref{s:toy_model}. For
    each $\eta$, $1000$ initial $\theta_0$ in zone III are evolved until just
    after separatrix encounter, where the outcome of the encounter is recorded.
    Shown in red is Eq.~\eqref{eq:P32_toy}.}\label{fig:1hist_toy}
\end{figure}

Finally, we remark that the calculation above is just a descriptive application
of \emph{Melnikov's Method} \citep{g_and_h}. Melnikov's Method is a general
calculation that gives the degree of splitting of a ``homoclinic orbit'' (here,
the separatrix) induced by a small, possibly time-dependent, perturbation. The
explicit connection between the evolution of the unperturbed Hamiltonian and
distances between curves in phase space (as seen in Fig.~\ref{fig:6equils})
is provided by Melnikov's Method.

\section{Spin Evolution with Weak Tidal Friction}\label{s:full_tide_prob}

\subsection{Tidal Model: Equilibrium Tides}\label{ss:weaktide}

To model the dissipative effect of times, we use the weak friction theory of
equilibrium tides \citep{lai2012}. In this model, tides cause both $\uv{s}$ and
$\Omega_{\rm s}$ to evolve on the characteristic tidal timescale $t_{\rm a}$:
\begin{align}
    \p{\rd{\uv{s}}{t}}_{\rm tide} &= \frac{1}{t_{\rm a}}
                \s{\frac{2n}{\Omega_{\rm s}} - \p{\uv{s} \cdot \uv{l}}}
                    \uv{s} \times \p{\uv{l} \times \uv{s}}\label{eq:dsdt_tide},\\
    \frac{1}{\Omega_{\rm s}}\p{\rd{\Omega_{\rm s}}{t}}_{\rm tide}
        &= \frac{1}{t_{\rm a}} \s{\frac{2n}{\Omega_{\rm s}}\p{\uv{s} \cdot
            \uv{l}} - 1 - \p{\uv{s} \cdot \uv{l}}^2},\label{eq:dWsdt_tide}
\end{align}
where $t_{\rm a}$ is given by
\begin{equation}
    \frac{1}{t_{\rm a}} \equiv \frac{L}{2S} \frac{\Omega_{\rm
        s}}{2n}\frac{3k_2}{Q}\p{\frac{M_\star}{m}}\p{\frac{R}{a}}^5 n.
\end{equation}
Here, $L = ma^2n$ and $S = kmR^2 \Omega_{\rm s}$ are the orbital and spin
angular momenta of the inner planet, respectively. We neglect orbital evolution
in this section since the time scale is longer than $t_{\rm a}$ by a factor of
$\sim L / S \gg 1$, and so $t_{\rm a}$ is a constant. We will continue mostly
consider the case where $\abs{gt_{\rm a}} \gg 1$. The full equations of motion
including weak tidal friction are
\begin{align}
    \rd{\theta}{t} &= g\sin I \sin \phi -
        \frac{1}{t_{\rm a}}\sin \theta\p{\frac{2n}{\Omega_{\rm s}} - \cos \theta}
            ,\label{eq:ds_fullq}\\
    \rd{\phi}{t} &= -\alpha\cos\theta
        - g\p{\cos I + \sin I \cot \theta \cos \phi}\label{eq:ds_fullphi},\\
    \frac{1}{\Omega_{\rm s}}\rd{\Omega_{\rm s}}{t}
        &= \frac{1}{t_{\rm a}} \s{\frac{2n}{\Omega_{\rm s}} \cos \theta
            - \p{1 + \cos^2\theta}}\label{eq:ds_fulls}.
\end{align}
The conditions for which the two tidal terms vanish can be respectively
expressed as:
\begin{align}
    \dot{\theta}_{\rm tide} = 0 &: \quad \frac{2n}{\Omega_{\rm s}} = \cos \theta,
        \label{eq:weaktide_dqzero}\\
    \dot{\Omega}_{\rm s} = 0 &: \quad \frac{n}{\Omega_{\rm s}}
        = \frac{1 + \cos^2\theta}{2\cos \theta}.\label{eq:weaktide_dWszero}
\end{align}

To understand the long-term behaviors of the system, we first consider its
behavior near a CS\@. Specifically, we wish to understand whether initial
conditions near a CS stay near the CS as the evolution of $\Omega_{\rm s}$
causes the CSs to evolve. We first note that the evolution of
$\Omega_{\rm s}$ does not drive spins towards or away from CSs: as long as it
evolves sufficiently slowly \citep[adiabatically,][]{su2020}, conservation of
phase space area ensures that trajectories will remain at a roughly fixed
distance to stable equilibria of the system. Thus, Eq.~\eqref{eq:dsdt_tide}
alone determines whether a point evolves towards or away from a nearby CS as
$\Omega_{\rm s}$ evolves. Then, evaluating Eq.~\eqref{eq:lambda_approx} with
$t_{\rm s} = t_{\rm a} / \p{2n / \Omega_{\rm s} - \cos \theta}$ (compare
Eqs.~\ref{eq:dsdt_tide_toy} and~Eq.~\ref{eq:dsdt_tide}), we see that CS2 is
still always stable, while CS1 is becomes unstable for $\Omega_{\rm s} > 2n$.

Using this result, we can then identify the long-term equilibria of the system
(i.e.\ when including the evolution of $\Omega_{\rm s}$), as the long-term
equilibria of the system must both satisfy $\dot{\Omega}_{\rm s} = 0$ and be a
stable CS\@. Figure~\ref{fig:6equils} describes
Eqs.~(\ref{eq:dsdt_tide}--\ref{eq:dWsdt_tide}) qualitatively in the coordinates
$(\Omega_{\rm s}, \theta)$, along with the locations of CS1 and CS2. The two
circled points in Fig.~\ref{fig:6equils} satisfy the criteria to be long-term
equilibria, and we call them \emph{tidal Cassini Equilibria} (tCE). We number
tCE1 and tCE2 the tCE that are in CSs 1 and 2 respectively. The obliquities of
the tCE depend on the system architecture, which can be quantified using the
parameter
\begin{equation}
    \eta_{\rm sync} \equiv \big(\eta\big)_{\Omega_{\rm s} = n},
        \label{eq:def_etasync}
\end{equation}
The tCE obliquity as a function of $\eta_{\rm sync}$ are shown for $I =
20^\circ$ in the top panel of Fig.~\ref{fig:probs20} and for $I = 5^\circ$ in
the top panel of Fig.~\ref{fig:probs5}.
\begin{figure}
    \centering
    \includegraphics[width=\columnwidth]{../initial/1_weaktide/6equils0_20.png}
    \caption{Schematic depiction of the effect of tidal friction on the planet's
    spin for $I = 20^\circ$ and $\eta_{\rm sync} = 0.2$. The black and blue lines
    denote where the tidal $\dot{\Omega}_{\rm s}$ and $\dot{\theta}$ change
    signs (Eqs.~\ref{eq:weaktide_dqzero}--\ref{eq:weaktide_dWszero}). The orange
    and green lines give the CS1 and CS2 obliquities respectively, which are the
    two CSs that are stable under the effect of tidal dissipation. Note that
    when $\dot{\theta}_{\rm tide} > 0$, CS1 becomes unstable, denoted by the
    dashed orange line. The points that are both CSs and satisfy
    $\dot{\Omega}_{\rm s} = 0$ are the tidal Cassini Equilibria (tCE), which are
    circled and labeled. }\label{fig:6equils}
\end{figure}

There are two important conditions that change the existence and stability of
the tCE\@. First, if $\eta_{\rm sync} > \eta_{\rm c}$, then tCE1 will not
exist\footnote{Strictly speaking, $\eta_{\rm sync}$ can be slightly smaller than
$\eta_{\rm c}$, as the planet's spin is slightly subsynchronous at tCE1 when
$\eta \approx \eta_{\rm c}$, see Fig.~\ref{fig:6equils}.}. Secondly, tCE2 may
not be stable if the tidal phase shift is too large. Applying the results of
Section~\ref{ss:tidal_equils}, we find that tCE2 is stable as long as $t_{\rm a}
\geq t_{\rm a, c}$ where
\begin{align}
    t_{\rm a, c} &\equiv \frac{\sin \theta_{\rm tCE2}}{\abs{g} \sin I}.
        \label{eq:def_ta_crit}
\end{align}
The tCE2 subscripts denote evaluation at tCE2. This can be further simplified
using that Eq.~\eqref{eq:weaktide_dWszero} and $\cos \theta \approx
\eta \cos I$ are satisfied at tCE2, giving:
\begin{equation}
    t_{\rm a,c} \approx \frac{1}{\abs{g}\sin I}\sqrt{\frac{2}{\eta_{\rm sync}
        \cos I}}.
\end{equation}

\subsection{Spin Obliquity Evolution as a Function of Initial Spin Orientation}

With this result, we can now consider the final fate of the inner planet's spin.
We assume that the planet is initially rotating supersynchronously and adopt the
fiducial initial spin frequency $\Omega_{\rm s, i} = 10n$. The final results are
not sensitive on the specific initial spin as long as $\Omega_{\rm s, i} \gg n$.
Then, for a given initial $\theta_0$ and $\phi_0$, the final outcome (either tCE1 and
tCE2) of the system can be calculated by direct integration of
Eqs.~(\ref{eq:dsdt_rot},~\ref{eq:dsdt_tide}--\ref{eq:dWsdt_tide}). In
Fig.~\ref{fig:Hhists_0_06}, we show the final outcome for many randomly chosen
$\theta_0$ and $\phi_0$ for $\eta_{\rm sync} = 0.06$ and $I = 20^\circ$. We see
that tCE1 is generally reached for spins initially in zone I, tCE2 is generally
reached for spins initially in zone II, and a probabilistic outcome is observed
for spins initially in zone III, very similar to the results found for the
alignment torque in Section~\ref{s:toy_model}. Figures~\ref{fig:Hhists_0_20}
and~\ref{fig:Hhists_0_70} show the same results but for $\eta_{\rm sync} = 0.2$
and $\eta_{\rm sync} = 0.5$. As $\eta_{\rm sync}$ is increased, more initial
conditions reach tCE2. This is both because there are more systems initially in
zone II and because more systems initially in zone III execute a III $\to$ II
transition upon separatrix encounter. Note also that in
Fig.~\ref{fig:Hhists_0_70}, even initial conditions in zone I are able to reach
tCE2; we comment on the origin of this behavior in the next section.
\begin{figure}
    \centering
    \includegraphics[width=\columnwidth]{../initial/1_weaktide/5Hhists0_06_20.png}
    \caption{\emph{Left:} Each dot indicates which tCE a given initial condition
    $(\theta_{\rm i}, \phi_{\rm i})$ evolve towards (labeled in legend), for
    $\eta_{\rm sync} = 0.06$ and $I = 20^\circ$. The
    separatrix is shown as the black line. Note that points above the
    separatrix evolve towards tCE1, points inside the separatrix evolve towards
    tCE2, and points below the separatrix have a probabilistic outcome.
    \emph{Right:} Histogram of which tCE a given initial obliquity $\theta_{\rm
    i}$ evolves towards.}\label{fig:Hhists_0_06}
\end{figure}
\begin{figure}
    \centering
    \includegraphics[width=\columnwidth]{../initial/1_weaktide/5Hhists0_20_20.png}
    \caption{Same as Fig.~\ref{fig:Hhists_0_06} but for $\eta_{\rm sync} =
    0.2$.}\label{fig:Hhists_0_20}
\end{figure}
\begin{figure}
    \centering
    \includegraphics[width=\columnwidth]{../initial/1_weaktide/5Hhists0_50_20.png}
    \caption{Same as Fig.~\ref{fig:Hhists_0_06} but for $\eta_{\rm sync} =
    0.7$. Note that even points above the separatrix can evolve towards tCE2
    here.}\label{fig:Hhists_0_70}
\end{figure}

\subsubsection{Analytical Calculation of Resonance Capture Probability
}\label{ss:phop_weaktide}

Even when including the evolution of $\Omega_{\rm s}$, and therefore the
spin-orbit precession frequency $\alpha$, the probabilities of the III $\to$ I
and III $\to$ II transitions upon separatrix encounter can be computed. The
calculation is more involved than that presented in
Section~\ref{ss:toy_outcomes}, and also incorporates the seminal resonance
capture theory of \citet{henrard1982}. In this section, we give an
overview of the approach.

In Section~\ref{ss:toy_outcomes}, we found that keeping track of the value of
$H$, the value of the
unperturbed Hamiltonian, allowed us to calculate the probabilities of the
various outcomes of separatrix encounter. Specifically, the outcome upon separatrix
encounter is determined by the value of $H$ at the start of the
separatrix-crossing orbit relative to $H_{\rm sep}$, the value of $H$ along the
separatrix. When the spin is also evolving, $H_{\rm sep}$ is also
changing during the separatrix-crossing orbit, and the discussion in
Section~\ref{ss:toy_outcomes} must be generalized to account for this. Instead
of focusing on the evolution of $H$ along a trajectory, we instead follow the
evolution of
\begin{equation}
    K \equiv H - H_{\rm sep}.
\end{equation}
Note that $K > 0$ inside the separatrix, and $K < 0$ outside. Then, the outcome
of the separatrix-crossing orbit is largely the same as discussed in
Section~\ref{ss:toy_outcomes}. First, we must compute the change in $K$ along
the legs of the separatrix. We define $\Delta K_{\pm}$ by generalizing
Eq.~\eqref{eq:def_dHpm} very naturally:
\begin{align}
    \Delta K_{\pm} &= \oint_{\mathcal{C}_{\pm}} \rd{H}{t}
        - \rd{H_{\rm sep}}{t}\;\mathrm{d}t.\label{eq:def_dK_weaktide}
\end{align}
Here, however, note that $\mathcal{C}_{\pm}$ depends on the value of
$\Omega_{\rm s}$ upon separatrix encounter. Since there is no closed form
solution for $\Omega_{\rm s}(t)$, the probabilities of the various outcomes can
only be determined as a function of the system properties at separatrix
encounter, and not as a simple function of the initial conditions.

Then, if we call $K_i$ the value of $K$ at the start ($\phi = 0$) of the
separatrix-crossing orbit, $K_i > -\Delta K_+ - \Delta K_-$ gives a III
$\to$ II transition and eventual evolution towards tCE2 while $-\Delta K_- < K_i
< -\Delta K_- - \Delta K_+$ gives a III $\to$ I transition and ultimate
evolution towards tCE1. Thus, we find that the probability of a III $\to$ II
transition is given by
\begin{equation}
    P_{\rm III \to II} = \frac{\Delta K_+ + \Delta K_-}{\Delta
        K_-}.\label{eq:def_pc_weaktide}
\end{equation}
Since $\Delta K_{\pm}$ are evaluated at resonance encounter, and
$\Omega_{\rm s}$ is evolving, there is no way to express $\Delta K_{\pm}$ as a
closed form of initial conditions. In fact, since many resonance encounters
occur when $\eta$ is substantial, even an approximate calculation using
Eq.~\eqref{eq:sep_theta} is inaccurate, and we instead calculate $\Delta
K_{\pm}$ along the two legs of the separatrix obtained numerically via a root
finding algorithm. Note that Eqs.~(\ref{eq:def_dK_weaktide},~\ref{eq:def_pc_weaktide}) are
equivalent to the separatrix capture result of \citet{henrard1982} when $\cos
\theta$ is not evolving \citep{henrard1987}. Here, we have argued that this
classic calculation can be unified with the calculation given in
Section~\ref{ss:toy_outcomes} to give an accurate prediction of separatrix
encounter outcome probabilities in the presence of both a dissipative
perturbation and a parametric variation of the Hamiltonian.

To validate the accuracy of Eq.~\eqref{eq:def_pc_weaktide}, we can compare with
the numerical integration of
Eqs.~(\ref{eq:dsdt_rot},~\ref{eq:dsdt_tide}--\ref{eq:dWsdt_tide}) for many
initial conditions in zone III while evaluating $P_{\rm III \to II}$ (and thus
also obtaining $P_{\rm III \to I} = 1 - P_{\rm III \to II}$) for each simulation
at the moment it encounters the separatrix. If the theory is correct, the total
numbers of systems converging to each of tCE1 and tCE2 should be equal to those
predicted by the sums of the calculated probabilities. In
Fig.~\ref{fig:pc_fits_0_06}, we show the agreement of this semi-analytic
procedure with the numerical results displayed earlier in the right panel of
Fig.~\ref{fig:Hhists_0_06}. Good agreement is observed.
Figs.~\ref{fig:pc_fits_0_20} shows the same for Figs.~\ref{fig:Hhists_0_20}.
Mostly satisfactory agreement is observed. Thus, we conclude that the outcomes
of separatrix encounter are accurately predicted by
Eq.~\eqref{eq:def_pc_weaktide}.
\begin{figure}
    \centering
    \includegraphics[width=\columnwidth]{../initial/1_weaktide/5pc_fits0_06_20.png}
    \caption{Comparison of the fraction of systems converging to tCE2 obtained
    via numerical simulation (red dots) and obtained via a semi-analytic
    calculation (blue line) for $\eta_{\rm sync} = 0.06$ and $I = 20^\circ$ (see
    right panel of Fig.~\ref{fig:Hhists_0_06}). The semi-analytic calculation is
    performed by numerically integrating
    Eqs.~(\ref{eq:dsdt_rot},~\ref{eq:dsdt_tide}--\ref{eq:dWsdt_tide}) on a grid
    of initial conditions uniform in $\cos \theta_{\rm i}$ and $\phi_{\rm i}$
    until the system reaches the separatrix, then calculating the probability of
    reaching tCE2 for each integration using
    Eq.~\eqref{eq:def_pc_weaktide}.}\label{fig:pc_fits_0_06}
\end{figure}
\begin{figure}
    \centering
    \includegraphics[width=\columnwidth]{../initial/1_weaktide/5pc_fits0_20_20.png}
    \caption{Same as Fig.~\ref{fig:pc_fits_0_06} but for $\eta_{\rm sync} =
    0.2$, corresponding to the right panel of
    Fig.~\ref{fig:Hhists_0_20}.}\label{fig:pc_fits_0_20}
\end{figure}

With the above calculation, we can understand why even initial conditions in
zone I can converge to tCE2. As long as the inital spin is sufficiently large
($\geq 2n$), Eq.~\eqref{eq:dsdt_tide} shows that sufficiently large initial
obliquities are \emph{increased} by weak tidal friction (see the bottom right
region of Fig.~\ref{fig:6equils}). This increase means that even initial
conditions in zone I can sometimes be driven by weak tidal friction to
experience separatrix encounter. When this is the case, it can be shown that
both $\Delta K_+ > 0$ and $\Delta K_- > 0$. This implies that both III $\to$ II
and I $\to$ II outcomes are guaranteed, and thus initial conditions in all three
zones can evolve into tCE2.

\subsection{Spin Obliquity Evolution as a Function of Precession Strength}

In the previous section, we considered the outcome as a function of the initial
spin orientation, specified by $\theta_0$ and $\phi_0$. In this section, we
consider the distribution of outcomes when averaging over a distribution of
initial spin orientations. For simplicity, we just consider $\uv{s}$ being
isotropically distributed. Figure~\ref{fig:probs20} shows this for $I
= 20^\circ$ as a function of $\eta_{\rm sync}$. It can be seen that tCE2 is
reached with substantial probability ($\sim 50^\circ$) while it is also at a
substantial obliquity ($\sim 60^\circ$) for $\eta_{\rm sync} \approx 0.4$.
Furthermore, the bottom panel illustrates that initial conditions other than
those beginning in zone II generate a significant fraction of systems that
converge to tCE2.
\begin{figure}
    \centering
    \includegraphics[width=\columnwidth]{../initial/1_weaktide/5probs_20.png}
    \caption{\emph{Top:} Obliquities of the two tCE where $I = 20^\circ$ for a
    range of $\eta_{\rm sync}$ (Eq.~\ref{eq:def_etasync}), averaged over an
    isotropic initial spin orientation. \emph{Bottom:} Total probability of
    ending up in tCE2 (red dots). The red shaded region denotes the contribution
    of initial conditions in zone II that converge to tCE2 directly, and the
    yellow shaded region denotes the contribution of initial conditions in zone
    III that converge to tCE2 after a separatrix encounter. The vertical dashed
    line denotes $\eta_{\rm c}$ (Eq.~\ref{eq:def_etac}).}\label{fig:probs20}
\end{figure}
\begin{figure}
    \centering
    \includegraphics[width=\columnwidth]{../initial/1_weaktide/5probs_5.png}
    \caption{Same as Fig.~\ref{fig:probs20} but for $I =
    5^\circ$.}\label{fig:probs5}
\end{figure}

\section{Applications}

\subsection{Application to Warm + Cold Jupiter Systems
}\label{ss:disc_sehj}

Consider a system consisting of an inner Super-Earth (SE) and an exterior cold
Jupiter (CJ); such systems are expected to be abundant \citep{zhu2018super}. For
this system, the relative tidal dissipation rate $\abs{gt_{\rm a}}^{-1}$ can be
computed
\begin{align}
    \frac{1}{\abs{g t_{\rm a}}}
        \approx{}& 0.003\frac{1}{\cos I}\p{\frac{2k_2/Q}{10^{-3}}}
            \p{\frac{m_{\rm p}}{M_{\rm J}}}^{-1}
            \p{\frac{a_{\rm p}}{5 \;\mathrm{AU}}}^{3}\nonumber\\
        &\times \p{\frac{a}{0.4\;\mathrm{AU}}}^{-6}
            \p{\frac{\rho}{3 \; \mathrm{g/cm}^3}}^{-1}
            \p{\frac{M_\star}{M_{\odot}}}^{2}.
\end{align}
In physical units, $t_{\rm a} \sim 2 \times 10^8\;\mathrm{yr}$ for the
parameters adopted above, and $\dot{a} / a \sim 10^{14}\;\mathrm{yr}$, so the
spin evolution of the inner planet occurs within the characteristic age of most
planetary systems, while the orbital evolution of the inner planet can indeed by
safely neglected. Thus, the approximations of slow tidal dissipation and
constant $a$ used throughout Section~\ref{s:full_tide_prob} is valid.
For the same fiducial parameters, Eq.~\eqref{eq:def_etasync} can be evaluated
\begin{align}
    \eta_{\rm sync} ={}& 0.33 \p{\frac{k}{k_{\rm q}}}
            \p{\frac{m_{\rm p}}{M_{\rm J}}}
            \p{\frac{a_{\rm p}}{5 \;\mathrm{AU}}}^{-3}\nonumber\\
        &\times \p{\frac{a}{0.4\;\mathrm{AU}}}^{6}
            \p{\frac{\rho}{3 \; \mathrm{g/cm}^3}}
            \p{\frac{M_\star}{M_{\odot}}}^{-2}.\label{eq:sehj_etasync}
\end{align}
Here, $\rho = m / \p{4\pi R^3/3}$ is the average density of the inner planet,
and $M_{\rm J}$ is the mass of Jupiter. From Figs.~\ref{fig:probs20}
and~\ref{fig:probs5}, we see that this value of $\eta_{\rm sync}$ gives a
high-obliquity tCE2 with significant probability. Thus, we predict that a
sizable fraction of SEs with exterior CJ companions can have significant
obliquities ($\sim \gtrsim 30^\circ$).

\subsection{Application to Ultra-short-period Planet Formation
}\label{ss:disc_usp}

In \citet{millholland2020formation}, the authors consider capture into the CS2
resonance as part of a mechanism to induce ultra-short period planet (USP)
formation. They envision a process consisting of three stages: (i) capture into
CS2, (ii) simultaneous inward tidal migration and increase of the obliquity of
CS2, and (iii) destruction of CS2 due to strong tidal dissipation and stalling
of inward migration. Below, we revisit this process and show that the parameter
space for this obliquity-driven tidal runaway formation channel of USPs is
larger than is given in their work.

First, we verify agreement of our criterion for breaking of CS2 with theirs.
Evaluating Eq.~\eqref{eq:mcs_shift_crit} for the weak tidal friction torque
given by Eq.~\eqref{eq:mcs_shift}, we find that tCE2 breaks when the semi-major
axis is smaller than $a_{\rm break}$ where \textcolor{red}{I haven't double
checked this calculation\dots}
\begin{align}
    a_{\rm break}
        \approx{}& 0.01\;\mathrm{AU}
            \p{\frac{k}{k_q}}^{-1/18}
            \p{\frac{2 k_2 / Q}{10^{-3}}}^{1/9}
            \p{\frac{m_{\rm p}}{M_{\rm \oplus}}}^{-1/6}\nonumber\\
            &\times \p{\frac{a_{\rm p}}{0.05\;\mathrm{AU}}}^{1/2}
            \p{\frac{\rho}{6\;\mathrm{g/cm^3}}}^{-1/6}
            \p{\frac{M_{\star}}{M_{\odot}}}^{1/3}.
\end{align}
We have calculated this for a mutual inclination of $I = 5^\circ$. This
corresponds to an orbital period of about half a day, qualifying as a USP\@.

Second, we note that \citet{millholland2020formation} focus on system
architectures for which $\eta_{\rm sync} \gtrsim \eta_{\rm c}$ (where CS1 does
not exist at spin-orbit synchronization; see Eqs.~\ref{eq:def_etac}
and~\ref{eq:def_etasync}), finding that architectures for which $\eta_{\rm sync}
\lesssim \eta_{\rm c}$ always generate evolution towards CS1. However, they
assumed an initial obliquity $\theta_{0} = 0^\circ$, while broader distributions
of initial planet obliquities will give some probability of evolution towards
CS2 and tCE2 (see Figs.~\ref{fig:Hhists_0_06}--\ref{fig:Hhists_0_70}). Thus, it
is not necessary to restrict formation of USPs via tidal runaway to system
architectures for which $\eta_{\rm sync} \gtrsim \eta_{\rm c}$. In fig
\begin{figure}
    \centering
    \includegraphics[width=\columnwidth]{../initial/99_misc/5millholland_actives.png}
    \caption{Plot showing allowable parameter space for USP formation for the
    fiducial parameters: $P_{j + 1} / P_j = 1.3$ and $M_\star = M_{\odot}$,
    mirroring Figure~5 of \citet{millholland2020formation}. Shaded in blue is
    their region of allowable parameter space for USP formation ($\eta_{\rm
    sync} \geq 1$ and $\dot{a} / a < \;\mathrm{Gyr}$), while shaded in
    green is the region where $\eta_{\rm sync} \geq 1/3$, for which the tCE2
    probability is $> 50\%$ for $I = 20^\circ$ (see Fig.~\ref{fig:probs20}).
    \textcolor{red}{I added a few fudge factors temporarily to be a little bit
    to be closer to the Millholland plot; I will double check their validity and
    update this plot to better reflect our actual numbers / parameters.}
    }\label{fig:millholland_actives}
\end{figure}

In particular, architectures with larger period ratios $P_{j + 1} / P_j$ (and
thus, semi-major axis ratios; see Eq.~\ref{eq:def_etasync}) are still able to
undergo tidal runaway. Such architectures can be advantageous for USP formation
as more of their total angular momentum budget is in the outer planet's orbit,
allowing for additional migration \citep{fabrycky_otides,
millholland2020formation}.

\subsection{Application to WASP-12b}\label{ss:disc_wasp12b}

Thoughts:
\begin{itemize}
    \item Significantly limited by conservation of angular momentum (spin of the
        star is a reservoir?).

    \item Initial conditions are solidly in the $\eta_{\rm sync} \gg 1$ regime
        (CS2 obliquity goes to near zero when $a$ is increased by $\sim30\%$),
        probably not a very useful application of our theory. For systems where
        the perturber is \emph{strong}, our theory produces extra tCE2 systems
        compared to neglecting the effect of resonance capture, but where the
        perturber is weak, everybody knows that CS2 is guaranteed.
\end{itemize}

\section{Summary and Discussion}\label{s:summary}

\section{Acknowledgements}

We thank Sarah Millholland for useful discussions. This work has been supported
in part by NSF grant AST1715246. YS is supported by the NASA FINESST grant
19-ASTRO19-0041.%chktex 8

\bibliography{Su_weak_tides}
\bibliographystyle{aasjournal}

\appendix

\onecolumn

\section{Convergence of Initial Conditions Inside the Separatrix to CS2
}\label{app:cs_stab2}

In Section~\ref{ss:tidal_equils}, we studied the stability of the CSs under of
tidal alignment torque given by Eq.~\eqref{eq:dsdt_tide_toy}, finding that CS2
is locally stable. Later, in Section~\ref{ss:toy_outcomes}, we found that all
initial conditions within the separatrix converge to CS2, which is not
guaranteed by local stability of CS2. In this section, we give an analytic
demonstration that all points inside the separatrix indeed converge to CS2,
focusing on the case where $\eta \ll 1$.

Similarly to the analytic calculation in Section~\ref{ss:toy_outcomes}, we seek
to compute the change in the unperturbed Hamiltonian over a single libration
cycle. To calculate the evolution of $H$, we first parameterize the unperturbed
trajectory (similarly to Eq.~\ref{eq:sep_theta}). For initial conditions inside
the separatrix, the value of $H$ can be written $H = H_{\rm sep} + \Delta H$
where $\Delta H > 0$, and the two legs of the libration trajectory can be
written:
\begin{align}
    \cos \theta_{\pm} &\approx
        \eta \cos I \pm \sqrt{2\eta\s{\sin I\p{1 - \cos \phi} - \Delta H}}.
        \label{eq:lib_cycle_toy}
\end{align}
We have taken $\sin \theta \approx 1$, a good approximation in zone II when
$\eta \ll 1$. Note that there are some values of $\phi$ for which no solutions
of $\theta$ exist, reflecting the fact that the libration cycle does not extend
over the full interval $\phi \in [0, 2\pi]$. During a libration cycle,
$\theta_-$ [$\theta_+$] is traversed while $\phi' > 0$ [$\phi' < 0$], i.e.\ the
trajectory librates counterclockwise in $(\cos \theta, \phi)$ phase space (see
Fig.~\ref{fig:1contours}).

The leading order change to $H$ over a single libration cycle can then computed by
integrating $\rdil{H}{t}$ along this trajectory, yielding:
\begin{align}
    \oint \rd{H}{t}\;\mathrm{d}t
        &= \oint \p{\rd{(\cos \theta)}{t}}_{\rm tide}
            \;\mathrm{d}\phi,\nonumber\\
        &= \int\limits_{\phi_{\min}}^{\phi_{\max}}
                \frac{1}{t_{\rm s}}
                \p{\sin^2\theta_- - \sin^2\theta_+} \;\mathrm{d}\phi\nonumber\\
        &= \frac{1}{t_{\rm s}}
            \int\limits_{\phi_{\min}}^{\phi_{\max}}
                4\eta \cos I \sqrt{2\eta\s{\sin I\p{1 - \cos \phi} - \Delta H}}
                \;\mathrm{d}\phi > 0.
\end{align}
Here, $\phi_{\min} > 0$ and $\phi_{\max} < 2\pi$ are defined such that the
trajectory librates over $\phi \in \s{\phi_{\min}, \phi_{\max}}$. Thus, $H$ is
strictly increasing for all initial conditions inside the separatrix, and they
all converge to CS2.

% \section{Separatrix Crossing Probability with Weak Tidal Friction
% }\label{app:sep_crossing_dynamics}

% In this appendix, we analytically calculate $\Delta K_{\pm}$ for use in
% Eq.~\eqref{eq:def_pc_weaktide} to calculate the probabilities of the two
% possible transitions upon separatrix encounter. We first rewrite the full
% equations of motion for the planet's spin including weak tidal friction in
% component form:
% \begin{align}
%     \rd{\theta}{t} &= g\sin I \sin \phi -
%         \frac{1}{t_{\rm a}} \sin \theta,\label{eq:ds_fullq}\\
%     \rd{\phi}{t} &= -\alpha\cos\theta
%         - g\p{\cos I + \sin I \cot \theta \cos \phi}\label{eq:ds_fullphi},\\
%     \rd{\Omega_{\rm s}}{t}
%         &= \frac{1}{t_{\rm a}} \s{2n \cos \theta
%             - g\Omega_{\rm s}\p{1 + \cos^2\theta}}\label{eq:ds_fulls}.
% \end{align}

% Note that Eq.~\eqref{eq:def_dK_weaktide} can be rewritten
% \begin{align}
%     \Delta K_{\pm} &= \oint_{\mathcal{C}_{\pm}} \rd{H}{t}
%         - \rd{H_{\rm sep}}{t}\;\mathrm{d}t
%         \nonumber\\
%         &= \oint_{\mathcal{C}_{\pm}}
%            \p{\rd{(\cos\theta)}{t}}_{\rm tide}
%             + \frac{\dot{\Omega}_{\rm s}}{\dot{\phi}}
%             \p{\pd{H}{s} - \pd{H_{\rm sep}}{s}}\;\mathrm{d}\phi
%                 \label{eq:app_dhpm}.
% \end{align}

% We then evaluate $\Delta K_{\pm}$ by integrating along the two legs of the
% separatrix $\mathcal{C}_{\pm}$ (see Fig.~\ref{fig:1contours}). Note that we must
% use the value of $\eta$ at the moment of separatrix encounter, as the evolution
% of $\Omega_{\rm s}$ changes the spin-orbit precession frequency $\alpha$
% \begin{align}
%     \Delta K_{\pm} ={}& \oint_{\mathcal{C}_{\pm}}
%         \frac{1}{t_{\rm s}}\sin^2\theta
%             \p{\frac{2n}{\Omega_{\rm s}} - \cos\theta}
%          +
%             \s{\frac{\alpha}{\Omega_{\rm s}}
%                 \frac{\cos^2\theta}{2} + g\frac{\Omega_{\rm
%             c}}{2\Omega_{\rm s}^2}\cos^2 I}\rd{\Omega_{\rm
%             s}}{t}\rd{t}{\phi}\;\mathrm{d}\phi\\
%      \approx{}& \oint_{\mathcal{C}_{\pm}}
%         \frac{1}{t_{\rm s}}\sin^2\theta
%             \p{\frac{2n}{\Omega_{\rm s}} - \cos\theta}
%         + \frac{1}{t_{\rm s}}\frac{n}{\Omega_{\rm s}\eta}
%             \s{\p{\cos \theta}_{\mathcal{C}_{\pm}} - \frac{\Omega_{\rm s}}{2n}}
%             \s{2 \cos I \pm \sqrt{2 \sin I\p{1 - \cos \phi} / \eta}}
%             \;\mathrm{d}\phi.\label{eq:app_deltaK}
%     t_{\rm s} \Delta K_{\pm} \approx{}&
%         -2\cos I\p{\pm 2\pi \eta \cos I + 8\sqrt{\eta \sin I}}
%         \pm 2\pi \frac{\Omega_{\rm s}}{n}\cos I
%         - 8\eta \cos I \sqrt{\sin I / \eta}
%             + \frac{4\Omega_{\rm s}}{n}\sqrt{\sin I/\eta}\nonumber\\
%         &+ \frac{2n}{\Omega_{\rm s}}\p{\mp 2\pi\p{1 - 2\eta \sin I}
%             + 16\cos I \eta^{3/2}\sqrt{\sin I}}
%             + 8\sqrt{\eta \sin I}
%             \pm 2 \pi \eta \cos I
%             - \frac{64}{3} \p{\eta \sin I}^{3/2}.\label{eq:app_deltaK}
% \end{align}
% At this point, since $\eta$ often is not small, it is necessary to integrate
% along the numerically computed $\mathcal{C}_{\pm}$ (instead of the approximate
% solution given by Eq.~\ref{eq:sep_theta}) to obtain sufficiently accurate values
% for $\Delta K_{\pm}$. These numerically integrated values are used in
% Eq.~\eqref{eq:def_pc_weaktide} to semi-analytically compute the tCE2
% probabilities (blue curves) in
% Figs.~\ref{fig:pc_fits_0_06}--\ref{fig:pc_fits_0_20}.

\label{lastpage} % chktex 24
\end{document}
