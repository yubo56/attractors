% vim 0_eta/1*.py 0_eta/3*.py 0_eta/6*.py 3_toy3/1sim.py 99_misc/0*.py 99_misc/2*.py 99_misc/5*.py 1_weaktide/5*.py 1_weaktide/6*.py
% cd 0_eta && python 1*.py && python 3*.py && python 6*.py && cd .. && cd 3_toy3 && python 1sim.py && cd .. && cd 99_misc && python 0*.py && python 2*.py && python 5*.py && cd .. && cd 1_weaktide && python 5*.py && python 6*.py
    \documentclass[
        fleqn,
        usenatbib,
        % referee,
    ]{mnras}
    \usepackage{
        amsmath,
        amssymb,
        newtxtext,
        newtxmath,
        ae, aecompl,
        graphicx,
        booktabs,
        xcolor,
    }

    \newcommand*{\scinot}[2]{#1\times10^{#2}}
    \newcommand*{\rd}[2]{\frac{\mathrm{d}#1}{\mathrm{d}#2}}
    \newcommand*{\rtd}[2]{\frac{\mathrm{d}^2#1}{\mathrm{d}#2^2}}
    \newcommand*{\pd}[2]{\frac{\partial#1}{\partial#2}}
    \newcommand*{\ptd}[2]{\frac{\partial^2#1}{\partial#2^2}}
    % inline
    \newcommand*{\mdil}[2]{\mathrm{D}#1/\mathrm{D}#2}
    \newcommand*{\pdil}[2]{\partial#1/\partial#2}
    \newcommand*{\rdil}[2]{\mathrm{d}#1/\mathrm{d}#2}
    \newcommand*{\at}[1]{\left.#1\right|}
    \newcommand*{\abs}[1]{\left|#1\right|}
    \newcommand*{\ev}[1]{\left\langle#1\right\rangle}
    \newcommand*{\p}[1]{\left(#1\right)}
    \newcommand*{\s}[1]{\left[#1\right]}
    \newcommand*{\z}[1]{\left\{#1\right\}}
    \newcommand*{\bm}[1]{\mathbf{#1}}
    \newcommand*{\uv}[1]{\hat{\boldsymbol{\mathbf{#1}}}}
    \newcommand*{\md}[0]{\mathrm{d}}
    \DeclareMathOperator*{\med}{med}
    \DeclareMathOperator*{\erf}{erf}

\title[Weak Tides and Cassini States]{Dynamics of Colombo's Top: Tidal
Dissipation and Resonance Capture}
\author[Y. Su and D. Lai.]{
Yubo Su,$^1$\thanks{E-mail: yubosu@astro.cornell.edu},
Dong Lai$^{1,2}$
\\
$^1$ Cornell Center for Astrophysics and Planetary Science, Department of
Astronomy, Cornell University, Ithaca, NY 14853, USA\\
$^2$ Tsung-Dao Lee Institute \& School of Physics and Astronomy, Shanghai Jiao
Tong University, 200240 Shanghai, China
}

\date{Accepted XXX\@. Received YYY\@; in original form ZZZ}

\pubyear{2021}

\begin{document}\label{firstpage}
\pagerange{\pageref{firstpage}--\pageref{lastpage}}
\maketitle

\begin{abstract}
    Abstract here
\end{abstract}

\begin{keywords}
planet-star interactions
\end{keywords}

\section{Introduction}\label{s:intro}

% (i) What is obliquity? Why is it important, esp in exoplanetary systems?
% (ii) what are the dynamics of obliquity?
% (iii) Introduce Paper I

It is well recognized that the obliquity of a planet, the angle between the spin
and orbital axes, likely reflects its dynamical history. In our Solar
System, planetary obliquities (hereafter just ``obliquities'') range from
$3.1^\circ$ for Jupiter to $26.7^\circ$ for Saturn to $98^\circ$ for Uranus. The
obliquities of exoplanets are challinging to measure, and so far only loose
constraints have been obtained for the obliquity of a faraway ($\gtrsim
50\;\mathrm{AU}$) planetary-mass companion \citep{bryan2020obliquity}.
Nevertheless, there are prospects for better constraints on exoplanetary
obliquities in the coming years, such as using high-resolution spectroscopy to
measure $v\sin i$ for a planet \citep{snellen2014fast, bryan2018constraints} and
using high-precision photometry to measure the asphericity of a planet
\citep{seager2002constraining}. Substantial obliquities are of increasing
theoretical interest for their proposed role in explaining peculiar thermal
phase curves \citep[see e.g.][]{millholland_signatures, ohno_infer_obl}, in
enhancing tidal dissipation in hot Jupiters \citep{millholland_wasp12b} and
super-Earths \citep{millholland2019obliquity}, and in the formation of
ultra-short-period planets \citep[USPs;][]{millholland2020formation} .

While nonzero obliquities are sometimes attributed to one or many giant impacts
\citep[e.g.]{original_gi, benz1989tilting, korycansky1990one, morbidelli_gi},
some studies suggest that large planetary obliquities may be due to encountering
spin-orbit resonances. In this scenario, a rotating planet is subjected to a
gravitational torque from its host star, making its spin axis precess around its
orbital (angular momentum) axis. At the same time, the orbital axis precesses
around another fixed axis under the gravitational influence of other masses in
the system, e.g.\ additional planets or a protoplanetary disk. When the two
precession frequencies become commensurate, a resonance can occur that excites
the obliquity to large values. This model is known as ``Colombo's Top'' after
the seminal work of \citet{colombo1966}, and subsequent works have investigated
the rich dynamics of this system \citep{peale1969, peale1974possible,
ward1975tidal, henrard1987}. Such resonances have been invoked to explain the
obliquities of both the Solar System gas giants \citep{ward2004I, ward2004II,
ward_jupiter, vokrouhlicky2015tilting, saillenfest2020future,
saillenfest2021large} and the ice giants \citep{hamilton_tilting_ice}.

In a previous paper \citep[hereafter Paper I]{su2020}, we presented a systematic
and general investigation of the dynamics of Colombo's Top when the two
precession frequencies of the system evolve through a commensurability. We
applied our results to investigate the generation of exoplanetary obliquities
via a dissipating protoplanetary disk. However, our model did not consider the
effect of additional torques in the system. Often, tidal dissipation in the
planet both causes the planet's spin frequency to approach its orbital frequency
and drives the planet's spin axis towards its orbital axis, complicating the
evolution of Colombo's Top \citep{fabrycky_otides, peale2008obliquity}. In this
paper, we extend the approach of Paper I to include tidal dissipation. Our new
results include a simplified formulation of the effect of tidal dissipation on
the equilibria of the system (Cassini States) and a novel, complete description
of the resonance encounter process.

Our paper is organized as follows. In Section~\ref{s:theory}, we briefly review
Colombo's Top. In Section~\ref{s:toy_model}, we investigate the effect of adding
a simple alignment torque to Colombo's Top. The resulting dynamics are
illustrative of the new behavior that emerges due to tidal dissipation. In
Section~\ref{s:full_tide_prob}, we solve for the dynamics of the system
including a realistic model of tidal dissipation. In Section~\ref{s:summary}, we
apply our results to three astrophysical scenarios of interest: (i) a super
Earth with a cold Jupiter companion, (ii) the formation of USPs, and (iii) the
rapid orbital decay of WASP-12b. We summarize and discuss in
Section~\ref{s:summary}.

\section{Spin Evolution Equations and Cassini States: Review}\label{s:theory}

In this section, we briefly introduce the spin dynamics of the planet (for a
more detailed treatment, see Paper I). We consider a star of mass $M_\star$
hosting an inner oblate planet of mass $m$ and radius $R$ on a circular orbit
with semi-major axis $a$ and an outer perturber of mass $m_{\rm p}$ on a
circular orbit with semi-major axis $a_{\rm p}$. We assume that the two orbits
are mutually inclined by the angle $I$. Denote $\bm{S}$ the spin angular
momentum and $\bm{L}$ the orbital angular momentum of the planet, and
$\bm{L}_{\rm p}$ the angular momentum of the perturber. The corresponding unit
vectors are $\uv{s} \equiv \bm{S} / S$, $\uv{l} \equiv \bm{L} / L$, and
$\uv{l}_{\rm p} \equiv \bm{L}_{\rm p} / L_{\rm p}$. The spin axis $\uv{s}$ of
the planet tends to precess around its orbital (angular momentum) axis $\uv{l}$,
driven by the gravitational torque from the host star acting on the planet's
rotational bulge. On the other hand, $\uv{l}$ and the disk axis $\uv{l}_{\rm p}$
precess around each other due to gravitational interactions. We assume $S \ll L
\ll L_{\rm p}$, so $\uv{l}_{\rm p}$ and $\uv{l}$ are nearly constant. The
equations of motion for $\uv{s}$ and $\uv{l}$ in this limit are
(\citealp{anderson2018teeter}, Paper I)
\begin{align}
    \rd{\uv{s}}{t}
        &= \omega_{\rm sl}\p{\uv{s} \cdot \uv{l}}\p{\uv{s} \times \uv{l}}
        \equiv \alpha\p{\uv{s} \cdot \uv{l}}\p{\uv{s} \times
        \uv{l}},\label{eq:dsdt1}\\
    \rd{\uv{l}}{t} &= \omega_{\rm lp}\p{\uv{l} \cdot \uv{l}_{\rm p}}\p{\uv{l}
        \times \uv{l}_{\rm p}} \equiv -g\p{\uv{l} \times \uv{l}_{\rm p}},
        \label{eq:dldt1}
\end{align}
where
\begin{align}
    \omega_{\rm sl} &\equiv \frac{3GJ_2 mR^2 M_\star}{2a^3 I\Omega_{\rm s}}
        = \frac{3k_q}{2k}\frac{M_\star}{m}\p{\frac{R}{a}}^3 \Omega_{\rm s},
            \label{eq:wsl}\\
    \omega_{\rm lp} &= \frac{3m_{\rm p}}{4M_\star}\p{\frac{a}{a_{\rm p}}}^3
        n.\label{eq:wlp}
\end{align}
In Eq.~\eqref{eq:wsl}, $\Omega_{\rm s}$ is the spin frequency of the inner
planet, $\mathcal{I} = k mR^2$ (with $k$ a constant) is its moment of inertia
and $J_2 = k_{\rm q}\Omega_{\rm s}^2 (R^3/Gm)$ (with $k_{q}$ a constant) is its
rotation-induced (dimensionless) quadrupole moment [for a body with uniform
density, $k=0.4, k_{\rm q}=0.5$; for the Earth, $k \simeq 0.331$ and $k_{\rm q}
\equiv k_2 / 3 \simeq 0.1$ \citep[e.g.][]{groten2004fundamental,
lainey2016quantification}]. In other studies, $3k_{\rm q} / 2 k$ is often
notated as $k_2 / 2C$ \citep[e.g.][]{millholland_disk}. In Eq.~\eqref{eq:wlp},
$n \equiv \sqrt{GM_\star/a^3}$ is the inner planet's orbital mean motion,  and
we have assumed $a_{\rm p}\gg a$ and included only the leading-order
(quadrupole) interaction between the inner planet and perturber. Following
standard notation, we define $\alpha = \omega_{\rm sl}$ and $g \equiv
-\omega_{\rm 1p} \cos I$ \citep[e.g.][]{colombo1966}.

As in Paper I, we combine Eqs.~(\ref{eq:dsdt1}--\ref{eq:dldt1}) into a
single equation by transforming into a frame rotating about $\uv{l}_{\rm p}$
with frequency $g$. In this frame, $\uv{l}_{\rm p}$ and $\uv{l}$ are both fixed,
and $\uv{s}$ evolves as:
\begin{equation}
    \p{\rd{\uv{s}}{t}}_{\rm rot}
        = \alpha\p{\uv{s} \cdot \uv{l}}\p{\uv{s} \times \uv{l}}
            + g\p{\uv{s} \times \uv{l}_{\rm p}}. \label{eq:dsdt_rot}
\end{equation}
In this reference frame, we choose the coordinate system such that $\uv{z} =
\uv{l}$ and $\uv{l}_{\rm p}$ lies in the $\uv{x}$-$\uv{z}$ plane. We describe
$\uv{s}$ in spherical coordinates using the polar angle $\theta$, the planet's
obliquity, and $\phi$, the precessional phase of $\uv{s}$ about $\uv{l}$,
defined so that when $\phi = 0^\circ$, $\uv{l}_{\rm p}$ and $\uv{s}$ are on
opposide sides of $\uv{l}$.

The equilibria of Eq.~\eqref{eq:dsdt_rot} are referred to as \emph{Cassini
States} \citep[CSs;][]{colombo1966, peale1969}. We follow the notation of
Paper I, where the parameter
\begin{equation}
    \eta \equiv -\frac{g}{\alpha},\label{eq:def_eta}
\end{equation}
is introduced. For a given value of $\eta$, there can be either two or four CSs,
all of which require $\uv{s}$ lie in the plane of $\uv{l}$ and $\uv{l}_{\rm p}$.
Following the standard nomenclature, $\uv{s}$ and $\uv{l}_{\rm p}$ are on
opposite sides of $\uv{l}$ for CSs 1, 3, and 4, and are on the same side for
CS2. We depart from the standard convention and simply label the CSs in
spherical coordinates: Figure~\ref{fig:cs_locs} shows the CS obliquities as a
function of $\eta$. CS1 and CS4 do not exist when $\eta > \eta_{\rm c}$, where
\begin{equation}
    \eta_{\rm c} \equiv \p{\sin^{2/3}\!I + \cos^{2/3}\!I}^{-3/2}.
        \label{eq:def_etac}
\end{equation}
\begin{figure}
    \centering
    \includegraphics[width=\columnwidth]{../initial/99_misc/2_cs_locs_phi.png}
    \caption{Cassini State obliquities $\theta$ as a function of $\eta \equiv
    -g/\alpha$ (Eq.~\ref{eq:def_eta}) for $I = 20^\circ$. The vertical dashed
    line denotes $\eta_{\rm c}$, where the number of Cassini States changes from
    four to just two (Eq.~\ref{eq:def_etac}). The y-axis labels on the right of
    the plot show the asymptotic obliquities for CS2 and CS3, $I$ and $180^\circ
    - I$ respectively. Note that $\theta$ does not follow the standard
    convention \citep[e.g.][]{colombo1966} and is simply the angle between
    $\uv{s}$ and $\uv{l}$.}\label{fig:cs_locs}
\end{figure}

The Hamiltonian corresponding to Eq.~\eqref{eq:dsdt_rot} is
\begin{align}
    H &= -\frac{\alpha}{2}\p{\uv{s} \cdot \uv{l}}^2
            - g\p{\uv{s} \cdot \uv{l}_{\rm d}}\nonumber\\
        &= -\frac{\alpha}{2} \cos^2\theta
            - g\p{\cos\theta \cos I - \sin I \sin\theta \cos \phi}.\label{eq:H}
\end{align}
Here, $\cos \theta$ and $\phi$ form a canonically conjugate pair of variables.
Figure~\ref{fig:1contours} shows the level curves of this Hamiltonian for $I =
20^\circ$, for which $\eta_{\rm c} \approx 0.574$ (Eq.~\ref{eq:def_etac}). When $\eta
< \eta_{\rm c}$, CS4 exists and is a saddle point. The infinite-period orbits
originating and ending at CS4 form the \emph{separatrix} and divide phase space
into three zones. $\phi$ librates for trajectories in zone II and circulates for
trajectories in zones I and III\@. On the other hand, when $\eta > \eta_{\rm
c}$, the separatrix is absent and all trajectories circulate. When the
separatrix exists, we divide it into two curves: $\mathcal{C}_+$, the boundary
between zones I and II, and $\mathcal{C}_-$, the boundary between zones II and
III\@.
\begin{figure}
    \centering
    \includegraphics[width=\columnwidth]{../initial/0_eta/1contours20.png}
    \caption{Level curves of the Cassini State Hamiltonian (Eq.~\ref{eq:H}) for
    $I = 20^\circ$, for which $\eta_{\rm c} \approx 0.57$
    (Eq.~\ref{eq:def_etac}). For $\eta < \eta_{\rm c}$, there are four Cassini
    States (labeled), while for $\eta > \eta_{\rm c}$ there are only two. In the
    former case, the existence of a \emph{separatrix} (solid black lines)
    separates phase space into three numbered zones (I/II/III, labeled). We
    denote the upper and lower legs of the separatrix by $\mathcal{C}_{\pm}$
    respectively, as shown in the upper two panels. }\label{fig:1contours}
\end{figure}

\section{Spin Evolution with Alignment Torque}\label{s:toy_model}

In this section, we consider a simplified dissipative torque that isolates the
important new phenomenon presented in this paper. We assume that the spin
magnitude of the planet is constant, so $\alpha$ and $g$ are both fixed, while
the spin orientation $\uv{s}$ experiences an alignment torque towards $\uv{l}$
on the alignment timescale $t_{\rm al}$:
\begin{equation}
    \p{\rd{\uv{s}}{t}}_{\rm tide}
        = \frac{1}{t_{\rm al}} \uv{s} \times \p{\uv{l} \times \uv{s}}.
        \label{eq:dsdt_tide_toy}
\end{equation}
The full equations of motion for $\uv{s}$ in the coordinates $\theta$ and $\phi$
can be written:
\begin{align}
    \rd{\theta}{t} &= -g\sin I \sin \phi - \frac{1}{t_{\rm al}} \sin \theta,
        \label{eq:dqdt_toy}\\
    \rd{\phi}{t} &= -\alpha \cos\theta
        - g\p{\cos I + \sin I \cot \theta \cos \phi}.\label{eq:dfdt_toy}
\end{align}

\subsection{Modified Cassini States and Linear Stability
Analysis}\label{ss:tidal_equils}

If the alignment torque is weak ($\abs{g}t_{\rm al} \gg 1$), then the fixed
points of Eqs.~(\ref{eq:dqdt_toy}--\ref{eq:dfdt_toy}) are slightly modified CSs.
To leading order, all of the CS obliquities $\theta_{\rm cs}$ are unchanged
while the azimuthal angle $\phi_{\rm cs}$ for each CS now satisfies
\begin{equation}
    \sin \phi_{\rm cs} = \frac{\sin\theta_{\rm cs}}{\sin I \abs{g}t_{\rm al}}.
        \label{eq:mcs_shift}
\end{equation}
We can see that if $t_{\rm al}$ is longer than some critical alignment
timescale $t_{\rm al, c}$, given by
\begin{align}
    t_{\rm al, c} &\equiv \frac{1}{\abs{g}\sin I},\label{eq:mcs_shift_crit}
\end{align}
then Eq.~\eqref{eq:mcs_shift} will always have solutions for $\phi_{\rm cs}$,
and the alignment torque does not change the number of fixed points of the
system. If $t_{\rm al}$ is decreased below $t_{\rm al, c}$, CS2 and CS4 cease to
be fixed points when $\eta \ll 1$ \citep[as first noted in][]{fabrycky_otides},
as $\theta_{\rm cs} \approx 90^\circ$ for these (see Fig.~\ref{fig:cs_locs}). On
the other hand, the other CSs have small $\sin \theta_{\rm cs}$ and are only
slightly perturbed. Figure~\ref{fig:mcs} shows the obliquity and azimuthal
angles for each of the CSs in the $\eta \ll 1$ case obtained via numerical
root finding of Eqs.~(\ref{eq:dqdt_toy}--\ref{eq:dfdt_toy}), where it can be
seen that CS2 and CS4 collide and annihilate when $t_{\rm al}$ reaches $t_{\rm
al, c}$. The phase shift $\phi_{\rm cs}$ for CS2 and CS4 for $t_{\rm al} >
t_{\rm al, c}$ can be predicted to good accuracy using Eq.~\eqref{eq:mcs_shift}
using $\theta_{\rm cs} \approx \pi/2 - \eta \cos (I) \approx 79^\circ$
\citep{su2020} for both, shown as the dashed lines in the bottom panel of
Fig.~\ref{fig:mcs}. For the remainder of this section, we will consider the case
where $t_{\rm al} \gg t_{\rm al, c}$ and the CSs only differ slightly from their
unperturbed locations.
\begin{figure}
    \centering
    \includegraphics[width=\columnwidth]{../initial/99_misc/0_stab.png}
    \caption{Modified CS obliquities (top) and azimuthal angles (bottom) for $I
    = 20^\circ$ and $\eta = 0.2$, where the CS1 and CS3 obliquities have been
    offset (labeled in top legend) to improve clarity of the plot. In both
    panels, the solid lines give the result when applying a numerical root
    finding algorithm to the full equations of motion,
    Eqs.~(\ref{eq:dqdt_toy}--\ref{eq:dfdt_toy}), while the dotted lines in the
    bottom panel give the CS2 and CS4 azimuthal angles according to
    Eq.~\eqref{eq:mcs_shift}. At $\abs{gt_{\rm al} \sin I} = 1$, CS2 and CS4
    collide and annihilate (Eq.~\ref{eq:mcs_shift_crit}).}\label{fig:mcs}
\end{figure}

We next seek to characterize the stability of small perturbations about each of
the CSs in the presence of the weak alignment torque. We can linearize
Eqs.~(\ref{eq:dqdt_toy}--\ref{eq:dfdt_toy}) about a shifted CS, yelding
\begin{align}
    \rd{}{t}\begin{bmatrix}
        \Delta \theta\\ \Delta \phi
    \end{bmatrix} &= \begin{bmatrix}
        -\frac{\cos \theta}{t_{\rm al}} &
        -g\sin I \cos \phi \\
        \alpha \sin \theta + g\frac{\sin I \cos \phi}{\sin^2\theta} &
        0
    \end{bmatrix}_{\rm cs}\begin{bmatrix}
        \Delta \theta \\ \Delta \phi
    \end{bmatrix},\label{eq:dsdt_hessian}
\end{align}
where the cs subscript indicates evaluating at a CS, $\Delta \theta = \theta -
\theta_{\rm cs}$, and $\Delta \phi = \phi - \phi_{\rm cs}$. The eigenvalues
$\lambda$ of Eq.~\eqref{eq:dsdt_hessian} satisfy the equation
\begin{equation}
    0 = \p{\lambda + \frac{\cos \theta_{\rm cs}}{t_{\rm al}}}\lambda
        - \lambda_0^2,\label{eq:lambda_orig}
\end{equation}
where
\begin{equation}
    \lambda_0^2 \equiv \p{\alpha
        \sin \theta_{\rm cs} + g\sin I \csc^2\theta_{\rm cs}\cos \phi_{\rm cs}}
            \p{- g \sin I \cos \phi_{\rm cs}}\label{eq:def_l0_sq}.
\end{equation}
When $t_{\rm al}$ is large, we can simplify Eq.~\eqref{eq:lambda_orig} to
\begin{equation}
    \lambda \approx -\frac{\cos \theta_{\rm cs}}{t_{\rm al}}
        \pm \sqrt{\lambda_0^2}. \label{eq:lambda_approx}
\end{equation}
The stability depends only on the real part of $\lambda$ in
Eq.~\eqref{eq:lambda_approx}. $\lambda_0^2$ is a generalization of Eq.~(A4)
in Paper I and generally has the same behavior: it is negative for CSs
1--3 and positive for CS4, as shown in Fig.~\ref{fig:lambda_full}. Thus, CS4 is
always unstable, as there will always be at least one positive solution for
$\lambda$, and the stability of CSs 1--3 are solely determined by the sign of
$\cos \theta_{\rm cs}$. Using Fig.~\ref{fig:cs_locs}, we conclude that CS1 and
CS2 are stable and attracting while trajectories near CS3 are driven away by the
alignment torque. These calculations justify results long used in CS literature
\citep[e.g.][]{ward1975tidal, fabrycky_otides}.
\begin{figure}
    \centering
    \includegraphics[width=\columnwidth]{../initial/99_misc/2_lambdas_full.png}
    \caption{$\lambda_0^2$ (Eq.~\ref{eq:def_l0_sq}) as a function of $\eta$ for
    the four CSs, for three different values of the shift in $\phi_{\rm cs}$
    (e.g.\ for $\Delta \phi_{\rm cs} = 60^\circ$, $\phi_{\rm cs} = 120^\circ$
    for CS2 and $\phi_{\rm cs} = 60^\circ$ for CSs 1, 3, and 4). The values of
    $\Delta \phi_{\rm cs}$ are labeled. }\label{fig:lambda_full}
\end{figure}

\subsection{Spin Obliquity Evolution Driven by Alignment
Torque}\label{ss:toy_outcomes}

With the above results, we are equipped to ask questions about the dynamics of
Eqs.~(\ref{eq:dqdt_toy}--\ref{eq:dfdt_toy}): what is the long term behavior for
a general initial $\uv{s}$? If $\eta > \eta_{\rm c}$, then the only stable
final spin state is CS2, and all initial conditions will evolve asympotically
towards it. As such, we consider only $\eta < \eta_{\rm c}$, where an initial
condition can asymptotically evolve towards either CS1 or CS2. Additionally, we
restrict our attention to the regime where the alignment torque is weak and CS2
remains stable; specifically, we use $\abs{g}t_{\rm al}= 10^{3}$. We
numerically integrate Eqs.~(\ref{eq:dqdt_toy}--\ref{eq:dfdt_toy}) for many
random initial conditions uniformly distributed in $\p{\cos \theta, \phi}$ and
record the nearest CS for each integration after $10t_{\rm al}$. In
Fig.~\ref{fig:toy_phop}, we show the results of this procedure for $\eta = 0.2$,
and $I = 20^\circ$. It is clear that initial conditions in zone I evolve into
CS1, those in zone II evolve into CS2, while those in zone III have a
probabilistic outcome. We aim to understand each of these in turn:
\begin{figure}
    \centering
    \includegraphics[width=\columnwidth]{../initial/0_eta/3stats3_5_0_2.png}
    \caption{Plot illustrating the asymptotic behavior of initial conditions for
    $\eta = 0.2$ and $I = 20^\circ$. Each dot represents an initial spin
    orientation, and the coloring of the dot indicates which Cassini State
    (legend) the system asymptotes towards.}\label{fig:toy_phop}
\end{figure}

For initial conditions in zone I, the spin circulates, and $\dot{\theta}$ is
negative everywhere during the cycle. Thus, for initial conditions in zone I,
$\theta$ decreases until the trajectory has converged to CS1. This is
intuitively reasonable, as CS1 is stable.

For initial conditions in zone II, our stability calculation in
Section~\ref{ss:tidal_equils} shows that initial conditions sufficiently near CS2
will converge to CS2. In fact, this result can be extended to all initial
conditions inside the separatrix, shown in Appendix~\ref{app:cs_stab2}.

For initial conditions in zone III, since there are no stable CSs in zone III,
the system must evolve through the separatrix to reach one of either CS1 or CS2.
The outcome of the separatrix encounter is effectively probabilistic and
determines the final CS\@. Intuitively, this can be understood as probabilistic
resonance capture, as first studied in the seminal work of \citet{henrard1982}:
since $\eta \ll \eta_{\rm c}$, $\alpha \gg -g$, but $\alpha \cos \theta$ can
become commensurate with $-g$ if $\cos \theta$ becomes small. This is achieved
as $\theta$ evolves from an initially retrograde obliquity through $90^\circ$
towards $0^\circ$ under the influence of the dissipative term in
Eq.~\eqref{eq:dqdt_toy}.

While similar in behavior to previous studies of probabilistic resonance capture
\citep{henrard1982, su2020}, the underlying mechanism is different: in these
previous studies, the phase space structure itself evolves and causes systems to
transition among phase space zones, while in the present scenario, a
non-Hamiltonian, dissipative perturbation causes systems to transition among
unchanging phase space zones. In the following section, we present a calculation
to understand analytically the probability distribution of outcomes upon
separatrix encounter in this new scenario.

\subsubsection{Analytical Calculation of Resonance Capture Probability}

Before discussing quantitative calculations, we first present a graphical
understanding of the separatrix encounter process.
Figure~\ref{fig:toy_hop_manifolds} shows how the perturbative alignment torque
generates the two outcomes upon separatrix encounter, a zone III-II or a zone
III-I transition. The critical trajectories in Fig.~\ref{fig:toy_hop_manifolds}
are calculated numerically by integrating a point infinitesimally close to CS4
forward and backward in time. In the absence of the alignment torque, these
trajectories would evolve along the separatrix, but in the presence of the
alignment torque, they are perturbed slightly and cease to overlap. It can be
seen in Fig.~\ref{fig:toy_hop_manifolds} that this splitting opens a path from
zone III into both zones I and II\@.

To understand this process more concretely, as well as associate probabilities
to the two possible outcomes, it is important to be quantitative. The correct
approach is to consider the evolution of the value of the \emph{unperturbed}
Hamiltonian (Eq.~\ref{eq:H}) as the spin evolves due to the alignment torque. A
point in zone III evolves such that $H$ is increasing until $H \approx H_{\rm
sep}$ where $H_{\rm sep}$ is the value of $H$ along the separatrix, given by
\begin{align}
    H_{\rm sep} &\equiv H\p{\cos \theta_{\rm 4}, \phi_{\rm 4}},\nonumber\\
        &\approx g\sin I + \frac{g^2}{2\alpha}\cos^2 I +
            \mathcal{O}\p{\eta^2},\label{eq:def_Hsep}
\end{align}
where $\theta_4$ and $\phi_4$ are the coordinates of CS4. As the system evolves
closer to the separatrix, the change in $H$ over each circulation cycle can be
approximated by $\Delta H_-$, the change in $H$ along $\mathcal{C}_-$ (see
Fig.~\ref{fig:1contours}). In general, we define the quantities $\Delta H_{\pm}$
\begin{equation}
    \Delta H_{\pm} \equiv \oint\limits_{\mathcal{C}_{\pm}}
        \rd{H}{t}\;\mathrm{d}t.\label{eq:def_dHpm}
\end{equation}
This can be simplified by using:
\begin{align}
    \rd{H}{t} &=
            \pd{H}{(\cos \theta)}\rd{(\cos \theta)}{t}
            + \pd{H}{\phi}\rd{\phi}{t},\nonumber\\
        &= \p{\rd{(\cos\theta)}{t}}_{\rm tide} \rd{\phi}{t},\\
    \Delta H_{\pm} &= \mp\frac{1}{t_{\rm al}}
        \int\limits_0^{2\pi} \sin^2\theta\;\mathrm{d}\phi.
\end{align}
In the first line, Hamilton's equations ensure that the non-tidal contributions
to $\rdil{H}{t}$ vanish. Thus, if we evaluate $H$ every time that a trajectory
originating in zone III crosses $\phi = 0$, we see that it will initially be $<
H_{\rm sep}$ and increase for each circulation cycle until the system encounters
the separatrix. At the beginning of the separatrix-crossing orbit, the initial
value of $H$, denoted $H_{\rm i}$, must be greater than $H_{\rm sep} - \Delta H$
to encounter the separatrix on the current orbit, and must be less than $H_{\rm
sep}$, otherwise the trajectory will be in zone II already. We thus have
\begin{equation}
    H_{\rm i} \in \s{ H_{\rm sep} - \Delta H_-,  H_{\rm sep} }
        \label{eq:Hi_range},
\end{equation}
The values of $\cos \theta$ corresponding to the lower and upper bounds on this
range are shown as the black and red dots on the left of
Fig.~\ref{fig:toy_hop_manifolds} respectively.

During the separatrix-crossing orbit, the trajectory first evolves
approximately along $\mathcal{C}_-$ and then along $\mathcal{C}_+$, after which
the final value of $H$, denoted $H_{\rm f}$ is approximately equal to
\begin{equation}
    H_{\rm f} = H_{\rm i} + \Delta H_+ + \Delta H_-.
\end{equation}
There can then be two outcomes depending on the value of $H_{\rm f}$:
\begin{itemize}
    \item If $H_{\rm f} > H_{\rm sep}$, then the trajectory enters the
        separatrix, following the red shaded region in
        Fig.~\ref{fig:toy_hop_manifolds}, and executes a zone III $\to$ II
        transition.

    \item If $H_{\rm f} < H_{\rm sep}$, then the trajectory exits the separatrix
        above CS4, following the yellow shaded region in
        Fig.~\ref{fig:toy_hop_manifolds} and executes a zone III $\to$ I
        transition.
\end{itemize}
These two possibilites can be re-expressed in terms of $H_{\rm i}$: we find that
if $H_{\rm i}$ is in the interval $\s{H_{\rm sep} - \Delta H_-, H_{\rm sep} -
\Delta H_{-} - \Delta H_+}$, then the system executes a III $\to$ I transition,
and if it is in the interval $\s{H_{\rm sep} - \Delta H_- - \Delta H_+, H_{\rm
sep}}$, then the system executes a III $\to$ II transition. We see that there is
a critical value of $H_{\rm i}$ in the interval given by
Eq.~\eqref{eq:Hi_range}, where $H_{\rm i} = H_{\rm sep} - \Delta H_{-} - \Delta
H_+$, that separates the two possible outcomes of the separatrix encounter
within the interval given by Eq.~\eqref{eq:Hi_range}. The value of $\cos \theta$
for which $H$ is equal to $H_{\rm sep} - \Delta H_{-} - \Delta H_+$ is shown as
the blue dot on the left of Fig.~\ref{fig:toy_hop_manifolds}. Finally, if the
alignment torque is weak, then $\Delta H_- \sim \mathcal{O}\p{t_{\rm al}^{-1}}$
is small compared to any variation in the value of $H$ (e.g.\ when changing
$\phi_0$ or $\theta_0$ by a small amount), and $H_{\rm i}$ can be effectively
considered as randomly chosen from a uniform distribution over $\s{H_{\rm sep} -
\Delta H_-, H_{\rm sep}}$. As a consequence we obtain the probability of a III
$\to$ II transition:
\begin{equation}
    P_{\rm III \to II} = \frac{\Delta H_- + \Delta H_+}{\Delta H_-}.
        \label{eq:def_P32_toy}
\end{equation}

\begin{figure}
    \centering
    \includegraphics[width=\columnwidth]{../initial/0_eta/6manifolds0_20.png}
    \caption{Plot illustrating the probabilistic origin of separatrix capture
    for $\eta = 0.2$, $I = 20^\circ$, and $\abs{gt_{\rm al}} = 10^{3}$. Orange
    regions converge to CS1, and green to CS2, while CS4 is labeled with the
    purple dots. The boundaries separating the CS1 and CS2-approaching regions
    consist of the critical trajectories (labeled in the legend) that, when
    evolved either forwards or backwards in time (superscripts in the legend
    labels), asymptotically approach CS4 where either $\phi = 0^\circ$ or $\phi
    = 360^\circ$ (subscripts in the legend labels). Within zone III, the regions
    of phase space reaching CS1 and CS2 both become very thin (shown
    qualitatively as the light green lines of decreasing width), reflecting the
    fact that the outcome for a particular initial condition can be approximated
    as probabilistic sufficiently far from the separatrix. Labeled in the blue
    and black dots at the left edge of the plot are the two backwards-in-time
    critical trajectories intersect $\phi = 0$, and are where $H$ is equal to
    $H_{\rm sep} - \Delta H_- - \Delta H_+$ and $H_{\rm sep} - \Delta H_-$
    respectively (Eqs.~\ref{eq:def_Hsep} and~\ref{eq:def_dHpm}).
    }\label{fig:toy_hop_manifolds}
\end{figure}

To actually evaluate Eq.~\eqref{eq:def_P32_toy}, we can use the parameterization
for the separatrix given in Paper I for $\eta \ll 1$:
\begin{equation}
    \p{\cos \theta}_{\mathcal{C}_{\pm}} \approx
        \eta \cos I \pm \sqrt{2\eta\sin I\p{1 - \cos \phi}}.
        \label{eq:sep_theta}
\end{equation}
It can then be shown that
\begin{align}
    \Delta H_- &\approx \frac{2\pi}{t_{\rm al}}\p{1
        - 2\eta \sin I} + \mathcal{O}(\eta^{3/2}),\\
    \Delta H_+ + \Delta H_- &\approx
        \frac{32 \eta^{3/2}\cos I \sqrt{\sin I}}{t_{\rm al}}
            + \mathcal{O}\p{\eta^{5/2}},\\
    P_{\rm III \to II} &\approx
        \frac{16 \eta^{3/2} \cos I \sqrt{\sin I}}{\pi
            \p{1  - 2\eta \sin I}}.\label{eq:P32_toy}
\end{align}

To directly compare Eq.~\eqref{eq:P32_toy} with numerical simulation, we perform
numerical integrations of Eqs.~(\ref{eq:dqdt_toy}--\ref{eq:dfdt_toy}) while
restricting our attention only initial conditions in zone III\@.
In Fig.~\ref{fig:1hist_toy}, we display Eq.~\eqref{eq:P32_toy} alongside the
computed $P_{\rm III \to II}$ using $1000$ initial conditions in zone III for
each of $60$ values of $\eta$. Excellent agreement is observed.
\begin{figure}
    \centering
    \includegraphics[width=\columnwidth]{../initial/3_toy3/1hist_toy.png}
    \caption{Plot of $P_{\rm III \to II}$ as a function of $\eta$ for the
    constant alignment torque model considered in Section~\ref{s:toy_model}. For
    each $\eta$, $1000$ initial $\theta_0$ in zone III are evolved until just
    after separatrix encounter, where the outcome of the encounter is recorded.
    Shown in red is Eq.~\eqref{eq:P32_toy}.}\label{fig:1hist_toy}
\end{figure}

Finally, we remark that the calculation above is just an of \emph{Melnikov's
Method} \citep{g_and_h}. Melnikov's Method is a general calculation that gives
the degree of splitting of a ``homoclinic orbit'' (here, the separatrix) induced
by a small, possibly time-dependent, perturbation. The explicit connection
between the evolution of the unperturbed Hamiltonian and distances between
curves in phase space (as seen in Fig.~\ref{fig:toy_hop_manifolds}) is provided by
Melnikov's Method.

\section{Spin Evolution with Weak Tidal Friction}\label{s:full_tide_prob}

\subsection{Tidal Model: Equilibrium Tides}\label{ss:weaktide}

Having understood the effect of introducing an additional dissipative torque to
the Colombo's Top system, we now implement a realistic model of tidal
dissipation. We use the weak friction theory of equilibrium tides
\citep{lai2012}. In this model, tides cause both the spin orientation $\uv{s}$
and frequency $\Omega_{\rm s}$ to evolve on the characteristic tidal timescale
$t_{\rm s}$ following:
\begin{align}
    \p{\rd{\uv{s}}{t}}_{\rm tide} &= \frac{1}{t_{\rm s}}
                \s{\frac{2n}{\Omega_{\rm s}} - \p{\uv{s} \cdot \uv{l}}}
                    \uv{s} \times \p{\uv{l} \times \uv{s}}\label{eq:dsdt_tide},\\
    \frac{1}{\Omega_{\rm s}}\p{\rd{\Omega_{\rm s}}{t}}_{\rm tide}
        &= \frac{1}{t_{\rm s}} \s{\frac{2n}{\Omega_{\rm s}}\p{\uv{s} \cdot
            \uv{l}} - 1 - \p{\uv{s} \cdot \uv{l}}^2},\label{eq:dWsdt_tide}
\end{align}
where $t_{\rm s}$ is given by
\begin{equation}
    \frac{1}{t_{\rm s}} \equiv \frac{1}{4k}
        \frac{3k_2}{Q}\p{\frac{M_\star}{m}}\p{\frac{R}{a}}^3 n.
\end{equation}
In this section, we neglect orbital evolution in this section since the time
scale is longer than $t_{\rm s}$ by a factor of $\sim L / S \gg 1$, and so
$t_{\rm s}$ is a constant. We will continue mostly consider the case where tidal
dissipation is slow, i.e.\ $\abs{g}t_{\rm s} \gg 1$. The full equations of
motion including weak tidal friction can be written in component form as
\begin{align}
    \rd{\theta}{t} &= g\sin I \sin \phi -
        \frac{1}{t_{\rm s}}\sin \theta\p{\frac{2n}{\Omega_{\rm s}} - \cos \theta}
            ,\label{eq:ds_fullq}\\
    \rd{\phi}{t} &= -\alpha\cos\theta
        - g\p{\cos I + \sin I \cot \theta \cos \phi}\label{eq:ds_fullphi},\\
    \frac{1}{\Omega_{\rm s}}\rd{\Omega_{\rm s}}{t}
        &= \frac{1}{t_{\rm s}} \s{\frac{2n}{\Omega_{\rm s}} \cos \theta
            - \p{1 + \cos^2\theta}}\label{eq:ds_fulls}.
\end{align}
The conditions for which the two tidal terms vanish are given respectively by:
\begin{align}
    \dot{\theta}_{\rm tide} = 0 &: \quad \frac{2n}{\Omega_{\rm s}} = \cos \theta,
        \label{eq:weaktide_dqzero}\\
    \dot{\Omega}_{\rm s} = 0 &: \quad \frac{n}{\Omega_{\rm s}}
        = \frac{1 + \cos^2\theta}{2\cos \theta}.\label{eq:weaktide_dWszero}
\end{align}

To understand the long-term behaviors of this system, we first consider its
behavior near a CS\@. Specifically, we wish to understand whether initial
conditions near a CS stay near the CS as the evolution of $\Omega_{\rm s}$
causes the CSs to evolve. We first note that the evolution of
$\Omega_{\rm s}$ does not drive spins towards or away from CSs: As long as it
evolves sufficiently slowly (adiabatically, Paper I), conservation of
phase space area ensures that trajectories will remain at a roughly fixed
distance to stable equilibria of the system. Thus, Eq.~\eqref{eq:dsdt_tide}
alone determines whether a point evolves towards or away from a nearby CS as
$\Omega_{\rm s}$ evolves. Then, evaluating Eq.~\eqref{eq:lambda_approx} with
$t_{\rm al} = t_{\rm s} / \p{2n / \Omega_{\rm s} - \cos \theta}$ (compare
Eqs.~\ref{eq:dsdt_tide_toy} and~Eq.~\ref{eq:dsdt_tide}), we see that CS2 is
still always stable, while CS1 is becomes unstable for $\Omega_{\rm s} > 2n$.

Using this result, we can then identify the long-term equilibria of the system
when $\Omega_{\rm s}$ is also evolving: these equilibria of the system must both
satisfy $\dot{\Omega}_{\rm s} = 0$ and be a CS that is stable in the presence of
tidal dissipation (i.e.\ satisfying $\rdil{\uv{s}}{t} = 0$).
Figure~\ref{fig:6equils006} describes the evolution of
Eqs.~(\ref{eq:dsdt_tide}--\ref{eq:dWsdt_tide}) qualitatively in the coordinates
$(\Omega_{\rm s}, \theta)$, along with the locations of CS1 and CS2. The circled
points in Fig.~\ref{fig:6equils006} satisfy the criteria to be long-term
equilibria, and we call them \emph{tidal Cassini Equilibria} (tCE). We number
tCE1 and tCE2 the tCE that are in CSs 1 and 2 respectively. The obliquities of
the tCE depend on the system architecture, which can be quantified using the
parameter
\begin{align}
    \eta_{\rm sync} &\equiv \big(\eta\big)_{\Omega_{\rm s} = n}
        = \eta \frac{\Omega_{\rm s}}{n}\nonumber\\
        &= \frac{1}{2}\frac{k}{k_{\rm q}}
            \frac{m_{\rm p}m}{M_\star^2}
            \p{\frac{a}{a_{\rm p}}}^3 \p{\frac{a}{R}}^3\cos(I).
            \label{eq:def_etasync}
\end{align}
In Fig.~\ref{fig:6equils006}, $\eta_{\rm sync} = 0.06$;
Figs.~\ref{fig:6equils050}--\ref{fig:6equils070} illustrate the change to the
tCE when $\eta_{\rm sync} = 0.5$ and $0.7$ respectively.

The tCE obliquity as a function of $\eta_{\rm sync}$ are shown for $I =
20^\circ$ in the top panel of Fig.~\ref{fig:probs20} and for $I = 5^\circ$ in
the top panel of Fig.~\ref{fig:probs5}. In fact, an analytical expression for
the tCE2 obliquity for $\eta_{\rm sync} \ll \eta_{\rm c}$ can be computed using
Eqs.~(\ref{eq:weaktide_dWszero}--\ref{eq:def_etasync}) and that $\cos \theta_2
\approx \eta \cos I$ to yield:
\begin{align}
    \cos \theta_{\rm tCE2} &\approx \sqrt{\frac{\eta_{\rm sync} \cos I}{2}},
        &
    \frac{\Omega_{\rm s, tCE2}}{n} &\approx
        \sqrt{2\cos I \eta_{\rm sync}}. \label{eq:def_tce2_approx}
\end{align}
This approximation for $\theta_{\rm tCE2}$ is shown as the black dashed line in
the top panels of Fig.~\ref{fig:probs20} and~\ref{fig:probs5}, where good
agreement is observed.
\begin{figure*}
    \centering
    \includegraphics[width=\textwidth]{../initial/1_weaktide/6equils0_06.png}
    \caption{Schematic depiction of the effect of tidal friction on the planet's
    spin for $I = 20^\circ$, corresponding to $\eta_{\rm c} \approx 0.574$
    (Eq.~\ref{eq:def_etac}), and $\eta_{\rm sync} = 0.06$. The black and blue
    lines denote where the tidal $\dot{\Omega}_{\rm s}$ and $\dot{\theta}$
    change signs (Eqs.~\ref{eq:weaktide_dqzero}--\ref{eq:weaktide_dWszero}). The
    orange and green lines give the CS1 and CS2 obliquities respectively, which
    are the two CSs that are stable under the effect of tidal dissipation. Note
    that when $\dot{\theta}_{\rm tide} > 0$, CS1 becomes unstable, denoted by
    the dashed orange line. The points that are both CSs and satisfy
    $\dot{\Omega}_{\rm s} = 0$ are the tidal Cassini Equilibria (tCE), which are
    circled and labeled. The various colored crosses and their associated
    colored lines represent a few characteristic examples of the evolution of
    Colombo's Top under weak tidal friction (for illustrative purposes, we have
    used $\abs{g}t_{\rm s} = 10^2$ and evolved each example for $5t_{\rm s}$).
    The phase space evolution of the two thicker evolutionary trajectories (cyan
    and pink; those beginning at $\theta_{\rm i} = 120^\circ$) are shown in
    Figs.~\ref{fig:trajs1}--\ref{fig:trajs2}.
    }\label{fig:6equils006}
\end{figure*}
\begin{figure*}
    \centering
    \includegraphics[width=\textwidth]{../initial/1_weaktide/6equils0_50.png}
    \caption{Same as Fig.~\ref{fig:6equils006} but for $\eta_{\rm sync} = 0.5$.
    The light-colored crosses and lines correspond to evolutionary trajectories
    using the same initial conditions as those shown in
    Fig.~\ref{fig:6equils006}.}\label{fig:6equils050}
\end{figure*}
\begin{figure*}
    \centering
    \includegraphics[width=\textwidth]{../initial/1_weaktide/6equils0_70.png}
    \caption{Same as Figs.~\ref{fig:6equils006} and~\ref{fig:6equils050} but for
    $\eta_{\rm sync} = 0.7$. Note that $\eta_{\rm sync} = 0.7 > \eta_{\rm c}
    \approx 0.574$ and tCE1 does not exist. The phase space evolution of the
    thick purple trajectory in the bottom panel (starting at $\theta_{\rm i}
    = 10^\circ$) is shown in Fig.~\ref{fig:trajs3}. }\label{fig:6equils070}
\end{figure*}
\begin{figure}
    \centering
    \includegraphics[width=\columnwidth]{../initial/1_weaktide/6equils_sims/0_06_2.png}
    \caption{Phase space evolution of the trajectory in
    Fig.~\ref{fig:6equils006} colored in pink, for which $\eta_{\rm sync} =
    0.06$ and $I = 20^\circ$ (for which $\eta_{\rm c} = 0.574$). The initial
    conditions are $\Omega_{\rm s, i} = 2.5n$, $\theta_{\rm i} = 120^\circ$, and
    $\phi_{\rm i} = 0^\circ$, and we have used $\abs{g}t_{\rm s} = 10^2$ and
    have evolved the system for $5t_{\rm s}$. In the left-most panel, the
    trajectory's evolution in $\p{\Omega_{\rm s}, \theta}$ coordinates along
    with the curves indicating CS1, 2, and $\dot{\Omega}_{\rm s} = 0$ are
    re-displayed from Fig.~\ref{fig:6equils006}. The vertical dashed lines
    denote the values of $\Omega_{\rm s}$ for which the phase space evolution of
    the system is displayed in the right four panels. In each of these right
    four panels, the trajectory's evolution for a single circulation/libration
    cycle is displayed in $\p{\cos \theta, \phi}$ phase space for the labeled
    value of $\Omega_{\rm s}$. The system encounters the separatrix, undergoes a
    III $\to$ I transition, and converges to tCE1.
    }\label{fig:trajs1}
\end{figure}
\begin{figure}
    \includegraphics[width=\columnwidth]{../initial/1_weaktide/6equils_sims/0_06_3.png}
    \caption{Same as Fig.~\ref{fig:trajs1} but for $\phi_{\rm i} = 286^\circ$,
    corresponding to the trajectory in Fig.~\ref{fig:6equils006} colored in
    cyan. The system encounters the separatrix, undergoes a III $\to$ II
    transition, and converges to tCE2. The small $\phi$ offset of tCE2 is
    visible (see Fig.~\ref{fig:mcs}). Note that the initial condition of this
    system and that displayed in Fig.~\ref{fig:trajs1} have the same initial
    $\theta_{\rm i}$ and $\Omega_{\rm s, i}$ and only different precessional
    phases $\phi_{\rm i}$.
    }\label{fig:trajs2}
\end{figure}
\begin{figure}
    \includegraphics[width=\columnwidth]{../initial/1_weaktide/6equils_sims/0_70_0.png}
    \caption{Same as Fig.~\ref{fig:trajs1} but for $\eta_{\rm sync} = 0.7$ and
    $\theta_{\rm i} = 10^\circ$, corresponding to the purple trajectory shown in
    the bottom panel of Fig.~\ref{fig:6equils070}. Here, the system evolves
    along CS1 until the separatrix disappears, upon which it experiences large
    obliquity variations that damp due to tidal dissipation. The highly
    asymmetric shape in the third panel arises due to the strong tidal
    dissipation used in this simulation. The system finally converges to tCE2,
    the only tCE that exists for this value of $\eta_{\rm sync}$ (see bottom
    panel of Fig.~\ref{fig:6equils070}).}\label{fig:trajs3}
\end{figure}

There are two important conditions that can change the existence and stability
of the tCE\@. First, if $\eta_{\rm sync} > \eta_{\rm c}$, then tCE1 will not
exist, as can be seen in Fig.~\ref{fig:6equils070}\footnote{Strictly speaking,
$\eta_{\rm sync}$ can be slightly smaller than $\eta_{\rm c}$, as the planet's
spin is slightly subsynchronous at tCE1 when $\eta \approx \eta_{\rm c}$, e.g.\
see Fig.~\ref{fig:6equils050}.}. Secondly, tCE2 may not be stable if the tidal
phase shift is too large. Applying the results of Section~\ref{ss:tidal_equils},
we find that tCE2 is stable as long as $t_{\rm s} \geq t_{\rm s, c}$ where
\begin{equation}
    t_{\rm s, c} \equiv \frac{\sin \theta_{\rm tCE2}}{\abs{g} \sin
            I}\p{\frac{2n}{\Omega_{\rm s, tCE2}} - \cos \theta_{\rm tCE2}}.
\end{equation}
The tCE2 subscripts denote evaluation at tCE2. When $\eta_{\rm sync} \ll
\eta_{\rm c}$, we can use Eq.~\eqref{eq:def_tce2_approx} to further simplify:
\begin{equation}
    t_{\rm s, c}\approx \frac{1}{\abs{g}\sin I}\sqrt{\frac{2}{\eta_{\rm sync}
        \cos I}}. \label{eq:def_ts_crit}
\end{equation}
Note that this result differs in form from that given in
\citet{fabrycky_otides}, which assume that the orbital precession of $\uv{l}$ is
driven by the central star. Our expression is simpler and is applicable to
systems where the orbital precession of the inner planet is dominated by an
exterior planet.

\subsection{Spin Obliquity Evolution as a Function of Initial Spin Orientation}

With this result, we can now consider the final fate of the inner planet's spin.
We begin by qualitatively examining the example trajectories shown in
Figs.~\ref{fig:6equils006}, for which we have integrated
Eqs.~(\ref{eq:dsdt_rot},~\ref{eq:dsdt_tide}--\ref{eq:dWsdt_tide}) and set $I =
20^\circ$. We discuss each of the six trajectories in turn:
\begin{itemize}
    \item The trajectory with initial conditions $\Omega_{\rm s, i} = 2.5n$ and
        $\theta_{\rm i} = 10^\circ$ (purple) has an initially prograde spin
        (i.e.\ in zone I, see Fig.~\ref{fig:1contours}) and directly evolves to
        tCE1 as the obliquity damps to zero.

    \item The trajectory with initial conditions $\Omega_{\rm s, i} = 2.5n$ and
        $\theta_{\rm i} = 90^\circ$ (red) has an initial condition inside the
        resonance (zone II) and evolves to tCE2. Note that the obliquity is
        trapped in a high value due to the stability of CS2 under weak tidal
        friction, as discussed in Section~\ref{ss:weaktide}.

    \item We have chosen two trajectories, both with initial conditions
        $\Omega_{\rm s, i} = 2.5n$ and $\theta_{\rm i} = 120^\circ$ but with
        different initial precessional phases $\phi_{\rm i}$; we first discuss
        the pink trajectory, for which $\phi_{\rm i} = 0^\circ$. This trajectory
        originates in zone III, evolves towards the separatrix as tidal friction
        damps the obliquity, and crosses the resonance without being captured,
        upon which the obliquity continues to damp until the system converges to
        tCE1. The detailed phase space evolution of this trajectory is shown in
        Fig.~\ref{fig:trajs1}, where the outcome of the separatrix encounter is
        very visible.

    \item The other trajectory with initial conditions $\Omega_{\rm s, i} = 2.5n$
        and $\theta_{\rm i} = 120^\circ$ (light blue) only differs from the pink
        trajectory as its initial precessional phase is $\phi_{\rm i} \approx
        286^\circ$. Nevertheless, this trajectory originates in zone III,
        encounters the separatrix and is captured into the resonance (zone II),
        upon which tidal friction drives the system towards tCE2. The detailed
        phase space evolution of this trajectory is shown in
        Fig.~\ref{fig:trajs2}, where the resonance capture is displayed. Also
        visible in the final panel of Fig.~\ref{fig:trajs2} is the slight phase
        offset of CS2, i.e.\ $\phi_{\rm cs} < 180^\circ$, in agreement with the
        discussion in Section~\ref{ss:tidal_equils} (see Fig.~\ref{fig:mcs}).

    \item For completeness, we also examine some trajectories for initially
        subsynchronous spin rates. The trajectory with initial conditions
        $\Omega_{\rm s, i} = 0.1n$ and $\theta_{\rm i} = 35^\circ$ (blue) has its
        obliquity rapidly damped to zero by tidal friction as it spins up to
        spin-orbit synchronization, eventually converging to tCE1.

        A subtlety of initially subsynchronous spins can be seen here: note that
        the initial $\eta_{\rm i} = 0.6$ here, so the separatrix and CS1 do not
        exist initially ($\eta_{\rm c} = 0.574$). As such, naively, one expects
        initial convergence to CS2 and subsequent obliquity along with CS2 as
        the spin increases. However, due to the strong tidal dissipation and the
        proximity of $\eta_{\rm i}$ to $\eta_{\rm c}$, CS1 appears within a
        single circulation cycle, and the obliquity quickly damps to, and
        continues to evolve along, CS1.

    \item The trajectory with initial conditions $\Omega_{\rm s, i} = 0.1n$ and
        $\theta_{\rm i} = 100^\circ$ (teal) also has its obliquity rapidly
        damped to zero as it approaches spin-orbit synchronization and tCE1. For
        these initial conditions, we remark that this initial condition
        converges to tCE2 when a tidal dissipation rate of $\abs{g}t_{\rm s} =
        10^3$ is used, satisfying the intuitive analysis given in the previous
        paragraph.
\end{itemize}
For each of these six initial conditions, we also display their evolutionary
trajectories for $\eta_{\rm sync} = 0.5$ and $\eta_{\rm sync} = 0.7$ in
Figs.~\ref{fig:6equils050}--\ref{fig:6equils070}. The qualitative evolution of
these six examples only changes in a few important ways, so we will discuss just
a few points of interest:
\begin{itemize}
    \item For both $\eta_{\rm sync} = 0.5$ and $\eta_{\rm sync} = 0.7$, we see
        that both initial conditions with $\theta_{\rm i} = 120^\circ$ converge
        to tCE2. In fact, for these values of $\eta_{\rm sync}$, all initial
        conditions with $\theta_{\rm i} = 120^\circ$ will converge to tCE2
        regardless of $\phi_{\rm i}$.

    \item The two subsynchronous initial conditions both evolve to tCE2 for both
        $\eta_{\rm sync} = 0.5$ and $\eta_{\rm sync} = 0.7$, as in both cases
        $\eta_{\rm i} \gg \eta_{\rm c}$ and CS2 is the only low-obliquity spin
        equilibrium. The system then continues to evolve along CS2 until
        reaching a state satisfying $\dot{\Omega}_{\rm s} = 0$ as well.

    \item Of particular interest is the evolution starting from the initial
        conditions $\Omega_{\rm s, i} = 2.5n$ and $\theta_{\rm i} = 10^\circ$
        (purple) when $\eta_{\rm sync} = 0.7$. Figure~\ref{fig:trajs3} shows the
        detailed phase space evolution of this system, where it can be seen that
        the system initially evolves along the stable CS1, but is ejected when
        $\Omega_{\rm s}$ is sufficiently small that CS1 ceases to exist, upon
        which large obliquity variations eventually lead to convergence to tCE2,
        the only tCE that exists.
\end{itemize}
It can be seen that the dynamics are complex and can vary greatly depending on
the parameters used. In the case where the initial spin is subsynchronous, the
detailed outcome depends sensitively on $\eta_{\rm i}$ and the tidal dissipation
rate. For tractability, we restrict our subsequent discussion to the more
astrophysically common regime $\Omega_{\rm s, i} \gg n$, and we adopt the
fiducial value $\Omega_{\rm s, i} = 10n$.

Having developed an intuition for a few different possible evolutionary
histories, we can next attempt to draw general conclusions about the final fate
of the inner planet's spin as a function as a function of its initial
conditions. We do this by again integrating
Eqs.~(\ref{eq:dsdt_rot},~\ref{eq:dsdt_tide}--\ref{eq:dWsdt_tide}) for many
initial $\theta_0$ and $\phi_0$ and examining which tCE a given initial
condition evolves into. For this study, we use a more gradual tidal dissipation
rate of $\abs{gt_{\rm s}} = 10^3$. In Fig.~\ref{fig:Hhists_0_06}, we show the
final outcome for many randomly chosen $\theta_0$ and $\phi_0$ for $\eta_{\rm
sync} = 0.06$ and $I = 20^\circ$. We see that the behavior seen in the example
trajectories in Fig.~\ref{fig:6equils006} are general: tCE1 is generally reached
for spins initially in zone I (like the purple trajectory in
Fig.~\ref{fig:6equils006}), tCE2 is generally reached for spins initially in
zone II (like the red trajectory in Fig.~\ref{fig:6equils006}), and a
probabilistic outcome is observed for spins initially in zone III (like the
light blue and pink trajectories in Fig.~\ref{fig:6equils006}).
Figures~\ref{fig:Hhists_0_20} and~\ref{fig:Hhists_0_50} show the same results
but for $\eta_{\rm sync} = 0.2$ and $\eta_{\rm sync} = 0.5$. As $\eta_{\rm
sync}$ is increased, more initial conditions reach tCE2. This is both because
there are more systems initially in zone II and because systems initially in
zone III have a higher probability of executing a III $\to$ II transition upon
separatrix encounter. Note also that in Fig.~\ref{fig:Hhists_0_50}, even initial
conditions in zone I are able to reach tCE2; we comment on the origin of this
behavior in the next section.
\begin{figure}
    \centering
    \includegraphics[width=\columnwidth]{../initial/1_weaktide/5Hhists0_06_20.png}
    \caption{\emph{Left:} Each dot indicates which tCE a given initial condition
    $(\theta_{\rm i}, \phi_{\rm i})$ evolve towards (labeled in legend), for
    $\eta_{\rm sync} = 0.06$ and $I = 20^\circ$. The planet's initial spin is
    chosen to be $\Omega_{\rm s, i} = 10n$, and the three zones of phase space
    (see Fig.~\ref{fig:1contours}) for the initial conditions are labeled,
    separated by the separatrix (black line). Note that initial conditions in
    zone I evolve towards tCE1, initial conditions in zone II evolve towards
    tCE2, and initial conditions in zone III have an effectively probabilistic
    outcome. \emph{Right:} Histogram of which tCE a given initial obliquity
    $\theta_{\rm i}$ evolves towards, averaged over $\phi_{\rm
    i}$.
    \textcolor{red}{NB\@: the excess of tCE2 initial conditions at $\cos
    \theta_{\rm i} \approx -1$ is likely an artifact of loss of numerical
    precision, I am working on fixing this.}
    }\label{fig:Hhists_0_06}
\end{figure}
\begin{figure}
    \centering
    \includegraphics[width=\columnwidth]{../initial/1_weaktide/5Hhists0_20_20.png}
    \caption{Same as Fig.~\ref{fig:Hhists_0_06} but for $\eta_{\rm sync} =
    0.2$.
    \textcolor{red}{NB\@: the excess of tCE2 initial conditions at $\cos
    \theta_{\rm i} \approx -1$ is likely an artifact of loss of numerical
    precision, I am working on fixing this.}
    }\label{fig:Hhists_0_20}
\end{figure}
\begin{figure}
    \centering
    \includegraphics[width=\columnwidth]{../initial/1_weaktide/5Hhists0_50_20.png}
    \caption{Same as Fig.~\ref{fig:Hhists_0_06} but for $\eta_{\rm sync} =
    0.5$. Note that even points above the separatrix can evolve towards tCE2
    here.}\label{fig:Hhists_0_50}
\end{figure}

\subsubsection{Analytical Calculation of Resonance Capture Probability
}\label{ss:phop_weaktide}

Even when including the evolution of $\Omega_{\rm s}$, and therefore the
spin-orbit precession frequency $\alpha$, the probabilities of the III $\to$ I
and III $\to$ II transitions upon separatrix encounter can still be computed.
The calculation resembles that presented in Section~\ref{ss:toy_outcomes} but is
more involved. We describe the calculation below.

In Section~\ref{ss:toy_outcomes}, we found that the evolution of $H$, the value
of the unperturbed Hamiltonian, allowed us to calculate the probabilities of the
various outcomes of separatrix encounter. Specifically, the outcome upon
separatrix encounter is determined by the value of $H$ at the start of the
separatrix-crossing orbit relative to $H_{\rm sep}$, the value of $H$ along the
separatrix. However, when the spin is also evolving, $H_{\rm sep}$ is also
changing during the separatrix-crossing orbit, and the discussion in
Section~\ref{ss:toy_outcomes} must be generalized to account for this. Instead
of focusing on the evolution of $H$ along a trajectory, we instead follow the
evolution of
\begin{equation}
    K \equiv H - H_{\rm sep}.
\end{equation}
Note that $K > 0$ inside the separatrix, and $K < 0$ outside. Then, the outcome
of the separatrix-crossing orbit is largely the same as discussed in
Section~\ref{ss:toy_outcomes}. First, we must compute the change in $K$ along
the legs of the separatrix. We define $\Delta K_{\pm}$ by generalizing
Eq.~\eqref{eq:def_dHpm} in the natural way:
\begin{align}
    \Delta K_{\pm} &= \oint_{\mathcal{C}_{\pm}} \rd{H}{t}
        - \rd{H_{\rm sep}}{t}\;\mathrm{d}t.\label{eq:def_dK_weaktide}
\end{align}
Here, however, note that the contours $\mathcal{C}_{\pm}$ depends on the value
of $\Omega_{\rm s}$ at separatrix encounter. Since there is no closed form
solution for $\Omega_{\rm s}(t)$, the probabilities of the various outcomes
cannot be expressed as a simple function of the initial conditions.

Continuing the argument presented in Section~\ref{ss:toy_outcomes}, we consider
the outcome of the separatrix-crossing orbit as a function of $K_i$, the value
of $K$ at the start ($\phi = 0$) of the separatrix-crossing orbit. We find that
if $K_i > -\Delta K_+ - \Delta K_-$, then the system undergoes a III $\to$ II
transition and eventual evolution towards tCE2; while if $-\Delta K_- < K_i <
-\Delta K_- - \Delta K_+$, then the system undergoes a III $\to$ I transition
and ultimate evolution towards tCE1. Thus, we find that the probability of a III
$\to$ II transition is given by
\begin{equation}
    P_{\rm III \to II} = \frac{\Delta K_+ + \Delta K_-}{\Delta
        K_-}.\label{eq:def_pc_weaktide}
\end{equation}
Again, since $\Delta K_{\pm}$ are evaluated at resonance encounter, and
$\Omega_{\rm s}$ is evolving, there is no way to express $\Delta K_{\pm}$ as a
closed form of initial conditions. In fact, since many resonance encounters
occur when $\eta$ is substantial, even an approximate calculation of $\Delta
K_{\pm}$ using Eq.~\eqref{eq:sep_theta} is inaccurate, and we instead calculate
$\Delta K_{\pm}$ along the numerically-calculated $\Delta K_{\pm}$ for arbitrary
$\eta$. Note that Eqs.~(\ref{eq:def_dK_weaktide},~\ref{eq:def_pc_weaktide}) are
equivalent to the separatrix capture result of \citet{henrard1982} when
$\dot{\theta}_{\rm tide} = 0$ \citep{henrard1987}. Here, we have argued that
this classic calculation can be unified with the calculation given in
Section~\ref{ss:toy_outcomes} to give an accurate prediction of separatrix
encounter outcome probabilities in the presence of both a dissipative
perturbation and a parametric evolution of the Hamiltonian.

To validate the accuracy of Eq.~\eqref{eq:def_pc_weaktide}, we can compare with
direct numerical integration of
Eqs.~(\ref{eq:dsdt_rot},~\ref{eq:dsdt_tide}--\ref{eq:dWsdt_tide}) for many
initial conditions while evaluating $P_{\rm III \to II}$ (and thus also
obtaining $P_{\rm III \to I} = 1 - P_{\rm III \to II}$) for each simulation at
the moment it encounters the separatrix, if it does so. If the theory is
correct, the total numbers of systems converging to each of tCE1 and tCE2 should
be equal to those predicted by the sums of the calculated probabilities. In
Fig.~\ref{fig:pc_fits_0_06}, we show the agreement of this semi-analytic
procedure with the numerical results displayed earlier in the right panel of
Fig.~\ref{fig:Hhists_0_06}. Good agreement is observed.
Figs.~\ref{fig:pc_fits_0_20} shows the same for Figs.~\ref{fig:Hhists_0_20}.
Mostly satisfactory agreement is observed. Thus, we conclude that the outcomes
of separatrix encounter are accurately predicted by
Eq.~\eqref{eq:def_pc_weaktide}.
\begin{figure}
    \centering
    \includegraphics[width=\columnwidth]{../initial/1_weaktide/5pc_fits0_06_20.png}
    \caption{Comparison of the fraction of systems converging to tCE2 obtained
    via numerical simulation (red dots) and obtained via a semi-analytic
    calculation (blue line) for $\eta_{\rm sync} = 0.06$, $I = 20^\circ$, and
    $\Omega_{\rm s, i} = 10n$ (see right panel of Fig.~\ref{fig:Hhists_0_06}).
    The semi-analytic calculation is performed by numerically integrating
    Eqs.~(\ref{eq:dsdt_rot},~\ref{eq:dsdt_tide}--\ref{eq:dWsdt_tide}) on a grid
    of initial conditions uniform in $\cos \theta_{\rm i}$ and $\phi_{\rm i}$
    until the system reaches the separatrix, then calculating the probability of
    reaching tCE2 for each integration using Eq.~\eqref{eq:def_pc_weaktide}. The
    green dashed line in the top panel shows the result of using the analytical
    expression (Eq.~\ref{eq:app_deltaK}) for $\Delta K_{\pm}$, and the bottom
    panel shows the distribution of values of $\eta_{\rm cross}$, the value of
    $\eta$ when a trajectory starting at $\theta_{\rm i}$ encounters the
    separatrix.
    \textcolor{red}{NB\@: the spike in tCE2 probability at at $\cos
    \theta_{\rm i} \approx -1$, as well as the large $\eta_{\rm cross}$ values,
    is likely an artifact of loss of numerical precision, I am working on fixing
    this.}
    }\label{fig:pc_fits_0_06}
\end{figure}
\begin{figure}
    \centering
    \includegraphics[width=\columnwidth]{../initial/1_weaktide/5pc_fits0_20_20.png}
    \caption{Same as Fig.~\ref{fig:pc_fits_0_06} but for $\eta_{\rm sync} =
    0.2$, corresponding to the right panel of
    Fig.~\ref{fig:Hhists_0_20}.}\label{fig:pc_fits_0_20}
\end{figure}

With the above calculation, we can understand why even initial conditions in
zone I can converge to tCE2. As long as the initial spin is sufficiently large
($\geq 2n$), Eq.~\eqref{eq:dsdt_tide} shows that sufficiently large initial
obliquities are \emph{increased} by weak tidal friction (the bottom right
region of panels of Fig.~\ref{fig:6equils006}). This increase means that even
initial conditions in zone I can sometimes be driven by tidal dissipation into
the separatrix. When this is the case, it can be shown that both $\Delta K_+ >
0$ and $\Delta K_- > 0$. This implies that both III $\to$ II and I $\to$ II
outcomes are guaranteed, and thus initial conditions in all three zones can
evolve into tCE2.

\subsection{Spin Obliquity Evolution as a Function of Precession Strength
}\label{ss:tce2_etasync}

In the previous section, we considered the outcome as a function of the initial
spin orientation, specified by $\theta_0$ and $\phi_0$. In this section, we
consider the probability of evolution into tCE2 when averaging over a
distribution of initial spin orientations, which we denote $P_{\rm tCE2}$. For
simplicity, we just consider $\uv{s}$ to be isotropically distributed; we
discuss the impact of more physically realistic distributions of $\uv{s}$ in
Section~\ref{s:summary}. The bottom panel of Fig.~\ref{fig:probs20} shows
$P_{\rm tCE2}$ for $I = 20^\circ$ as a function of $\eta_{\rm sync}$. It can be
seen that tCE2 is reached with substantial probability ($\sim 50\%$) while it is
also at a substantial obliquity ($\sim 60^\circ$) for $\eta_{\rm sync} \approx
0.4$.
\begin{figure}
    \centering
    \includegraphics[width=\columnwidth]{../initial/1_weaktide/5probs_20.png}
    \caption{\emph{Top:} Obliquities of the two tCE where $I = 20^\circ$ for a
    range of $\eta_{\rm sync}$ (Eq.~\ref{eq:def_etasync}); the blue dashed line
    denotes the analytical approximation Eq.~\eqref{eq:def_tce2_approx} and is
    only displayed in its regime of validity, $\eta_{\rm sync} \ll \eta_{\rm c}$
    (vertical dashed line; Eq.~\ref{eq:def_etac}). \emph{Bottom:} Total
    probability of ending up in tCE2 ($P_{\rm tCE2}$; red dots), averaged over
    an isotropic initial spin orientation and taking $\Omega_{\rm s, i} = 10n$.
    The three shaded regions denote the contribution of initial conditions in
    zones I/II/III (labeled) to the total tCE2 probability in the isotropic
    average. For example, among systems that converge to tCE2 for $\eta_{\rm
    sync} = 0.06$, more originate in zone III than zone II, and none originate
    in zone I.}\label{fig:probs20}
\end{figure}
\begin{figure}
    \centering
    \includegraphics[width=\columnwidth]{../initial/1_weaktide/5probs_5.png}
    \caption{Same as Fig.~\ref{fig:probs20} but for $I =
    5^\circ$.}\label{fig:probs5}
\end{figure}

When $\eta_{\rm sync} \ll 1$ and $\Omega_{\rm s} \gtrsim n$, an approximate
analytical formula for $P_{\rm tCE2}$ can be obtained (see
Appendix~\ref{app:ptce2}):
\begin{align}
    P_{\rm tCE2} &\simeq
            \frac{4\sqrt{\eta _{\rm sync} \sin I}}{\pi}\s{
                \sqrt{n / \Omega_{\rm s, i}}
                + \frac{3}{2\p{1
                + \sqrt{n / \Omega_{\rm s, i}}}}}.\label{eq:app_tce2_p_tot}
\end{align}
Eq.~\eqref{eq:app_tce2_p_tot} is shown in
Figs.~\ref{fig:probs20}--\ref{fig:probs5} as the red dashed lines, where it
agrees well with the numerical results (red dots) for $\eta_{\rm sync} \lesssim
0.4$. To illustrate the predicted values of $P_{\rm tCE2}$ for small $\eta_{\rm
sync}$, we display $P_{\rm tCE2}$ for $\eta_{\rm sync} \in \s{10^{-4}, 0.4}$ for
both $I = 20^\circ$ and $I = 5^\circ$ in Fig.~\ref{fig:anal_ptce}. Note that no
numerical results are presented for $\eta_{\rm sync} \leq 10^{-2}$, as the
spin-orbit coupling becomes very strong and the integration slows dramatically
due to the rapid precession of $\uv{s}$ about $\uv{l}$.
\begin{figure}
    \centering
    \includegraphics[width=\columnwidth]{../initial/1_weaktide/7anal_ptce.png}
    \caption{Plot of $P_{\rm tCE2}$ for $I = 5^\circ$ and $I = 20^\circ$
    (legend) shown on a log-log plot, to emphasize the scaling at small
    $\eta_{\rm sync}$. The crosses are the results of numerical simulations as
    shown in Figs.~\ref{fig:probs20}--\ref{fig:probs5}, the solid lines are
    Eq.~\eqref{eq:app_tce2_p_tot} for $\Omega_{\rm s, i} = 10n$ and the dashed
    lines are for $\Omega_{\rm s, i} = 3n$. \textcolor{red}{Note: The agreement
    with the data are will change slightly once the numerical
    precision issue is resolved.}}\label{fig:anal_ptce}
\end{figure}

\section{Applications}\label{s:applications}

\subsection{Application to Super Earth + Cold Jupiter Systems
}\label{ss:disc_sehj}

Consider a system consisting of an inner Super-Earth (SE) and an exterior cold
Jupiter (CJ); such systems are expected to be abundant \citep{zhu2018super}. A
phase of giant impacts may occur in the formation of such SEs
\citep{inamdar2015formation, izidoro2017breaking}, which can scramble the
initial obliquity distribution into one resembling the isotropic distribution
considered in Section~\ref{ss:tce2_etasync}. For this system, the spin of the
planet approaches a tCE on the timescale
% 1 / (1.5 * 1e-3 * (Msun / (4 Mearth)) * ((2 Rearth) / (0.4 AU))^3 * 2 * pi / (0.4)^(3/2))
% = 3.34e7
\begin{align}
    \frac{1}{t_{\rm s}} ={}& \frac{1}{3 \times 10^7\;\mathrm{yr}}
            \p{\frac{1}{4k}}
            \p{\frac{2k_2/Q}{10^{-3}}}
            \p{\frac{M_\star}{M_{\odot}}}^{3/2}
            \p{\frac{m}{4M_{\oplus}}}^{-1}\nonumber\\
        &\times \p{\frac{R}{2R_{\oplus}}}^3
            \p{\frac{a}{0.4\;\mathrm{AU}}}^{-9/2}.
            \label{eq:sehj_ts}
\end{align}
This occurs well within the age of most SE-CJ systems. On the other hand,
the orbital evolution of the SE occurs on the timescale \citep{lai2012}
% 1 / (1.5 * 1e-3 * (Msun / (4 Mearth)) * ((2 Rearth) / (0.4 AU))^5 * 2 * pi / (0.4)^(3/2))
% = 7.36e14
\begin{align}
    -\frac{\dot{a}}{a} ={}& \frac{3k_2}{Q}\frac{M_\star}{m}
            \p{\frac{R}{a}}^5n \p{1 - \frac{\Omega_{\rm s}}{n}\cos \theta}\nonumber\\
        ={}& \frac{1}{7 \times 10^{14}\;\mathrm{yr}}
            \p{1 - \frac{\Omega_{\rm s}}{n}\cos \theta}
            \p{\frac{2k_2/Q}{10^{-3}}}
            \p{\frac{M_\star}{M_{\odot}}}^{3/2}\nonumber\\
        &\times \p{\frac{m}{4M_{\oplus}}}^{-1}
            \p{\frac{R}{2R_{\oplus}}}^5
            \p{\frac{a}{0.4\;\mathrm{AU}}}^{-13/2}.\label{eq:sehj_adot}
\end{align}
Thus, $a$ does not evolve within the age of the SE-CJ system, and the
approximation where $a$ is held constant is valid. For the same fiducial
parameters, Eq.~\eqref{eq:def_etasync} can be evaluated
% 1/2 * Mjup * (4 * Mearth) / (Msun)^2 * ((0.4 AU) / (2 Rearth))^3 * (0.4 / 5)^3
% = 0.303884
\begin{align}
    \eta_{\rm sync} ={}& 0.303 \cos I
            \p{\frac{k}{k_{\rm q}}}
            \p{\frac{m_{\rm p}}{M_{\rm J}}}
            \p{\frac{m}{4M_{\oplus}}}
            \p{\frac{M_\star}{M_{\odot}}}^{-2}
            \p{\frac{a}{0.4\;\mathrm{AU}}}^{6}\nonumber\\
        &\times \p{\frac{a_{\rm p}}{5\;\mathrm{AU}}}^{-3}
            \p{\frac{R}{2R_{\oplus}}}^{-3}.\label{eq:sehj_etasync}
\end{align}
Here, $M_{\rm J}$ is the mass of Jupiter. We see from Figs.~\ref{fig:probs20}
and~\ref{fig:probs5} that this value of $\eta_{\rm sync}$ gives a high-obliquity
tCE2 with significant probability. Eq.~\eqref{eq:def_ts_crit} shows that tCE2 is
only unstable if $t_{\rm s} \lesssim t_{\rm s, c}$ where
% 1 / (3/8 * ((Mjup^3 * 4 * Mearth) / Msun^4 * (0.4 AU)^12 / ((5 AU)^9 * (2 Rearth)^3))^(1/2) * 2 * pi / (0.4)^(3/2))
% = 2.8189e5
\begin{align}
    \frac{1}{t_{\rm s, c}} ={}& \frac{\sin I \cos^2 I}{3 \times
        10^5\;\mathrm{yr}}
            \frac{k}{k_{\rm q}}
            \p{\frac{m_{\rm p}}{M_{\rm J}}}^{3/2}
            \p{\frac{m}{4 M_{\oplus}}}^{1/2}\nonumber\\
        &\times \p{\frac{M_\star}{M_{\odot}}}^{-3/2}
            \p{\frac{a}{0.4\;\mathrm{AU}}}^6
            \p{\frac{a_{\rm p}}{5\;\mathrm{AU}}}^{-9/2}
            \p{\frac{R}{2 R_{\oplus}}}^{-3/2}.
            \label{eq:sehj_tsc}
\end{align}
Since $t_{\rm s} \gg t_{\rm s, c}$, we see that tCE2 is stable. In
Fig.~\ref{fig:sehj_region}, we show the value of $\eta_{\rm sync}$ in the
regions of $\p{a, a_{\rm p}}$ parameter space that satisfy the stability
condition for tCE2. It can be seen that a generous portion of parameter space is
able to generate and sustain inclined SEs. In summary, we predict that a
significant fraction of SEs with exterior CJ companions can have long-lived,
significant obliquities ($\gtrsim 60^\circ$) due to being trapped in tCE2.
\begin{figure}
    \centering
    \includegraphics[width=\columnwidth]{../initial/99_misc/5sehj_region.png}
    \caption{Depiction of the values of $\eta_{\rm sync}$ for the Super Earth +
    Cold Jupiter systems as a function of $a$ and $a_{\rm p}$ for $I = 20^\circ$
    (top) and $I = 5^\circ$ (bottom). The SE is taken to have
    $m = 4M_{\oplus}$ and $R = 2R_{\oplus}$ while the HJ is taken to have
    $m_{\rm p} = M_{\rm J}$, and we have taken $k \approx k_{\rm q}$. We only
    show the regions satisfying $t_{\rm s} \geq t_{\rm s, c}$ (the stability
    condition for tCE2; Eqs.~\ref{eq:sehj_ts}--\ref{eq:sehj_tsc}). The line
    satisfying $\eta_{\rm sync} = \eta_{\rm c}$ (Eq.~\ref{eq:def_etac}) is shown
    as the black dashed line.}\label{fig:sehj_region}
\end{figure}

\subsection{Application to Ultra-short-period Planet Formation
}\label{ss:disc_usp}

In \citet{millholland2020formation}, the authors consider capture into the CS2
resonance as part of a mechanism to form ultra-short period planets (USPs).
Their proposed mechanism consists of three stages:
\begin{itemize}
    \item Rapid initial guaranteed capture into CS2.

    \item Runaway inward tidal migration due to the simultaneous increase of the
        CS2 obliquity during migration.

    \item Stalling of inward migration when tidal dissipation becomes
        sufficiently strong to destroy CS2.
\end{itemize}
They consider an inner proto-USP with two external perturbing planets, and
imagine that the spin of the planet quickly evolves to its equilibrium value,
i.e.\ satisfying Eq.~\eqref{eq:weaktide_dWszero}. In our language, this
corresponds to immediate evolution into tCE2. In this section, we revisit this
mechanism with the results presented earlier in this paper.

First, a proto-USP evolves from its initial condition into one of the tCE on the
timescale
% 1 / (1.5 * 1e-3 * (Msun / (8 Mearth)) * ((2 Rearth) / (0.035 AU))^3 * 2 * pi / (0.035)^(3/2))
% = 1157
\begin{align}
    \frac{1}{t_{\rm s}} ={}& \frac{1}{1200\;\mathrm{yr}}
            \p{\frac{1}{4k}}
            \p{\frac{2k_2/Q}{10^{-3}}}
            \p{\frac{M_\star}{M_{\odot}}}^{3/2}
            \p{\frac{\rho}{\rho_{\oplus}}}^{-1}
            \p{\frac{a}{0.035\;\mathrm{AU}}}^{-9/2}.
\end{align}
This is much shorter than the age of the system, and so the proto-USP can be
treated as starting in one of the tCE\@.

Next, we investigate the outcome of the initial, rapid spin evolution. For
simplicity, we will first consider the case with a single external perturber.
First, since the considered systems are very compact, we must generalize
Eq.~\eqref{eq:wlp} to small semi-major axis ratios \citep[see e.g.][]{lai_2017}:
\begin{equation}
    \omega_{\rm lp} = \frac{3m_{\rm p}}{4M_{\star}}
        \p{\frac{a}{a_{\rm p}}}^3 n f(\alpha),
        \label{eq:def_g_usp}
\end{equation}
where $\alpha = a / a_{\rm p}$ and
\begin{equation}
    f(\alpha) \equiv \frac{b_{\rm 3/2}^{(1)}(\alpha)}{3\alpha}
        \approx 1 + \frac{15}{8}\alpha^2 + \dots
\end{equation}
where $b_{\rm 3/2}^{(1)}$ is a Laplace coefficient. With this modification,
$\eta_{\rm sync}$ (Eq.~\ref{eq:def_etasync}) can be computed:
% (10 * (Mearth)^2 / (Msun)^2) / 2 * (1.3)^(-3) * ((0.035 AU) / (Rearth))^3
% 0.011389
\begin{align}
    \eta_{\rm sync} ={}& 0.011 \cos I f(\alpha)
            \p{\frac{k}{k_{\rm q}}}
            \p{\frac{\rho}{\rho_{\oplus}}}
            \p{\frac{a}{0.035 \;\mathrm{AU}}}^{3}\nonumber\\
        &\times
            \p{\frac{m_{\rm p}}{10 M_{\oplus}}}
            \p{\frac{a / a_{\rm p}}{1/1.3}}^{3}
            \p{\frac{M_\star}{M_{\odot}}}^{-2}
            .\label{eq:etasync_usp}
\end{align}
We have evaluated using $a / a_{\rm p}$ corresponding to a period ratio $P_{\rm
p} / P = 1.5$, for which $f(\alpha) \approx 5.5$. Note furthermore that $k /
k_{\rm q} \approx 3$ for Earth-like planets \citep{groten2004fundamental,
lainey2016quantification}. For $I = 5^\circ$, roughly
consistent with typical proto-USP systems \citep{dai2018larger},
Eq.~\eqref{eq:def_etac} shows that $\eta_{\rm sync} \ll \eta_{\rm c}$, and thus
there are four CSs at spin-orbit synchronization. As such, if we take the
initial planetary obliquity to be zero, the planet is guaranteed to evolve into
tCE1, and not tCE2 (see Fig.~\ref{fig:Hhists_0_06}). If we assume instead a
randomly oriented initial planetary spin, Fig.~\ref{fig:anal_ptce} suggests that
the probability of capture into tCE2 is still only modest ($\lesssim 20\%$). A
more sophisticated calculation including the effect of a third planet does not
greatly modify these results.
% Note that this calculation is in tension with
% Fig.~4 of \citet{millholland2020formation}, where they find that $\eta \sim 1$
% for the same physical parameters when $\Omega_{\rm s}$ is at its equilibrium
% value (i.e.\ satisfying Eq.~\ref{eq:weaktide_dWszero}, or $\Omega_{\rm s}
% \approx n$ when the obliquity is small).

The second stage of the proposed mechanism, runaway inward migration after
attaining tCE2, requires that the initial orbital decay time scale be
sufficiently fast. Evaluating Eq.~\eqref{eq:sehj_adot} for the relevant physical
parameters, we find that:
% 1 / (1.5 * 1e-3 * (Msun / (8 Mearth)) * ((2 Rearth) / (0.035 AU))^5 * 2 * pi / (0.035)^(3/2))
% = 7.8e8
\begin{align}
    -\frac{\dot{a}}{a} ={}& \frac{1}{8 \times 10^8\;\mathrm{yr}}
            \p{1 - \frac{\Omega_{\rm s}}{n}\cos \theta}
            \p{\frac{2k_2/Q}{10^{-3}}}
            \p{\frac{M_\star}{M_{\odot}}}^{3/2}\nonumber\\
        &\times \p{\frac{m}{M_{\oplus}}}^{-1}
            \p{\frac{R}{R_{\oplus}}}^5
            \p{\frac{a}{0.035\;\mathrm{AU}}}^{-13/2}.
            \label{eq:adot_usp}
\end{align}
For $\eta_{\rm sync} \ll \eta_{\rm c}$, Eqs.~\eqref{eq:def_tce2_approx} imply
that $\Omega_{\rm s}\cos \theta / n \ll 1$ in tCE2, so indeed the orbit of the
proto-USP is able to decay within the lifetime of the system. On the other hand,
in tCE1, $\omega_{\rm s} \approx n$ and $\cos \theta \approx 1 - \eta_{\rm
sync}^2 \sin^2 I / 2$, so $\dot{a} / a$ is suppressed by a factor of $\sim
\eta_{\rm sync}^2 \sin^2 I$. This shows that a proto-USP in tCE1 is unable to
initiate runaway orbital decay within the age of the system. Note that this
constraint also implies $\eta_{\rm sync}$ (Eq.~\ref{eq:etasync_usp}) cannot be
increased by considering proto-USPs with larger values of $a$, as the initial
orbital decay will become too slow.

Finally, we compute the orbital separation at which tCE2 becomes unstable due to
sufficiently strong tidal dissipation. Evaluating Eq.~\eqref{eq:def_ts_crit}, we
find that tCE2 breaks ($t_{\rm s} \lesssim t_{\rm s, c}$) when the semi-major
axis is smaller than $a_{\rm break}$, where
% (1e-3 * (Msun / (3.16 Mearth))^3 * ((Rearth) * (0.05 AU))^(9/2))^(1/9) / (1 AU)
% = 0.031994
\begin{align}
    a_{\rm break} \approx{}& 0.032 \;\mathrm{AU}
        \p{\frac{2k_2/Q}{10^{3}}}^{-1/9}
        \p{\frac{M_\star}{M_{\odot}}}^{1/3}
        \p{\frac{m}{10M_{\oplus}}}^{-1/6}\nonumber\\
        &\times \p{\frac{\rho}{\rho_{\oplus}}}^{-1/6}
        \p{\frac{a_{\rm p}}{0.05\;\mathrm{AU}}}^{1/2}
        \p{\frac{1}{\sin I \cos^2 I}}^{1/9}.
\end{align}
This corresponds to a 2.7-day orbit. Once the system exits tCE2, it rapidly
evolves to tCE1, which suppresses its orbital decay (Eq.~\ref{eq:adot_usp}).
This final orbital separation does not qualify as a USP ($P \lesssim
\mathrm{day}$).

In summary, our results suggest that only proto-USPs with a large primordial
obliquity have a probability of evolving into tCE2 initially. Furthermore,
proto-USPs that successfully initiate runaway tidal migration after reaching
tCE2 will cease their inward migration before becoming a USP\@.

\subsection{Application to WASP-12b}\label{ss:disc_wasp12b}

In \citet{millholland2019obliquity}, the authors consider the possibility that
the measured orbital decay of the HJ WASP-12b \citep[$P / \dot{P} = -3.2
\;\mathrm{Myr}$; ][]{maciejewski2016departure, patra2017apparently} is caused by
tidal dissipation in the HJ enhanced by being trapped in a high-obliquity CS due
to an undetected exterior planet. We validate the plausibility of this scenario
by the relevant physical parameters as was done in the previous sections using
the following current-day system parameters \citet{hebb2009wasp,
maciejewski2013multi, millholland_wasp12b}: $R_\star = 1.63R_{\odot}$, $M_\star
= 1.36M_{\odot}$, $R = 1.89R_{\rm J}$, $m = 1.41M_{\rm J}$, and $a = 0.023
\;\mathrm{AU}$. We begin with the spin evolution timescale, which is given by:
% 1 / (1.5 * 1e-6 * ((1.36 Msun) / (1.41 Mjup)) * ((1.89 Rjup) / (0.023 AU))^3 * 2 * pi / (0.023)^(3/2))
% 6466.022710
\begin{align}
    \frac{1}{t_{\rm s}} ={}& \frac{1}{6000\;\mathrm{yr}}
            \p{\frac{1}{4k}}
            \p{\frac{2k_2/Q}{10^{-6}}}
            \p{\frac{M_\star}{1.36 M_{\odot}}}^{3/2}\nonumber\\
        &\times \p{\frac{m}{1.41 M_{\rm J}}}^{-1}
            \p{\frac{R}{1.89 R_{\rm J}}}^{3}
            \p{\frac{a}{0.023\;\mathrm{AU}}}^{-9/2}.
\end{align}
Thus, the spin of WASP-12b has plenty of time to find a tCE\@. We also wish to
calculate $\eta_{\rm sync}$, but there are three complications: (i) the system
likely instead has angular momentum hierarchy $L \gg L_{\rm p}$ (to be verified
after the fact), opposite to that considered in the paper thus far; (ii) the
properties of the hypothetical perturber are unknown; and (iii) we should
evaluate $\eta_{\rm sync}$ using the value of $a$ for WASP-12b at the start of
its orbital migration, not its present day value. Towards resolving complication
(i), we express the precession of $\uv{l}$ about $\bm{J} = J\uv{\jmath} \equiv
\bm{L} + \bm{L}_{\rm p}$, the total angular momentum axis, as
\begin{equation}
    \rd{\uv{l}}{t} = \omega_{\rm lp}\cos I \frac{J}{L_{\rm p}}
        \p{\uv{l} \times \uv{\jmath}},
        \label{eq:wasp12b_g}
\end{equation}
where $\omega_{\rm lp}$ is still given by Eq.~\eqref{eq:wlp}. Thus, we see that
the precession frequency $\abs{g}$ is enhanced by a factor of $J / L_{\rm p}
\approx L / L_{\rm p}$. Towards addressing complications (ii--iii),% chktex 8
we use the fiducial values for the initial semi-major axis $a_{\rm i} =
0.038\;\mathrm{AU}$ and initial semi-major axis ratio $a_{\rm p} / a_{\rm i} =
1.29$, to be justified after the fact. We obtain:
% 1/2 * (1.41 Mjup)^2 / (1.36 Msun)^2 * ((0.038 AU) / (1.89 Rjup))^3 * (0.038 / 0.05)^(3.5)
% 0.014916
\begin{align}
    \eta_{\rm sync, i} ={}& 0.015 \cos I f(\alpha)
            \p{\frac{m}{1.41M_J}}^2
            \p{\frac{M_\star}{1.36M_{\odot}}}^{-2}
            \nonumber\\
        &\times
            \p{\frac{a_{\rm i}}{0.038\;\mathrm{AU}}}^{3}
            \p{\frac{a_{\rm p} / a_{\rm i}}{1.29}}^{-7/2}
            \p{\frac{R}{1.89R_{\rm J}}}^{-3}.\label{eq:hj_etasync}
\end{align}
For these values of $a_{\rm i}$ and $a_{\rm p}$, $f(\alpha) \approx 5$.

We next work towards justifying these choices of fiducial parameters. There are
four physical and observational constraints on the WASP-12b + perturber system:
\begin{itemize}
    \item The HJ must have had a sufficiently small initial semimajor axis such
        that its orbital decay timescale is less than the age of the system.

    \item The exterior planet must be sufficiently massive to keep the HJ in a
        high-obliquity tCE today.

    \item The RV signal of the exterior planet must be smaller than the
        residuals of the published RVs, $\sim 16\;\mathrm{m/s}$
        \citep{hebb2009wasp, husnoo2011orbital, knutson2014friends,
        bonomo2017gaps}.

    \item The initial configuration of the HJ and exterior planet must be
        dynamically stable.
\end{itemize}

First, we require that the initial HJ semimajor axis, denoted $a_{\rm i}$, is
sufficiently small that decay occurs within the age of the system. The semimajor
axis decay rate can be expressed as
% 1 / (1.5 * 1e-7 * (1.36 Msun) / (1.41 Mjup) * ((1.89 Rjup) / (0.038 AU))^5 * (G * (1.36 Msun) / (0.038 AU)^3)^(1/2) * (1 yr))
% = 9.82e8
\begin{align}
    -\frac{\dot{a}}{a_{\rm i}}
        =& \frac{1}{\mathrm{Gyr}}
            \p{\frac{2k_2/Q}{10^{-6}}}
            \p{\frac{M_\star}{1.36 M_{\odot}}}^{3/2}
            \p{\frac{m}{1.41M_J}}^{-1}\nonumber\\
        &\times \p{\frac{R}{1.89R_j}}^5
            \p{\frac{a_{\rm i}}{0.038\;\mathrm{AU}}}^{-13/2}
            \p{1 - \frac{\Omega_{\rm s}}{n}\cos \theta}.
            \label{eq:hj_adot}
\end{align}
Again, $t_{\rm s} \ll a_{\rm i} / \dot{a}$, so we can treat the HJ as
immediately evolving into a tCE\@. Thus, we see that the initial semimajor axis
for the HJ cannot exceed $0.038\;\mathrm{AU}$ even when $\p{1 - \Omega_{\rm
s}\cos \theta/n} \approx 1$, as is the case for rapid evolution into tCE2 while
$\eta_{\rm sync} \ll 1$.

Second, we address the constraint that tCE2 is stable under the influence of the
exterior planet. Using the amended precession frequency $\abs{g}$ given by
Eq.~\eqref{eq:wasp12b_g}, we compute that the stability of tCE2 requires
\begin{equation}
    \frac{1}{t_{\rm s}} \lesssim \omega_{\rm lp}\cos I \frac{L}{L_{\rm p}}
        \sin i \sqrt{\frac{\eta_{\rm sync}\cos i}{2}}.
\end{equation}
Here, $\cos i \equiv \uv{l} \cdot \uv{\jmath}$. This then yields that
\begin{align}
    \frac{a_{\rm p}}{a}
        \lesssim{}& 3.3 \p{\sin i \cos^{1/2} i \cos^{3/2} I}^{4/21}
            \p{\frac{m}{1.41 M_J}}^{12/21}
            \p{\frac{M_\star}{1.36 M_{\odot}}}^{-12/21}\nonumber\\
        &\times \p{\frac{2k_2/Q}{10^{-6}}}^{-4/21}
            \p{\frac{a}{0.023\;\mathrm{AU}}}^{6/7}
            \p{\frac{R}{1.89 R_J}}^{-6/7}.\label{eq:hj_tce2_stab}
\end{align}
Notably, this expression does not depend on the perturber mass but only on
$a_{\rm p}$.

Third, we address the constraint that the RV signal of the exterior planet be
smaller than $16\;\mathrm{m/s}$, which requires
% (G * (1.36 Msun) / (0.076 AU))^(1/2) * (80 Mearth) / (1.36 Msun) / 2^(1/2)
% 15.737236 m\s
\begin{equation}
    \p{\frac{a_{\rm p}}{0.076\;\mathrm{AU}}}^{-1/2}
    \p{\frac{m_{\rm p}}{80M_{\oplus}}}
    \p{\frac{M_\star}{1.36M_{\odot}}}^{-1/2}
    \p{\frac{\sin i}{1 / \sqrt{2}}} \lesssim 1,
    \label{eq:hj_rv}
\end{equation}
where $i$ is the line-of-sight inclination angle. Here, we have taken $a_{\rm
p}$ to be the maximum value permitted by Eq.~\eqref{eq:hj_tce2_stab}. For these
values of $a_{\rm p}$ and $m_{\rm p}$, we still have $L / L_{\rm p} \sim 0.4$,
and so $L \gg L_{\rm p}$ is well satisfied for the permitted parameter space as
claimed.

Finally, we require that the initial orbital configuration be dynamically
stable. We use the Hill stability criterion \citep[e.g.][]{petit2020path}:
\begin{equation}
    \frac{a_{\rm p} - a_{\rm i}}{a_{\rm i}} > 2\sqrt{3}\frac{a_{\rm p} + a_{\rm
        i}}{2a_{\rm i}} \p{\frac{m + m_{\rm p}}{3M_{\star}}}^{1/3}.
\end{equation}
Assuming $m_{\rm p} \ll m$, this yields:
\begin{equation}
    \frac{a_{\rm p}}{a_{\rm i}} > 1.29.\label{eq:hj_orbstab}
\end{equation}
The combination of the two constraints in
Eqs.~(\ref{eq:hj_adot},~\ref{eq:hj_orbstab}) justify the fiducial parameters
used in Eq.~\eqref{eq:hj_tce2_stab}.

We next address the implications of the $\eta_{\rm sync}$ value found. When
evaulating $\eta_{\rm sync}$ in Eq.~\eqref{eq:hj_tce2_stab}, it is possible
that $R$ is larger today than its primordial value due to inflation induced by
increased stellar irradiation. However, if a smaller value of $R$ is used in
Eq.~\eqref{eq:hj_etasync}, the value of $a_{\rm i}$ must also be decreased such
that $R^5 / a_{\rm i}^{13/2}$ is constant (Eq.~\ref{eq:hj_adot}), which further
decreases $\eta_{\rm sync, i}$.

For $\eta_{\rm sync} = 0.075$, we find that there are four CSs unless $i$ is
extremely small. From Fig.~\ref{fig:Hhists_0_06}, we can infer that prograde
primordial obliquities will evolve towards tCE1 in the regime where $\eta_{\rm
sync} \ll \eta_{\rm c}$. On the other hand, if the primordial obliquity of the
HJ is assumed to be isotropically distributed, then Fig.~\ref{fig:probs20}
suggests that the probability of entry into tCE2 is $\lesssim 25\%$ even if
the perturbing planet is misaligned by $I = 20^\circ$.

In summary, the joint constraints we have placed on the WASP-12b \& perturber
system imply that the initial $\eta_{\rm sync, i}$ is unlikely to have been
large. This suggests that early capture into tCE2 is unlikely from either an
isotropic or prograde-favoring initial obliquity distribution. However, it is
possible that additional mechanisms can generate a favorable initial obliquity
distribution, such as the presence of a dissipating protoplanetary disk in the
early lifetime of the system quickly driving the obliquity of WASP-12b to $\sim
90^\circ$ \citep{millholland_disk, su2020}.

\section{Summary and Discussion}\label{s:summary}

In this work, we have studied the evolution of an interior planet's spin due to
the joint effect of tidal dissipation and gravitational interactions with an
exterior perturber. Our results extend our study of Colombo's Top in a previous
paper \citep{su2020} to include a dissipative torque, for which we have adopted
the weak friction theory of the equilibrium tide \citep{lai2012}. Our primary
esults are as follows:
\begin{itemize}
    \item We calculate the modification of the equilibria of Colombo's Top
        (Cassini States; CSs) due to tidal dissipation and evaluate the
        stability of the CSs in the presence of the tidal dissipation.

    \item We show that, for a given orbital architecture, the spin of the inner
        planet has at most two spin configurations (spin frequency and
        orientation) that are stable under the effect of tidal dissipation. We
        call these equilibria \emph{tidal Cassini Equilibria} (tCE\@; see
        Fig.~\ref{fig:6equils006}). The locations of these equilibria are
        determined by the system architecture and are parameterized by
        $\eta_{\rm sync}$ (Eq.~\ref{eq:def_etasync}).

    \item We show that if tCE1 exists ($\eta_{\rm sync} < \eta_{\rm c}$,
        Eqs.~\ref{eq:def_etasync} and~\ref{eq:def_etac}; see discussion in
        Section~\ref{ss:weaktide}), which tCE a given initial planetary spin
        configuration asymptotically evolves towards depends on which of the
        phase space zones (see Fig.~\ref{fig:1contours}) the initial spin
        orientation belongs to: (i) If the spin originates in zone I, then it
        generally evolves towards tCE1 (unless $\eta_{\rm sync}$ very near
        $\eta_{\rm c}$, e.g.\ see Fig.~\ref{fig:Hhists_0_50}); (ii) if the spin
        originates in zone II, then it evolves towards tCE2; and (iii) if the
        spin originates in zone III, the outcome is generally probabilistic.

    \item For initial conditions in zone III, the probabilities of approaching
        either tCE can be determined by careful study of the dynamics upon
        separatrix encounter (Sections~\ref{ss:toy_outcomes}
        and~\ref{ss:phop_weaktide}). As an example, if the initial spin
        orientation is assumed to be isotropically distributed, the
        probabilities of entering tCE1 and tCE2 are given in
        Figs.~\ref{fig:probs20} and~\ref{fig:probs5} as a function of $\eta_{\rm
        sync}$ for two different mutual inclinations $I$.

    \item In three exoplanetary scenarios of interest, application of our
        results yields the following conclusions: (i) a super-Earth with an
        exterior cold Jupiter has substantial probability of being trapped in a
        high-obliquity tCE\@; (ii) formation of ultra-short-period planets via
        tidal runaway is unlikely due to low probability of capture into a
        high-obliquity tCE\@; and (iii) WASP-12b is unlikely to be undergoing
        enhanced orbital decay due to obliquity tides, as the probability of
        entry into tCE2 is also low.
\end{itemize}

Above, we have studied tidal dissipation in the planet via the weak friction
theory of the equilibrium tide \citep{lai2012}. However, other mechanisms of
tidal dissipation may be dominant, depending on the system parameters
\citep[e.g.][]{papaloizou_ivanov_inertial, teyssandier2019formation}. While the
detailed outcomes may change, a different tidal mechanism is amenable to the
same analysis presented in this paper: The tCE can still be found by an analysis
similar to that shown in Fig.~\ref{fig:6equils006}, and the probabilistic
outcome of a separatrix encounter can still be solved using the techniques of
Sections~\ref{ss:toy_outcomes} and~\ref{ss:phop_weaktide}.

Additionally, we have primarily focused on an isotropic distribution of initial
spin orientation, assuming that giant impacts will effectively randomize a
planet's spin. More physically accurate distributions can be used in the case of
planetary mergers \citep{li2020planetary} or many smaller impacts
\citep{dones1993does}. Figures~\ref{fig:probs20}--\ref{fig:probs5} can be
updated accordingly by convolving any initial obliquity distribution with the
tCE2 probability distributions, such as those shown in the right panels of
Figs.~\ref{fig:Hhists_0_06}--\ref{fig:Hhists_0_50}. The qualitative results are
unlikely to change, though the detailed probabilities for tCE2 capture can
increase [decrease] if the initial obliquity distribution favors [disfavors]
$\theta_{\rm i} \approx 90^\circ$ compared to the isotropic distribution.

\section{Acknowledgements}

We thank Sarah Millholland for useful discussions. This work has been supported
in part by NSF grant AST1715246. YS is supported by the NASA FINESST grant
19-ASTRO19-0041.%chktex 8

\bibliography{Su_weak_tides}
\bibliographystyle{aasjournal}

\appendix

\onecolumn

\section{Convergence of Initial Conditions Inside the Separatrix to CS2
}\label{app:cs_stab2}

In Section~\ref{ss:tidal_equils}, we studied the stability of the CSs under of
tidal alignment torque given by Eq.~\eqref{eq:dsdt_tide_toy}, finding that CS2
is locally stable. Later, in Section~\ref{ss:toy_outcomes}, we found that all
initial conditions within the separatrix converge to CS2, which is not
guaranteed by local stability of CS2. In this section, we give an analytic
demonstration that all points inside the separatrix indeed converge to CS2,
focusing on the case where $\eta \ll 1$.

Similarly to the analytic calculation in Section~\ref{ss:toy_outcomes}, we seek
to compute the change in the unperturbed Hamiltonian over a single libration
cycle. To calculate the evolution of $H$, we first parameterize the unperturbed
trajectory (similarly to Eq.~\ref{eq:sep_theta}). For initial conditions inside
the separatrix, the value of $H$ can be written $H = H_{\rm sep} + \Delta H$
where $\Delta H > 0$, and the two legs of the libration trajectory can be
written:
\begin{align}
    \cos \theta_{\pm} &\approx
        \eta \cos I \pm \sqrt{2\eta\s{\sin I\p{1 - \cos \phi} - \Delta H}}.
        \label{eq:lib_cycle_toy}
\end{align}
We have taken $\sin \theta \approx 1$, a good approximation in zone II when
$\eta \ll 1$. Note that there are some values of $\phi$ for which no solutions
of $\theta$ exist, reflecting the fact that the libration cycle does not extend
over the full interval $\phi \in [0, 2\pi]$. During a libration cycle,
$\theta_-$ [$\theta_+$] is traversed while $\phi' > 0$ [$\phi' < 0$], i.e.\ the
trajectory librates counterclockwise in $(\cos \theta, \phi)$ phase space (see
Fig.~\ref{fig:1contours}).

The leading order change to $H$ over a single libration cycle can then computed by
integrating $\rdil{H}{t}$ along this trajectory, yielding:
\begin{align}
    \oint \rd{H}{t}\;\mathrm{d}t
        &= \oint \p{\rd{(\cos \theta)}{t}}_{\rm tide}
            \;\mathrm{d}\phi,\nonumber\\
        &= \int\limits_{\phi_{\min}}^{\phi_{\max}}
                \frac{1}{t_{\rm s}}
                \p{\sin^2\theta_- - \sin^2\theta_+} \;\mathrm{d}\phi\nonumber\\
        &\approx \frac{1}{t_{\rm s}}
            \int\limits_{\phi_{\min}}^{\phi_{\max}}
                4\eta \cos I \sqrt{2\eta\s{\sin I\p{1 - \cos \phi} - \Delta H}}
                \;\mathrm{d}\phi > 0.
\end{align}
Here, $\phi_{\min} > 0$ and $\phi_{\max} < 2\pi$ are defined such that the
trajectory librates over $\phi \in \s{\phi_{\min}, \phi_{\max}}$. Thus, $H$ is
strictly increasing for all initial conditions inside the separatrix, and they
all converge to CS2.

\section{Approximate TCE2 Probability for Small $\eta_{\rm sync}$
}\label{app:ptce2}

In this appendix, we seek a tentative analytic understanding for the
probability of convergence to tCE2 when $\eta_{\rm sync}$ is small, i.e.\ the
left extremes of Figs.~\ref{fig:probs20} and~\ref{fig:probs5}. In this regime,
following the discussions in Sections~\ref{ss:toy_outcomes}
and~\ref{ss:phop_weaktide}, we understand that initial conditions (ICs) in zone I
always converge to tCE1, ICs in zone II always converge to tCE2,
and ICs in zone III experience separatrix encounter and
probabilistically converge to either one of the tCE\@. To further proceed, we
will assume an isotropic distribution of initial spin orientations; different
distributions again will only change the quantitative but not qualitative
character of the discussion. Then the tCE2 probability, which we denote by
$P_{\rm tCE2}$, can be expressed as the sum of: (i) the probability that an IC
is in zone II, and (ii) the probability that an IC is both in zone III and
undergoes a III $\to$ II transition. To simplify the discussion, we will
approximate that $P_{\rm tCE2}$ can be calculated as
\begin{equation}
    P_{\rm tCE2} \sim \frac{A_{\rm II}}{4\pi}
            + \frac{A_{\rm III}}{4\pi}\ev{P_{\rm III \to II}},
            \label{eq:app_ptce2_schematic}
\end{equation}
where $A_{\rm II}$ and $A_{\rm III}$ are the phase space areas of zones II and
III respectively, and $\ev{P_{\rm III \to II}}$ is the \emph{average} III $\to$
II transition probability for a random IC in zone III\@. Next, we evaluate each
of the expressions in Eq.~\eqref{eq:app_ptce2_schematic}.

We first consider $A_{\rm II}$ and $A_{\rm III}$. Exact analytic forms for both
$A_{\rm II}$ and $A_{\rm III}$ is known (\citealp{ward2004I}, Paper I), but an
accurate approximation can be obtained using Eq.~\eqref{eq:sep_theta} since
$\eta_{\rm i} \ll 1$. We obtain that:
\begin{align}
    \frac{A_{\rm II}}{4\pi} &= \frac{4}{\pi}\sqrt{\eta_{\rm i} \sin I},\\
    \frac{A_{\rm III}}{4\pi}
        &= \frac{1 + \eta_{\rm i} \cos I}{2} -
            \frac{2}{\pi}\sqrt{\eta_{\rm i} \sin I}.
\end{align}

Next, we need to evaluate $\ev{P_{\rm III \to II}}$, for which we must
understand the outcomes of the separatrix encounters that ICs in zone III
experience. We proceed by analytically calculating $\Delta K_{\pm}$
(Eq.~\ref{eq:def_dK_weaktide}) for use in Eq.~\eqref{eq:def_pc_weaktide} to
obtain the probabilities of the outcomes of separatrix encounter. We first
rewrite Eq.~\eqref{eq:def_dK_weaktide} as:
\begin{align}
    \Delta K_{\pm} &= \oint_{\mathcal{C}_{\pm}} \rd{H}{t}
        - \rd{H_{\rm sep}}{t}\;\mathrm{d}t
        \nonumber\\
        &= \oint_{\mathcal{C}_{\pm}}
           \p{\rd{(\cos\theta)}{t}}_{\rm tide}
            + \frac{\dot{\Omega}_{\rm s}}{\dot{\phi}}
            \p{\pd{H}{\Omega_{\rm s}} - \pd{H_{\rm sep}}{\Omega_{\rm s}}}\;\mathrm{d}\phi
                \label{eq:app_dhpm}.
\end{align}
Then, using the full equations of motion for the planet's spin including weak
tidal friction in component form, given by
Eqs.~(\ref{eq:ds_fullq}--\ref{eq:ds_fulls}), we can evaluate $\Delta K_{\pm}$ by
integrating along the two legs of the separatrix $\mathcal{C}_{\pm}$ (see
Fig.~\ref{fig:1contours}). Note that we must use the value of $\eta$ at the
moment of separatrix encounter, which we denote $\eta_{\rm cross}$, as the
evolution of $\Omega_{\rm s}$ changes the spin-orbit precession frequency
$\alpha$ and thus $\eta$ itself:
\begin{align}
    % \Delta K_{\pm} ={}& \oint_{\mathcal{C}_{\pm}}
    %     \frac{1}{t_{\rm s}}\sin^2\theta
    %         \p{\frac{2n}{\Omega_{\rm s}} - \cos\theta}
    %      +
    %         \s{\frac{\alpha}{\Omega_{\rm s}}
    %             \frac{\cos^2\theta}{2} + g\frac{\Omega_{\rm
    %         c}}{2\Omega_{\rm s}^2}\cos^2 I}\rd{\Omega_{\rm
    %         s}}{t}\rd{t}{\phi}\;\mathrm{d}\phi\nonumber\\
    %  \approx{}& \oint_{\mathcal{C}_{\pm}}
    %     \frac{1}{t_{\rm s}}\sin^2\theta
    %         \p{\frac{2n}{\Omega_{\rm s}} - \cos\theta}
    %     + \frac{1}{t_{\rm s}}\frac{n}{\Omega_{\rm s}\eta_{\rm cross}}
    %         \s{\p{\cos \theta}_{\mathcal{C}_{\pm}} - \frac{\Omega_{\rm s}}{2n}}
    %         \s{2 \cos I \pm \sqrt{2 \sin I\p{1 - \cos \phi} / \eta_{\rm cross}}}
    %         \;\mathrm{d}\phi,\\
    t_{\rm s} \Delta K_{\pm} \approx{}&
        \frac{\eta_{\rm cross}^2}{\eta_{\rm sync}}\s{
            -2\cos I\p{\pm 2\pi \eta_{\rm cross} \cos I
                + 8\sqrt{\eta_{\rm cross} \sin I}}
            \mp 4\pi \sin I
            - 8 \cos I \sqrt{\eta_{\rm cross}\sin I}
                + \frac{4\eta_{\rm sync}}{\eta_{\rm cross}}
                    \sqrt{\sin I/\eta_{\rm cross}}}\nonumber\\
        &+ \frac{2\eta_{\rm cross}}{\eta_{\rm sync}}
            \p{\mp 2\pi\p{1 - 2\eta_{\rm cross} \sin I}
            + 16\cos I \eta_{\rm cross}^{3/2}\sqrt{\sin I}}
            + 8\sqrt{\eta_{\rm cross} \sin I}
            \pm 2 \pi \eta_{\rm cross} \cos I
            - \frac{64}{3} \p{\eta_{\rm cross} \sin I}^{3/2}.
                \label{eq:app_deltaK}
\end{align}
The resulting $P_{\rm III \to II}$ obtained using this analytic $\Delta K_{\pm}$
in Eq.~\eqref{eq:def_pc_weaktide} is shown as the green dashed line in the top
panel of Fig.~\ref{fig:pc_fits_0_06}, where it can be seen that agreement is
reasonable for $\eta_{\rm cross} \lesssim 0.05$. For the purposes of this
section, we drop all but the leading order terms in both the numerator and
denominator of Eq.~\eqref{eq:def_pc_weaktide} and obtain:
% proof:
% from utils import *; I = np.radians(20); s_c = 0.01; s = np.linspace(1, 10, 100); top, bottom = get_ps_anal(I, s_c, s); plt.plot(s_c/s, (bottom + top) / bottom, 'k'); plt.plot(s_c/s, 6 * s_c / np.pi * np.sqrt(np.sin(I) / (s_c / s)), 'g')
\begin{equation}
    P_{\rm III \to II} \approx
        \frac{6\eta_{\rm sync}}{\pi} \sqrt{\frac{\sin I}{\eta_{\rm cross}}}.
\end{equation}

However, $\eta_{\rm cross}$ cannot be expressed in closed form as a function of
the ICs. Based on the bottom panel of Fig.~\ref{fig:pc_fits_0_06}, we make the
crude approximation that $\eta_{\rm cross}$ is uniformly distributed between
$\eta_{\rm i}$ and $\eta_{\rm sync}$. Note that if $\Omega_{\rm s} \simeq n$,
then this approximation is invalid: since nearly antialigned spins ($\theta_{\rm
i} \approx 180^\circ$) will undergo significant spin-down before tidal friction
can realign the spin orientation, $\Omega_{\rm s, i}$ being too close to $n$
results in $\eta_{\rm cross} \ll \eta_{\rm sync}$. We thus obtain:
\begin{align}
    \ev{P_{\rm III \to II}}
        &\sim \frac{1}{\eta_{\rm sync} - \eta_{\rm i}}
            \int\limits_{\eta_{\rm i}}^{\eta_{\rm sync}}
            P_{\rm III \to II}\;\mathrm{d}\eta_{\rm cross}\nonumber\\
        &= \frac{12 \sqrt{\eta_{\rm sync}\sin I}}{\pi
            \p{1 + \sqrt{n / \Omega_{\rm s, i}}}}.
\end{align}

With this result, we can finally express Eq.~\eqref{eq:app_ptce2_schematic} as:
\begin{align}
    P_{\rm tCE2} &\simeq
            \frac{4\sqrt{\eta _{\rm sync} \sin I}}{\pi}\s{
                \sqrt{n / \Omega_{\rm s, i}}
                + \frac{3}{2\p{1
                + \sqrt{n / \Omega_{\rm s, i}}}}}
            + \mathcal{O}\p{\eta_{\rm sync}}.
\end{align}
This is exactly Eq.~\eqref{eq:app_tce2_p_tot}. We remark again that this is
valid in the regime where $\eta_{\rm sync} \ll 1$ and $\Omega_{\rm s} \gtrsim
n$.

\label{lastpage} % chktex 24
\end{document}
