    \documentclass[dvipsnames]{beamer}
    \usetheme{Madrid}
    \usefonttheme{professionalfonts}
    \usepackage{
        amsmath,
        amssymb,
        fancyhdr, % page styling
        lastpage, % footer fanciness
        hyperref, % various links
        setspace, % line spacing
        amsthm, % newtheorem and proof environment
        mathtools, % \Aboxed for boxing inside aligns, among others
        float, % Allow [H] figure env alignment
        enumerate, % Allow custom enumerate numbering
        graphicx, % allow includegraphics with more filetypes
        wasysym, % \smiley!
        upgreek, % \upmu for \mum macro
        listings, % writing TrueType fonts and including code prettily
        tikz, % drawing things
        booktabs, % \bottomrule instead of hline apparently
        cancel % can cancel things out!
    }
    \usepackage[
        labelfont=bf, % caption names are labeled in bold
        font=scriptsize % smaller font for captions
    ]{caption}
    \usepackage[font=scriptsize]{subcaption} % subfigures

    \newcommand*{\scinot}[2]{#1\times10^{#2}}
    \newcommand*{\dotp}[2]{\left<#1\,\middle|\,#2\right>}
    \newcommand*{\rd}[2]{\frac{\mathrm{d}#1}{\mathrm{d}#2}}
    \newcommand*{\pd}[2]{\frac{\partial#1}{\partial#2}}
    \newcommand*{\rdil}[2]{\mathrm{d}#1 / \mathrm{d}#2}
    \newcommand*{\pdil}[2]{\partial#1 / \partial#2}
    \newcommand*{\rtd}[2]{\frac{\mathrm{d}^2#1}{\mathrm{d}#2^2}}
    \newcommand*{\ptd}[2]{\frac{\partial^2 #1}{\partial#2^2}}
    \newcommand*{\md}[2]{\frac{\mathrm{D}#1}{\mathrm{D}#2}}
    \newcommand*{\pvec}[1]{\vec{#1}^{\,\prime}}
    \newcommand*{\svec}[1]{\vec{#1}\;\!}
    \newcommand*{\bm}[1]{\boldsymbol{\mathbf{#1}}}
    \newcommand*{\uv}[1]{\hat{\bm{#1}}}
    \newcommand*{\ang}[0]{\;\text{\AA}}
    \newcommand*{\mum}[0]{\;\upmu \mathrm{m}}
    \newcommand*{\at}[1]{\left.#1\right|}
    \newcommand*{\bra}[1]{\left<#1\right|}
    \newcommand*{\ket}[1]{\left|#1\right>}
    \newcommand*{\abs}[1]{\left|#1\right|}
    \newcommand*{\ev}[1]{\langle#1\rangle}
    \newcommand*{\p}[1]{\left(#1\right)}
    \newcommand*{\s}[1]{\left[#1\right]}
    \newcommand*{\z}[1]{\left\{#1\right\}}

    \let\Re\undefined
    \let\Im\undefined
    \DeclareMathOperator{\Res}{Res}
    \DeclareMathOperator{\Re}{Re}
    \DeclareMathOperator{\Im}{Im}
    \DeclareMathOperator{\Log}{Log}
    \DeclareMathOperator{\Arg}{Arg}
    \DeclareMathOperator{\Tr}{Tr}
    \DeclareMathOperator{\E}{E}
    \DeclareMathOperator{\Var}{Var}
    \DeclareMathOperator*{\argmin}{argmin}
    \DeclareMathOperator*{\argmax}{argmax}
    \DeclareMathOperator{\sgn}{sgn}
    \DeclareMathOperator{\diag}{diag\;}

    % \everymath{\displaystyle} % biggify limits of inline sums and integrals
    \tikzstyle{circ} % usage: \node[circ, placement] (label) {text};
        = [draw, circle, fill=white, node distance=3cm, minimum height=2em]
    \definecolor{commentgreen}{rgb}{0,0.6,0}
    \lstset{
        basicstyle=\ttfamily\footnotesize,
        frame=single,
        numbers=left,
        showstringspaces=false,
        keywordstyle=\color{blue},
        stringstyle=\color{purple},
        commentstyle=\color{commentgreen},
        morecomment=[l][\color{magenta}]{\#}
    }

\begin{document}

\title{Spin-Orbit Resonances in 3+ Planet Systems}
\subtitle{Dong Lai Group Meeting Presentation}
\author{Yubo Su}
\date{June 25, 2021}

\maketitle

\begin{frame}
    \frametitle{Review}
    \framesubtitle{Cassini States (Planet + Perturber)}

    \begin{columns}
        \begin{column}{0.5\columnwidth}
            \begin{itemize}
                \item In 1 + 1 system:
                    \begin{align*}
                        \rd{\uv{s}}{t} &= \underbrace{\omega_{\rm sl}}_\alpha
                            \p{\uv{s} \cdot \uv{l}}
                            \p{\uv{s} \times \uv{l}_{\rm p}},\\
                        \rd{\uv{l}}{t}
                            &= \underbrace{\omega_{\rm lp}
                                \p{\uv{l} \cdot \uv{J}}}_{-g}
                                \p{\uv{l} \times \uv{J}}.
                    \end{align*}
                    Here, $\bm{J} = \bm{l} + \bm{l}_{\rm p}$, or $\uv{J}$ is the
                    invariable plane.

                \item Can be a resonance when $\alpha \sim -g$.
            \end{itemize}
        \end{column}
        \begin{column}{0.5\columnwidth}
            \begin{figure}
                \centering
                \begin{tikzpicture}[scale=0.7]
                    \draw[->, black, thick] (0, 0) -- (0, 5);
                    \draw[->, blue, thick] (0, 0) -- (-1, 4.9);
                    \draw[->, red, thick] (0, 0) -- (4, 3);
                    \node[above] at (0, 5) {$\uv{J}$};
                    \node[left, blue] at (-1, 4.9) {$\uv{l}$};
                    \node[right, red] at (4, 3) {$\uv{s}$};
                \end{tikzpicture}
            \end{figure}
        \end{column}
    \end{columns}
\end{frame}

\begin{frame}
    \frametitle{Review}
    \framesubtitle{Cassini States + Tides}

    \begin{columns}
        \begin{column}{0.5\columnwidth}
            \begin{itemize}
                \item Tides drive $\uv{s} \to \uv{l}$.

                \item Two stable Cassini States: $\theta_{\rm sl} \approx 0$,
                    $\theta_{\rm sl} \approx 90^\circ$.

                \item Choose random $\uv{s}$, where does it go?
                    \begin{itemize}
                        \item If very prograde, $\to$ CS1.
                        \item If inside Cassini State resonance, $\to$ CS2.
                        \item If very retrograde, \emph{probabilistic}.
                    \end{itemize}
            \end{itemize}
        \end{column}
        \begin{column}{0.5\columnwidth}
            \begin{figure}
                \centering
                \includegraphics[width=0.7\columnwidth]{../../initial/4nplanet/3paramtide/outcomes00.png}
                \includegraphics[width=0.7\columnwidth]{../../initial/4nplanet/3paramtide/outcomes00_hist.png}
            \end{figure}
        \end{column}
    \end{columns}
\end{frame}

\begin{frame}
    \frametitle{More Planets}
    \framesubtitle{Precession Equations}

    \begin{columns}
        \begin{column}{0.6\columnwidth}
            \begin{itemize}
                \item In 1 + $n$ system?
                    \begin{align*}
                        \rd{\uv{s}}{t} &= \underbrace{\omega_{\rm sl}}_\alpha
                            \p{\uv{s} \cdot \uv{l}}
                            \p{\uv{s} \times \uv{l}_{\rm p}},\\
                        \mathcal{I}(t) &= \sum\limits_{k = 1}^{n}
                            \mathcal{I}_k \exp\s{ig_kt + \phi_k},\\
                        \uv{l}(t) &= \Re\p{\mathcal{I}}\uv{x}
                            + \Im\p{\mathcal{I}}\uv{y}
                            + \sqrt{1 - \mathcal{I}^2}\uv{J}.
                    \end{align*}
                    Laplace-Lagrange.

                \item We focus on $n = 2$, so two modes. Likely two CSs?

                \item \emph{Chaos} when resonance overlap (existing work by
                    Laskar).
            \end{itemize}
        \end{column}
        \begin{column}{0.4\columnwidth}
            \begin{figure}
                \centering
                \begin{tikzpicture}[scale=0.7]
                    \draw[->, black, thick] (0, 0) -- (0, 5);
                    \draw[->, blue, thick] (0, 0) -- (-1, 4.9);
                    \draw[->, red, thick] (0, 0) -- (4, 3);
                    \node[above] at (0, 5) {$\uv{J}$};
                    \node[left, blue] at (-1, 4.9) {$\uv{l}$};
                    \node[right, red] at (4, 3) {$\uv{s}$};
                \end{tikzpicture}
            \end{figure}
        \end{column}
    \end{columns}
\end{frame}

\begin{frame}
    \frametitle{More Planets}
    \framesubtitle{Plus Tides}

    \begin{itemize}
        \item NEW\@: with tides, where are stable / long-lived equilibria?

        \item Naively: at best, each of CS1/CS2 (two stable CSs) for
            each $g_i$?

        \item Chaos can likely change which ICs converge to which equilbria.
    \end{itemize}
\end{frame}

\begin{frame}
    \frametitle{More Planets}
    \framesubtitle{$I_1 = 10^\circ$, $I_2 = 1^\circ$, $g_1 = 0.1 \alpha$,
        $g_2 = 0.1g_1$}

    \begin{figure}
        \centering
        \includegraphics[width=0.45\columnwidth]{../../initial/4nplanet/3paramtide/outcomes01.png}
        \includegraphics[width=0.45\columnwidth]{../../initial/4nplanet/3paramtide/outcomes01_hist.png}
    \end{figure}
\end{frame}

\begin{frame}
    \frametitle{More Planets}
    \framesubtitle{$I_1 = 10^\circ$, $I_2 = 1^\circ$, $g_1 = 0.1 \alpha$,
        $g_2 = 1.5g_1$}

    \begin{figure}
        \centering
        \includegraphics[width=0.45\columnwidth]{../../initial/4nplanet/3paramtide/outcomes15.png}
        \includegraphics[width=0.45\columnwidth]{../../initial/4nplanet/3paramtide/outcomes15_hist.png}
    \end{figure}
\end{frame}

\begin{frame}
    \frametitle{More Planets}
    \framesubtitle{$I_1 = 10^\circ$, $I_2 = 1^\circ$, $g_1 = 0.1 \alpha$,
        $g_2 = 2.5g_1$}

    \begin{figure}
        \centering
        \includegraphics[width=0.45\columnwidth]{../../initial/4nplanet/3paramtide/outcomes25.png}
        \includegraphics[width=0.45\columnwidth]{../../initial/4nplanet/3paramtide/outcomes25_hist.png}
    \end{figure}
\end{frame}

\begin{frame}
    \frametitle{More Planets}
    \framesubtitle{$I_1 = 10^\circ$, $I_2 = 1^\circ$, $g_1 = 0.1 \alpha$,
        $g_2 = 3.5g_1$}

    \begin{figure}
        \centering
        \includegraphics[width=0.45\columnwidth]{../../initial/4nplanet/3paramtide/outcomes35.png}
        \includegraphics[width=0.45\columnwidth]{../../initial/4nplanet/3paramtide/outcomes35_hist.png}
    \end{figure}
\end{frame}

\begin{frame}
    \frametitle{More Planets}
    \framesubtitle{$I_1 = 10^\circ$, $I_2 = 1^\circ$, $g_1 = 0.1 \alpha$,
        $g_2 = 10g_1$}

    Mixed mode? Frequency is none of CSs!
    \begin{figure}
        \centering
        \includegraphics[width=0.45\columnwidth]{../../initial/4nplanet/3paramtide/outcomes010.png}
        \includegraphics[width=0.45\columnwidth]{../../initial/4nplanet/3paramtide/outcomes010_hist.png}
    \end{figure}
\end{frame}

\begin{frame}
    \frametitle{More Planets}
    \framesubtitle{Mixed Mode}

    Resonance angle is $\phi_{\rm res} = \phi_{\rm sJ} + \p{g_1 + g_2}t/2$. $I_1
    = 10^\circ$, $I_2 = 1^\circ$, $g_1 = 0.1 \alpha$, and $g_2 = 10g_1$.
    \begin{figure}
        \centering
        \includegraphics[width=0.8\columnwidth]{../../initial/4nplanet/3paramtide/mm_tide.png}
    \end{figure}
\end{frame}

\begin{frame}
    \frametitle{More Planets}
    \framesubtitle{Mixed Mode}

    Resonance angle is $\phi_{\rm res} = \phi_{\rm sJ} + \p{g_1 + g_2}t/2$. $I_1
    = 10^\circ$, $I_2 = 2^\circ$, $g_1 = 0.1 \alpha$, and $g_2 = 15g_1$.
    \begin{figure}
        \centering
        \includegraphics[width=0.8\columnwidth]{../../initial/4nplanet/3paramtide/mm_I22_tide.png}
    \end{figure}
\end{frame}

\end{document}

