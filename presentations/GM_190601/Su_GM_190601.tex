    \documentclass[dvipsnames, 11pt]{beamer}
    \usetheme{Madrid}
    \usefonttheme{professionalfonts}
    \usepackage{
        amsmath,
        amssymb,
        fouriernc, % fourier font w/ new century book
        fancyhdr, % page styling
        lastpage, % footer fanciness
        hyperref, % various links
        setspace, % line spacing
        amsthm, % newtheorem and proof environment
        mathtools, % \Aboxed for boxing inside aligns, among others
        float, % Allow [H] figure env alignment
        enumerate, % Allow custom enumerate numbering
        graphicx, % allow includegraphics with more filetypes
        wasysym, % \smiley!
        upgreek, % \upmu for \mum macro
        listings, % writing TrueType fonts and including code prettily
        tikz, % drawing things
        booktabs, % \bottomrule instead of hline apparently
        cancel % can cancel things out!
    }
    \usepackage[
        labelfont=bf, % caption names are labeled in bold
        font=scriptsize % smaller font for captions
    ]{caption}
    \usepackage[font=scriptsize]{subcaption} % subfigures

    \newcommand*{\scinot}[2]{#1\times10^{#2}}
    \newcommand*{\dotp}[2]{\left<#1\,\middle|\,#2\right>}
    \newcommand*{\rd}[2]{\frac{\mathrm{d}#1}{\mathrm{d}#2}}
    \newcommand*{\pd}[2]{\frac{\partial#1}{\partial#2}}
    \newcommand*{\rtd}[2]{\frac{\mathrm{d}^2#1}{\mathrm{d}#2^2}}
    \newcommand*{\ptd}[2]{\frac{\partial^2 #1}{\partial#2^2}}
    \newcommand*{\md}[2]{\frac{\mathrm{D}#1}{\mathrm{D}#2}}
    \newcommand*{\pvec}[1]{\vec{#1}^{\,\prime}}
    \newcommand*{\svec}[1]{\vec{#1}\;\!}
    \newcommand*{\bm}[1]{\boldsymbol{\mathbf{#1}}}
    \newcommand*{\ang}[0]{\;\text{\AA}}
    \newcommand*{\mum}[0]{\;\upmu \mathrm{m}}
    \newcommand*{\at}[1]{\left.#1\right|}

    \let\Re\undefined
    \let\Im\undefined
    \DeclareMathOperator{\Res}{Res}
    \DeclareMathOperator{\Re}{Re}
    \DeclareMathOperator{\Im}{Im}
    \DeclareMathOperator{\Log}{Log}
    \DeclareMathOperator{\Arg}{Arg}
    \DeclareMathOperator{\Tr}{Tr}
    \DeclareMathOperator{\E}{E}
    \DeclareMathOperator{\Var}{Var}
    \DeclareMathOperator*{\argmin}{argmin}
    \DeclareMathOperator*{\argmax}{argmax}
    \DeclareMathOperator{\sgn}{sgn}
    \DeclareMathOperator{\diag}{diag\;}

    \DeclarePairedDelimiter\bra{\langle}{\rvert}
    \DeclarePairedDelimiter\ket{\lvert}{\rangle}
    \DeclarePairedDelimiter\abs{\lvert}{\rvert}
    \DeclarePairedDelimiter\ev{\langle}{\rangle}
    \DeclarePairedDelimiter\p{\lparen}{\rparen}
    \DeclarePairedDelimiter\s{\lbrack}{\rbrack}
    \DeclarePairedDelimiter\z{\lbrace}{\rbrace}

    % \everymath{\displaystyle} % biggify limits of inline sums and integrals
    \tikzstyle{circ} % usage: \node[circ, placement] (label) {text};
        = [draw, circle, fill=white, node distance=3cm, minimum height=2em]
    \definecolor{commentgreen}{rgb}{0,0.6,0}
    \lstset{
        basicstyle=\ttfamily\footnotesize,
        frame=single,
        numbers=left,
        showstringspaces=false,
        keywordstyle=\color{blue},
        stringstyle=\color{purple},
        commentstyle=\color{commentgreen},
        morecomment=[l][\color{magenta}]{\#}
    }

\begin{document}

\title[Cassini S-O Misalignment]{High Spin-Orbit Misalignment as an Attractor in
Cassini State Systems with Weak Tidal Friction}
\subtitle{Meeting XX/XX/XXXX}
\author{Yubo Su}
\date{Some day}

\maketitle

\begin{frame}
    \frametitle{Problem 1}
    \framesubtitle{Introduction}

    \begin{itemize}
        \item Close-in planet to a star w/ $\vec{l}$ gstarts with random spin
            $\hat{s}$ (e.g.\ collision). Evolves under tides + precession around
            perturber $\hat{l}_p$.

        \item \textbf{Toy Problem:} Assume constant tidal dissipation, fate?

        \item Cassini States: $H^{(0)} = \frac{\p*{\hat{s} \cdot \hat{l}}^2}{2}
            + \eta \hat{s} \cdot \hat{l}_p$. CS4 is saddle point,
            \emph{separatrix}.

        \begin{figure}[t]
            \centering
            \includegraphics[width=0.5\textwidth]{../../initial/0_eta/1contours.png}
        \end{figure}
    \end{itemize}
\end{frame}

\begin{frame}
    \frametitle{Problem 1}
    \framesubtitle{Constant Tides}

    \begin{itemize}
        \item Constant tides $\rd{\theta}{t} = \epsilon \sin\theta$,
            EOM ($\mu = \cos \theta$):
            \begin{align*}
                \rd{\hat{s}}{\tau}
                    &= \p*{\hat{s} \cdot \hat{l}}\p*{\hat{s} \times \hat{l}}
                        - \eta\hat{s} \times \hat{l}_p +
                        \epsilon \hat{s} \times \p*{\hat{l} \times \hat{s}}
                        ,\\
                \pd{\phi}{t} &= \mu - \eta
                    \p*{\cos I + \sin I \frac{\mu}{\sqrt{1 - \mu^2}}
                        \cos \phi},\\
                \pd{\mu}{t} &= -\eta \sin I \sin \phi
                    + \epsilon \p*{1 - \mu^2},
            \end{align*}

        \item \emph{Review:} Last meeting, found probability that initial state
            $\cos\theta < 0$ jumps into separatrix $\propto
            \eta^{3/2}\epsilon^0$.
    \end{itemize}
\end{frame}

\begin{frame}
    \frametitle{Problem 1}
    \framesubtitle{Flow Boundaries}

    \begin{columns}
        \begin{column}{0.5\textwidth}
            \begin{figure}[t]
                \centering
                \includegraphics[width=0.9\columnwidth]{../../initial/0_eta/6manifolds0_20.png}
            \end{figure}
        \end{column}
        \begin{column}{0.5\textwidth}
            \begin{itemize}
                \item Notation: subscripts \emph{stable, unstable} manifolds,
                    superscripts are left/right copy of CS4.

                    \begin{itemize}
                        \item Below black, continues orbit $\mu < \mu_4$.

                        \item Black/green, escapes.

                        \item Green/red, captures.
                    \end{itemize}

                \item Evaluate $\Delta \mu$ at $\phi = \pi$ reproduces
                    probability (numerical was $P_{hop}(\eta = 0.2) = 0.251$).
            \end{itemize}
        \end{column}
    \end{columns}
\end{frame}

\begin{frame}
    \frametitle{Problem 1}
    \framesubtitle{Melnikov's Method (coarsely)}

    \begin{columns}
        \begin{column}{0.5\textwidth}
            \begin{itemize}
                \item \textbf{Goal:} Under weak perturbation, how much does
                    trajectory deviate from level curve of $H$?

                \item Procedure:

                    \begin{itemize}
                        \item Compute $\Delta H^{(0)}$ unperturbed  over
                            trajectory.

                        \item Locate new level curve to find $\Delta q$ along
                            coordinate: $\Delta H^{(0)} =
                            \pd{H^{(0)}}{q}\Delta q$.
                    \end{itemize}
                \item So looks something like:
                    {\tiny
                    \begin{equation*}
                        \rd{H^{(0)}}{t} =
                            \underbrace{
                            \cancel{\pd{H^{(0)}}{\mu}\rd{\mu^{(0)}}{t}
                                + \pd{H^{(0)}}{\phi}\rd{\phi^{(0)}}{t}}}
                                _{\dot{\phi}^{(0)}\dot{\mu}^{(0)}
                                    - \dot{\mu}^{(0)}\dot{\phi}^{(0)}}
                            + \pd{H^{(0)}}{\mu} \rd{\mu^{(1)}}{t}.
                    \end{equation*}}
            \end{itemize}
        \end{column}
        \begin{column}{0.5\textwidth}
            \begin{figure}[t]
                \centering
                \includegraphics[width=0.6\columnwidth]{../../initial/0_eta/6manifolds0_20.png}
            \end{figure}
            \begin{itemize}
                \item Gives rise to Melnikov distance:

                \begin{itemize}
                    \item $\Delta \mu(\phi) = \frac{1}{\dot{\phi}(\phi)}
                            \oint \dot{\phi}^{(0)} \epsilon\p*{1 - \mu^2}
                                \;\mathrm{d}t.$
                    \item $\mu(\phi) \approx
                        \eta \cos I \pm \mathcal{O}\p*{\sqrt{\eta}}$.

                    \item $W_s^{(1)}, W_u^{(0)}$: dominated by $1$.

                    \item $W_s^{(0)}, W_u^{(0)}$: dominated by $\eta^{3/2}$!
                \end{itemize}
            \end{itemize}
        \end{column}
    \end{columns}
\end{frame}

\begin{frame}
    \frametitle{Problem 2}
    \framesubtitle{Realistic Tides}

    \begin{columns}
        \begin{column}{0.5\textwidth}
            \begin{itemize}
                \item In realistic tides, $\eta$ can evolve as $s$ spins down.

                \item $\eta = \frac{s_c}{s}$, full expression below.

                \item First, consider tides alone, phase portrait at right:

                \begin{itemize}
                    \item Like ignoring $\rd{\mu}{\phi}$ over precession orbit.

                    \item Breaks down near $\mu_4$!
                \end{itemize}
            \end{itemize}
        \end{column}
        \begin{column}{0.5\textwidth}
            \begin{figure}[t]
                \centering
                \includegraphics[width=\columnwidth]{../../initial/1_weaktide/2quiver0_7.png}
            \end{figure}
        \end{column}
    \end{columns}

    \begin{align}
        \rd{\hat{s}}{\tau}
            &= \frac{s}{s_c}\p*{\hat{s} \cdot \hat{l}}
                \p*{\hat{s} \times \hat{l}}
                - \hat{s} \times \hat{l}_p
                + \frac{\epsilon 2\Omega}{s}
                    \p*{1 - \frac{s}{2\Omega}\p*{\hat{l} \cdot \hat{s}}}
                        \hat{s} \times \p*{\hat{l} \times \hat{s}},\\
        \rd{s}{\tau}
            &= \epsilon 2\Omega \p*{\hat{s} \cdot \hat{l} - \frac{s}{2\Omega}\p*{1
                + \p*{\hat{s} \cdot \hat{l}}^2}}.
    \end{align}
\end{frame}

\begin{frame}
    \frametitle{Problem 2}
    \framesubtitle{Intuitive Guess}

    \begin{itemize}
        \item Ignore perturber for now ($\eta \ll 1$). Maybe: \emph{points
            incident on CS4} have $P_{hop}$ probability of entering separatrix?

        \item For instance, shaded green = capture region:
    \end{itemize}

    \begin{figure}[t]
        \centering
        \includegraphics[width=0.6\textwidth]{../../initial/1_weaktide/2phase_portrait0_7.png}
    \end{figure}
\end{frame}

\begin{frame}
    \frametitle{Problem 2}
    \framesubtitle{Near CS4}

    \begin{columns}
        \begin{column}{0.5\textwidth}
            \begin{figure}[t]
                \centering
                \includegraphics[width=0.7\columnwidth]{../../initial/0_eta/1contours_02.png}
            \end{figure}

            \begin{itemize}
                \item When $\eta$ non-negligible, full phase space $(\mu, \phi,
                    s)$.

                \item Study $(\mu_0, s)$ where $\mu_0 \equiv \mu(\phi = 0)$.
                    \begin{itemize}
                        \item $\mu_0 \approx \mu_4$ = $P_{hop}$ candidate.
                    \end{itemize}
            \end{itemize}
        \end{column}
        \begin{column}{0.5\textwidth}
            \begin{itemize}
                \item Update model:
                    \begin{itemize}
                        \item $\abs*{\mu_0 - \mu_4} \gtrsim \sqrt{\eta \sin I}$,
                            then $\mu(\phi) \approx \mu_0$.

                        \item $\abs*{\mu_0 - \mu_4} \lesssim
                            \sqrt{\eta \sin I}$, turns out

                            \begin{equation}
                                \Delta \mu \sim
                                    \frac{\epsilon}{\mu_4 - \mu_0}
                                        \dot{\mu}^{T}\p*{\mu_{eff}}.
                            \end{equation}

                        \item $\dot{\mu}^{T}$ is just the tidal component.

                        \item $\mu_{eff}$ such that $\dot{\mu}^{T}(\mu_{eff})
                            \equiv \ev*{\dot{\mu}^{T}(\mu(t))}$.
                    \end{itemize}
            \end{itemize}
        \end{column}
    \end{columns}
\end{frame}

\begin{frame}
    \frametitle{Problem 2}
    \framesubtitle{Effective Model}

    \begin{columns}
        \begin{column}{0.5\textwidth}
            \begin{equation*}
                \Delta \mu \sim \frac{\epsilon}{\mu_4 - \mu_0}
                        \dot{\mu}^{T}\p*{\mu_{eff}}.
            \end{equation*}
            \begin{itemize}
                \item Two key properties:
                    \begin{itemize}
                        \item $\Delta \mu$'s sign is set at $\mu_{eff}$, which
                            changes steeply $\mu_0 \approx \mu_4$.

                        \item $\mu_0 \approx \mu_4$ means large $\Delta \mu$.
                    \end{itemize}

                \item Trajectories will cross CS4 \emph{multiple times}!

                \item Effective ``phase portrait''.
                    \begin{itemize}
                        \item Shade $\abs*{\mu_4 - \mu_0} < \sqrt{\eta\sin I}$.
                            \emph{Attracting}!
                    \end{itemize}
            \end{itemize}
        \end{column}
        \begin{column}{0.5\textwidth}
            \begin{figure}[t]
                \centering
                \includegraphics[width=0.7\columnwidth]{../../initial/1_weaktide/2quiver_int0_7.png}
            \end{figure}
            \begin{figure}[t]
                \centering
                \includegraphics[width=0.7\columnwidth]{../../initial/1_weaktide/2quiver_pw0_7.png}
            \end{figure}
        \end{column}
    \end{columns}
\end{frame}

\begin{frame}
    \frametitle{Problem 2}
    \framesubtitle{Capture Region}

    \begin{itemize}
        \item Effective CS4 cross section attracts onto CS4, trajectories cross
            CS4 multiple times, $P_{hop} \approx 1$ to CS2.

        \item Capture region of cross section? Almost all!

        \begin{figure}[t]
            \centering
            \includegraphics[width=0.6\textwidth]{../../initial/1_weaktide/2phase_portrait_pw0_7.png}
        \end{figure}
    \end{itemize}
\end{frame}

\begin{frame}
    \frametitle{Problem 2}
    \framesubtitle{Summary}

    \begin{columns}
        \begin{column}{0.5\textwidth}
            \begin{figure}[t]
                \centering
                \includegraphics[width=0.8\columnwidth]{../../initial/1_weaktide/1sim_0_7xn0_9.png}
            \end{figure}
            \begin{itemize}
                \item Slightly outside of capture region in previous slide,
                    barely intersects $\mu_4$ before $\mu_4$ disappears. Near
                    capture boundary.
            \end{itemize}
        \end{column}
        \begin{column}{0.5\textwidth}
            \begin{itemize}
                \item Capture region
                    \begin{itemize}
                        \item Flow CS4 effective cross section backwards in
                            time.

                        \item 100\% goes to CS2.
                    \end{itemize}

                \item Above: stays near CS1, tides stay weak.

                \item Below: cannot catch up to CS4, go on level curves of
                    large-$\eta$ Hamiltonian.
            \end{itemize}
        \end{column}
    \end{columns}
\end{frame}

\end{document}

