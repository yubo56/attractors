    \documentclass[dvipsnames, 11pt]{beamer}
    \usetheme{Madrid}
    \usefonttheme{professionalfonts}
    \usepackage{
        amsmath,
        amssymb,
        fouriernc, % fourier font w/ new century book
        fancyhdr, % page styling
        lastpage, % footer fanciness
        hyperref, % various links
        setspace, % line spacing
        amsthm, % newtheorem and proof environment
        mathtools, % \Aboxed for boxing inside aligns, among others
        float, % Allow [H] figure env alignment
        enumerate, % Allow custom enumerate numbering
        graphicx, % allow includegraphics with more filetypes
        wasysym, % \smiley!
        upgreek, % \upmu for \mum macro
        listings, % writing TrueType fonts and including code prettily
        tikz, % drawing things
        booktabs, % \bottomrule instead of hline apparently
        cancel % can cancel things out!
    }
    \usepackage[
        labelfont=bf, % caption names are labeled in bold
        font=scriptsize % smaller font for captions
    ]{caption}
    \usepackage[font=scriptsize]{subcaption} % subfigures

    \newcommand*{\scinot}[2]{#1\times10^{#2}}
    \newcommand*{\dotp}[2]{\left<#1\,\middle|\,#2\right>}
    \newcommand*{\rd}[2]{\frac{\mathrm{d}#1}{\mathrm{d}#2}}
    \newcommand*{\pd}[2]{\frac{\partial#1}{\partial#2}}
    \newcommand*{\rtd}[2]{\frac{\mathrm{d}^2#1}{\mathrm{d}#2^2}}
    \newcommand*{\ptd}[2]{\frac{\partial^2 #1}{\partial#2^2}}
    \newcommand*{\md}[2]{\frac{\mathrm{D}#1}{\mathrm{D}#2}}
    \newcommand*{\pvec}[1]{\vec{#1}^{\,\prime}}
    \newcommand*{\svec}[1]{\vec{#1}\;\!}
    \newcommand*{\bm}[1]{\boldsymbol{\mathbf{#1}}}
    \newcommand*{\ang}[0]{\;\text{\AA}}
    \newcommand*{\mum}[0]{\;\upmu \mathrm{m}}
    \newcommand*{\at}[1]{\left.#1\right|}

    \let\Re\undefined
    \let\Im\undefined
    \DeclareMathOperator{\Res}{Res}
    \DeclareMathOperator{\Re}{Re}
    \DeclareMathOperator{\Im}{Im}
    \DeclareMathOperator{\Log}{Log}
    \DeclareMathOperator{\Arg}{Arg}
    \DeclareMathOperator{\Tr}{Tr}
    \DeclareMathOperator{\E}{E}
    \DeclareMathOperator{\Var}{Var}
    \DeclareMathOperator*{\argmin}{argmin}
    \DeclareMathOperator*{\argmax}{argmax}
    \DeclareMathOperator{\sgn}{sgn}
    \DeclareMathOperator{\diag}{diag\;}

    \DeclarePairedDelimiter\bra{\langle}{\rvert}
    \DeclarePairedDelimiter\ket{\lvert}{\rangle}
    \DeclarePairedDelimiter\abs{\lvert}{\rvert}
    \DeclarePairedDelimiter\ev{\langle}{\rangle}
    \DeclarePairedDelimiter\p{\lparen}{\rparen}
    \DeclarePairedDelimiter\s{\lbrack}{\rbrack}
    \DeclarePairedDelimiter\z{\lbrace}{\rbrace}

    % \everymath{\displaystyle} % biggify limits of inline sums and integrals
    \tikzstyle{circ} % usage: \node[circ, placement] (label) {text};
        = [draw, circle, fill=white, node distance=3cm, minimum height=2em]
    \definecolor{commentgreen}{rgb}{0,0.6,0}
    \lstset{
        basicstyle=\ttfamily\footnotesize,
        frame=single,
        numbers=left,
        showstringspaces=false,
        keywordstyle=\color{blue},
        stringstyle=\color{purple},
        commentstyle=\color{commentgreen},
        morecomment=[l][\color{magenta}]{\#}
    }

\begin{document}

\title[Cassini S-O Misalignment]{High Spin-Orbit Misalignment is Sometimes
Attracting: Cassini State Systems with Weak Tidal Friction}
\subtitle{Group Meeting}
\author{Yubo Su}
\date{May 31, 2019}

\maketitle

\begin{frame}
    \frametitle{Problem 1}
    \framesubtitle{Introduction}

    \begin{itemize}
        \item Close-in planet to a star w/ $\vec{l}$ gstarts with random spin
            $\hat{s}$ (e.g.\ collision). Evolves under tides + precession around
            perturber $\hat{l}_p$.

        \item \textbf{Toy Problem:} Assume constant tidal dissipation, fate?

        \item Cassini States: $H^{(0)} = \frac{\p*{\hat{s} \cdot \hat{l}}^2}{2}
            + \eta \hat{s} \cdot \hat{l}_p$. CS4 is saddle point,
            \emph{separatrix}.

        \begin{figure}[t]
            \centering
            \includegraphics[width=0.5\textwidth]{../../initial/0_eta/1contours.png}
        \end{figure}
    \end{itemize}
\end{frame}

\begin{frame}
    \frametitle{Problem 1}
    \framesubtitle{Constant Tides}

    \begin{itemize}
        \item Constant tides $\rd{\theta}{t} = \epsilon \sin\theta$,
            EOM ($\mu = \cos \theta$):
            \begin{align*}
                \rd{\hat{s}}{\tau}
                    &= \p*{\hat{s} \cdot \hat{l}}\p*{\hat{s} \times \hat{l}}
                        - \eta\hat{s} \times \hat{l}_p +
                        \epsilon \hat{s} \times \p*{\hat{l} \times \hat{s}}.
            \end{align*}

        \item \emph{Review:} Last meeting, found $P_{hop} \propto
            \eta^{3/2}\epsilon^0$.

        \begin{figure}
            \centering
            \includegraphics[width=0.6\textwidth]{../../initial/0_eta/3stats3_5_0_2.png}
        \end{figure}
    \end{itemize}
\end{frame}

\begin{frame}
    \frametitle{Problem 1}
    \framesubtitle{Flow Boundaries (optional)}

    \begin{columns}
        \begin{column}{0.5\textwidth}
            \begin{figure}[t]
                \centering
                \includegraphics[width=0.9\columnwidth]{../../initial/0_eta/6manifolds0_20.png}
            \end{figure}
        \end{column}
        \begin{column}{0.5\textwidth}
            \begin{itemize}
                \item Key result:
                    \begin{equation*}
                        P_{hop} = \frac{16\eta^{3/2}\cos I \sqrt{\sin I}}{\pi}.
                    \end{equation*}

                \item Analytical, similar to MMR capture probability.
            \end{itemize}
        \end{column}
    \end{columns}
\end{frame}

\begin{frame}
    \frametitle{Problem 2}
    \framesubtitle{Realistic Tides}

    \begin{itemize}
        \item In realistic tides, $\eta$ can evolve as $s$ spins down.

        \item $\eta \equiv \frac{s_c}{s}$, so $s_c$ is \emph{critical spin at
            which perturber strength is of order spin-orbit coupling}.
            \begin{align}
                \rd{\hat{s}}{\tau}
                    &= \frac{s}{s_c}\p*{\hat{s} \cdot \hat{l}}
                        \p*{\hat{s} \times \hat{l}}
                        - \hat{s} \times \hat{l}_p
                        + \frac{\epsilon 2\Omega}{s}
                            \p*{1 - \frac{s}{2\Omega}\p*{\hat{l} \cdot \hat{s}}}
                                \hat{s} \times \p*{\hat{l} \times \hat{s}},\\
                \rd{s}{\tau}
                    &= \epsilon 2\Omega \p*{\hat{s} \cdot \hat{l} -
                        \frac{s}{2\Omega}\p*{1 + \p*{\hat{s} \cdot \hat{l}}^2}}.
            \end{align}

        \item Compare to Problem 1:
            \begin{equation*}
                \rd{\hat{s}}{\tau}
                    = \p*{\hat{s} \cdot \hat{l}}\p*{\hat{s} \times \hat{l}}
                        - \eta\hat{s} \times \hat{l}_p +
                        \epsilon \hat{s} \times \p*{\hat{l} \times \hat{s}}.
            \end{equation*}
    \end{itemize}
\end{frame}

\begin{frame}
    \frametitle{Problem 2}
    \framesubtitle{Outcomes}

    \begin{itemize}
        \item Three zones:
            \begin{figure}[t]
                \centering
                \includegraphics[width=0.5\textwidth]{../../initial/1_weaktide/3zones.png}
            \end{figure}

        \item According to work from Problem 1, expect:
            \begin{description}
                \item[I] Goes to CS1/alignment.
                \item[II] Goes to CS2/misalignment.
                \item[III] $P_{hop}$ to CS2 or CS1.
            \end{description}
    \end{itemize}
\end{frame}

\begin{frame}
    \frametitle{Problem 2}
    \framesubtitle{$s_c = 0.2$, Zone I}

    \begin{columns}
        \begin{column}{0.5\textwidth}
            \begin{figure}
                \centering
                \begin{subfigure}{\columnwidth}
                    \centering
                    \includegraphics[width=0.7\columnwidth]{../../initial/1_weaktide/3zones.png}
                \end{subfigure}

                \begin{subfigure}{\columnwidth}
                    \centering
                    \includegraphics[width=0.9\columnwidth]{../../initial/1_weaktide/4plots/0_200x0_300x0_000_ind.png}
                \end{subfigure}
            \end{figure}
        \end{column}
        \begin{column}{0.5\textwidth}
            \begin{figure}[t]
                \centering
                \includegraphics[width=\columnwidth]{../../initial/1_weaktide/4plots/0_200x0_300x0_000.png}
            \end{figure}
        \end{column}
    \end{columns}
\end{frame}

\begin{frame}
    \frametitle{Problem 2}
    \framesubtitle{$s_c = 0.2$, Zone II}

    \begin{columns}
        \begin{column}{0.5\textwidth}
            \begin{figure}
                \centering
                \begin{subfigure}{\columnwidth}
                    \centering
                    \includegraphics[width=0.7\columnwidth]{../../initial/1_weaktide/3zones.png}
                \end{subfigure}

                \begin{subfigure}{\columnwidth}
                    \centering
                    \includegraphics[width=0.9\columnwidth]{../../initial/1_weaktide/4plots/0_200x0_050x2_094_ind.png}
                \end{subfigure}
            \end{figure}
        \end{column}
        \begin{column}{0.5\textwidth}
            \begin{figure}[t]
                \centering
                \includegraphics[width=\columnwidth]{../../initial/1_weaktide/4plots/0_200x0_050x2_094.png}
            \end{figure}
        \end{column}
    \end{columns}
\end{frame}

\begin{frame}
    \frametitle{Problem 2}
    \framesubtitle{$s_c = 0.2$, Zone III, Separatrix Hopping}

    \begin{columns}
        \begin{column}{0.5\textwidth}
            \begin{figure}
                \centering
                \begin{subfigure}{\columnwidth}
                    \centering
                    \includegraphics[width=0.7\columnwidth]{../../initial/1_weaktide/3zones.png}
                \end{subfigure}

                \begin{subfigure}{\columnwidth}
                    \centering
                    \includegraphics[width=0.9\columnwidth]{../../initial/1_weaktide/4plots/0_200xn0_800x0_000_ind.png}
                \end{subfigure}
            \end{figure}
        \end{column}
        \begin{column}{0.5\textwidth}
            \begin{figure}[t]
                \centering
                \includegraphics[width=\columnwidth]{../../initial/1_weaktide/4plots/0_200xn0_800x0_000.png}
            \end{figure}
        \end{column}
    \end{columns}
\end{frame}

\begin{frame}
    \frametitle{Problem 2}
    \framesubtitle{$s_c = 0.2$, Zone III, Separatrix Crossing}

    \begin{columns}
        \begin{column}{0.5\textwidth}
            \begin{figure}
                \centering
                \begin{subfigure}{\columnwidth}
                    \centering
                    \includegraphics[width=0.7\columnwidth]{../../initial/1_weaktide/3zones.png}
                \end{subfigure}

                \begin{subfigure}{\columnwidth}
                    \centering
                    \includegraphics[width=0.9\columnwidth]{../../initial/1_weaktide/4plots/0_200xn0_820x0_000_ind.png}
                \end{subfigure}
            \end{figure}
        \end{column}
        \begin{column}{0.5\textwidth}
            \begin{figure}[t]
                \centering
                \includegraphics[width=\columnwidth]{../../initial/1_weaktide/4plots/0_200xn0_820x0_000.png}
            \end{figure}
        \end{column}
    \end{columns}
\end{frame}

\begin{frame}
    \frametitle{Problem 2}
    \framesubtitle{$s_c = 0.7$, Zone I, Separatrix Crossing!}

    \begin{columns}
        \begin{column}{0.5\textwidth}
            \begin{figure}
                \centering
                \begin{subfigure}{\columnwidth}
                    \centering
                    \includegraphics[width=0.7\columnwidth]{../../initial/1_weaktide/3zones.png}
                \end{subfigure}

                \begin{subfigure}{\columnwidth}
                    \centering
                    \includegraphics[width=0.9\columnwidth]{../../initial/1_weaktide/4plots/0_700x0_800x0_000_ind.png}
                \end{subfigure}
            \end{figure}
        \end{column}
        \begin{column}{0.5\textwidth}
            \begin{figure}[t]
                \centering
                \includegraphics[width=\columnwidth]{../../initial/1_weaktide/4plots/0_700x0_800x0_000.png}
            \end{figure}
        \end{column}
    \end{columns}
\end{frame}

\begin{frame}
    \frametitle{Problem 2}
    \framesubtitle{$s_c = 0.7$, Zone III, Always Separatrix Crossing!}

    \begin{columns}
        \begin{column}{0.5\textwidth}
            \begin{figure}
                \centering
                \begin{subfigure}{\columnwidth}
                    \centering
                    \includegraphics[width=0.7\columnwidth]{../../initial/1_weaktide/3zones.png}
                \end{subfigure}

                \begin{subfigure}{\columnwidth}
                    \centering
                    \includegraphics[width=0.9\columnwidth]{../../initial/1_weaktide/4plots/0_700xn0_800x0_000_ind.png}
                \end{subfigure}
            \end{figure}
        \end{column}
        \begin{column}{0.5\textwidth}
            \begin{figure}[t]
                \centering
                \includegraphics[width=\columnwidth]{../../initial/1_weaktide/4plots/0_700xn0_800x0_000.png}
            \end{figure}
        \end{column}
    \end{columns}
\end{frame}

\begin{frame}
    \frametitle{Problem 2}
    \framesubtitle{Outcome Distribution}

    \begin{columns}
        \begin{column}{0.5\textwidth}
            \begin{itemize}
                \item In summary, going from $s_c = 0.2$ to $s_c = 0.7$ makes
                    CS2 attracting in zone I, sets $P_{hop} = 1$ for zone III\@.

                \item Interesting histories
                    \begin{itemize}
                        \item CS1, no sep crossing (I).
                        \item CS2, no sep crossing (II).
                        \item Cross to CS1 (III).
                        \item Hop to CS2 (III).
                    \end{itemize}

                \item At $s_c \lesssim 0.3$, $P_{hop} \propto \sqrt{s}$, so
                    closer to sep $\Rightarrow$ higher $P_{hop}$.
            \end{itemize}
        \end{column}
        \begin{column}{0.5\textwidth}
            \begin{figure}[t]
                \centering
                \includegraphics[width=\columnwidth]{../../initial/1_weaktide/4_stats0_2.png}
            \end{figure}
        \end{column}
    \end{columns}
\end{frame}

\begin{frame}
    \frametitle{Problem 2}
    \framesubtitle{Evolution of Outcome Distribution with $s_c$}

    \begin{figure}[t]
        \centering
        \begin{subfigure}{0.4\textwidth}
            \centering
                \includegraphics[width=\columnwidth]{../../initial/1_weaktide/4_stats0_3.png}
        \end{subfigure}
        \begin{subfigure}{0.4\textwidth}
            \centering
                \includegraphics[width=\columnwidth]{../../initial/1_weaktide/4_stats0_4.png}
        \end{subfigure}

        \begin{subfigure}{0.4\textwidth}
            \centering
                \includegraphics[width=\columnwidth]{../../initial/1_weaktide/4_stats0_5.png}
        \end{subfigure}
        \begin{subfigure}{0.4\textwidth}
            \centering
                \includegraphics[width=\columnwidth]{../../initial/1_weaktide/4_stats0_7.png}
        \end{subfigure}
    \end{figure}
\end{frame}

\begin{frame}
    \frametitle{Problem 2}
    \framesubtitle{Phase Portrait}

    \begin{columns}
        \begin{column}{0.5\textwidth}
            \begin{itemize}
                \item In the absence of perturber (just the weak tide
                    components), phase portrait takes on shape:

                \item Green is $\mu_4$ Cassini State 4, above red is $\dot{\mu}
                    < 0$.
            \end{itemize}
        \end{column}
        \begin{column}{0.5\textwidth}
            \begin{figure}[t]
                \centering
                \includegraphics[width=\columnwidth]{../../initial/1_weaktide/2quiver0_7.png}
            \end{figure}
        \end{column}
    \end{columns}
\end{frame}

\begin{frame}
    \frametitle{Problem 2}
    \framesubtitle{Sign of $\ev*{\rd{\mu}{t}}$}

    \begin{columns}
        \begin{column}{0.5\textwidth}
            \begin{figure}[t]
                \centering
                \begin{subfigure}{0.8\columnwidth}
                    \centering
                    \includegraphics[width=\textwidth]{../../initial/1_weaktide/3diags0_3.png}
                \end{subfigure}
                \begin{subfigure}{0.8\columnwidth}
                    \centering
                    \includegraphics[width=\textwidth]{../../initial/1_weaktide/3diags0_7.png}
                \end{subfigure}
            \end{figure}
        \end{column}
        \begin{column}{0.5\textwidth}
            \begin{figure}[t]
                \centering
                \includegraphics[width=\columnwidth]{../../initial/1_weaktide/3crits.png}
            \end{figure}
        \end{column}
    \end{columns}
\end{frame}

\end{document}

