    \documentclass[dvipsnames, 9pt]{beamer}
    \usetheme{Madrid}
    \usefonttheme{professionalfonts}
    \usepackage{
        amsmath,
        amssymb,
        fouriernc, % fourier font w/ new century book
        fancyhdr, % page styling
        lastpage, % footer fanciness
        hyperref, % various links
        setspace, % line spacing
        amsthm, % newtheorem and proof environment
        mathtools, % \Aboxed for boxing inside aligns, among others
        float, % Allow [H] figure env alignment
        enumerate, % Allow custom enumerate numbering
        graphicx, % allow includegraphics with more filetypes
        wasysym, % \smiley!
        upgreek, % \upmu for \mum macro
        listings, % writing TrueType fonts and including code prettily
        tikz, % drawing things
        booktabs, % \bottomrule instead of hline apparently
        cancel % can cancel things out!
    }
    \usepackage[
        labelfont=bf, % caption names are labeled in bold
        font=scriptsize % smaller font for captions
    ]{caption}
    \usepackage[font=scriptsize]{subcaption} % subfigures

    \newcommand*{\scinot}[2]{#1\times10^{#2}}
    \newcommand*{\dotp}[2]{\left<#1\,\middle|\,#2\right>}
    \newcommand*{\rd}[2]{\frac{\mathrm{d}#1}{\mathrm{d}#2}}
    \newcommand*{\pd}[2]{\frac{\partial#1}{\partial#2}}
    \newcommand*{\rdil}[2]{\mathrm{d}#1 / \mathrm{d}#2}
    \newcommand*{\pdil}[2]{\partial#1 / \partial#2}
    \newcommand*{\rtd}[2]{\frac{\mathrm{d}^2#1}{\mathrm{d}#2^2}}
    \newcommand*{\ptd}[2]{\frac{\partial^2 #1}{\partial#2^2}}
    \newcommand*{\md}[2]{\frac{\mathrm{D}#1}{\mathrm{D}#2}}
    \newcommand*{\pvec}[1]{\vec{#1}^{\,\prime}}
    \newcommand*{\svec}[1]{\vec{#1}\;\!}
    \newcommand*{\bm}[1]{\boldsymbol{\mathbf{#1}}}
    \newcommand*{\uv}[1]{\hat{\bm{#1}}}
    \newcommand*{\ang}[0]{\;\text{\AA}}
    \newcommand*{\mum}[0]{\;\upmu \mathrm{m}}
    \newcommand*{\at}[1]{\left.#1\right|}
    \newcommand*{\bra}[1]{\left<#1\right|}
    \newcommand*{\ket}[1]{\left|#1\right>}
    \newcommand*{\abs}[1]{\left|#1\right|}
    \newcommand*{\ev}[1]{\langle#1\rangle}
    \newcommand*{\p}[1]{\left(#1\right)}
    \newcommand*{\s}[1]{\left[#1\right]}
    \newcommand*{\z}[1]{\left\{#1\right\}}

    \let\Re\undefined
    \let\Im\undefined
    \DeclareMathOperator{\Res}{Res}
    \DeclareMathOperator{\Re}{Re}
    \DeclareMathOperator{\Im}{Im}
    \DeclareMathOperator{\Log}{Log}
    \DeclareMathOperator{\Arg}{Arg}
    \DeclareMathOperator{\Tr}{Tr}
    \DeclareMathOperator{\E}{E}
    \DeclareMathOperator{\Var}{Var}
    \DeclareMathOperator*{\argmin}{argmin}
    \DeclareMathOperator*{\argmax}{argmax}
    \DeclareMathOperator{\sgn}{sgn}
    \DeclareMathOperator{\diag}{diag\;}

    % \everymath{\displaystyle} % biggify limits of inline sums and integrals
    \tikzstyle{circ} % usage: \node[circ, placement] (label) {text};
        = [draw, circle, fill=white, node distance=3cm, minimum height=2em]
    \definecolor{commentgreen}{rgb}{0,0.6,0}
    \lstset{
        basicstyle=\ttfamily\footnotesize,
        frame=single,
        numbers=left,
        showstringspaces=false,
        keywordstyle=\color{blue},
        stringstyle=\color{purple},
        commentstyle=\color{commentgreen},
        morecomment=[l][\color{magenta}]{\#}
    }

\begin{document}

\title{Cassini States and Tidal Dissipation}
\subtitle{Group Meeting Presentation}
\author{Yubo Su \& Dong Lai}
\date{June 03, 2021}

\maketitle

\begin{frame}
    \frametitle{Motivation}
    \framesubtitle{Super Earth (SE) + Cold Jupiter (CJ)}

    \begin{itemize}
        \item Zhu \& Wu 2018 find that many SEs have CJ companions.

        \item SE may have experienced giant impacts, giving it a
            nontrivial initial obliquity (spin-orbit misalignment
            angle).

        \item Spin of SE evolves under tidal interactions with host
            star.

        \item SE also experiences Cassini State dynamics (spin-orbit and
            orbit-orbit precession).

        \item \emph{What is the final outcome of the spin of the SE?}
    \end{itemize}
\end{frame}

\begin{frame}
    \frametitle{Dynamics}
    \framesubtitle{Cassini States}

    \begin{columns}
        \begin{column}{0.5\columnwidth}
            \begin{itemize}
                \item Central star $M_\star$, inner planet $m$, and perturber
                    $m_{\rm p}$ mildly inclined by $I$.

                \item Two precession effects on inner planet:
                \begin{itemize}
                    \item Spin-orbit coupling:
                        \begin{align*}
                            \rd{\uv{s}}{t}
                                &= \omega_{\rm sl}
                                    \p{\uv{s} \cdot \uv{l}}
                                    \p{\uv{s} \times \uv{l}},\\
                            \omega_{\rm sl}
                                &\equiv
                                    \frac{3GJ_2 mR^2 M_\star}
                                    {2a^3 I}\textcolor{red}{\Omega_{\rm s}}.
                        \end{align*}

                    \item Orbit-orbit coupling
                        \begin{align*}
                            \rd{\uv{l}}{t}
                                &= \omega_{\rm lp}
                                    \p{\uv{l} \cdot \uv{l}_{\rm p}}
                                    \p{\uv{l} \times \uv{l}_{\rm p}},\\
                            \omega_{\rm lp}
                                &= \frac{3m_{\rm p}}{4M_\star}
                                \p{\frac{a}{a_{\rm p}}}^3 n.
                        \end{align*}
                \end{itemize}

                \item Equilibria (Cassini States) depend on $\eta \equiv
                    \omega_{\rm lp} / \omega_{\rm sl} \propto \Omega_{\rm
                    s}^{-1}$. \end{itemize}
        \end{column}
        \begin{column}{0.5\columnwidth}
            \begin{figure}
                \centering
                \includegraphics[width=0.65\columnwidth]{../../initial/99_misc/2_3vec_new.png}
                \includegraphics[width=0.65\columnwidth]{../../initial/99_misc/2_cs_locs_phi.png}
            \end{figure}
        \end{column}
    \end{columns}
\end{frame}

\begin{frame}
    \frametitle{Dynamics}
    \framesubtitle{Weak Tidal Friction}

    \begin{columns}
        \begin{column}{0.5\columnwidth}
            \begin{itemize}
                \item \textbf{Objective:} Introduce dissipation into the system.
                    For a given $\bm{s}_{\rm i}$ and $\Omega_{\rm s, i}$, what
                    is the final outcome?

                \item Weak tidal friction: encourages (i) spin-orbit alignment,
                    and (ii) spin-orbit synchronization:
                \begin{align*}
                    \p{\rd{\uv{s}}{t}}_{\rm tide} &= \frac{1}{t_{\rm s}}
                        \textcolor{gray}{\s{\frac{2n}{\Omega_{\rm s}} - \p{\uv{s} \cdot \uv{l}}}
                        \uv{s} \times \p{\uv{l} \times
                        \uv{s}}},\\
                    \frac{1}{\Omega_{\rm s}}\p{\rd{\Omega_{\rm s}}{t}}_{\rm
                            tide}
                        &= \frac{1}{t_{\rm s}} \textcolor{gray}
                        {\s{\frac{2n}{\Omega_{\rm
                        s}}\p{\uv{s} \cdot \uv{l}} - 1 - \p{\uv{s} \cdot
                        \uv{l}}^2}}.
                \end{align*}
            \end{itemize}
        \end{column}
        \begin{column}{0.5\columnwidth}
            \begin{figure}
                \centering
                \includegraphics[width=\columnwidth]{../../initial/1_weaktide/6equils0_06_notraj.png}
            \end{figure}
        \end{column}
    \end{columns}
\end{frame}

\begin{frame}
    \frametitle{Dynamics}
    \framesubtitle{Weak Tidal Friction}

    \begin{columns}
        \begin{column}{0.5\columnwidth}
            \begin{itemize}
                \item \textbf{Objective:} Introduce dissipation into the system.
                    For a given $\bm{s}_{\rm i}$ and $\Omega_{\rm s, i}$, what
                    is the final outcome?

                \item Weak tidal friction: encourages (i) spin-orbit alignment,
                    and (ii) spin-orbit synchronization:
                \begin{align*}
                    \p{\rd{\uv{s}}{t}}_{\rm tide} &= \frac{1}{t_{\rm s}}
                        \textcolor{gray}{\s{\frac{2n}{\Omega_{\rm s}} -
                        \p{\uv{s} \cdot \uv{l}}} \uv{s} \times \p{\uv{l} \times
                        \uv{s}}},\\
                    \frac{1}{\Omega_{\rm s}}\p{\rd{\Omega_{\rm s}}{t}}_{\rm
                            tide}
                        &= \frac{1}{t_{\rm s}} \textcolor{gray}
                        {\s{\frac{2n}{\Omega_{\rm
                        s}}\p{\uv{s} \cdot \uv{l}} - 1 - \p{\uv{s} \cdot
                        \uv{l}}^2}}.
                \end{align*}
            \end{itemize}
        \end{column}
        \begin{column}{0.5\columnwidth}
            \begin{figure}
                \centering
                \includegraphics[width=\columnwidth]{../../initial/1_weaktide/6equils0_06.png}
            \end{figure}
        \end{column}
    \end{columns}
\end{frame}

\begin{frame}
    \frametitle{Dynamics}
    \framesubtitle{Outcome Probability}
    \begin{columns}
        \begin{column}{0.5\columnwidth}
            \begin{itemize}
                \item Choose the initial conditions $\Omega_{\rm s, i} = 10n$,
                    and $\uv{s}_{\rm i}$ isotropically distributed.

                \item \emph{What is the tCE2 probability?}

                \item Depends on the parameter
                    \begin{equation*}
                        \eta_{\rm sync} \equiv \eta \frac{\Omega_{\rm s}}{n}
                            \propto \Omega_{\rm s}^0
                    \end{equation*}

                \item Analytically, is approximately
                \begin{align}
                    P_{\rm tCE2} &\simeq
                            \frac{4\sqrt{\eta _{\rm sync} \sin I}}{\pi}
                                f\p{\frac{\Omega_{\rm s, i}}{n}}.
                \end{align}
            \end{itemize}
        \end{column}
        \begin{column}{0.5\columnwidth}
            \begin{figure}
                \centering
                \includegraphics[width=\columnwidth]{../../initial/1_weaktide/5probs_20.png}
            \end{figure}
        \end{column}
    \end{columns}
\end{frame}

\begin{frame}
    \frametitle{Applications}
    \framesubtitle{SE + HJ}

    \begin{columns}
        \begin{column}{0.5\columnwidth}
            \begin{itemize}
                \item Zhu \& Wu 2018 find that many SEs have CJ companions.

                \item SE may have experienced giant impacts, $\sim$ isotropic
                    $\uv{s}$.

                \item Spin of SE evolves under tidal interactions with host
                    star ($t_{\rm s} \sim \scinot{3}{7}\;\mathrm{yr}$).

                \item SE also experiences Cassini State dynamics:
                    {\small\begin{align*}
                        \eta_{\rm sync} ={}& 0.303 \cos I
                                \p{\frac{k}{k_{\rm q}}}
                                \p{\frac{m_{\rm p}}{M_{\rm J}}}\nonumber\\
                            &\times \p{\frac{m}{4M_{\oplus}}}
                                \p{\frac{M_\star}{M_{\odot}}}^{-2}
                                \p{\frac{a}{0.4\;\mathrm{AU}}}^{6}\nonumber\\
                            &\times \p{\frac{a_{\rm p}}{5\;\mathrm{AU}}}^{-3}
                                \p{\frac{R}{2R_{\oplus}}}^{-3}.
                    \end{align*}}

                \item Maybe many high-obliquity SEs!
            \end{itemize}
        \end{column}
        \begin{column}{0.5\columnwidth}
            \begin{figure}
                \centering
                \includegraphics[width=\columnwidth]{../../initial/1_weaktide/5probs_5.png}
            \end{figure}
        \end{column}
    \end{columns}
\end{frame}

% \begin{frame}
%     \frametitle{Applications}
%     \framesubtitle{Ultra-Short-Period Planet (USP) Formation}

%     \begin{columns}
%         \begin{column}{0.5\columnwidth}
%             \begin{itemize}
%                 \item USPs have $P \lesssim 1\;\mathrm{day}$ and cannot have
%                     formed \emph{in situ} (e.g.\ low-$e$ migration, Pu
%                     \& Lai 2018).

%                 \item Millholland \& Spalding 2020 propose obliquity
%                     tides-driven migration:
%                     \begin{equation*}
%                         \frac{\dot{a}}{a} \propto
%                                 \p{1 - \frac{\Omega_{\rm s}}{n}\cos \theta}
%                     \end{equation*}

%                 \item In tCE1, $\cos \theta \approx \Omega_{\rm s} / n \approx
%                     1$, so $\dot{a}$ is small.

%                 \item In tCE2, $\Omega_{\rm s} / n, \cos \theta \lesssim 1$:
%                     \textbf{enhanced migration}!

%                 \item $\eta_{\rm sync} \sim 0.06$ when $P_{j + 1} / P_j = 1.5$.
%             \end{itemize}
%         \end{column}
%         \begin{column}{0.5\columnwidth}
%             \begin{figure}
%                 \centering
%                 \includegraphics[width=\columnwidth]{../../initial/1_weaktide/7anal_ptce.png}
%             \end{figure}

%             \begin{itemize}
%                 \item \textbf{2nd Caveat:} When $a$ is sufficiently small, tides
%                     breaks the tCE\@. We find $a_{\rm break} \simeq
%                     0.032\;\mathrm{AU}$, or $P = 2.7\;\mathrm{day}$, no USP\@!
%             \end{itemize}
%         \end{column}
%     \end{columns}
% \end{frame}

\end{document}

