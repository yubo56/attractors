    \documentclass[11pt,
        usenames, % allows access to some tikz colors
        dvipsnames % more colors: https://en.wikibooks.org/wiki/LaTeX/Colors
    ]{article}
    \usepackage{
        amsmath,
        amssymb,
        fancyhdr, % page styling
        lastpage, % footer fanciness
        hyperref, % various links
        setspace, % line spacing
        amsthm, % newtheorem and proof environment
        mathtools, % \Aboxed for boxing inside aligns, among others
        float, % Allow [H] figure env alignment
        enumerate, % Allow custom enumerate numbering
        graphicx, % allow includegraphics with more filetypes
        wasysym, % \smiley!
        upgreek, % \upmu for \mum macro
        listings, % writing TrueType fonts and including code prettily
        tikz, % drawing things
        booktabs, % \bottomrule instead of hline apparently
        xcolor, % colored text
        cancel % can cancel things out!
    }
    \usepackage[margin=1in]{geometry} % page geometry
    \usepackage[
        labelfont=bf, % caption names are labeled in bold
        font=scriptsize % smaller font for captions
    ]{caption}
    \usepackage[font=scriptsize]{subcaption} % subfigures

    \newcommand*{\scinot}[2]{#1\times10^{#2}}
    \newcommand*{\dotp}[2]{\left<#1\,\middle|\,#2\right>}
    \newcommand*{\rd}[2]{\frac{\mathrm{d}#1}{\mathrm{d}#2}}
    \newcommand*{\pd}[2]{\frac{\partial#1}{\partial#2}}
    \newcommand*{\rdil}[2]{\mathrm{d}#1 / \mathrm{d}#2}
    \newcommand*{\pdil}[2]{\partial#1 / \partial#2}
    \newcommand*{\rtd}[2]{\frac{\mathrm{d}^2#1}{\mathrm{d}#2^2}}
    \newcommand*{\ptd}[2]{\frac{\partial^2 #1}{\partial#2^2}}
    \newcommand*{\md}[2]{\frac{\mathrm{D}#1}{\mathrm{D}#2}}
    \newcommand*{\pvec}[1]{\vec{#1}^{\,\prime}}
    \newcommand*{\svec}[1]{\vec{#1}\;\!}
    \newcommand*{\bm}[1]{\boldsymbol{\mathbf{#1}}}
    \newcommand*{\uv}[1]{\hat{\bm{#1}}}
    \newcommand*{\ang}[0]{\;\text{\AA}}
    \newcommand*{\mum}[0]{\;\upmu \mathrm{m}}
    \newcommand*{\at}[1]{\left.#1\right|}
    \newcommand*{\bra}[1]{\left<#1\right|}
    \newcommand*{\ket}[1]{\left|#1\right>}
    \newcommand*{\abs}[1]{\left|#1\right|}
    \newcommand*{\ev}[1]{\left\langle#1\right\rangle}
    \newcommand*{\p}[1]{\left(#1\right)}
    \newcommand*{\s}[1]{\left[#1\right]}
    \newcommand*{\z}[1]{\left\{#1\right\}}

    \newtheorem{theorem}{Theorem}[section]

    \let\Re\undefined
    \let\Im\undefined
    \DeclareMathOperator{\Res}{Res}
    \DeclareMathOperator{\Re}{Re}
    \DeclareMathOperator{\Im}{Im}
    \DeclareMathOperator{\Log}{Log}
    \DeclareMathOperator{\Arg}{Arg}
    \DeclareMathOperator{\Tr}{Tr}
    \DeclareMathOperator{\E}{E}
    \DeclareMathOperator{\Var}{Var}
    \DeclareMathOperator*{\argmin}{argmin}
    \DeclareMathOperator*{\argmax}{argmax}
    \DeclareMathOperator{\sgn}{sgn}
    \DeclareMathOperator{\diag}{diag\;}

    \colorlet{Corr}{red}

    % \everymath{\displaystyle} % biggify limits of inline sums and integrals
    \tikzstyle{circ} % usage: \node[circ, placement] (label) {text};
        = [draw, circle, fill=white, node distance=3cm, minimum height=2em]
    \definecolor{commentgreen}{rgb}{0,0.6,0}
    \lstset{
        basicstyle=\ttfamily\footnotesize,
        frame=single,
        numbers=left,
        showstringspaces=false,
        keywordstyle=\color{blue},
        stringstyle=\color{purple},
        commentstyle=\color{commentgreen},
        morecomment=[l][\color{magenta}]{\#}
    }

\begin{document}

\tableofcontents

\section{06/02/21---Simple Parameterized Model}

\subsection{Orbital Dynamics}

We consider the 3-planet case with an inner test mass, as Dong did, for
simplicity. This is not quantitatively exact but gives us an intuitive
parameterization (to leading order) of realistic dynamics.

Consider the mutual precession of an inner test particle and two outer planets.
Define the complex inclination $\mathcal{I}_j = I_j\exp\p{i\Omega_j}$, then
these evolve following
\begin{align}
    \rd{}{t}\begin{bmatrix}
        \mathcal{I}_1\\
        \mathcal{I}_2\\
        \mathcal{I}_3
    \end{bmatrix} &=
        \begin{bmatrix}
            -\omega_{21} - \omega_{31} & \omega_{21} & \omega_{31}\\
            0 & -\omega_{32} & \omega_{32}\\
            0 & \omega_{23} & -\omega_{23}
        \end{bmatrix}
    \begin{bmatrix}
        \mathcal{I}_1\\
        \mathcal{I}_2\\
        \mathcal{I}_3
    \end{bmatrix},\label{eq:06/02/21.LL_matrix}\\
    \omega_{kj}
        &= -\frac{3 m_>}{4 M_\star}\p{\frac{a_>}{a_<}}^3 n_<
            f\p{\frac{a_<}{a_>}}
            \times \min\p{\frac{L_k}{L_j}, 1},\\
    f\p{\alpha} &= \frac{b_{3/2}^{(1)}}{3\alpha}\nonumber\\
        &= \frac{1}{3\alpha}
            \frac{1}{\pi}\int\limits_0^{2\pi}
                \frac{\cos \p{t}}{
                \p{\alpha^2 + 1 - 2\alpha \cos t}^{3/2}}\;\mathrm{d}t\nonumber\\
        &\approx 1 + \frac{15}{8}\alpha^2 + \mathcal{O}\p{\alpha^4}.
\end{align}
Note that this $\min$ term means e.g.\ that $\omega_{23} =
\omega_{32}\frac{L_2}{L_3}$.

We seek the eigenmodes of Eq.~\eqref{eq:06/02/21.LL_matrix}. It is expected that
there will only be two: a third is if $\mathcal{I}_i = 0$, i.e.\ if everything
is aligned with the total angular momentum. Otherwise:
\begin{itemize}
    \item The first obvious, non-trivial eigenmode is $\mathcal{I}_2 =
        \mathcal{I}_3 = 0$, then $\mathcal{I}_1$ executes free precession with
        the precession frequency $g_1 \equiv -\omega_{21} - \omega_{31}$.

    \item We expect the second eigenmode to describe precession of
        $\mathcal{I}_2$ and $\mathcal{I}_3$ about their total angular momentum
        axis. Upon inspection, we find that the second eigenvector must indeed
        satisfy:
        \begin{equation}
            \mathcal{I}_2 = -\frac{L_3}{L_2}\mathcal{I}_3.
        \end{equation}
        The eigenvalue can also be directly read off in this case to be
        $\omega_{32}\p{-1 - L_2/L_3} \approx -\omega_{32}(J / L_3) \equiv g_2$,
        where $J \approx L_2 + L_3$. Then, we can find that the corresponding
        component of $\mathcal{I}_1$ must satisfy
        \begin{align}
            g_1\mathcal{I}_1
                + \omega_{21}\mathcal{I}_2 + \omega_{31}\mathcal{I}_3
                &= g_2\mathcal{I}_1,\\
            \p{g_2 - g_1}\mathcal{I}_1 &=
                \p{\omega_{21}\frac{L_3}{J}
                    + \omega_{31}\frac{L_2}{J}}\mathcal{I}_{23},\\
            \mathcal{I}_1 &= \frac{\p{\omega_{21}L_3
                    + \omega_{31}L_2} / J}{g_2 - g_1}\mathcal{I}_{23}.
        \end{align}
\end{itemize}
Here, $\mathcal{I}_{23} = \mathcal{I}_3 - \mathcal{I}_2$ and corresponds to the
complexified mutual inclination. Thus, the general solution to
$\mathcal{I}_1(t)$ is given by:
\begin{equation}
    \mathcal{I}_1(t) = i_1e^{ig_1t} + i_{\rm 1f}e^{i(g_2t + \phi_0)},
\end{equation}
where $i_1$, $i_{\rm 1f}$, and $\phi_0$ are real. Then, of course, we can
construct the vector solution to $\uv{l}_1$ via:
\begin{align}
    \uv{l}_1 &= \Re \mathcal{I}_1\uv{x}
            + \Im \mathcal{I}_1 \uv{y}
            + \sqrt{1 - \abs{\mathcal{I}_1}^2}
            \uv{z},\label{eq:06/02/21.l1param}\\
        &= \begin{bmatrix}
            i_1 \cos (g_1t) + i_{\rm 1f} \cos\p{g_2 t + \phi_0}\\
            i_1 \sin (g_1t) + i_{\rm 1f} \sin\p{g_2 t + \phi_0}\\
            \cos\abs{\mathcal{I}_1}.\nonumber
        \end{bmatrix}
\end{align}
Note that $i_{\rm 1f}$ is effectively set by the initial $\mathcal{I}_{23}$
(overlap with the eigenvector), and $i_{1}$ is set by whatever remaining IC is
not described by the forced component.

\subsection{Spin Dynamics}

We now have the parameterized form for $\uv{l}_1$ following
Eq.~\eqref{eq:06/02/21.l1param} and the solution for $\mathcal{I}_1$, and the
spin evolves following
\begin{equation}
    \rd{\uv{s}}{t} = \alpha\p{\uv{s} \cdot \uv{l}_1}\p{\uv{s} \times \uv{l}_1}.
\end{equation}

\subsubsection{Numerical Approach}

We seek to understand the stability of CS2 in the regime $\alpha \gg g_1, g_2$.
WLOG, we choose to perturb about the initial condition where $\uv{s}$ points
exactly along CS2, and $i_{\rm 1f} = 0$ (as is in the case of aligned outer
planets; note that if the inner planet is not a perfect test mass, aligned outer
planets does not completely suppress this mode). There are then three
perturbations we can make:
\begin{itemize}
    \item We can increase $i_{\rm 1f}$; the two relevant regimes are expected to
        be where $i_{\rm 1f} \ll i_1$ and $i_{\rm 1f} \gg i_1$.

    \item We can change $g_2$; the two relevant regimes are expected to be where
        $g_2 \gg g_1$ and $g_2 \ll g_1$, with possible resonance when they are
        close (since $i_{\rm 1f}$ is prescribed explicitly and held constant, we
        can imagine that the mutual inclination is decreased in proportion to
        $g_{2} - g_1$ such that $i_{\rm 1f}$ is constant).

    \item We can perturb the initial system about the initial CS2; we do this by
        adding an angle offset $\Delta \theta_{\rm i}$ to the IC, so that there
        is a nonzero initial libration amplitude.
\end{itemize}

\subsubsection{Analytical Approach}

It seems like we might be seeing some resonance type behaviors. For convenience,
we go to the co-rotating frame with $g_1$ where we then have
\begin{equation}
    \p{\rd{\uv{s}}{t}}_{\rm rot, 1}
        = \alpha\p{\uv{s} \cdot \uv{l}_1}\p{\uv{s} \times \uv{l}_1}
            + g_1\p{\uv{s} \cdot \uv{J}}.
\end{equation}
Here, now, the mode frequencies are $g_1 \Rightarrow 0$ and $g_2 \Rightarrow g_2
- g_1$. Decompose then $\uv{l}$ (we drop the subscript) into a mean and fluctuating
piece $\bar{\bm{l}}$ and $\bm{l}'$ (where the former is still a vector of unit
length), then:
\begin{equation}
    \p{\rd{\uv{s}}{t}}_{\rm rot, 1}
        = \alpha\p{\uv{s} \cdot \bar{\bm{l}}}\p{\uv{s} \times \bar{\bm{l}}}
            + g_1\p{\uv{s} \cdot \uv{J}}
            + \alpha\s{
                \p{\uv{s} \cdot \bar{\bm{l}}}
                \p{\uv{s} \times \bm{l}'} +
                \p{\uv{s} \cdot \bm{l}'}
                \p{\uv{s} \times \bar{\bm{l}}}
            }.
\end{equation}
This is already looking painful. Let's write down the Hamiltonian to have a
faster path to EOM in coordinates:
\begin{align}
    H_{\rm rot, 1} ={}& -\frac{\alpha}{2}\p{\uv{s} \cdot \uv{l}}^2
            - g\p{\uv{s} \cdot \uv{J}},\\
        ={}& -\frac{\alpha}{2}\p{\uv{s} \cdot \bar{\bm{l}}}^2
            - \alpha\p{\uv{s} \cdot \bar{\bm{l}}}
                \p{\uv{s} \cdot \bm{l}'}
            - g\p{\uv{s} \cdot \uv{J}},\\
        \approx{}& -\frac{\alpha}{2}\cos^2\theta
            - g\p{\cos \theta \cos I - \sin I \sin \theta \cos \phi}\nonumber\\
        & - \alpha \cos \theta\p{
            \cos \theta \cos i_{\rm 1f}
            - \sin i_{\rm 1f} \sin \theta \cos \p{\phi - \phi_0 - \p{g_2 -
            g_1}t}}.
\end{align}
Is this right, or is $i_{\rm 1f}$ also changing?

\end{document}

