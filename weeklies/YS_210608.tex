    \documentclass[11pt,
        usenames, % allows access to some tikz colors
        dvipsnames % more colors: https://en.wikibooks.org/wiki/LaTeX/Colors
    ]{article}
    \usepackage{
        amsmath,
        amssymb,
        fouriernc, % fourier font w/ new century book
        fancyhdr, % page styling
        lastpage, % footer fanciness
        hyperref, % various links
        setspace, % line spacing
        amsthm, % newtheorem and proof environment
        mathtools, % \Aboxed for boxing inside aligns, among others
        float, % Allow [H] figure env alignment
        enumerate, % Allow custom enumerate numbering
        graphicx, % allow includegraphics with more filetypes
        wasysym, % \smiley!
        upgreek, % \upmu for \mum macro
        listings, % writing TrueType fonts and including code prettily
        tikz, % drawing things
        booktabs, % \bottomrule instead of hline apparently
        xcolor, % colored text
        cancel % can cancel things out!
    }
    \usepackage[margin=1in]{geometry} % page geometry
    \usepackage[
        labelfont=bf, % caption names are labeled in bold
        font=scriptsize % smaller font for captions
    ]{caption}
    \usepackage[font=scriptsize]{subcaption} % subfigures

    \newcommand*{\scinot}[2]{#1\times10^{#2}}
    \newcommand*{\dotp}[2]{\left<#1\,\middle|\,#2\right>}
    \newcommand*{\rd}[2]{\frac{\mathrm{d}#1}{\mathrm{d}#2}}
    \newcommand*{\pd}[2]{\frac{\partial#1}{\partial#2}}
    \newcommand*{\rdil}[2]{\mathrm{d}#1 / \mathrm{d}#2}
    \newcommand*{\pdil}[2]{\partial#1 / \partial#2}
    \newcommand*{\rtd}[2]{\frac{\mathrm{d}^2#1}{\mathrm{d}#2^2}}
    \newcommand*{\ptd}[2]{\frac{\partial^2 #1}{\partial#2^2}}
    \newcommand*{\md}[2]{\frac{\mathrm{D}#1}{\mathrm{D}#2}}
    \newcommand*{\pvec}[1]{\vec{#1}^{\,\prime}}
    \newcommand*{\svec}[1]{\vec{#1}\;\!}
    \newcommand*{\bm}[1]{\boldsymbol{\mathbf{#1}}}
    \newcommand*{\uv}[1]{\hat{\bm{#1}}}
    \newcommand*{\ang}[0]{\;\text{\AA}}
    \newcommand*{\mum}[0]{\;\upmu \mathrm{m}}
    \newcommand*{\at}[1]{\left.#1\right|}
    \newcommand*{\bra}[1]{\left<#1\right|}
    \newcommand*{\ket}[1]{\left|#1\right>}
    \newcommand*{\abs}[1]{\left|#1\right|}
    \newcommand*{\ev}[1]{\left\langle#1\right\rangle}
    \newcommand*{\p}[1]{\left(#1\right)}
    \newcommand*{\s}[1]{\left[#1\right]}
    \newcommand*{\z}[1]{\left\{#1\right\}}

    \newtheorem{theorem}{Theorem}[section]

    \let\Re\undefined
    \let\Im\undefined
    \DeclareMathOperator{\Res}{Res}
    \DeclareMathOperator{\Re}{Re}
    \DeclareMathOperator{\Im}{Im}
    \DeclareMathOperator{\Log}{Log}
    \DeclareMathOperator{\Arg}{Arg}
    \DeclareMathOperator{\Tr}{Tr}
    \DeclareMathOperator{\E}{E}
    \DeclareMathOperator{\Var}{Var}
    \DeclareMathOperator*{\argmin}{argmin}
    \DeclareMathOperator*{\argmax}{argmax}
    \DeclareMathOperator{\sgn}{sgn}
    \DeclareMathOperator{\diag}{diag\;}

    \colorlet{Corr}{red}

    % \everymath{\displaystyle} % biggify limits of inline sums and integrals
    \tikzstyle{circ} % usage: \node[circ, placement] (label) {text};
        = [draw, circle, fill=white, node distance=3cm, minimum height=2em]
    \definecolor{commentgreen}{rgb}{0,0.6,0}
    \lstset{
        basicstyle=\ttfamily\footnotesize,
        frame=single,
        numbers=left,
        showstringspaces=false,
        keywordstyle=\color{blue},
        stringstyle=\color{purple},
        commentstyle=\color{commentgreen},
        morecomment=[l][\color{magenta}]{\#}
    }

\begin{document}

\onehalfspacing

\pagestyle{fancy}
\rfoot{Yubo Su}
\rhead{}
\cfoot{\thepage/\pageref{LastPage}}

\title{Yubo Weekly June 8, 2021}

\maketitle

We analyze Dong's equations, where:
\begin{align}
    \rd{\uv{s}}{t} &= \alpha\p{\uv{s} \cdot \uv{l}_1}\p{\uv{s} \times
        \uv{l}_1},\\
    \uv{l}_1 &= \Re \mathcal{I}_1\uv{x}
            + \Im \mathcal{I}_1 \uv{y}
            + \sqrt{1 - \abs{\mathcal{I}_1}^2}
            \uv{z},\label{eq:06/02/21.l1param}\\
        &= \begin{bmatrix}
            i_1 \cos (g_1t) + i_{\rm 1f} \cos\p{g_2 t + \phi_0}\\
            i_1 \sin (g_1t) + i_{\rm 1f} \sin\p{g_2 t + \phi_0}\\
            \cos\abs{\mathcal{I}_1}
        \end{bmatrix}.
\end{align}
I've taken $\phi_0 = 0$ for simplicity. We take $\alpha / g_1 = 10$
consistently. We permit $i_{\rm 1f}$ and $g_{1,2}$ to vary independently for
now, even though $i_{\rm 1f} \propto \p{g_2 - g_1}^{-1}$.

\section{Preliminary Numerical Work}

We analyze the dynamics of $\phi_{\rm rot}$, the azimuthal angle in the
co-rotating frame. This lets us both understand the locations of equilibria
($\phi_{\rm rot} \approx \phi_{\rm rot, 0}$) and the distortion of the
separatrix (circulation-libration transition). We suppress the suffix. In
particular, we consider the quantity:
\begin{equation}
    \Delta \phi\p{i_{\rm 1f}; g_2; \Delta \theta_{\rm i}}
        \equiv \phi_{\max} - \phi_{\min}.
\end{equation}
Here, we fix the initial $\phi_{\rm rot} = 180^\circ$, and $\Delta \theta_{\rm
i}$ is the angular distance from CS2. For instance, $\Delta \phi\p{0, 0, 0} =
0$. We show a few plots in Figs.~\ref{fig:scan_I2}--\ref{fig:scan_dq_res}.
\begin{figure}[h]
    \centering
    \includegraphics[width=0.7\columnwidth]{../initial/4nplanet/3paramsim/scan_I2.png}
    \caption{Scanning $\Delta \phi$ over $i_{\rm 1f}$. Note that $i_1 =
    10^\circ$ for all of these.}\label{fig:scan_I2}
\end{figure}
\begin{figure}[h]
    \centering
    \includegraphics[width=0.7\columnwidth]{../initial/4nplanet/3paramsim/scan_frequency.png}
    \caption{Scanning $\Delta \phi$ over $g_2$.}\label{fig:scan_dq_res}
\end{figure}
\begin{figure}[h]
    \centering
    \includegraphics[width=0.7\columnwidth]{../initial/4nplanet/3paramsim/scan_dq.png}
    \caption{Scanning $\Delta \phi$ over $\Delta \theta_{\rm
    i}$.}\label{fig:scan_dq}
\end{figure}
\begin{figure}[h]
    \centering
    \includegraphics[width=0.7\columnwidth]{../initial/4nplanet/3paramsim/scan_dq_res.png}
    \caption{Scanning $\Delta \phi$ over $\Delta \theta_{\rm
    i}$ for many frequencies near the $g_2 = 2g_1$ resonance.}\label{fig:scan_dq_res}
\end{figure}

It is clear that there are some resonances for $g_1 \simeq g_2$! In fact, there
are some interesting behaviors. If we examine the phase space evolution (in
$\cos \theta, \phi$ space as usual) of ICs near CS2 as we vary $g_2 / g_1$, we
obtain:
\begin{figure}
    \centering
    \includegraphics[width=0.45\columnwidth]{../initial/4nplanet/3paramsim/phase_portrait0.png}
    \includegraphics[width=0.45\columnwidth]{../initial/4nplanet/3paramsim/phase_portrait01.png}

    \includegraphics[width=0.45\columnwidth]{../initial/4nplanet/3paramsim/phase_portrait1.png}
    \includegraphics[width=0.45\columnwidth]{../initial/4nplanet/3paramsim/phase_portrait2.png}

    \includegraphics[width=0.45\columnwidth]{../initial/4nplanet/3paramsim/phase_portrait3.png}
    \includegraphics[width=0.45\columnwidth]{../initial/4nplanet/3paramsim/phase_portrait10.png}
    \caption{Evolution of some ICs for $g_2 / g_1 = \z{0, 0.1, 1, 2, 3, 10}$.
    Note that for $g_2 = 0$, we've set $i_{\rm 1f} = 0^\circ$, else we set
    $i_{\rm f} = 1^\circ$, and $i_1 = 10^\circ$.
    }\label{fig:phase_portraits}
\end{figure}

\section{Analytical Approach}

It is clear that there are some resonance behaviors, and in particular that $g_2
/ g_1 = 1$ is \emph{not} a resonance; it is suspiciously quiet (see
Fig.~\ref{fig:phase_portraits}). Here is a simple explanation:

Consider the Hamiltonian
\begin{equation}
    H = - \frac{\alpha}{2}\p{\uv{s} \cdot \uv{l}}^2.
\end{equation}
Decompose $\uv{l} = \bar{\bm{l}} + \bm{l}'$ where:
\begin{align}
    \bar{\bm{l}} &= \begin{bmatrix}
        i_1 \cos (g_1t)
        i_1 \sin (g_1t)
        \cos i_1
    \end{bmatrix},\\
    \bm{l}' &\equiv \uv{l} - \bar{\bm{l}},\\
        &\approx \begin{bmatrix}
        i_{\rm 1f} \cos\p{g_2 t + \phi_0}\\
        i_{\rm 1f} \sin\p{g_2 t + \phi_0}\\
        -\frac{i_{\rm 1f}i_1}{\cos^2 i_1} \cos\p{g_2t + \phi_0}
    \end{bmatrix}.
\end{align}
Note that $l'_z \approx 0$.

We next go to the corotating frame with $g_1$ about $\uv{\jmath}$, so that
\begin{align}
    H_{\rm rot} &= - \frac{\alpha}{2}\p{\uv{s} \cdot \uv{l}}^2
        - g\p{\uv{s} \cdot \uv{\jmath}},\\
        &\equiv H_0 + H_1,\\
    H_0 &\equiv - \frac{\alpha}{2}\p{\uv{s} \cdot \bar{\bm{l}}}^2
        - g\p{\uv{s} \cdot \uv{\jmath}},\\
    H_1 &\approx -\alpha\p{\uv{s} \cdot \bar{\bm{l}}}
        \p{\uv{s} \cdot \bm{l}'} + \mathcal{O}\p{(l')^2}.
\end{align}
In the corotating frame, we find that
\begin{equation}
    \p{\bm{l}'}_{\rm rot} \approx \begin{bmatrix}
        i_{\rm 1f} \cos\p{(g_2 - g_1) t + \phi_0}\\
        i_{\rm 1f} \sin\p{(g_2 - g_1) t + \phi_0}\\
        0
    \end{bmatrix}.
\end{equation}
Thus, in the corotating frame, when $g_2 = g_1$, CS2 is simply \emph{shifted},
since $\bm{l}'_{\rm rot}$ is fixed in space. This isn't entirely realistic, as
$i_{\rm 1f} \propto \p{g_2 - g_1}^{-1}$.

\section{Numerical Searching for Resonant Behavior (WIP)}

We look for resonances by fixing all frequencies of the system and varying
$\Delta \theta_{\rm i}$, proximity to CS2.

\textbf{What are the origin of any resonances?} My hypothesis was that if $g_2 -
g_1 \simeq \omega_{\rm lib}$, the libration frequency about CS2, then resonances
can appear. Thus, we calculate $\omega_{\rm lib}$ as a function of $\Delta
\theta_{\rm i}$ numerically.

\textbf{Separately, is evolution chaotic?} For every $\Delta \theta_{\rm i}$, we
can compute the evolution for $\phi_{\rm i} = 0$ and $\phi_{\rm i} = 10^{-5}$,
then record
\begin{equation}
    \max \Delta \uv{s} \equiv \max_t \abs{\uv{s}_1 - \uv{s}_2}.
\end{equation}
If $\max \Delta \uv{s} \simeq 1$, then chaos is likely (I spot checked that
these seemed to correspond to the case of exponential growth).

We can plot both of these criteria in Fig.~\ref{fig:resonances}. It seems like a
plausible explanation that resonances occur when $g_2 - g_1 \simeq \omega_{\rm
lib}$, and that these resonances give rise to chaotic behavior (likely when
interacting with the separatrix).
\begin{figure}
    \centering
    \includegraphics[width=0.4\columnwidth]{../initial/4nplanet/3paramsim/resonances.png}
    \includegraphics[width=0.4\columnwidth]{../initial/4nplanet/3paramsim/resonances13.png}

    \includegraphics[width=0.4\columnwidth]{../initial/4nplanet/3paramsim/resonances15.png}
    \includegraphics[width=0.4\columnwidth]{../initial/4nplanet/3paramsim/resonances3.png}
    \caption{Top panels: $\omega_{\rm lib}$ (in the absence of any perturbation)
    as a function of distance to CS2. Horizontal line is the naive $2\pi /
    \p{g_2 - g_1}$. Bottom: $\max \Delta \uv{s}$, where large values suggest
    chaos. $i_1 = 10^\circ$ and $i_{\rm 1f} = 1^\circ$ as usual.}\label{fig:resonances}
\end{figure}

\end{document}

