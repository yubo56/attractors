    \documentclass[11pt,
        usenames, % allows access to some tikz colors
        dvipsnames % more colors: https://en.wikibooks.org/wiki/LaTeX/Colors
    ]{article}
    \usepackage{
        amsmath,
        amssymb,
        fouriernc, % fourier font w/ new century book
        fancyhdr, % page styling
        lastpage, % footer fanciness
        hyperref, % various links
        setspace, % line spacing
        amsthm, % newtheorem and proof environment
        mathtools, % \Aboxed for boxing inside aligns, among others
        float, % Allow [H] figure env alignment
        enumerate, % Allow custom enumerate numbering
        graphicx, % allow includegraphics with more filetypes
        wasysym, % \smiley!
        upgreek, % \upmu for \mum macro
        listings, % writing TrueType fonts and including code prettily
        tikz, % drawing things
        booktabs, % \bottomrule instead of hline apparently
        xcolor, % colored text
        cancel % can cancel things out!
    }
    \usepackage[margin=1in]{geometry} % page geometry
    \usepackage[
        labelfont=bf, % caption names are labeled in bold
        font=scriptsize % smaller font for captions
    ]{caption}
    \usepackage[font=scriptsize]{subcaption} % subfigures

    \newcommand*{\scinot}[2]{#1\times10^{#2}}
    \newcommand*{\dotp}[2]{\left<#1\,\middle|\,#2\right>}
    \newcommand*{\rd}[2]{\frac{\mathrm{d}#1}{\mathrm{d}#2}}
    \newcommand*{\pd}[2]{\frac{\partial#1}{\partial#2}}
    \newcommand*{\rdil}[2]{\mathrm{d}#1 / \mathrm{d}#2}
    \newcommand*{\pdil}[2]{\partial#1 / \partial#2}
    \newcommand*{\rtd}[2]{\frac{\mathrm{d}^2#1}{\mathrm{d}#2^2}}
    \newcommand*{\ptd}[2]{\frac{\partial^2 #1}{\partial#2^2}}
    \newcommand*{\md}[2]{\frac{\mathrm{D}#1}{\mathrm{D}#2}}
    \newcommand*{\pvec}[1]{\vec{#1}^{\,\prime}}
    \newcommand*{\svec}[1]{\vec{#1}\;\!}
    \newcommand*{\bm}[1]{\boldsymbol{\mathbf{#1}}}
    \newcommand*{\uv}[1]{\hat{\bm{#1}}}
    \newcommand*{\ang}[0]{\;\text{\AA}}
    \newcommand*{\mum}[0]{\;\upmu \mathrm{m}}
    \newcommand*{\at}[1]{\left.#1\right|}
    \newcommand*{\bra}[1]{\left<#1\right|}
    \newcommand*{\ket}[1]{\left|#1\right>}
    \newcommand*{\abs}[1]{\left|#1\right|}
    \newcommand*{\ev}[1]{\left\langle#1\right\rangle}
    \newcommand*{\p}[1]{\left(#1\right)}
    \newcommand*{\s}[1]{\left[#1\right]}
    \newcommand*{\z}[1]{\left\{#1\right\}}

    \newtheorem{theorem}{Theorem}[section]

    \let\Re\undefined
    \let\Im\undefined
    \DeclareMathOperator{\Res}{Res}
    \DeclareMathOperator{\Re}{Re}
    \DeclareMathOperator{\Im}{Im}
    \DeclareMathOperator{\Log}{Log}
    \DeclareMathOperator{\Arg}{Arg}
    \DeclareMathOperator{\Tr}{Tr}
    \DeclareMathOperator{\E}{E}
    \DeclareMathOperator{\Var}{Var}
    \DeclareMathOperator*{\argmin}{argmin}
    \DeclareMathOperator*{\argmax}{argmax}
    \DeclareMathOperator{\sgn}{sgn}
    \DeclareMathOperator{\diag}{diag\;}

    \colorlet{Corr}{red}

    % \everymath{\displaystyle} % biggify limits of inline sums and integrals
    \tikzstyle{circ} % usage: \node[circ, placement] (label) {text};
        = [draw, circle, fill=white, node distance=3cm, minimum height=2em]
    \definecolor{commentgreen}{rgb}{0,0.6,0}
    \lstset{
        basicstyle=\ttfamily\footnotesize,
        frame=single,
        numbers=left,
        showstringspaces=false,
        keywordstyle=\color{blue},
        stringstyle=\color{purple},
        commentstyle=\color{commentgreen},
        morecomment=[l][\color{magenta}]{\#}
    }

\begin{document}

\section{3-Planet Cassini State Modeling}

\subsection{Code Description}

I have a code that solves the following equations for $N$ planets, $1 \leq i
\leq N$:
\begin{align}
    \rd{\uv{s}_i}{t}
        &= \alpha_i \p{\uv{s}_i \cdot \uv{l}_i}\p{\uv{s}_i \times \uv{l}_i}
            + \epsilon_{\rm tide}\p{\uv{s}_i \times \p{\uv{l}_i \times \uv{s}_i}
                },\\
    \rd{\uv{l}_i}{t}
        &= \sum\limits_{j \neq i}
            \omega_{ji} \p{\uv{l}_j \cdot \uv{l}_i}
                \p{\uv{l}_j \cdot \uv{l}_i},\\
    \alpha_i &= \frac{3k_{\rm q}}{2k}
        \frac{M_\star}{m_i}
        \p{\frac{R_i}{a_i}}^3
        \Omega_{\rm s, i},\\
    \omega_{ji} &= \frac{3m_>}{4M_\star}
        \p{\frac{a_<}{a_>}}^3
        n_<
        f(\alpha) \times
        \begin{cases}
            1 & j > i\\
            \frac{L_j}{L_i} & j < i.
        \end{cases},\\
    \rd{\Omega_{\rm s, i}}{t} &= \epsilon_{\rm spin} \Omega_{\rm s, i}.
\end{align}
These vector+spin equations are evolved in an inertial frame. Define $\uv{j}$ to
be aligned with the total angular momentum axis, then we define the coordinate
system for each possible CS of each spin:
\begin{align}
    \uv{Z}_i &= \uv{l}_i,\\
    \uv{X}_i &\propto \uv{j} - \p{\uv{j} \cdot \uv{l}_i}\uv{l}_i,\\
    \uv{Y}_i &= \uv{Z}_i \times \uv{X}_i,\\
    \cos \theta_i &= \uv{s}_i \cdot \uv{Z}_i,\\
    \tan \phi_i &= \frac{\uv{s}_i \cdot \uv{X}_i}{\uv{s}_i \cdot \uv{Y}_i}.
\end{align}
In other words, $\phi_i$ is defined in the frame \emph{corotating} with
$\uv{l}_i$ about $\uv{j}$.

\subsection{Dynamics}

Note that there are two options to introduce dynamics: either $\epsilon_{\rm
spin}$ (not a physical model; we will use both spinup and spindown), or
$\epsilon_{\rm tide}$, an alignment torque. For each set of parameters, we will
show four groups of panels: one with no dynamics, one with only $\epsilon_{\rm
spin}$, one with only $\epsilon_{\rm tide}$, then one with both an
$\epsilon_{\rm spin}$ and an $\epsilon_{\rm tide}$. In all cases, the $\epsilon
\sim 10^{-3}\;\mathrm{/yr}$.
\begin{figure}
    \centering
    \includegraphics[width=0.48\columnwidth]{../initial/4nplanet/2_3p_mode1.png}
    \includegraphics[width=0.48\columnwidth]{../initial/4nplanet/2_dissipative1.png}

    \includegraphics[width=0.48\columnwidth]{../initial/4nplanet/2_dynamical1.png}
    \includegraphics[width=0.48\columnwidth]{../initial/4nplanet/2_dynamicaltide1.png}
    \caption{Case 1. Modes are as labeled in Panel 4. Corresponding CS2
    obliquities are labeled in Panel 2.}\label{fig:1}
\end{figure}
\begin{figure}
    \centering
    \includegraphics[width=0.48\columnwidth]{../initial/4nplanet/2_3p_mode1_2.png}
    \includegraphics[width=0.48\columnwidth]{../initial/4nplanet/2_dissipative1_2.png}

    \includegraphics[width=0.48\columnwidth]{../initial/4nplanet/2_dynamical1_2.png}
    \includegraphics[width=0.48\columnwidth]{../initial/4nplanet/2_dynamicaltide1_2.png}
    \caption{Case 2, offset from Case 1 by $\phi_{\rm i} = 180^\circ$.}\label{fig:2}
\end{figure}
\begin{figure}
    \centering
    \includegraphics[width=0.32\columnwidth]{../initial/4nplanet/2_3p_mode2.png}
    \includegraphics[width=0.32\columnwidth]{../initial/4nplanet/2_dynamical2.png}
    \includegraphics[width=0.32\columnwidth]{../initial/4nplanet/2_dynamicaltide2.png}
    \caption{Case 3; dissipative case is missing.}\label{fig:5}
\end{figure}
\begin{figure}
    \centering
    \includegraphics[width=0.32\columnwidth]{../initial/4nplanet/2_3p_mode3.png}
    \includegraphics[width=0.32\columnwidth]{../initial/4nplanet/2_dynamical3.png}
    \includegraphics[width=0.32\columnwidth]{../initial/4nplanet/2_dynamicaltide3.png}
    \caption{Case 4, offset from case 3 by $\phi_{\rm i} = 180^\circ$;
    dissipative case is missing.}\label{fig:6}
\end{figure}
\begin{figure}
    \centering
    \includegraphics[width=0.32\columnwidth]{../initial/4nplanet/2_no_spinup3.png}
    \includegraphics[width=0.32\columnwidth]{../initial/4nplanet/2_spinup3.png}
    \includegraphics[width=0.32\columnwidth]{../initial/4nplanet/2_spinupdisp3.png}
    \caption{Spinup case; pure dissipative case is missing}\label{fig:3}
\end{figure}
\begin{figure}
    \centering
    \includegraphics[width=0.32\columnwidth]{../initial/4nplanet/2_no_spinup4.png}
    \includegraphics[width=0.32\columnwidth]{../initial/4nplanet/2_spinup4.png}
    \includegraphics[width=0.32\columnwidth]{../initial/4nplanet/2_spinupdisp4.png}
    \caption{Another spinup case; pure dissipative case is missing}\label{fig:4}
\end{figure}

Among these, there are a few major takeaways:
\begin{itemize}
    \item There is a limit cycle where the angle $\phi_{\rm inertial} + g_>t =
        \phi_{\rm rot} - g_<t$ is circulating (see e.g.\ third group of panels
        in Fig.~\ref{fig:1}). I suspect that this is an interesting limit cycle,
        but may not strictly be a ``resonance''? I'm not sure.

    \item It seems to be hard to be caught in the $g_<$ resonance when both
        $\eta$'s are initially small: compare Figs.~\ref{fig:1}--\ref{fig:2}. On
        the other hand, when the large $\eta$ is already large, the resonances
        are well separated, and it's possible to be advected along, see
        Fig.~\ref{fig:5}.

    \item During spinup, parameters can be finely tuned to be ``caught'' by
        either resonance (Figs.~\ref{fig:3}--\ref{fig:4}).
\end{itemize}
However, no resonant angle appears during these resonance advection phases\dots
why is that? Only the ``limit cycle'' seems to have a resonant angle, but it
doesn't seem to advect at all.

\end{document}

